%Master File:lectures.tex

\lesson{Make it Generic}
\vspace*{-2cm}
\begin{center}
  \includegraphics[height=11cm]{swissarmy}
\end{center}
\vspace*{-1cm}
\keywords{Inheritance, generic programming, Java Collection classes, STL}

%%%%%%%%%%%%%%%%%%%%%%% Next Slide %%%%%%%%%%%%%%%%%%%%%%%

\renewcommand{\Outline}{%
\begin{slide}
\section{Outline}
\begin{minipage}{10cm}
  \vfill
  \begin{enumerate}
    \outlineitem{Generic Programming}{intro}
    \outlineitem{Old Style}{old}
    \outlineitem{New Style}{new}
    \outlineitem{Collection Classes and STL}{collections}
  \end{enumerate}
  \vfill
\end{minipage}\hfill
\begin{minipage}{10cm}
  \includegraphics[width=10cm]{swissarmy}
\end{minipage}
\end{slide}
\addtocounter{outlineitem}{1}
}
\setcounter{outlineitem}{1}
\Outline
\toptarget{firstoutline}

%%%%%%%%%%%%%%%%%%%%%%% Next Slide %%%%%%%%%%%%%%%%%%%%%%%

\begin{slide}
\section{Why Generic Programming?}

\begin{PauseHighLight}
  \begin{itemize}
\item We don't want to write a data structure for every type
  \begin{itemize}
  \item IntArrayList
  \item DoubleArrayList
  \item WidgetArrayList
  \end{itemize}\pause
\item Nor do we want to have different sort algorithms for integers, doubles,
  strings, etc.\pause
  \item Want to create data structures and algorithms which can work for
    any object type\pause
  \end{itemize}
\end{PauseHighLight}

\end{slide}


%%%%%%%%%%%%%%%%%%%%%%% Next Slide %%%%%%%%%%%%%%%%%%%%%%%

\Outline
%%%%%%%%%%%%%%%%%%%%%%% Next Slide %%%%%%%%%%%%%%%%%%%%%%%

\begin{slide}
\section[-1]{Generics Through Inheritance}

\begin{PauseHighLight}
  \begin{itemize}
  \item One way to implement generics is using inheritance hierarchies\pause
  \item Inheritance is used to model the \emph{is a} relationship
  \end{itemize}
  \begin{center}
    \includegraphics[width=0.9\linewidth]{Object}\pause
    \llap{\includegraphics[width=0.9\linewidth]{Object1}}\pauseb
  \end{center} 
\end{PauseHighLight}
\end{slide}

%%%%%%%%%%%%%%%%%%%%%%% Next Slide %%%%%%%%%%%%%%%%%%%%%%%

\begin{slide}
\section{Inheriting Methods}
\vspace{-2cm}
\begin{PauseHighLight}
  \begin{itemize}\squeeze
  \item A \texttt{Square} is a \texttt{Rectangle} which is a
    \texttt{Shape} which is a \texttt{Object}\pause
  \item \texttt{Square} may have more member functions than
    \texttt{Rectangle} and \texttt{Shape}\pause
  \item The implementation for the member function of \texttt{Square}
    may be different from \texttt{Rectangle} and \texttt{Shape}\pause
  \item The classes \texttt{Shape} and \texttt{Regular} may not contain
    implementations of all their methods in which case they are
    \emph{abstract classes}\pause
  \item A class that contains no implementations is an
    \emph{interface}\pause
  \end{itemize}
\end{PauseHighLight}

\end{slide}

%%%%%%%%%%%%%%%%%%%%%%% Next Slide %%%%%%%%%%%%%%%%%%%%%%%

\begin{slide}
\section{Polymorphism}

\begin{PauseHighLight}
  \begin{itemize}
  \item We can create an instance of a \texttt{Shape}
    \begin{java}
      Shape shape1 = new Square();
    \end{java}
    (note that you can only call \texttt{new} on a concrete
    class---i.e. a non-abstract class)\pause
  \item We can then call any method of \texttt{Shape}
    (e.g. \jl$area()$) ---this will call
    the method of that name for \texttt{Square}\pause
  \item We can also convert \texttt{Shape} back to an \texttt{Square}
    using a `cast'
    \begin{java}
      Square square = (Square) shape1;
    \end{java}\pause
  \end{itemize}
\end{PauseHighLight}

\end{slide}


%%%%%%%%%%%%%%%%%%%%%%% Next Slide %%%%%%%%%%%%%%%%%%%%%%%

\begin{slide}
\section{Object}
\vspace{-2cm}
\begin{PauseHighLight}
  \begin{itemize}
  \item In Java all classes inherit the property of \jl$Object$\pause
  \item This allows us to create a generic container\pause
  \item We did this in the previous lecture
\begin{java}
public class MyArrayList implements MyList
{
    protected Object[] data;
    protected int no_elements;

    public MyArrayList()
    {
        data = new Object[10];
        no_elements = 0;
    }

    // etc.
}
\end{java}\pause
  \end{itemize}
\end{PauseHighLight}
\end{slide}

%%%%%%%%%%%%%%%%%%%%%%% Next Slide %%%%%%%%%%%%%%%%%%%%%%%

\begin{slide}
\section{Comparable}

\begin{PauseHighLight}
  \begin{itemize}
  \item To perform \texttt{sort} we have to compare two objects to see if
    the lhs is `less than', `equal to', or `greater than' the rhs\pause
  \item Java defined an interface \jl$Comparable$ with the single method
    \begin{java}
      public int compareTo(Object rhs)
    \end{java}
    which returns a negative number, zero or a positive number if the
    lhs is `less than', 'equal to', or `greater than' the rhs\pause
  \item Sort can be carried out on any type that implements the
    \jl$Comparable$ interface\pause
  \end{itemize}
\end{PauseHighLight}
\end{slide}

%%%%%%%%%%%%%%%%%%%%%%% Next Slide %%%%%%%%%%%%%%%%%%%%%%%

\begin{slide}
\section{Wrapper Classes}

\begin{PauseHighLight}
  \begin{itemize}
  \item Primitive types, \texttt{byte, short, int, long, float, double,
      char} and \texttt{boolean} aren't classes and so aren't
    \texttt{Object}\pause
  \item To create collections of these types Java has Wrapper classes,
    \texttt{Integer, Double}, etc.\ which are classes and store
    primitive types\pause
  \item Java 1.5+ performs `autoboxing and unboxing' which converts from
    \texttt{Integer} to \texttt{int} automatically, etc.\ and the other
    way around without using casts\pause
  \end{itemize}
\end{PauseHighLight}
\end{slide}

%%%%%%%%%%%%%%%%%%%%%%% Next Slide %%%%%%%%%%%%%%%%%%%%%%%

\begin{slide}
\section{Disadvantage with Using Inheritance}

\vspace{-2cm}
\begin{PauseHighLight}
  \begin{itemize}
  \item Java is a strongly typed programming language\pause
  \item This means that all object know what type they are at compile
    time\pause
  \item This catches many errors before a program runs\pause
  \item Deep inheritance breaks strong typing\pause
  \item Errors that should be caught at compile time become run
    time errors
\begin{java}
        MyList a = new MyArrayList();
        a.add(new Double(2.3));
        if (blueMoon)
            Integer b = (Integer) a.get(0);  // causes a runtime error
\end{java}\pause
  \end{itemize}
\end{PauseHighLight}
\end{slide}


%%%%%%%%%%%%%%%%%%%%%%% Next Slide %%%%%%%%%%%%%%%%%%%%%%%

\Outline
%%%%%%%%%%%%%%%%%%%%%%% Next Slide %%%%%%%%%%%%%%%%%%%%%%%

\begin{slide}
\section{Generics Programming}

\begin{PauseHighLight}
  \begin{itemize}
  \item Java 1.5 onwards allows you to write generic classes\pause
  \item \jl$public class MyClass<T>$
    \begin{itemize}
    \item Depends on some unspecified type \jl$T$
    \item \jl$MyClass<String>$ creates an instance of a specific class
    \end{itemize}\pause
  \item Follows similar syntax to \textit{templates} in C++\pause
  \end{itemize}
\end{PauseHighLight}

\end{slide}

%%%%%%%%%%%%%%%%%%%%%%% Next Slide %%%%%%%%%%%%%%%%%%%%%%%

\begin{slide}
\section{A Generic List}

\begin{java}
public interface GenericList<T>
{
    public void add(T obj);
    // post: add new element to the end of List

    public T get(int index);
    // pre: 0 <= index < size()
    // post: return element in location given by index

    public void set(int index, T obj);
    // pre: 0 <= index < size()
    // post: set element in location given by index to obj

    public T removeLast();
    // pre: non-empty List
    // post: return last element and remove this from List

    public int size();
    //post: return size of List
}
\end{java}

\end{slide}


%%%%%%%%%%%%%%%%%%%%%%% Next Slide %%%%%%%%%%%%%%%%%%%%%%%

\begin{slide}
\section{A Generic ArrayList}
\pb
\begin{java}
public class GenericArrayList<T> implements GenericList<T>
{
    private T[] data;
    private int no_elements;

    $\pause$
    @SuppressWarnings("unchecked") $\pause \pauselevel{=1}$
    public GenericArrayList()
    {
        data = (T[]) new Object[10]; // unchecked cast
        no_elements = 0;
    }

  // etc.
}
\end{java}\pause

\end{slide}

%%%%%%%%%%%%%%%%%%%%%%% Next Slide %%%%%%%%%%%%%%%%%%%%%%%

\begin{slide}
\section{Using Generics}

\vspace*{-2cm}
\begin{PauseHighLight}
  \begin{itemize}\squeeze
  \item To make a class generic add \jl$<T>$ to its name\pause---we
    could use any identifier rather than  \jl$T$\pause
  \item Replace \jl$Object$ by the identifier \jl$T$\pause
  \item You cannot create an array of type \jl$T$ so we create an array
    of its bound (i.e. \jl$Object$) and then cast to \jl$T$, e.g.
    \begin{java}
      data = (T[]) new Object[10];
    \end{java}\pause
    \vspace*{-1cm}
  \item You still can't create generics for primitive types\pause
  \item You can also write member functions that act on a class with an
    unspecified type\pause
  \item We achieve this using a wildcard \jl$<?>$, e.g.
    \begin{java}
      public static double doSomething(ArrayList<?> arr);
    \end{java}\pause
    \vspace*{-1cm}
  \end{itemize}
\end{PauseHighLight}
\end{slide}

%%%%%%%%%%%%%%%%%%%%%%% Next Slide %%%%%%%%%%%%%%%%%%%%%%%


\begin{slide}
\section{Bounded Wildcard}

\begin{PauseHighLight}
  \begin{itemize}
  \item We may require the type to extend some other type\pause
  \item Suppose we want to compute the total area of some shapes
\begin{java}
  public static double totalArea(ArrayList<? extends Shape> arr)
  {
    double total = 0.0;

    for(Shape s: arr)
        total += s.area();

    return total;
  }$\pause$
\end{java}
  \item If \jl$Rectangle$ extends the \jl$Shape$ interface this
    would work for \jl$ArrayList<Rectangle>$\pause
  \end{itemize}
\end{PauseHighLight}
\end{slide}


%%%%%%%%%%%%%%%%%%%%%%% Next Slide %%%%%%%%%%%%%%%%%%%%%%%

\begin{slide}
\section[-1.5]{Generic Methods Type Name}

\begin{PauseHighLight}
  \begin{itemize}
  \item Sometimes we need an identifier for the generic type in a
    method\pause
  \item This can be important when
    \begin{itemize}
    \item The type is used in more than one parameter type
    \item The type is used as a return type
    \item The type is used to declare a local variable\pause
    \end{itemize}
  \item We can create a generic method using
    \begin{java}
      public static <T> double myMethod(T arg1, T arg2) {
        // do something
      }$\pause$

      public static <T> T firstElement(List<T> list) {
        return list.get(0);
      }$\pause$
    \end{java}
  \end{itemize}
\end{PauseHighLight}
\end{slide}

%%%%%%%%%%%%%%%%%%%%%%% Next Slide %%%%%%%%%%%%%%%%%%%%%%%

\begin{slide}
\section[-1.5]{Contains}

\begin{PauseHighLight}
Here is a generic \texttt{contains} method
\begin{java}
public class ArrayStuff
{
    public static <T> boolean contains(T[] arr, T x)
    {
        for (T val: arr)
            if ( x.equals(val) )
                return true;
        
        return false;
    }
}$\pause$
\end{java}
so that
\begin{java}
        Integer[] arr = {1, 4 ,7, 2};
        System.out.println(ArrayStuff.contains(arr, 4));
        System.out.println(ArrayStuff.contains(arr, 5));  
\end{java}
returns \texttt{true} and \texttt{false} \pause
\end{PauseHighLight}
\end{slide}


%%%%%%%%%%%%%%%%%%%%%%% Next Slide %%%%%%%%%%%%%%%%%%%%%%%

\begin{slide}
\section[-1]{Difficult Type}

\begin{PauseHighLight}
  \begin{itemize}
  \item Occasionally generics become a bit complicated\pause
  \item Suppose we want to implement a \texttt{findMax} method that works on
    any type
\begin{java}
    public static <T> T findMax(T[] arr)
    {
        int maxIndex = 0;
        for(int i=1; i< arr.length; i++)
            if (arr[i].compareTo(arr[maxIndex])>0)
                maxIndex = i;

        return arr[maxIndex];
    }$\pause$      
\end{java}
  \item Won't compile because doesn't know that type \texttt{T}
  implements the Comparable interface\pause
  \end{itemize}
\end{PauseHighLight}

\end{slide}

%%%%%%%%%%%%%%%%%%%%%%% Next Slide %%%%%%%%%%%%%%%%%%%%%%%

\begin{slide}
\section[-1]{Type Bound}

\begin{PauseHighLight}
  \begin{itemize}
  \item We need an array of \texttt{Comparable}s
\begin{java}
public static <T extends Comparable<T>> T findMax(T[] arr)
\end{java}\pause\vspace*{-3ex}
\item Works when
  \begin{center}
    \includegraphics[width=0.3\linewidth]{findmax}\pause
  \end{center}\vspace*{-2ex}
\item But fails if
  \begin{center}
    \includegraphics[width=0.3\linewidth]{findmax1}\pause
  \end{center}\vspace*{-2ex}
\item Need
\begin{java}
public static <T extends Comparable<? super T>> T findMax(T[] arr)$\pause$
\end{java}
  \end{itemize}
\end{PauseHighLight}

\end{slide}

%%%%%%%%%%%%%%%%%%%%%%% Next Slide %%%%%%%%%%%%%%%%%%%%%%%

\begin{slide}
\section[-2]{Get and Put Principles}

\begin{PauseHighLight}
  \begin{itemize}
  \item The wildcard \jl$<?>$ allows any type\pause
  \item \jl$<? extends T>$\hfill\includegraphics[width=0.3\linewidth]{extends}
    \begin{itemize}
    \item used when we want to \emph{get} a value from a structure
    \item we need the type \jl$?$ to implement a method of type
      \jl$T$
    \item \jl$A extends B$ if \jl$A$ is a subclass of \jl$B$ (\jl$A$ might
      extend or implement \jl$B$)\pause
    \end{itemize}
  \item \jl$<? super T>$\hfill\includegraphics[width=0.2\linewidth]{super}
    \begin{itemize}
    \item used when we want to \emph{put} a value into a structure\pause
    \end{itemize}
  \end{itemize}
\begin{java}
public static <T> void copy(List<? super T> dest, List<? extends T> src)
\end{java}\pause
\end{PauseHighLight}

\end{slide}



%%%%%%%%%%%%%%%%%%%%%%% Next Slide %%%%%%%%%%%%%%%%%%%%%%%

\begin{slide}
\section{C++ Templates}

\begin{PauseHighLight}
  \begin{itemize}
  \item C++ templates look like Java generics\pause
  \item They are implemented very differently\pause
    \begin{itemize}
    \item In Java generics uses old style inheritance behind the scenes
      (erasure)\pause
    \item In C++ a template defines how a class should look---for each
      instance of the class with a different template parameter a
      different class is compiled (you can't pre-compile template
      classes)\pause
    \end{itemize}
  \item C++ templates work with primitive types\pause
  \item C++ doesn't use any kind of inheritance\pause
  \end{itemize}
\end{PauseHighLight}

\end{slide}


%%%%%%%%%%%%%%%%%%%%%%% Next Slide %%%%%%%%%%%%%%%%%%%%%%%

\Outline
%%%%%%%%%%%%%%%%%%%%%%% Next Slide %%%%%%%%%%%%%%%%%%%%%%%

\begin{slide}
\section{Don't roll your own}

\begin{PauseHighLight}
  \begin{itemize}
  \item Most language provide commonly used data structures and
    algorithms\pause
  \item Java has the Collection Framework\pause
  \item C++ has the Standard Template Library STL\pause
  \item These are efficient, well tested, container classes\pause
  \item Use them!\pause
  \end{itemize}
\end{PauseHighLight}
\end{slide}

%%%%%%%%%%%%%%%%%%%%%%% Next Slide %%%%%%%%%%%%%%%%%%%%%%%

\begin{slide}
\section{Java Collection Framework}

\begin{PauseHighLight}
  \begin{itemize}
  \item The Java Collection Framework provides many classes including
    \begin{itemize}
    \item \jl$ArrayList<T>$---A list implemented by an array\pause
    \item \jl$Vector<T>$---Another list implemented by an array
      (depreciated)\pause
    \item \jl$LinkedList<T>$---A list implemented by a linked list\pause
    \item \jl$TreeSet<T>$---A set implemented by a binary tree\pause
    \item \jl$HashSet<T>$---A set implemented by a hash table\pause
    \item \jl$TreeMap<K,V>$---A map implemented by a binary tree\pause
    \item \jl$HashMap<K,V>$---A map implemented by a hash table\pause
    \end{itemize}
  \item Also have bunch of algorithms in \jl$Collections$\pause
  \end{itemize}
\end{PauseHighLight}

\end{slide}

%%%%%%%%%%%%%%%%%%%%%%% Next Slide %%%%%%%%%%%%%%%%%%%%%%%

\begin{slide}
\section{Collection Interface}

\vspace{-1cm}
\begin{center}
  \includegraphics[height=14cm]{java-collection}
\end{center}
\vspace{-1cm}
\end{slide}


%%%%%%%%%%%%%%%%%%%%%%% Next Slide %%%%%%%%%%%%%%%%%%%%%%%

\begin{slide}
\section{Collection Classes}
\begin{center}
  \includegraphics[width=\linewidth]{collection}
\end{center}
\begin{PauseHighLight}
  \begin{itemize}
  \item Shows some of the interfaces and classes\pause
  \item There are 5 \jl$BlockingQueues<T>$'s, a \jl$PriorityQueue<T>$, and
    \jl$ConcurrentLinkedQueue<T>$\pause
  \item  There are lots of intermediate abstract classes\pause
  \end{itemize}
\end{PauseHighLight}

\vspace{-1cm}

\vspace{-1cm}
\end{slide}

%%%%%%%%%%%%%%%%%%%%%%% Next Slide %%%%%%%%%%%%%%%%%%%%%%%

\begin{slide}
\section{Lists and Sets}
\pb
\pause
\begin{center}
  \includegraphics[height=14cm]{java-list}\mypl{1}\pause
  \llap{\includegraphics[height=14cm]{java-set}}\mypl{2}
\end{center}
\end{slide}

%%%%%%%%%%%%%%%%%%%%%%% Next Slide %%%%%%%%%%%%%%%%%%%%%%%

\begin{slide}
\section{Maps}
\vspace{-1cm}
\begin{center}
  \includegraphics[height=14cm]{java-map}
\end{center}
\vspace{-1cm}
\end{slide}

%%%%%%%%%%%%%%%%%%%%%%% Next Slide %%%%%%%%%%%%%%%%%%%%%%%

\begin{slide}
\section[-1.5]{STL}
\begin{PauseHighLight}
  \begin{itemize}
  \item C++ provides an equivalent set of classes known as the Standard
    Template Library\pause
  \item Includes
    \begin{itemize}
    \item \jl$vector<T>$---dynamic resizable array
    \item \jl$deque<T>$---a ``deck'' or double ended array
    \item \jl$list<T>$---a doubly linked list
    \item \jl$set<T>$ and \jl$multiset<T>$---a set and multiset
      implemented by a binary tree
    \item \jl$map<K,V>$ and \jl$multimap<K,V>$--a map and multimap
      implemented by a binary tree
    \item \jl$unordered_X$ hash based \jl$set$, \jl$multiset$,
      \jl$map$,  \jl$multimap$
    \item queues, stacks, priority queues, \ldots
    \end{itemize}\pause
  \end{itemize}
\end{PauseHighLight}
\end{slide}

%%%%%%%%%%%%%%%%%%%%%%% Next Slide %%%%%%%%%%%%%%%%%%%%%%%

\begin{slide}
\section[-2]{Lessons}

\begin{PauseHighLight}
  \begin{itemize}\squeeze
  \item For data structures and algorithms to be really useful they must
    be made generic\pause
  \item You can make classes generic using inheritance (i.e. by making
    them act on \texttt{Object})\pause
  \item This is bad form because it breaks strong typing---errors are
    missed at compile time\pause
  \item Java 1.5 introduced a new way to make things generic ``generic
    classes''\pause
  \item Generic classes are simple to implement (although there are a
    few subtleties to master)\pause
  \item Generic classes for common data structures and algorithms exist
    both for Java and C++\pause
  \end{itemize}
\end{PauseHighLight}
\end{slide}
