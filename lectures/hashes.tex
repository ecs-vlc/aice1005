%Master File:lectures.tex

\lesson{Make a hash of it}

\vspace{-2cm}
\begin{center}
  \includegraphics[height=10cm]{hashsign}
\end{center}

\keywords{Hash tables, separate chaining, open addressing, linear/quadratic probing, double hashing}

%%%%%%%%%%%%%%%%%%%%%%% Next Slide %%%%%%%%%%%%%%%%%%%%%%%
\renewcommand{\Outline}{%
\begin{slide}
\section{Outline}

\begin{minipage}{10cm}
  \vfill
  \begin{enumerate}
    \outlineitem{Why Hash?}{hash}
    \outlineitem{Separate Chaining}{sc}
    \outlineitem{Open Addressing}{oa}
    \begin{itemize}
    \item \toplink{qprob}{Quadratic Probing}
    \item \toplink{dhash}{Double Hashing}
    \end{itemize}
    \outlineitem{Hash Set and Map}{hshm}
  \end{enumerate}
  \vfill
\end{minipage}\hfill
\begin{minipage}{10cm}
  \includegraphics[width=8cm]{hashsign}
\end{minipage}
\end{slide}
\addtocounter{outlineitem}{1}
}

\setcounter{outlineitem}{1}
\Outline
\toptarget{firstoutline}
%%%%%%%%%%%%%%%%%%%%%%% Next Slide %%%%%%%%%%%%%%%%%%%%%%%

\begin{slide}
\section{Content Addressable Memory}

\begin{PauseHighLight}
  \begin{itemize}
  \item Suppose we have a list of objects which we want to look up
    according to its contents\pause
  \item This is often referred to as \emph{associative memory}
    structures\pause
  \item A classical example would be a telephone directory\pause
    \begin{itemize}
    \item We look up a name
    \item We want to know the number\pause
    \end{itemize}
  \item What data structure should we use?\pause
  \end{itemize}
\end{PauseHighLight}

\end{slide}

%%%%%%%%%%%%%%%%%%%%%%% Next Slide %%%%%%%%%%%%%%%%%%%%%%%

\begin{slide}
\section{Lists and Trees}

\begin{PauseHighLight}
  \begin{itemize}
  \item To find an entry in a normal list takes $\Theta(n)$
    operations\pause 
  \item If we had a sorted list we could use ``binary search'' to reduce
    this to $\Theta(\log(n))$\pause
    \begin{itemize}
    \item We will study binary search later\pause
    \item Maintaining an ordered list is costly ($\Theta(n)$ insertions)\pause
    \end{itemize}
  \item We could use a binary search tree\pause
    \begin{itemize}
    \item Search is $\Theta(\log(n))$\pause
    \item Insertion/deletion is $\Theta(\log(n))$\pause
    \end{itemize}
  \end{itemize}
\end{PauseHighLight}

\end{slide}

%%%%%%%%%%%%%%%%%%%%%%% Next Slide %%%%%%%%%%%%%%%%%%%%%%%

\begin{slide}
\section[-1.5]{Thinking Outside the Box}

\begin{PauseHighLight}
  \begin{itemize}
  \item As with many data structures thinking about the problem
    differently can lead to much better solutions\pause
  \item Let us consider the content we want to search on as a
    \emph{key}\pause
  \item For telephone numbers the key would be the name of the person we
    want to phone\pause
  \item We could get $O(1)$ search, insertion and deletion if we used
    the key as an index into a big array\pause
  \item That is the key is a string of, say, 100 characters so can be
    represented by an 800 digit binary number\pause
  \item We could look up the key in a table of $2^{800}$ items\pause
  \end{itemize}
\end{PauseHighLight}

\end{slide}

%%%%%%%%%%%%%%%%%%%%%%% Next Slide %%%%%%%%%%%%%%%%%%%%%%%

\begin{slide}
\section[-2]{Hashing}

\pausebuild
\color{TwoColor}
\begin{itemize}
\item This approach is slightly wasteful of memory\pauseh
\item Almost all memory locations would be empty\pauseh
\item We can save on memory by folding up the table up onto
  itself\pauseh
\end{itemize}
\begin{center}
  \begin{picture}(220,100)\pauselevel{=1 :4}
    \put(0,0){\includegraphics[width=22cm]{nonhash}}\pause\pauselevel{=5}
    \put(0,0){\includegraphics[width=22cm]{hashed}}\pause
  \end{picture}
\end{center}

\end{slide}


%%%%%%%%%%%%%%%%%%%%%%% Next Slide %%%%%%%%%%%%%%%%%%%%%%%

\begin{slide}
\section{Hashing Codes}

\begin{PauseHighLight}
  \begin{itemize}
  \item A \emph{hashing function} \texttt{hashCode(x)} takes an object,
    \texttt{x}, and returns a positive integer, the \emph{hash code}\pause
  \item To turn the hash code into an address take the modulus of the
    table size
    \begin{java}
      int index = abs(hashCode(x) % tableSize);$\pause$
    \end{java}
  \item If \texttt{tableSize} $=2^n$ we can compute this more
    efficiently using a mask
    \begin{java}
      int index = abs(hashCode(x) & (tableSize -1));$\pause$
    \end{java}
  \end{itemize}
\end{PauseHighLight}
\end{slide}

%%%%%%%%%%%%%%%%%%%%%%% Next Slide %%%%%%%%%%%%%%%%%%%%%%%

\begin{slide}
\section[-2]{Hashing Functions}

\begin{PauseHighLight}
  \begin{itemize}\squeeze
  \item Hashing functions take an object and return an integer\pause
  \item Hashing functions aren't magic\pause
    \begin{itemize}
    \item They tend to add up integers representing the parts of the
      object\pause
    \end{itemize}
  \item We want the integers to be close to random so that similar
    objects are mapped to different integers\pause
  \item Sometimes two objects will be mapped to the same
    address\pause---this is known as a \emph{collision}\pause
  \item Collision resolution is an important part of hashing\pause
  \end{itemize}
\end{PauseHighLight}

\end{slide}

%%%%%%%%%%%%%%%%%%%%%%% Next Slide %%%%%%%%%%%%%%%%%%%%%%%

\begin{slide}
\section{Hashing Strings}

\begin{PauseHighLight}
  \begin{itemize}
  \item A strings might be hashed using a function\pause
    \begin{cpp}
      unsigned long long hash(string const& s) {
        unsigned long long results = 12345;
        
        for (auto ch = s.begin(); ch != s.end(); ++ch) {
          results = 127*results + static_cast<unsigned char>(*ch);
        }
        return results;
      }$\pause$
    \end{cpp}
  \item The numbers 12345 and 127 is to try to prevent
    clashes\pause---there are lots of alternatives\pauseb
  \item What we want is that strings that might be similar receive very
    different hash codes\pauseb
  \end{itemize}
\end{PauseHighLight}

\end{slide}

%%%%%%%%%%%%%%%%%%%%%%% Next Slide %%%%%%%%%%%%%%%%%%%%%%%

\begin{slide}
\section{DIY}

\begin{PauseHighLight}
  \begin{itemize}
  \item The \texttt{unordered\_set<T, Hash<T> >} allows you to define
    your own hash function\pause
  \item By default this is set to \texttt{std::hash<T>(T)}\pause
  \item Not all classes have hash function defined so you will need to
    do this\pause
  \item Care is needed to make you hash function produce near random
    hash codes\pause
  \end{itemize}
\end{PauseHighLight}

\end{slide}


%%%%%%%%%%%%%%%%%%%%%%% Next Slide %%%%%%%%%%%%%%%%%%%%%%%
\Outline % Separate Chaining
%%%%%%%%%%%%%%%%%%%%%%% Next Slide %%%%%%%%%%%%%%%%%%%%%%%

\begin{slide}
\section{Collision Resolution}

\begin{PauseHighLight}
  \begin{itemize}
  \item Collisions are inevitable and must be dealt with\pause
  \item There are two commonly used strategies\pause
    \begin{itemize}
    \item Separate chaining---make a hash table of lists\pause
    \item Open addressing---find a new position in the hash table\pause
    \end{itemize}
  \item Collisions add computational cost\pause
  \item They occur when the hash table becomes full\pause
  \item If the hash table becomes too full then it may need to be
    resized\pause
  \end{itemize}
\end{PauseHighLight}

\end{slide}

%%%%%%%%%%%%%%%%%%%%%%% Next Slide %%%%%%%%%%%%%%%%%%%%%%%

\begin{slide}
\section[-1]{Resizing a Hash Table}

\begin{PauseHighLight}
  \begin{itemize}
  \item Resizing a hash table is easy\pause
    \begin{itemize}
    \item Create a new hash table of, say, twice the size\pause
    \item Iterate through the old hash table adding each element to the
      new hash table\pause
    \end{itemize}
  \item Note that you have to recompute all the hash codes\pause
  \item Resizing a hash table has a modest amortised cost, but can give
    you a very hiccupy performance\pause
  \item The size of a hash table is a classic example of a memory-space
    versus execution time trade off\pause---using bigger (sparser) hash
    tables speeds up performance\pause
  \end{itemize}
\end{PauseHighLight}

\end{slide}


%%%%%%%%%%%%%%%%%%%%%%% Next Slide %%%%%%%%%%%%%%%%%%%%%%%

\begin{slide}
\section{Separate Chaining}

\pausebuild
\color{TextColor}
\begin{itemize}
\item In separate chaining we build a singly-linked list at each table
  entry\pauseh
\end{itemize}
\begin{center}
  \multiinclude[graphics={width=\linewidth}]{hash-anim}\pause
\end{center}
\end{slide}

%%%%%%%%%%%%%%%%%%%%%%% Next Slide %%%%%%%%%%%%%%%%%%%%%%%

\begin{slide}
\section[-1]{Search}

\begin{PauseHighLight}
  \begin{itemize}
  \item To find an entry in a hash table we again use the hash function
    on a key to find the table entry and then we search the list\pause
  \item The time complexity depends on where objects are hashed\pause
  \item If the objects are evenly dispersed in the table, search (and
    insertion) is $\Omega(1)$\pause
  \item If the objects are hashed to the same entry in the hash table
    then search is $O(n)$\pause
  \item Provided you have a good hashing function and the hash table
    isn't too full you can expect $\Theta(1)$ average case performance\pause
  \end{itemize}
\end{PauseHighLight}

\end{slide}

%%%%%%%%%%%%%%%%%%%%%%% Next Slide %%%%%%%%%%%%%%%%%%%%%%%

\begin{slide}
\section{Iterating Over a Hash Table}

\begin{PauseHighLight}
  \begin{minipage}{12cm}
    \raggedright
  \begin{itemize}
  \item To iterate over a hash table we
    \begin{itemize}
    \item Iterate through the array\pause
    \item At each element we iterate through the linked list\pause
    \end{itemize}
  \item The order of the elements appears random\pause
  \item This becomes more efficient as the table becomes fuller\pause 
  \end{itemize}
  \end{minipage}\hfill
  \begin{minipage}{10cm}
    \includegraphics[width=10cm]{hash-ex.mps}\\
    55, 17, 34, 63, 12
  \end{minipage}
\end{PauseHighLight}

\end{slide}


%%%%%%%%%%%%%%%%%%%%%%% Next Slide %%%%%%%%%%%%%%%%%%%%%%%
\Outline % Open Addressing
%%%%%%%%%%%%%%%%%%%%%%% Next Slide %%%%%%%%%%%%%%%%%%%%%%%

\begin{slide}
\section{Open Addressing}

\begin{PauseHighLight}
  \begin{itemize}
  \item In open addressing we have a single table of objects (without a
    linked-list)\pause
  \item In the case of a collision a new location in the table is
    found\pause 
  \item The simplest mechanism is known as \emph{linear probing} where
    we move the entry to the next available location\pause
  \end{itemize}
\end{PauseHighLight}

\end{slide}
%%%%%%%%%%%%%%%%%%%%%%% Next Slide %%%%%%%%%%%%%%%%%%%%%%%

\begin{slide}
\section{Linear Probing}

\pausebuild
\begin{center}
  \multiinclude[graphics={width=\linewidth}]{linear-probing}\pause
\end{center}
\end{slide}

%%%%%%%%%%%%%%%%%%%%%%% Next Slide %%%%%%%%%%%%%%%%%%%%%%%

\begin{slide}
\section{Linear Probing Pile Up}

\begin{PauseHighLight}
  \begin{minipage}{16cm}
    \raggedright
  \begin{itemize}
  \item The entries will tend to pile up or cluster---this is sometimes
    referred to as \emph{primary clustering}\pause
  \item Clusters become worse as the number of entries grow\pause
  \item Clusters will increase the number of probes needed to find an
    insert location\pause
  \item The proportion of full entries in the table is known as the
    \emph{loading factor}\pause
  \end{itemize}
  \end{minipage}\hfill
  \begin{minipage}{7cm}
    \includegraphics[width=7cm]{pile-up.mps}    
  \end{minipage}
\end{PauseHighLight}
\end{slide}

%%%%%%%%%%%%%%%%%%%%%%% Next Slide %%%%%%%%%%%%%%%%%%%%%%%

\begin{slide}
\section[-1]{Reducing Number of Probes}
\pb\pause\pauselevel{=1}
\begin{center}
  \multipdf[width=0.9\textwidth]{probe_no}\pause
\end{center}
\begin{itemize}\small
\item To avoid clustering we can use \emph{quadratic probing}\pause{}
  or \emph{double hashing}\pause
\end{itemize}

\end{slide}



%%%%%%%%%%%%%%%%%%%%%%% Next Slide %%%%%%%%%%%%%%%%%%%%%%%

\begin{slide}
\section[-2]{Quadratic Probing}
\toptarget{qprob}

\begin{PauseHighLight}
  \begin{itemize}
  \item In quadratic probing we try the locations $h(x)+d_i$ where
    $h(x)$ is the original hash code and $d_i=i^2$\pause
  \item That is we takes steps 1, 4, 9, 16, \ldots\pause
  \item Quadratic probing prevents primary clustering so dramatically
    decreases the number of probes needed to find a free location when
    the table is reasonably full\pause
  \item One problem is that if we are unlucky we might not be able to
    add an element to the hash table even if the table isn't full\pause
  \item However, if the size of the table is prime then quadratic
    probing will always find a free position provided it is not more
    than half full\pause
  \end{itemize}
\end{PauseHighLight}

\end{slide}

%%%%%%%%%%%%%%%%%%%%%%% Next Slide %%%%%%%%%%%%%%%%%%%%%%%

\begin{slide}
\section[-1]{Double Hashing}
\toptarget{dhash}

\begin{PauseHighLight}
  \begin{itemize}
  \item An alternative strategy is to known as double hashing where the
    locations tried are $h(x)+d_i$ where $d_i = i\times h_2(x)$\pause
  \item $h_2(x)$ is a second hash function that depends on the
    key\pause
  \item A good choice is $h_2(x) = R - (x \mod R)$ where $R$ is a
    prime smaller than the table size\pause
  \item It is important that $h_2(x)$ is not a divisor of the
    table size\pause
    \begin{itemize}
    \item Either make sure the table size is prime or\pause
    \item Set the step size to 1 if $h_2(x)$ is a divisor of the table
      size\pause
    \end{itemize}
  \end{itemize}
\end{PauseHighLight}

\end{slide}

%%%%%%%%%%%%%%%%%%%%%%% Next Slide %%%%%%%%%%%%%%%%%%%%%%%

\begin{slide}
\section[-1]{Problems with Remove}

\begin{PauseHighLight}
  \begin{itemize}
  \item For all open addressing hash systems removing an entry is a
    problem\pause
  \item Remember our strategy to find an input $x$ is\pause
    \begin{enumerate}
    \item Compute the array index based on the hash code of $x$\pause
    \item If the array location is empty then the search fails\pause
    \item If the array location contains the key the search succeeds\pause
    \item otherwise find a new location using an open addressing strategy
      and go to 2\pause
    \end{enumerate}
  \item If we remove an entry then find might reach an empty location
    which was previously full\pause
  \item This can prevent us finding a true entry\pause
  \end{itemize}
\end{PauseHighLight}

\end{slide}

%%%%%%%%%%%%%%%%%%%%%%% Next Slide %%%%%%%%%%%%%%%%%%%%%%%

\begin{slide}
\section{Linear Probing Example}

\pausebuild
\begin{center}
  \multiinclude[format=mps,graphics={width=\linewidth}]{hashfind}\pause
\end{center}
\end{slide}

%%%%%%%%%%%%%%%%%%%%%%% Next Slide %%%%%%%%%%%%%%%%%%%%%%%

\begin{slide}
\section{Lazy Remove}

\begin{PauseHighLight}
  \begin{itemize}
  \item One easy fix is to mark the deleted table with a special
    entry\pause
  \item A find method would consider this entry as full\pause
  \item An iterator would ignore this entry\pause
  \item An insert operator could insert a new entry in these special
    locations\pause
  \end{itemize}
\end{PauseHighLight}

\end{slide}

%%%%%%%%%%%%%%%%%%%%%%% Next Slide %%%%%%%%%%%%%%%%%%%%%%%

\begin{slide}
\section{Lazy Remove in Action}

\pausebuild
\begin{center}
  \multiinclude[graphics={width=\linewidth}]{lazyremove}\pause
\end{center}
\end{slide}


%%%%%%%%%%%%%%%%%%%%%%% Next Slide %%%%%%%%%%%%%%%%%%%%%%%
\Outline % HashSets and HashMaps
%%%%%%%%%%%%%%%%%%%%%%% Next Slide %%%%%%%%%%%%%%%%%%%%%%%


\begin{slide}
\section{What Strategy to Use?}

\begin{PauseHighLight}
  \begin{itemize}
  \item Most libraries including the STL (and the Java Collection
    class) use separate chaining\pause
  \item This has the advantage that its performance does not degrade
    badly as the number of entries increase\pause
  \item This reduces the need to resize the hash table\pause
  \item The C++ standard did not include a hash table until C++11
    \Frowny\pause---although very good hash tables existed in C++\pause
  \end{itemize}
\end{PauseHighLight}

\end{slide}

%%%%%%%%%%%%%%%%%%%%%%% Next Slide %%%%%%%%%%%%%%%%%%%%%%%

\begin{slide}
\section[-1]{Hash Sets and Maps}

\begin{PauseHighLight}
  \begin{itemize}
  \item C++ also provides an \texttt{unordered\_map<Key, V>} class
  \item It's performance is asymptotically superior to \texttt{map},
    $O(1)$ rather than $O(\log(n))$\pause
  \item Hash functions can take time to compute (it is often
    $O(\log(n))$) so \texttt{unordered\_set}s might not be faster than
    \texttt{set}s\pause
  \item One major difference is that the iterator for \texttt{set}s
    return the elements in order, \texttt{undordered\_set}'s iterator
    doesn't\pause 
  \end{itemize}
\end{PauseHighLight}

\end{slide}


%%%%%%%%%%%%%%%%%%%%%%% Next Slide %%%%%%%%%%%%%%%%%%%%%%%

\begin{slide}
\section{Applications}

\begin{PauseHighLight}
  \begin{itemize}
  \item Hash tables are used everywhere\pause
  \item E.g. most databases use hash tables to speed up search\pause
  \item In many document applications hash tables will be being
    generated in the background\pause
  \item Content addressability is ubiquitous to many application where
    hash tables are used as standard\pause
  \end{itemize}
\end{PauseHighLight}

\end{slide}

%%%%%%%%%%%%%%%%%%%%%%% Next Slide %%%%%%%%%%%%%%%%%%%%%%%

\begin{slide}
\section{Lessons}

\begin{PauseHighLight}
  \begin{itemize}
  \item Hash tables are one of the most useful tools you have
    available\pause
  \item They aren't particularly difficult to understand\pause, but you
    need to know about
    \begin{itemize}
    \item hashing functions\pause
    \item collision strategies\pause
    \item performance (i.e. when they work)\pause
    \end{itemize}
  \end{itemize}
\end{PauseHighLight}

\end{slide}
