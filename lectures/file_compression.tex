%Master File:lectures.tex

\lesson{Use Smart Encoding!}

\vspace{-2cm}
\begin{center}
  \includegraphics[height=10cm]{2dwavelet}
\end{center}
\keywords{File compression, Huffman codes, wavelets}

%%%%%%%%%%%%%%%%%%%%%%% Next Slide %%%%%%%%%%%%%%%%%%%%%%%
\renewcommand{\Outline}{%
\begin{slide}
\section[1]{Outline}

\begin{minipage}{12cm}
  \begin{enumerate}\squeeze
    \outlineitem{Huffman codes}{huffman}
    \outlineitem{Wavelets}{wavelet}
  \end{enumerate}
\end{minipage}\hfill
\begin{minipage}{10cm}
  \includegraphics[width=10cm]{2dwavelet}
\end{minipage}
\end{slide}
\addtocounter{outlineitem}{1}
}

\setcounter{outlineitem}{1}
%%%%%%%%%%%%%%%%%%%%%%% Next Slide %%%%%%%%%%%%%%%%%%%%%%%
\Outline % Huffman coding
%%%%%%%%%%%%%%%%%%%%%%% Next Slide %%%%%%%%%%%%%%%%%%%%%%%
\toptarget{firstoutline}

\begin{slide}
\section[-1]{File Compression}

\begin{PauseHighLight}
  \begin{itemize}
  \item File compression comes in two varieties
    \begin{itemize}
    \item Exact compression (e.g. zip used on text files)
    \item Lossy compression (e.g. jpeg used on pictures---jpeg can also
      be loss-less or exact)\pause
    \end{itemize}
  \item Good exact compression (also known as entropy encodings) can
    give a compression ratio around 25\%\pause
  \item Lossy compression can give a compression ratio from 10-1\%\pause
  \item Important for saving space, but lossy compression can also be
    used for noise reduction\pause
  \item Even used for plagiarism detection!\pauseb
  \end{itemize}
\end{PauseHighLight}


\end{slide}

%%%%%%%%%%%%%%%%%%%%%%% Next Slide %%%%%%%%%%%%%%%%%%%%%%%

\begin{slide}
\section[-2]{Entropy Encoding}

\begin{PauseHighLight}
  \begin{itemize}
  \item Exact encodings use the principle of using short words for
    frequently occurring sequences (symbols) and longer words for
    sequences that occur less often\pause
  \item Claude Shannon showed that for an alphabet of $n$ symbols where the
    probability of symbol $i$ occurring is $p_i$ no code exists which
    can transmit information in less than
    \begin{align*}
      -\sum_{i=1}^n p_i \log_2(p_i) \,\,\text{bits}
    \end{align*}
    asymptotically this compression can be achieved\pause
  \item Different encoding schemes differ in the way they identify
    symbols of the alphabet\pause---this is rather specialist and we
    won't go into this\pauseb
  \end{itemize}
\end{PauseHighLight}


\end{slide}

%%%%%%%%%%%%%%%%%%%%%%% Next Slide %%%%%%%%%%%%%%%%%%%%%%%

\begin{slide}
\section[-1]{Huffman Coding}

\begin{PauseHighLight}
  \begin{itemize}
  \item Given a sequence of symbols and their probabilities of
    occurance, Huffman code provides a way of coding up the
    information\pause
  \item It is an example of a \emph{greedy} strategy that happens to be
    optimal\pause
  \item Like many greedy strategies it is easily implemented using a
    priority queue\pause
  \item It is used in the UNIX compress program and in the exact part of
    JPEG\pause
  \item The idea is to assign short codes to commonly used symbols\pause
  \end{itemize}
\end{PauseHighLight}

\end{slide}

%%%%%%%%%%%%%%%%%%%%%%% Next Slide %%%%%%%%%%%%%%%%%%%%%%%

\begin{slide}
\section[-1]{Symbol Frequency}

\begin{PauseHighLight}
  \begin{itemize}
  \item We start from an alphabet describing the original document\pause
    \begin{itemize}
    \item This might be the set of characters\pause
    \item For an image it might be the set of pixel values\pause
    \item It might be pairs of pixel values\pause
    \end{itemize}
  \item We compute the number of occurrences of each symbol
    \begin{center}
      \begin{tabular}{|c|c|} \hline
        Symbol & \# Occurrences \\ \hline
        a & 145 \\ \hline
        b & 67  \\ \hline
        \vdots & \vdots  \\ \hline
      \end{tabular}\pause
    \end{center}
  \end{itemize}
\end{PauseHighLight}

\end{slide}


%%%%%%%%%%%%%%%%%%%%%%% Next Slide %%%%%%%%%%%%%%%%%%%%%%%

\begin{slide}
\section[-2]{Encoding}

\begin{PauseHighLight}
  \begin{itemize}
  \item We want to assign a code to each symbol\pause
  \item To save space we want to assign short codes to frequently used
    symbols\pause
  \item There is a problem:\pause\ decoding\pause
  \item If we assigned a code
    \begin{align*}
      e& \rightarrow 0 & a &\rightarrow 1 & r &\rightarrow 01 \\
      o& \rightarrow 10 & i &\rightarrow 11 & t &\rightarrow 000
    \end{align*}
    etc.  we could compress a document very efficiently but we could
    never decode it uniquely\pause
  \end{itemize}
\end{PauseHighLight}

\end{slide}

%%%%%%%%%%%%%%%%%%%%%%% Next Slide %%%%%%%%%%%%%%%%%%%%%%%

\begin{slide}
\section[-2.5]{Huffman Trees}

\begin{PauseHighLight}
  \begin{itemize}\squeeze
  \item Once again tree come to the rescue!\pause
  \item We assign each symbol to a leaf of a binary tree\pause
  \item We use the position of the branch as an encoding of the symbol\pause
    \begin{center}
      \multiinclude[graphics={width=0.9\linewidth}]{huffman}\pause
    \end{center}
  \item The decoding is unique\pause
  \end{itemize}
\end{PauseHighLight}

\end{slide}

%%%%%%%%%%%%%%%%%%%%%%% Next Slide %%%%%%%%%%%%%%%%%%%%%%%

\begin{slide}
\section[-2.5]{Generating the Huffman Tree}

\pausebuild
\color{TwoColor}
\begin{itemize}\squeeze
\item We are left with the problem of constructing the Huffman tree
  such that frequently occurring letters have short codes\pauseh
\item A greedy approach is to iteratively build a tree by\pauseh
  \begin{enumerate}
  \item combine the two most infrequent symbols into a subtree\pauseh
  \item Add their scores and treat them as a single symbol\pauseh
  \end{enumerate}
  \begin{center}
    \multiinclude[graphics={width=0.8\linewidth}]{huffman-construct}\pause
  \end{center}
\end{itemize}


\end{slide}

%%%%%%%%%%%%%%%%%%%%%%% Next Slide %%%%%%%%%%%%%%%%%%%%%%%

\begin{slide}
\section[-2]{English Letters}
\begin{center}
  \includegraphics[width=\linewidth]{huffman}
\end{center}
\end{slide}

%%%%%%%%%%%%%%%%%%%%%%% Next Slide %%%%%%%%%%%%%%%%%%%%%%%

\begin{slide}
\section{Implementing Huffman Encoding}

\begin{PauseHighLight}
  \begin{itemize}
  \item To implement Huffman encoding you need
    \begin{enumerate}
    \item A class to build Huffman trees by combining subtrees\pause
    \item A way to find the least frequently used symbols or symbol
      combinations\pause
    \end{enumerate}
  \item Priority queues are ideal for this application\pause
  \item They allow you to find the least frequently used symbols
    (\texttt{removeMin}) and to add new symbols (\texttt{add})\pause
  \item To decode you follow the Huffman tree\pause
  \end{itemize}
\end{PauseHighLight}

\end{slide}


%%%%%%%%%%%%%%%%%%%%%%% Next Slide %%%%%%%%%%%%%%%%%%%%%%%

% \begin{slide}
% \section{Code Outline}

% \begin{java}
% public abstract class HuffmanNode implements Comparable<HuffmanNode>
% {
%     protected int count;
%     protected HuffmanNode parent;

%     public int getCount()
%     {
%         return count;
%     }

%     public int compareTo(HuffmanNode rhs)
%     {
%         return getCount()-rhs.getCount();
%     }

%     public void setParent(HuffmanNode p)
%     {
%         parent = p;
%     }
% }
% \end{java}
% \end{slide}

% %%%%%%%%%%%%%%%%%%%%%%% Next Slide %%%%%%%%%%%%%%%%%%%%%%%

% \begin{slide}
% \section[-2]{Nodes and Leaves}

% \begin{java}
% public class HuffmanSubTree extends HuffmanNode {
%     private HuffmanNode left;
%     private HuffmanNode right;
    
%     HuffmanSubTree(HuffmanNode l, HuffmanNode r)
%     {   left = l;
%         right = r;
%         count = l.getCount() + r.getCount();
%         l.setParent(this);
%         r.setParent(this);
%     }
% } $\pause$

% public class HuffmanLeaf extends HuffmanNode {
%     private char ch;

%     HuffmanLeaf(int s, int frequency)
%     {   ch = (char)(s);
%         count = frequency;
%     }
% } $\pause$
% \end{java}
% \end{slide}

% %%%%%%%%%%%%%%%%%%%%%%% Next Slide %%%%%%%%%%%%%%%%%%%%%%%

% \begin{slide}
% \section{Constructing the Huffman Tree}

% \begin{java}
%   Map<Integer,Integer> charCount = new TreeMap<Integer,Integer>();
%   int ch;
%   while ( (ch=input.read()) != -1) {
%       int cnt = 1;
%       if (charCount.containsKey(ch))
%           cnt += charCount.get(ch);
%       charCount.put(ch, cnt);
%   } $\pause$
%   Set<Map.Entry<Integer, Integer>> setView = charCount.entrySet(); $\pause$

%   PQ<HuffmanNode> pq = new HeapPQ<HuffmanNode>();
%   for (Map.Entry<Integer, Integer> entry: setView)
%       pq.add(new HuffmanLeaf(entry.getKey(), entry.getValue())); $\pause$

%   while (pq.size()>1) {
%       HuffmanNode ht1 = pq.removeMin();
%       HuffmanNode ht2 = pq.removeMin();
%       pq.add(new HuffmanSubTree(ht1,ht2));
%   }
%   HuffmanNode ht = pq.removeMin(); $\pause$
% \end{java}
% \end{slide}



%%%%%%%%%%%%%%%%%%%%%%% Next Slide %%%%%%%%%%%%%%%%%%%%%%%

\begin{slide}
\section{Greedy Strategy}

\begin{PauseHighLight}
  \begin{itemize}
  \item Huffman encoding is an example of a \emph{Greedy solution
      pattern}\pause
  \item That is we look for local optimality (i.e. we combine the two
    least frequently used symbols)\pause
  \item In this case, we obtain global optimality (i.e. the Huffman tree
    obtained gives an optimal Huffman code)\pause
  \item There are a number of important problems where greedy algorithms
    lead to global optimality (we saw this earlier)\pause
  \item For these algorithms priority queues commonly are used for
    implementing the algorithm\pause
  \end{itemize}
\end{PauseHighLight}

\end{slide}

%%%%%%%%%%%%%%%%%%%%%%% Next Slide %%%%%%%%%%%%%%%%%%%%%%%

\begin{slide}
\section{Advanced Techniques}

\begin{PauseHighLight}
  \begin{itemize}
  \item Huffman code is optimal given the frequency of symbols\pause
  \item However, there is considerable art in identifying which
    'symbols' to use\pause
  \item Advanced compression algorithms (LZ78, LZW Lempel-Ziv-Welch) build
    dictionaries of sequences seen in the files---they tend to be rather
    specialised\pause
  \item Some recent algorithms (e.g. Burrows-Wheeler) transform the file
    in such a way that similar symbols are mapped to adjacent
    sites---depends on the generating mechanism of the language\pause
  \end{itemize}
\end{PauseHighLight}

\end{slide}
%%%%%%%%%%%%%%%%%%%%%%% Next Slide %%%%%%%%%%%%%%%%%%%%%%%

\begin{slide}
\section{File Compression and Plagiarism Detection}

\begin{PauseHighLight}
  \begin{itemize}
  \item One way of spotting plagiarism is to compare the compressed
    lengths of two files and the length of the compressed file when the
    two files are concatenated first\pause
  \item If the files have the same structure the concatenated version
    can often be significantly reduced\pause
  \item Also used in identifying closeness of species in constructing
    phylogenetic trees\pause
  \end{itemize}
\end{PauseHighLight}

\end{slide}


%%%%%%%%%%%%%%%%%%%%%%% Next Slide %%%%%%%%%%%%%%%%%%%%%%%
\Outline % Wavelets
%%%%%%%%%%%%%%%%%%%%%%% Next Slide %%%%%%%%%%%%%%%%%%%%%%%

\begin{slide}
\section[-1]{Signals and Energies}

\begin{PauseHighLight}
  \begin{itemize}
  \item We consider compressing a signal $\bm{x} = (x_0,\,x_1,\,
    \ldots,\, x_{n-1})$\pause
    \begin{center}
      \includegraphics[width=0.7\linewidth]{signal}
    \end{center}
  \item We can define the ``energy'' as the squared deviations
    \begin{align*}
      E = \sum_{i=1}^n x_i^2\pause
    \end{align*}
  \item Our strategy in lossy compression is to transmit as much
    ``energy'' in as few bits as possible\pause
  \item There are different strategies to achieve good compress\pause
  \end{itemize}
\end{PauseHighLight}
\end{slide}

%%%%%%%%%%%%%%%%%%%%%%% Next Slide %%%%%%%%%%%%%%%%%%%%%%%

\begin{slide}
\section[-2.5]{Wavelets}

\begin{PauseHighLight}
  \begin{itemize}
  \item With wavelets we try to re-represent the signal so as to squeeze
    as much energy as possible into fewer bits\pause
  \item The easiest way to do this is with Haar wavelets
    \begin{align*}
      a_i &= \frac{x_{2i} + x_{2i+1}}{\sqrt{2}} &
      d_i &= \frac{x_{2i} - x_{2i+1}}{\sqrt{2}}\pause
    \end{align*}
  \item Define new signal $(a_0, a_1, a_2, \ldots, a_{n/2-1}, d_0, d_1,
    \ldots, d_{n/2-1})$
  \end{itemize}
  \begin{center}
    \includegraphics[width=0.7\linewidth]{haar}\pause   
  \end{center}
\end{PauseHighLight}

\end{slide}

%%%%%%%%%%%%%%%%%%%%%%% Next Slide %%%%%%%%%%%%%%%%%%%%%%%

\begin{slide}
\section[-2]{Carrier and Difference Signals}

\begin{PauseHighLight}
  \begin{itemize}
  \item The terms $a_i = (x_{2i} + x_{2i+1})/\sqrt{2}$ takes the
    ``average'' of the signal, but compresses it in half the space\pause
  \item The terms $d_i = (x_{2i} - x_{2i+1})/\sqrt{2}$ takes the
    difference and is small if the signal does not change much\pause
  \item The energy is conserved since 
    {\small
      \begin{align*}
        a_i^2 + d_i^2 &= \left(\frac{x_{2i} + x_{2i+1}}{\sqrt{2}}\right)^2
        + \left(\frac{x_{2i} - x_{2i+1}}{\sqrt{2}}\right)^2 \\
        &= \frac{x_{2i}^2 + 2 x_{2i}x_{2i+1} + x_{2i+1}^2 + x_{2i}^2 - 2
          x_{2i}x_{2i+1} + x_{2i+1}^2}{2} = x_{2i}^2 + x_{2i+1}^2\pause
      \end{align*}}
  \item Attempt to push all the energy into the carrier signal, $a_i$\pause
  \end{itemize}
\end{PauseHighLight}

\end{slide}

%%%%%%%%%%%%%%%%%%%%%%% Next Slide %%%%%%%%%%%%%%%%%%%%%%%

\begin{slide}
\section{Inverse Transform}

\begin{PauseHighLight}
  \begin{itemize}
  \item The wavelet transform can be easily reversed
    {\small
      \begin{align*}
        a_i &= \frac{x_{2i} + x_{2i+1}}{\sqrt{2}} &
        d_i &= \frac{x_{2i} - x_{2i+1}}{\sqrt{2}}
        \\
        x_{2i} &= \frac{a_i + d_i}{\sqrt{2}} &
        x_{2i+1} &= \frac{a_i - d_i}{\sqrt{2}}\pause
      \end{align*}}
  \item Can compute transform using vectors (wavelets)
    \begin{align*}
      a_i &= \bm{V_i} \cdot \bm{x} & d_i &= \bm{W_i}\cdot \bm{x}\pause
    \end{align*}
  \item These vectors are orthogonal to each other ($\bm{V_i} \cdot
    \bm{V_j}=0$, $\bm{V_i} \cdot \bm{W_j}=0$, etc.)\pause
  \end{itemize}
\end{PauseHighLight}

\end{slide}


%%%%%%%%%%%%%%%%%%%%%%% Next Slide %%%%%%%%%%%%%%%%%%%%%%%

\begin{slide}
\section[-2]{And So On\ldots}

\begin{PauseHighLight}
  \begin{itemize}
  \item We can repeat the process again to concentrate the energy
    further\pause
  \item We apply the Haar transform just to the carry part $\bm{a} =
    (a_0,a_1,\ldots, a_{n/2-1})$
  \end{itemize}
  \begin{center}
    \includegraphics[width=0.7\linewidth]{haar2}\pause
  \end{center}
\end{PauseHighLight}

\end{slide}

%%%%%%%%%%%%%%%%%%%%%%% Next Slide %%%%%%%%%%%%%%%%%%%%%%%

\begin{slide}
\section[-2]{Daubechies Wavelets}

\begin{PauseHighLight}
  \begin{itemize}
  \item Ingrid Daubechies suggested a host of wavelets which do better
    than Haar for smooth signals\pause
  \item The simplest is Daub4 defined by
    {\small
      \begin{align*}
        a_i &= c_0 x_{2i} + c_1 x_{2i+1} + c_2 x_{2i+2} + c_3 x_{2i+3} \\
        d_i &= c_3 x_{2i} - c_2 x_{2i+1} + c_1 x_{2i+2} - c_0 x_{2i+3}
      \end{align*}
      \begin{align*}
        c_0 &= \frac{1+\sqrt{3}}{4\sqrt{2}} &
        c_1 &= \frac{3+\sqrt{3}}{4\sqrt{2}} &
        c_2 &= \frac{3-\sqrt{3}}{4\sqrt{2}} &
        c_3 &= \frac{1-\sqrt{3}}{4\sqrt{2}}\pause
      \end{align*}
    }
  \item Again conserves energy
    \begin{align*}
      \sum_{i=1}^{n/2} a_i^2 + b_i^2 = \sum_{i=1}^{n} x_i^2\pause
    \end{align*}
  \end{itemize}
\end{PauseHighLight}

\end{slide}

%%%%%%%%%%%%%%%%%%%%%%% Next Slide %%%%%%%%%%%%%%%%%%%%%%%

\begin{slide}
\section{Properties of Daub4}

\begin{PauseHighLight}
  \begin{itemize}
  \item Similar to the Haar transform
    \begin{align*}
      c_0 + c_1 + c_2 + c_3 &= \sqrt{2}, &
      c_3 - c_2 + c_1 - c_0 &= 0
    \end{align*}
    so the carrier signal ($a_i$) is approximately $\sqrt{2}$ times the
    original and the difference part ($d_i$) is equal to 0 for a flat
    signal, $\bm{x}$\pause
  \item However in addition
    \begin{align*}
      0 c_3 - 1 c_2 + 2 c_1 - 3 c_0 = 0
    \end{align*}
    so the difference part ($d_i$) is equal to 0 for any linear signal,
    $\bm{x}$ \pause
  \end{itemize}
\end{PauseHighLight}
\end{slide}

%%%%%%%%%%%%%%%%%%%%%%% Next Slide %%%%%%%%%%%%%%%%%%%%%%%

\begin{slide}
\section{Daub4}

\begin{center}
\includegraphics[width=0.9\linewidth]{daub4}  
\end{center}

\end{slide}

%%%%%%%%%%%%%%%%%%%%%%% Next Slide %%%%%%%%%%%%%%%%%%%%%%%

\begin{slide}
\section{Signal Compression}

\begin{PauseHighLight}
  \begin{itemize}
  \item To compress the signal we can set all components of the
    transformed signal whose magnitude lies below a threshold to 0\pause
  \item We transmit the non-zero magnitude together with a binary mask
    showing the position of the non-zero magnitude\pause
  \item We can reduce the accuracy (number of decimal places) of the
    non-zero magnitudes (quantisation)---this is repaired on inverting
    transform\pause
  \item We can compress the binary mask using Huffman encoding or other
    scheme\pause
  \end{itemize}
\end{PauseHighLight}

\end{slide}

%%%%%%%%%%%%%%%%%%%%%%% Next Slide %%%%%%%%%%%%%%%%%%%%%%%

\begin{slide}
\section{Daub6}
\pb\pause\pauselevel{=1}
\begin{center}
  \multipdf[width=0.9\linewidth]{daub6}\pause
\end{center}
\end{slide}


%%%%%%%%%%%%%%%%%%%%%%% Next Slide %%%%%%%%%%%%%%%%%%%%%%%

\begin{slide}
\section[-1]{Noise Reduction}

\pb\pause\pauselevel{=1}
\begin{itemize}
\item Can also be used in noise reduction\pause
\end{itemize}
\begin{center}
  \multipdf[width=0.9\linewidth]{noiseReduction}\pause
\end{center}

\end{slide}

%%%%%%%%%%%%%%%%%%%%%%% Next Slide %%%%%%%%%%%%%%%%%%%%%%%

\begin{slide}
\section{Other Wavelets}

\begin{PauseHighLight}
  \begin{itemize}
  \item Can use high-order wavelets which captures more energy in the
    carrier signal, e.g. Daub10 or Daub20\pause
  \item Many other wavelets capture other properties (e.g. Coiflets
    capture properties of a continuous signal sampled at discrete
    points)\pause
  \item Efficiency of wavelets depend on how well the capture underlying
    properties of signals\pause
  \item Can also construct 2-d wavelets for image compression
    (jpeg-2000)\pause
  \end{itemize}
\end{PauseHighLight}

\end{slide}

%%%%%%%%%%%%%%%%%%%%%%% Next Slide %%%%%%%%%%%%%%%%%%%%%%%

\begin{slide}
\section[-2]{2-D Wavelets}

\begin{center}
  \includegraphics[width=0.6\linewidth]{2dwavelet}
\end{center}
\end{slide}



%%%%%%%%%%%%%%%%%%%%%%% Next Slide %%%%%%%%%%%%%%%%%%%%%%%

\begin{slide}
\section{Summary}

\begin{PauseHighLight}
  \begin{itemize}
  \item File compression is an important task in its own right\pause
  \item Files may either be compressed losslessly or lossily\pause
  \item Lossy compression is typically much more efficient (e.g. an
    order of magnitude smaller)\pause
  \item Huffman encoding often lies at the lowest level in many
    compression algorithms\pause
  \item Wavelets illustrate a strategy of changing the representation to
    concentrate the energy of a signal\pause
  \end{itemize}
\end{PauseHighLight}

\end{slide}

