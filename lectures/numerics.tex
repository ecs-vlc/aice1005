%Master File:lectures.tex

\lesson{Understand Numerics}

\vspace{-2cm}
\begin{center}
  \includegraphics[height=9cm]{newtonRaphson-2}
\end{center}
\keywords{Representing reals, rounding error, convergence, stability,
  conditioning}
%%%%%%%%%%%%%%%%%%%%%%% Next Slide %%%%%%%%%%%%%%%%%%%%%%%
\renewcommand{\Outline}{%
\begin{slide}
\section[1]{Outline}

\begin{minipage}{12cm}
  \begin{enumerate}\squeeze
    \outlineitem{Numerical Approximations}{numerics}
    \outlineitem{Iterating to a Solution}{solvingEquations}
    \outlineitem{Linear Algebra}{linalg}
  \end{enumerate}
\end{minipage}\hfill
\begin{minipage}{10cm}
  \includegraphics[width=10cm]{newtonRaphson-2}
\end{minipage}
\end{slide}
\addtocounter{outlineitem}{1}
}

\setcounter{outlineitem}{1}
%%%%%%%%%%%%%%%%%%%%%%% Next Slide %%%%%%%%%%%%%%%%%%%%%%%
\Outline
\toptarget{firstoutline}
%%%%%%%%%%%%%%%%%%%%%%% Next Slide %%%%%%%%%%%%%%%%%%%%%%%


\begin{slide}
\section{Numerical Analysis}

\begin{PauseHighLight}
  \begin{itemize}
  \item Numerical algorithms are usually taught separately from
    the ``discrete algorithms'' we have predominantly looked at\pause
  \item The main difference stems from the fact that numerical
    algorithms model continuous variables\pause
  \item Computers can only approximate continuous variables\pause
  \item Numerical algorithms have to take into account this
    approximation\pause
  \end{itemize}
\end{PauseHighLight}

\end{slide}

%%%%%%%%%%%%%%%%%%%%%%% Next Slide %%%%%%%%%%%%%%%%%%%%%%%

\begin{slide}
\section{Representing Reals}

\begin{PauseHighLight}
  \begin{itemize}
  \item All real numbers are approximated by a binary encoding
    \begin{center}
      \includegraphics[width=0.9\linewidth]{realRepresentation}\pause
    \end{center}
  \item $x = m \times 2^{e-t}$\pause
  \item $t$ is precision so that if $e=t$, then $0.5 \leq x < 1$\pause
  \item For IEEE double $t=1023$, $expon_{\min} = -1021$, $expon_{\max}=1024$\pause
  \item Typical rounding error is $u=1\times10^{-16}$\pause
  \end{itemize}
\end{PauseHighLight}

\end{slide}

%%%%%%%%%%%%%%%%%%%%%%% Next Slide %%%%%%%%%%%%%%%%%%%%%%%

\begin{slide}
\section[-1]{The Number Line}

\pb
\begin{itemize}
\item We approximate the continuous number line by a set of discrete
  values\pauseh
\item Imagine using a mantissa of 3 bits and an exponent of 2 bits (and
  a sign)\pauseh
\end{itemize}\pauselevel{=2}
\begin{center}
  \multipdf[width=0.9\linewidth]{numberline}\pauseb
\end{center}
\begin{itemize}
\item The rounding error is half the gap between the discrete values\pauseh
\end{itemize}

\end{slide}

%%%%%%%%%%%%%%%%%%%%%%% Next Slide %%%%%%%%%%%%%%%%%%%%%%%

\begin{slide}
\section{Overflow and Underflow}

\begin{PauseHighLight}
  \begin{itemize}
  \item An overflow will cause a program to fall over at run time\pause
  \item An underflow is ignored\pause
  \item This is usually innocuous, but can lead to trouble\pause
  \item If you call \jl$log(x)$ or \jl$1.0/x$ but \jl$x$ has underflowed
    then your program will crash\pause
  \end{itemize}
\end{PauseHighLight}

\end{slide}

%%%%%%%%%%%%%%%%%%%%%%% Next Slide %%%%%%%%%%%%%%%%%%%%%%%

\begin{slide}
\section[-2]{Rounding Error}

\begin{PauseHighLight}
  \begin{center}
    \includegraphics[width=0.9\linewidth]{numberline-0}
  \end{center}
  \vspace*{-3cm}
  \begin{itemize}
  \item The distance between two real numbers $\Delta x$ grows with the
    number such that $\Delta x/x \leq u$ where $u\approx 10^{-16}$ for
    doubles\pause
  \item Measure relative error
    \begin{align*}
      \text{Relative error} = \left\vert \frac{\text{Approx} -
          \text{Exact}}{\text{Exact}} \right \vert\pause
    \end{align*}
  \item Thus almost every operation is only accurate up to this small
    (relative) rounding error\pause
  \item Most operations are carefully designed that these rounding errors are
    unbiased so that the sum of random errors grows sub-linearly\pause
  \end{itemize}
\end{PauseHighLight}

\end{slide}

%%%%%%%%%%%%%%%%%%%%%%% Next Slide %%%%%%%%%%%%%%%%%%%%%%%

\begin{slide}
\section[-2]{Losing Precision}

\begin{PauseHighLight}
  \begin{itemize}
  \item There seems to be plenty of precision, so what's the problem?\pause
  \item One issue is that its easy to lose precision\pause
  \item Consider estimating derivatives by finite differencing
    \begin{align*}
      f'(x) \approx \frac{f(x+\epsilon) -
        f(x-\epsilon)}{2\,\epsilon}\pause
      = \frac{\scriptstyle 0.841470984861927-0.841470984753866}
      {2.0\times 10^{-10}}\pauseb
    \end{align*}
  \item The problem is $f(x+\epsilon)$ and $f(x-\epsilon)$ are very
    close so in taking their difference we lose precision\pause
  \item $f(x) = \sin(x)$, $f'(x) = \cos(x)$ at $x=1.0$
    \begin{center}\small
      \begin{tabular}{|c|c|c|c|c|c|}\hline
        $\epsilon$ & $10^{-6}$ & $10^{-8}$ & $10^{-10}$ & $10^{-12}$ & $10^{-14}$  \\ \hline
        relative error & $5\times 10^{-11}$ & $5\times10^{-9}$ &
        $1\times 10^{-7}$ & $2\times10^{-5}$ &$ 6\times10^{-3}$ \\ \hline
      \end{tabular}\pause
    \end{center}
  \end{itemize}
\end{PauseHighLight}

\end{slide}

%%%%%%%%%%%%%%%%%%%%%%% Next Slide %%%%%%%%%%%%%%%%%%%%%%%

\begin{slide}
\section[-1]{Solving Quadratic Equations}

\begin{PauseHighLight}
  \begin{itemize}
  \item A classic example where you can lose precision is in solving a
    quadratic equation \mbox{$a\,x^2 + b\,x + c =0$}
    \begin{align*}
      x = \frac{-b \pm \sqrt{b^2 - 4\,a\,c\,}}{2\,a}\pause
    \end{align*}
  \item If $b^2 \gg |4\,a\,x|$ then for one solution we end up
    subtracting numbers very close\pause
  \item We rather use this equation to compute one solution
    \begin{align*}
      x_1 = \frac{-b - \mathrm{sgn}(b) \sqrt{b^2 - 4\,a\,c\,}}{2\,a}\pause
    \end{align*}
  \item Use the identity $x_1\,x_2 = c/a$ to find $x_2$\pause{}
    (i.e. $x_2= c/(a\,x_1)$)\pauseb
  \end{itemize}
\end{PauseHighLight}

\end{slide}


%%%%%%%%%%%%%%%%%%%%%%% Next Slide %%%%%%%%%%%%%%%%%%%%%%%

\begin{slide}
\section[-1]{Accumulation of Rounding Error}

\begin{PauseHighLight}
  \begin{itemize}
  \item With many significant figures surely we can afford to lose some
    accuracy?\pause
  \item This is sometimes true, but we often use ``for loops'' where we
    might be losing accuracy all the time
    \begin{minipage}{0.5\linewidth}
    \begin{java}
      x = 1.6;
      for (i=0; i<50; i++)
         x = sqrt(x);
      for (i=0; i<50; i++)
         x = x*x;$\pause$
    \end{java}
    \end{minipage}\hfill
    \begin{minipage}{0.15\linewidth}
      \includegraphics[width=\linewidth]{truncationError}\pauseb
    \end{minipage}
    \item Gave the answer 1.2840 (if I run the for loop 60 times it
      gives the answer 1 for almost any input)\pause
  \end{itemize}
\end{PauseHighLight}

\end{slide}



%%%%%%%%%%%%%%%%%%%%%%% Next Slide %%%%%%%%%%%%%%%%%%%%%%%

\begin{slide}
\section{Coping With Truncation Errors}

\begin{PauseHighLight}
  \begin{itemize}
  \item Nothing is exact so to check that $x = y$ we use
    \begin{java}
      Math.abs(x-y) < 1.0e-10$\pause$  // a small constant
    \end{java}
  \item Sometimes sums that add up to 1 don't quite so we have not to
    rely on anything being exact\pause
  \item Avoid operations that are likely to lose accuracy (e.g. by
    taking the difference of similar numbers) where possible\pause
  \item Sometimes it pays to do some operations using higher precision
    \jl$long double$\pause
  \item Make sure that errors are unbiased\pause
  \end{itemize}
\end{PauseHighLight}

\end{slide}

%%%%%%%%%%%%%%%%%%%%%%% Next Slide %%%%%%%%%%%%%%%%%%%%%%%
\Outline % Solving Equations
%%%%%%%%%%%%%%%%%%%%%%% Next Slide %%%%%%%%%%%%%%%%%%%%%%%

\begin{slide}
\section{Iterative Algorithms}

\begin{PauseHighLight}
  \begin{itemize}
  \item We solve many numerical tasks by obtaining successively
    better solutions
    \begin{align*}
      x^{(0)},\, x^{(1)},\, x^{(2)},\, x^{(3)},\, x^{(4)},\,\ldots\pause
    \end{align*}
  \item We often stop when the change in solution is below some
    threshold, e.g. $|x^{(i+1)}-x^{(i)}|\leq \epsilon \approx u$\pause
  \item The time complexity depends on the speed of convergence\pause
  \item This can range from very fast to miserably slow\pause
  \end{itemize}
\end{PauseHighLight}

\end{slide}

%%%%%%%%%%%%%%%%%%%%%%% Next Slide %%%%%%%%%%%%%%%%%%%%%%%

\begin{slide}
\section[-2]{Bisection}
\pb
\begin{itemize}
\item Suppose we want to compute $\sqrt{2}$ (without using
  \jl$sqrt(2)$)\pauseh
  \begin{align*}
    f(x) = x^2 - 2 = 0
  \end{align*}
\item One of the classic methods of solving $f(x)=0$ is
  \emph{bisection}\pauseh\pauselevel{=1}
  \begin{center}
    \multipdf[width=0.8\linewidth]{numericalBisection}\pause
  \end{center}
\end{itemize}

\end{slide}

%%%%%%%%%%%%%%%%%%%%%%% Next Slide %%%%%%%%%%%%%%%%%%%%%%%

\begin{slide}
\section[-2]{Newton Raphson}

\pb
\begin{itemize}
\item A second classic method to solve $f(x)=0$ is Newton-Raphson's
  method
  \begin{align*}
    x^{(i+1)} = x^{(i)} - \frac{f(x^{(i)})}{f'(x^{(i)})}\pauseh 
  \end{align*}
\item For $f(x)=x^2-2$ so $x^{(i+1)} = ((x^{(i)})^2-1)/(2\,x^{(i)})$ 
  \pauseh\pauselevel{=1}
  \begin{center}
    \multipdf[width=0.8\linewidth]{newtonRaphson}\pause
  \end{center}
\end{itemize}

\end{slide}

%%%%%%%%%%%%%%%%%%%%%%% Next Slide %%%%%%%%%%%%%%%%%%%%%%%

\begin{slide}
\section[-2]{Convergence}

\pb\pause\pauselevel{=1}
\begin{center}
  \multipdf[width=0.8\linewidth]{convergence}\pause
\end{center}
\begin{itemize}
\item Bisection shows linear convergence (exponential increase in
  accuracy)\pause
\item Newton Raphson shows quadratic convergence\pause
\end{itemize}

\end{slide}

%%%%%%%%%%%%%%%%%%%%%%% Next Slide %%%%%%%%%%%%%%%%%%%%%%%

\begin{slide}
\section[-2]{Beware of Asymptotic Convergence}

\pb\pause\pauselevel{=1}
\begin{itemize}
\item Newton Raphson only converges quadratically if you start close
  enough to the solution\pauseh
\item Consider solving $x^6-1=0$ starting with $x^{(0)}=0.5$\pauseh
  \pauselevel{=1}
  \begin{center}
    \multipdf[width=0.8\linewidth]{newtonRaphson10}\pause
  \end{center}
\end{itemize}
\end{slide}

%%%%%%%%%%%%%%%%%%%%%%% Next Slide %%%%%%%%%%%%%%%%%%%%%%%

\begin{slide}
\section[-2]{Evaluating Functions}

\begin{PauseHighLight}
  \begin{itemize}
  \item We can evaluate many functions using a series expansion
    \begin{align*}
      \e{x} = 1 + x + \frac{x^2}{2!} + \frac{x^3}{3!} + \cdots\pause
    \end{align*}
    \begin{center}
      \includegraphics[width=0.8\linewidth]{expTaylor}\pause
    \end{center}
  \item For large $i$ this converges since $i! \gg x^i$ \pause
  \end{itemize}
\end{PauseHighLight}

\end{slide}

%%%%%%%%%%%%%%%%%%%%%%% Next Slide %%%%%%%%%%%%%%%%%%%%%%%

\begin{slide}
\section[-2]{Slow convergence}

\begin{PauseHighLight}
  \begin{itemize}
  \item Some expansions converge rather slowly (or even diverge)
    \begin{align*}
      \log(1+x) =  x - \frac{x^2}{2} + \frac{x^3}{3} - \cdots\pause
    \end{align*}
  \item Converges for $-1 < x \leq 1$, but converges slowly for $x=1$
    \begin{center}
      \includegraphics[width=0.8\linewidth]{logTaylor}\pause
    \end{center}
  \end{itemize}
\end{PauseHighLight}

\end{slide}

%%%%%%%%%%%%%%%%%%%%%%% Next Slide %%%%%%%%%%%%%%%%%%%%%%%

\begin{slide}
\section{Convergence}

\begin{PauseHighLight}
  \begin{itemize}
  \item Many functions can be approximated by a sum\pause
  \item We get a truncation error by taking only a finite number of
    elements\pause
  \item We want the truncation error to be around machine accuracy\pause
  \item For quick evaluation we need a strongly convergent series\pause
  \item This often depend on the value of the argument we give to the
    function\pause
  \item Most special functions are approximated by different series
    depending on the input argument\pause
  \end{itemize}
\end{PauseHighLight}

\end{slide}

%%%%%%%%%%%%%%%%%%%%%%% Next Slide %%%%%%%%%%%%%%%%%%%%%%%

\begin{slide}
\section{Differential Equations}

\begin{PauseHighLight}
  \begin{itemize}
  \item Differential equations are used in many applications, for
    example in modelling the motion of object\pause
  \item A typical equation of motion might be
    \begin{align*}
      \frac{\dd^2 x(t)}{\dd t^2} = 2\,\frac{\dd x(t)}{\dd t} + 3\,x(t)
    \end{align*}
  \item Which has a general solution $x(t) = c_1 \e{-t} + c_2 \e{3\,t}$\pause
  \item The constants are determined by initial conditions, for example,
    if $x(0) =1$ and $\dot{x}(0) = -1$ then $x(t) = \e{-t}$\pause
  \end{itemize}
\end{PauseHighLight}

\end{slide}

%%%%%%%%%%%%%%%%%%%%%%% Next Slide %%%%%%%%%%%%%%%%%%%%%%%

\begin{slide}
\section[-2]{Euler's Method}

\pb\pause\pauselevel{=1}
\begin{itemize}
\item To solve a differential equation we use an approximate
  update equation
  \begin{align*}
    x(t+\epsilon) &\approx x(t) + \epsilon \dot{x}(t) \\
    \dot{x}(t+\epsilon) &\approx \dot{x}(t) + \epsilon \ddot{x}(t)\pauseh
  \end{align*}
\item This becomes more exact as $\epsilon\rightarrow 0$\pauseh
  \begin{center}
    \multipdf[width=0.8\linewidth]{stability}\pause
  \end{center}
\end{itemize}

\end{slide}

%%%%%%%%%%%%%%%%%%%%%%% Next Slide %%%%%%%%%%%%%%%%%%%%%%%

\begin{slide}
\section{Stability}

\begin{PauseHighLight}
  \begin{itemize}
  \item Some iterative equations are unstable\pause
  \item Round off errors can push a system of equations towards an
    unstable solution\pause
  \item This can sometimes be overcome by cunning (e.g. running the
    equations backwards)\pause
  \item Finding stable algorithms and avoiding unstable algorithms can
    be key to getting accurate predictions\pause
  \end{itemize}
\end{PauseHighLight}


\end{slide}

%%%%%%%%%%%%%%%%%%%%%%% Next Slide %%%%%%%%%%%%%%%%%%%%%%%
\Outline % Linear Algebra
%%%%%%%%%%%%%%%%%%%%%%% Next Slide %%%%%%%%%%%%%%%%%%%%%%%

\begin{slide}
\section[-1]{Solving Simultaneous Equations}

\begin{PauseHighLight}
  \begin{itemize}
  \item When problems involve many variables it is convenient to use
    matrices and vectors to store the numbers\pause\\
    \begin{minipage}{0.4\linewidth}
    \vspace*{-13mm}
    \begin{align*}
      3 x + 2y &= 5\\
      7 x - 8y &= -11
    \end{align*}\pause
  \end{minipage}
  \begin{minipage}[b]{0.4\linewidth}
    \begin{align*}
      \begin{pmatrix}
        3 & 2 \\ 7 & -8
      \end{pmatrix}
      \begin{pmatrix}
        x \\ y
      \end{pmatrix} &=
      \begin{pmatrix}
        5 \\ -11
      \end{pmatrix}\pause
    \end{align*}
  \end{minipage}
  \vspace*{-2cm}
\item Or $\mat{A} \bm{x} = \bm{b}$ with solution $\bm{x} = \mat{A}^{-1}
  \mat{b}$\pause
\item Linear algebra is an abstraction allowing mathematicians,
  scientists and engineers to write solutions at a higher level\pause
\item The job of the numerical analyst is to write the code that does
  this\pause 
\end{itemize}
\end{PauseHighLight}

\end{slide}

%%%%%%%%%%%%%%%%%%%%%%% Next Slide %%%%%%%%%%%%%%%%%%%%%%%

\begin{slide}
\section[-2]{Linear Algebra}

\begin{PauseHighLight}
  \begin{itemize}
  \item There are a large number of problems with matrices that people
    care about\pause
  \item The solution often depends on the problem\pause
  \item These include
    \begin{itemize}\squeeze
    \item Multiply matrices together
    \item Solving linear equations $\mat{A} \bm{x} = \bm{b}$
    \item Finding eigenvalues of symmetric and non-symmetric matrices
    \item Performing singular valued decomposition\pause
    \end{itemize}
  \item These are important tasks that need to be done efficiently and
    reliably\pause
  \end{itemize}
\end{PauseHighLight}

\end{slide}

%%%%%%%%%%%%%%%%%%%%%%% Next Slide %%%%%%%%%%%%%%%%%%%%%%%

\begin{slide}
\section[-2]{Solving Linear Equations}

\begin{PauseHighLight}
  \begin{itemize}
  \item We consider the classic problem of solving $\mat{A} \bm{x} =
    \bm{b}$\pause
  \item Although we can solve this by computing $\mat{A}^{-1} \bm{b}$,
    finding the inverse of a matrix is typically a $\Theta(n^3)$
    operation\pause
  \item It is preferable to decompose $\mat{A}$ into a product of a lower
    triangular matrix $\mat{L}$ and an upper triangular matrix $\mat{U}$
    which takes $\Theta(n^2)$ operations
    \begin{align*}
      \mat{A} =
      \begin{pmatrix}
        4 & 2 & 6 \\
        3 & 5 & 9 \\
        1 & 2 & 3
      \end{pmatrix}
      =
      \begin{pmatrix}
        1 & 0 & 0 \\
        0.75 & 1 & 0 \\
        0.25 & 0.428 & 1
      \end{pmatrix}
      \begin{pmatrix}
        4 & 2 & 6\\
        0 & 3.5 & 4.5\\
        0 & 0 & -4.28\\
      \end{pmatrix}
      = \mat{L} \mat{U}\pause
    \end{align*}
  \item Solving $\bm{x} = \mat{U}^{-1} (\mat{L}^{-1} \bm{b})$ is also
    $\Theta(n^2)$ because of the structure of $\mat{L}$ and
    $\mat{U}$\pause
  \end{itemize}
\end{PauseHighLight}

\end{slide}

%%%%%%%%%%%%%%%%%%%%%%% Next Slide %%%%%%%%%%%%%%%%%%%%%%%

\begin{slide}
\section{LU-Decomposition}

\begin{PauseHighLight}
  \begin{itemize}
  \item LU-decomposition is achieved by Gaussian-elimination\pause
  \item This is a straightforward procedure, but if done carelessly can
    lead to large rounding errors\pause
  \item The standard solution is to permute the rows of the
    matrix (aka pivoting) to prevent loss of accuracy\pause
  \item In addition we can ``polish'' solutions
    \begin{align*}
      \mat{A} (\bm{x} + \delta\bm{x}) - \bm{b} = \bm{\epsilon}\pause
    \end{align*}
  \item Thus $\delta\bm{x} = \mat{A}^{-1} \bm{\epsilon}$ which we can
    use to get an improved estimate of $\bm{x}$\pause
  \end{itemize}
\end{PauseHighLight}

\end{slide}

%%%%%%%%%%%%%%%%%%%%%%% Next Slide %%%%%%%%%%%%%%%%%%%%%%%

\begin{slide}
\section{Norms}

\begin{PauseHighLight}
  \begin{itemize}
  \item With some work we can get a good approximation to $\bm{x}$ such
    that $\mat{A} \bm{x} = \bm{b}$\pause
  \item But what if we have some error in $\bm{b}$, this induces an
    error $\delta \bm{x} = \mat{A}^{-1} \, \delta\bm{b}$\pause
  \item How big is $\delta \bm{x}$?\pause
  \item To measure the size of a vector we use a norm $\|\delta
    \bm{x}\|$, which is a number encoding the size of $\delta \bm{x}$
    \pause
  \item There are a number of different norms, e.g.
    \begin{align*}
      \|\delta \bm{x}\|_2 &= \sqrt{\delta x_1^2 + \cdots +
        \delta x_n^2}, \pause &
      \|\delta \bm{x}\|_1 &= |\delta x_1| + \cdots +
        |\delta x_n| \pause
    \end{align*}
  \end{itemize}
\end{PauseHighLight}

\end{slide}

%%%%%%%%%%%%%%%%%%%%%%% Next Slide %%%%%%%%%%%%%%%%%%%%%%%

\begin{slide}
\section{Conditioning}

\begin{PauseHighLight}
  \begin{itemize}
  \item The size of the error in $\bm{x}$: $\mat{A}\,\bm{x}=\bm{b}$
    when $\bm{b}$ has error $\delta \bm{b}$ is
    \begin{align*}
      \|\delta \bm{x}\| = \| \mat{A}^{-1} \delta \bm{b}\| \pause \leq
      \| \mat{A}^{-1} \| \, \|\delta \bm{b}\|\pause
    \end{align*}
  \item Where $\| \mat{A}^{-1} \|$ provides a measure of the size of
    the error in the worst case\pause
  \item For large matrices $\| \mat{A}^{-1} \|$ can be large meaning
    that any error in $\bm{b}$ is potentially magnified significantly\pause
  \item Such matrices are said to be ill-conditions\pause
  \item Ill-conditioning is not to due with rounding errors but the
    structure of the matrix\pause
  \end{itemize}
\end{PauseHighLight}

\end{slide}

%%%%%%%%%%%%%%%%%%%%%%% Next Slide %%%%%%%%%%%%%%%%%%%%%%%

\begin{slide}
\section[-1]{Linear Algebra}

\begin{PauseHighLight}
  \begin{itemize}
  \item Linear algebra packages provide an important set of tools used
    for solving linear equations\pause
  \item Care has to be taken to ensure that needless operations (such as
    inverting a matrix) are not done\pause
  \item Algorithms must ensure that as little accuracy as possible is
    lost (e.g. by permuting rows in LU-decomposition)\pause
  \item Even when the algorithms are precise, small errors can get
    amplified in some operations, which requires care in formulating the
    problem\pause
  \item The idea of poor conditioning (errors being amplified) is useful
    in understanding many numerical tasks\pause
  \end{itemize}
\end{PauseHighLight}

\end{slide}

%%%%%%%%%%%%%%%%%%%%%%% Next Slide %%%%%%%%%%%%%%%%%%%%%%%

\begin{slide}
\section{Lessons}

\begin{PauseHighLight}
  \begin{itemize}
  \item Be wary of numerical algorithms, because computers approximate
    real numbers you don't always get what you expect\pause
  \item Don't avoid numerical algorithms, they are hugely important with
    vast areas of applications\pause
  \item This is a well studied area with large libraries of reliable
    algorithms that work most of the time\pause
  \item There are some good books such as ``Numerical Recipes'' by
    Press, \textit{et al.}, which describes the issues and provides
    code\pause
  \end{itemize}
\end{PauseHighLight}

\end{slide}
