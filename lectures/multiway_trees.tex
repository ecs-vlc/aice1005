%Master File:lectures.tex

\begin{slide}
\lesson{Sometimes It Pays Not to Be Binary}

\vspace*{-2cm}
\begin{center}
  \includegraphics[width=\textwidth]{btree}
\end{center}

\keywords{B-Trees, Tries, Suffix Trees}
\end{slide}

%%%%%%%%%%%%%%%%%%%%%%% Next Slide %%%%%%%%%%%%%%%%%%%%%%%
\renewcommand{\Outline}{%
\begin{slide}
\section{Outline}

\begin{minipage}{10cm}
  \vfill
  \begin{enumerate}
    \outlineitem{B-Trees}{btrees}
    \outlineitem{Tries}{tries}
    \outlineitem{Suffix Tree}{suffix}
  \end{enumerate}
  \vfill
\end{minipage}\hfill
\begin{minipage}{14cm}
  \includegraphics[width=14cm]{btree}
\end{minipage}
\end{slide}
\addtocounter{outlineitem}{1}
}

\setcounter{outlineitem}{1}
\Outline
\toptarget{firstoutline}

%%%%%%%%%%%%%%%%%%%%%%% Next Slide %%%%%%%%%%%%%%%%%%%%%%%

\begin{slide}
\section{B-Trees}

\begin{PauseHighLight}
  \begin{itemize}
  \item \emph{B-trees} are balanced trees for fast search, finding successors
    and predecessors, insert, delete, maximum, minimum, etc.\pause
  \item Not to be confused with binary trees\pause
  \item They are designed to keep related data close to each other in
    (disk) memory to minimise retrieval time\pause
  \item Important when working with large amount of data that is stored
    on secondary storage (e.g. disks)\pause
  \item Used extensively in databases\pause
  \end{itemize}
\end{PauseHighLight}

\end{slide}

%%%%%%%%%%%%%%%%%%%%%%% Next Slide %%%%%%%%%%%%%%%%%%%%%%%

\begin{slide}
\section{When Big-O Doesn't Work}

\begin{PauseHighLight}
  \begin{itemize}
  \item An underlying assumption of Big-O is that all elementary
    operations take roughly the same amount of time\pause
  \item This just isn't true of disk look-up\pause
  \item The typical time of an elementary operation on a modern
    processor is $10^{-9}$ seconds\pause
  \item But a typical hard disk might do 7\,200 revolutions per
    minute or 120 revolutions per second\pause
  \item The typical time it takes to locate a record is around 10ms or
    $10^7$ times slower than an elementary operation\pause
  \end{itemize}
\end{PauseHighLight}

\end{slide}

%%%%%%%%%%%%%%%%%%%%%%% Next Slide %%%%%%%%%%%%%%%%%%%%%%%

\begin{slide}
\section{Accessing Data from Disk}

\begin{PauseHighLight}
  \begin{itemize}
  \item When accessing data from disk minimising the number of disk
    accesses is critical for good performance\pause
  \item In database applications we want to store data as large
    sets\pause
  \item Storing data in binary trees is disastrous as we typically need
    around $\log_2(n)$ disk accesses before we locate our data\pause
  \item It is not unusual in databases for $n=10\,000\,000$ so that
    $\log_2(n) \approx 24$\pause
  \item Using binary trees it would often take several seconds to find a
    record\pause
  \end{itemize}
\end{PauseHighLight}

\end{slide}

%%%%%%%%%%%%%%%%%%%%%%% Next Slide %%%%%%%%%%%%%%%%%%%%%%%

\begin{slide}
\section{Multiway-Trees}

\begin{PauseHighLight}
  \begin{itemize}
  \item To remedy this we can use M-way trees so that the access time is
    \begin{align*}
      \log_M(n) = \frac{\log_2(n)}{\log_2(M)}\pause
    \end{align*}
  \item In practice we might use $M\approx 200 \approx 2^8$ so we can
    reduce the depth of the tree by around a factor of 8\pause
  \item The basic data structures for doing this is the B-tree\pause
  \item There are many variants of B-tree, all trying to squeeze a bit
    more performances from the basic structure\pause
  \end{itemize}
\end{PauseHighLight}

\end{slide}

%%%%%%%%%%%%%%%%%%%%%%% Next Slide %%%%%%%%%%%%%%%%%%%%%%%

\begin{slide}
\section{$\mathbf{B}^{+}$ Tree}

\begin{PauseHighLight}
  \begin{itemize}
  \item A pretty basic implementation would obey the following rules
    \begin{enumerate}\squeeze
    \item The data items are stored at leaves\pause
    \item The non-leaf nodes store up to M-1 keys to guide the
      search: key $i$ represents the smallest key in subtree
      $i+1$\pause
    \item The root is either a leaf or has between 2 and $M$ children\pause
    \item All non-leaf nodes except the root have between $\lceil M/2
      \rceil$ and $M$ children\pause
    \item All leaves are at the same depth and have between $\lceil L/2
      \rceil$ and $L$ data entries\pause
    \end{enumerate}
  \end{itemize}
\end{PauseHighLight}

\end{slide}

%%%%%%%%%%%%%%%%%%%%%%% Next Slide %%%%%%%%%%%%%%%%%%%%%%%

\begin{slide}
\section{Choosing $M$ and $L$}

\begin{PauseHighLight}
  \begin{itemize}
  \item The choice of $M$ and $L$ depends on the block size (the
    information read in one go from disk)\pause
  \item It also depends on the type of data that is being stored
    (integer, reals, strings, etc.)\pause
  \item $M$ and $L$ might be in the hundreds or thousands\pause
  \item In the examples below we consider tiny $M = L = 5$ which is
    unrealistic, but drawable\pause
  \end{itemize}
\end{PauseHighLight}

\end{slide}


%%%%%%%%%%%%%%%%%%%%%%% Next Slide %%%%%%%%%%%%%%%%%%%%%%%

\begin{slide}
\section{B-Tree Example}
\begin{itemize}
\item $M=5$, $L=5$
\end{itemize}
\pb\pause
\begin{center}
\multipdf[width=\linewidth]{bTree} \pause
\end{center}
\end{slide}

%%%%%%%%%%%%%%%%%%%%%%% Next Slide %%%%%%%%%%%%%%%%%%%%%%%

\begin{slide}
\section{Other Changes}

\begin{PauseHighLight}
  \begin{itemize}
  \item If the root is full then it can be split into two and a new root
    created\pause
  \item B-trees also have to allow the removal of records without losing
    its structure\pause
  \item There are a number of variant strategies (e.g. neighbouring
    nodes can adopt a child if the current node cannot expand any more)\pause
  \item The actual implementation of B-trees is tricky because there are
    many special cases\pause
  \end{itemize}
\end{PauseHighLight}

\end{slide}

%%%%%%%%%%%%%%%%%%%%%%% Next Slide %%%%%%%%%%%%%%%%%%%%%%%

\begin{slide}
\section{B-Tree Summary}

\begin{PauseHighLight}
  \begin{itemize}
  \item B-trees are an important data structure for databases where
    reducing the number of disk searches is vital\pause
  \item They tend to be much more complex than the other data structures
    we have seen\pause
  \item The problem of disk access can be improved by replacing disk
    memory with solid-state drives (still slow compared to memory)\pause
  \item For massive databases new data structures have been developed to
    allow faster (although less flexible) information access
    (e.g. NOSQL, MongoDB, Neo4j)\pause
  \end{itemize}
\end{PauseHighLight}
\end{slide}



%%%%%%%%%%%%%%%%%%%%%%% Next Slide %%%%%%%%%%%%%%%%%%%%%%%
\Outline % Tries
%%%%%%%%%%%%%%%%%%%%%%% Next Slide %%%%%%%%%%%%%%%%%%%%%%%

\begin{slide}
\section[-1]{Tries}

\begin{PauseHighLight}
  \begin{itemize}
  \item A \emph{Trie} (pron. `try') or \emph{digital tree} is a multiway
    tree often used for storing large sets of words\pause
  \item They are trees with a possible branch for every letter of an
    alphabet\pause
  \item Their names comes from the word \textit{re\emph{trie}val}\pause
  \item Tries usually compactify the edges in the tree\pause
  \item All words end with a special letter ``\$''\pause
  \end{itemize}
\end{PauseHighLight}

\end{slide}


%%%%%%%%%%%%%%%%%%%%%%% Next Slide %%%%%%%%%%%%%%%%%%%%%%%

\begin{slide}
\section[-1]{Trie}
\pause
\pb
\begin{center}
  \includegraphics[width=\linewidth]{trie1}\mypl{1}
  \multido{\i=2+1}{2}{%
    \llap{\includegraphics[width=\linewidth]{trie\i}}\mypl{\i}}
\end{center}

\end{slide}

%%%%%%%%%%%%%%%%%%%%%%% Next Slide %%%%%%%%%%%%%%%%%%%%%%%

\begin{slide}
\section{Uses of Tries}

\begin{PauseHighLight}
  \begin{itemize}
  \item Tries are yet another way of implementing sets\pause
  \item They provide quick insertion, deletion and find\pause
  \item Typically considerably quicker than binary trees and hash
    tables\pause
  \item They are particularly good for spell checkers, completion
    algorithms, longest-prefix matching, hyphenation\pause
  \item Each search finds the longest match between the words in the set
    and the query\pause
  \end{itemize}
\end{PauseHighLight}

\end{slide}

%%%%%%%%%%%%%%%%%%%%%%% Next Slide %%%%%%%%%%%%%%%%%%%%%%%

\begin{slide}
\section[-2]{Trie for 31 Most Common English Words}
\pause
\pb
\pauselevel{=1}
\begin{center}
  \multipdf[width=0.99\linewidth]{trieTable}\pause
\end{center}
\end{slide}

%%%%%%%%%%%%%%%%%%%%%%% Next Slide %%%%%%%%%%%%%%%%%%%%%%%

\begin{slide}
\section{Disadvantage of Tries}

\begin{PauseHighLight}
  \begin{itemize}
  \item Table-based tries typically waste large amounts of memory\pause
  \item Often table-based tries are used for the first few layers, while
    lower levels use a less memory intensive data structure\pause
  \item These days memory is less of a problem so table-based tries are
    acceptable for some applications\pause
  \item There are many implementations of tries each suited to a
    particular task\pause
  \end{itemize}
\end{PauseHighLight}

\end{slide}

%%%%%%%%%%%%%%%%%%%%%%% Next Slide %%%%%%%%%%%%%%%%%%%%%%%

\begin{slide}
\section{Binary Tries}

\begin{PauseHighLight}
  \begin{itemize}
  \item One extreme (though not uncommon) solution to address memory
    issues is to build a bit-level trie so the data-structure is a
    binary tree\pause
  \item It differs from a binary tree in that the decisions to go left
    or right depends on the current bit\pause
  \item Although you lose the advantage of a multiway tree (of reducing
    the depth) it does find the longest match and it speeds up finds
    which fail\pause
  \end{itemize}
\end{PauseHighLight}

\end{slide}

%%%%%%%%%%%%%%%%%%%%%%% Next Slide %%%%%%%%%%%%%%%%%%%%%%%

\begin{slide}
\section{Why Tries?}

\begin{PauseHighLight}
  \begin{itemize}
  \item Tries are a classic example of a trade-off between memory and
    computational complexity\pause
  \item Tries are slightly specialist and tend to get used in very
    particular applications
    \begin{itemize}
    \item Finding longest matches
    \item Completion, spell checking, etc.\pause
    \end{itemize}
  \item A basic trie is not too complicated, however, \ldots\pause
  \item There are many implementation which try to overcome the
    difficulty of wasting too much memory\pause
  \end{itemize}
\end{PauseHighLight}

\end{slide}


%%%%%%%%%%%%%%%%%%%%%%% Next Slide %%%%%%%%%%%%%%%%%%%%%%%
\Outline % Suffix Trees
%%%%%%%%%%%%%%%%%%%%%%% Next Slide %%%%%%%%%%%%%%%%%%%%%%%
\begin{slide}
\section[-1]{Suffix Tree}

\begin{PauseHighLight}
  \begin{itemize}
  \item Suffix tree is a trie of all suffixes of a string\pause
  \item E.g. banana
    \begin{center}
      \includegraphics[width=\linewidth]{suffixBanana}\pause
    \end{center}
  \end{itemize}
\end{PauseHighLight}
\end{slide}

%%%%%%%%%%%%%%%%%%%%%%% Next Slide %%%%%%%%%%%%%%%%%%%%%%%

\begin{slide}
\section{Importance of Suffix Tree}

\begin{PauseHighLight}
  \begin{itemize}
  \item The first linear-time algorithm for computing suffix trees was
    proposed by Peter Weiner in 1973, a more space efficient algorithm
    was proposed by Edward M. McCreight in 1976\pause
  \item Esko Ukkonen in 1995 proposed a variant of McCreight's
    algorithm, but in a way that was much easier to understand\pause
  \item It really only got implemented after this\pause
  \item They are very important for string-based algorithms\pause
  \item The classic application is in finding a match for a query
    string, $Q$, in a text, $T$\pause
  \end{itemize}
\end{PauseHighLight}

\end{slide}

%%%%%%%%%%%%%%%%%%%%%%% Next Slide %%%%%%%%%%%%%%%%%%%%%%%

\begin{slide}
\section{String Matching}

\begin{PauseHighLight}
  \begin{itemize}
  \item To find a match of a query string, $Q$, in a text, $T$, we can
    first construct the suffix tree of the string $T$ we then simple look
    up the query, $Q$, using the trie
  \end{itemize}
    \begin{center}
      \includegraphics[width=\linewidth]{suffixBanana}\pause
    \end{center}
\end{PauseHighLight}

\end{slide}

%%%%%%%%%%%%%%%%%%%%%%% Next Slide %%%%%%%%%%%%%%%%%%%%%%%

\begin{slide}
\section{Complexity of Suffix Trees}

\begin{PauseHighLight}
  \begin{itemize}
  \item Using a regular trie for a suffix tree would typically use far
    too much memory to be useful\pause
  \item However, by using pointers to the original text it is possible
    to build a suffix tree using $O(n)$ memory where $n$ is the length
    of the text\pause
  \item Furthermore (and rather incredibly) there is a linear time
    ($O(n)$) algorithm to construct the trie\pause
  \item The algorithm is not however trivial to understand\pause
  \end{itemize}
\end{PauseHighLight}

\end{slide}

%%%%%%%%%%%%%%%%%%%%%%% Next Slide %%%%%%%%%%%%%%%%%%%%%%%

\begin{slide}
\section{Uses of Suffix Trees}

\begin{PauseHighLight}
  \begin{itemize}
  \item Suffix trees are efficient whenever it is likely that you will
    do multiple searches\pause
  \item Exact word matching is in itself a very important application\pause
  \item Suffix trees in combination with dynamic programming (which we
    will eventually get to) can be used to do inexact matching (finding
    the match with the smallest edit distance)\pause
  \item Suffix trees get used in bioinformatics, advanced machine
    learning algorithms, \ldots\pause
  \end{itemize}
\end{PauseHighLight}

\end{slide}

%%%%%%%%%%%%%%%%%%%%%%% Next Slide %%%%%%%%%%%%%%%%%%%%%%%

\begin{slide}
\section{Lessons}

\begin{PauseHighLight}
  \begin{itemize}
  \item Multiway trees can considerable speed up search over binary
    trees\pause
  \item They are very important in some specialised applications
    (e.g. databases, spell-checking, completion, suffix trees)\pause
  \item They are not as general purpose as binary search trees and are
    more complicated to implement\pause
  \item But they can give the best performance\pause---sometimes
    performance matters enough to make it worthwhile implementing
    multiway trees\pauseb
  \end{itemize}
\end{PauseHighLight}

\end{slide}
