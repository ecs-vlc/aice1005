%Master File:lectures.tex

\lesson{Follow a Pattern}
\vspace{-1cm}
\begin{center}
  \includegraphics[height=10cm]{pattern}
\end{center}
\keywords{Design patterns, constructor-destructor pattern, iterators}

%%%%%%%%%%%%%%%%%%%%%%% Next Slide %%%%%%%%%%%%%%%%%%%%%%%
\renewcommand{\Outline}{%
\begin{slide}
\section{Outline}
\begin{minipage}{15cm}
  \vfill
  \begin{enumerate}
    \outlineitem{Design Patterns}{intro}
    \begin{itemize}
    \item \toplink{condes}{Constructor-Destructor Pattern}
    \end{itemize}
    \outlineitem{Comparable/Comparator Pattern}{compare}
    \outlineitem{The Iterator Pattern}{iterators}
    \outlineitem{A Linked List Iterator}{lliter}
  \end{enumerate}
  \vfill
\end{minipage}\hfill
\begin{minipage}{8cm}
  \includegraphics[height=8cm,angle=90]{pattern}
\end{minipage}
\end{slide}
\addtocounter{outlineitem}{1}
}

\setcounter{outlineitem}{1}
\Outline
\toptarget{firstoutline}

%%%%%%%%%%%%%%%%%%%%%%% Next Slide %%%%%%%%%%%%%%%%%%%%%%%

\begin{slide}
\section{Software Design Patterns}

\begin{PauseHighLight}
  \begin{itemize}
  \item A design pattern is a solution method for a commonly met
    problem\pause
  \item In the early 90's it was realised that many problems reoccur in
    computing\pause
  \item Rather than re-invent the wheel, good solutions were collected
    as design patterns\pause
  \item Using design patterns provides a consistency to code writing\pause
  \end{itemize}
\end{PauseHighLight}
\end{slide}

%%%%%%%%%%%%%%%%%%%%%%% Next Slide %%%%%%%%%%%%%%%%%%%%%%%

\begin{slide}
\section{Simple Design Patterns}

\begin{PauseHighLight}
  \begin{itemize}
  \item In many situations a method returns two results.\pause\ E.g.
    \begin{itemize}
    \item a minimisation method calculates the location of the
      minimum and its value\pause
    \item In integrating a function we want to know the value of the
      integral and its error\pause
    \end{itemize}
  \item One common solution is the ``pair pattern'', i.e. the method
    returns a Pair
    \begin{java}
      public class Pair<T,S>
      {
        public T first;
        public S second;
      }
    \end{java}\pause
  \end{itemize}
\end{PauseHighLight}

\end{slide}

%%%%%%%%%%%%%%%%%%%%%%% Next Slide %%%%%%%%%%%%%%%%%%%%%%%

\begin{slide}
\section[-1]{The Decorator Pattern}

\begin{PauseHighLight}
  \begin{itemize}
  \item The decorator pattern is is used when a class could have a lot of
    functionality\pause
  \item In this pattern we dynamically add functionality\pause
  \item Rather than have one complicated class in the decorator pattern we
    create a hierarchy of classes\pause
  \item This pattern is used for input and output streams
    \begin{java}
      FileOutputStream fout = new FileOutputStream("filename");
      BufferedOutputStream bout = new BufferedOutputStream(fout);
      GZIPOutputStream gout = new GZIPOutputStream(bout);
      ObjectOutputStream oout = new ObjectOutputStream(gout);
      oout.writeObject(someObject);
    \end{java}\pause
  \end{itemize}
\end{PauseHighLight}
\end{slide}

%%%%%%%%%%%%%%%%%%%%%%% Next Slide %%%%%%%%%%%%%%%%%%%%%%%

\begin{slide}
\section[-2]{Freeing Resources}
\toptarget{condes}
\begin{PauseHighLight}
  \begin{itemize}
  \item A common problem in writing large programs is to make sure that
    no resources get lost\pause
  \item For example, there is often a limited number of files that can
    be open\pause
  \item Closing files therefore frees a resource\pause
  \item This is a particular issue in C++ which does not have automatic
    garbage collection\pause
    \begin{itemize}
    \item Any time you reserve memory using \jl$new$ you have to
      free it using \jl$delete$\pause
    \item Failure to do so results in a \textit{memory leak} which can
      result in a program using up all available memory\pause
    \end{itemize}
  \end{itemize}
\end{PauseHighLight}
\end{slide}


%%%%%%%%%%%%%%%%%%%%%%% Next Slide %%%%%%%%%%%%%%%%%%%%%%%

\begin{slide}
\section[-2]{Creating Objects in C++ and Java}

\begin{PauseHighLight}
  \begin{itemize}
  \item In Java declaring a \texttt{Widget} creates enough memory for a
    reference to a widget
    \begin{java}
      Widget widget;$\pause$
      widget = new Widget();
    \end{java}
    To create a \texttt{Widget} we you use \texttt{new}\pause
  \item In C++ declaring a \texttt{Widget} creates a \texttt{Widget}
    \begin{java}
      Widget widget;
    \end{java}
    where the default constructor is called\pause
  \item To declare a \texttt{Widget} with a non-default constructor in
    C++ we write
    \begin{java}
      Widget widget(parameters);$\pause$
    \end{java}
  \end{itemize}
\end{PauseHighLight}

\end{slide}

%%%%%%%%%%%%%%%%%%%%%%% Next Slide %%%%%%%%%%%%%%%%%%%%%%%

\begin{slide}
\section{References in C++}

\begin{PauseHighLight}
  \begin{itemize}
  \item You can use references in C++ but the syntax is slightly
    different\pause
  \item To create a reference to a \texttt{Widget} in C++ we would use
    \begin{java}
      Widget& widget;
      widget = new Widget();$\pause$
    \end{java}
  \item But, because there is no garbage collection is C++, you, the
    programmer, are responsible for destroying the Widget\pause
  \end{itemize}
\end{PauseHighLight}

\end{slide}


%%%%%%%%%%%%%%%%%%%%%%% Next Slide %%%%%%%%%%%%%%%%%%%%%%%

\begin{slide}
\section[-2]{Destructors in C++}

\begin{PauseHighLight}
  \begin{itemize}\squeeze
  \item In C++ you can use either references to objects or objects
    themselves\pause
  \item This makes the objects more like primitive types\pause
  \item In C++ you tend to create objects (rather than references)\pause
  \item When an object goes out of scope the object destructor is
    called\pause
  \item There is therefore no need to destroy objects
    \begin{java}
      function() {
        vector<double> v(10);
        $\vdots$
      }
    \end{java}
    the destructor for the \texttt{vector<double>} will be called\pause
  \item This destructor frees all the memory used by
  \texttt{vector<double>}\pause
  \end{itemize}
\end{PauseHighLight}

\end{slide}

%%%%%%%%%%%%%%%%%%%%%%% Next Slide %%%%%%%%%%%%%%%%%%%%%%%

\begin{slide}
\section[-2]{RAII:Constructor-Destructor Pattern}

\begin{PauseHighLight}
  \begin{itemize}\squeeze
  \item One common pattern to get around the problem of losing resources
    is to wrap up all resources in a class\pause
  \item The resource is grabbed by the constructor method\pause
  \item The resource is freed in the destructor method\pause
  \item In Java the \texttt{finalize()} method is called when the object
    is removed by the garbage collector\pause---this is not very useful
    as this doesn't necessarily happen\pause
  \item In C++ the destructor is called when the objects goes out of
    scope\pause
  \item This constructor-destructor pattern is used in the standard
    template library so you don't need to worry about freeing
    memory\pause---one very good reason for using the STL!\pause
  \end{itemize}
\end{PauseHighLight}
\end{slide}


%%%%%%%%%%%%%%%%%%%%%%% Next Slide %%%%%%%%%%%%%%%%%%%%%%%
\Outline
%%%%%%%%%%%%%%%%%%%%%%% Next Slide %%%%%%%%%%%%%%%%%%%%%%%

\begin{slide}
\section{Comparing Objects}

\begin{PauseHighLight}
  \begin{itemize}
  \item If you want to sort any objects then you have to specify an
    \emph{ordering}\pause
  \item This is also true of finding the maximum or minimum element in a
    collection, etc.\pause
  \item That is, you need some way of comparing two objects and saying
    that object 1 comes before object 2\pause
  \item Java has two patterns to achieve this either
    \begin{itemize}
    \item The objects themselves are \emph{Comparable}\pause\ or
    \item You supply a \emph{Comparator} class to impose an ordering\pause
    \end{itemize}
  \end{itemize}
\end{PauseHighLight}

\end{slide}

%%%%%%%%%%%%%%%%%%%%%%% Next Slide %%%%%%%%%%%%%%%%%%%%%%%



\begin{slide}
\section[-1]{Comparable}

\begin{PauseHighLight}
  \begin{itemize}
  \item In Java to make a class Widget comparable you make it implement
    \jl$Comparable<Widget>$\pause
  \item The comparable interface contains a single method
    \begin{java}
      public interface Comparable<T>
      {
         public int compareTo(T o);
      }$\pause$
    \end{java}
  \item where \jl$x.compareTo(y)$ returns
    \begin{itemize}
    \item \textbf{0} if $x=y$
    \item \textbf{a negative integer} if $x$ precedes $y$
    \item \textbf{a positive integer} if $x$ succeeds $y$\pause
    \end{itemize}
  \item Write \jl$x.compareTo(y)<0$ to mean $x<y$\pause
  \end{itemize}
\end{PauseHighLight}

\end{slide}

%%%%%%%%%%%%%%%%%%%%%%% Next Slide %%%%%%%%%%%%%%%%%%%%%%%

\begin{slide}
\section[-2]{Implementing an Integer class}
\begin{PauseHighLight}
  \begin{itemize}
  \item To implement the class Integer we have
    \begin{java}
      class Integer implements Comparable<Integer>
      {
         private int value;
         ...
         public int compareTo(Integer that){
           if ( this.value < that.value )
              return -1;
           if ( this.value.equals(that.value) )
              return 0;
           return 1;
         }
         ... 
      }$\pause$
    \end{java}
  \item This defines the \emph{natural ordering} of the class\pause
 \item This should be sensible (antisymmetric, transitive and usually
    compatible with equals)\pause
  \end{itemize}
\end{PauseHighLight}

\end{slide}

%%%%%%%%%%%%%%%%%%%%%%% Next Slide %%%%%%%%%%%%%%%%%%%%%%%

\begin{slide}
\section[-2]{Comparator}

\begin{PauseHighLight}
  \begin{itemize}
  \item Sometimes you want to order a class which isn't Comparable or
    you want to use a different ordering to the natural ordering\pause
  \item To do this you create a new class \jl$WidgetCompartor$ which
    implements the interface \jl$Comparator<Widget>$ where
    \begin{java}
      public interface Comparator<T>
      {
         public int compare(T o1, T o2);
      }$\pause$
    \end{java}
  \item \jl$compare(T o1, To2)$ returns a value that is negative, zero
    or positive depending on whether $o1<o2$, $o1=o2$ or $o1>o2$\pause
  \item You pass the comparator to your sort method, etc.
    \begin{java}
      Collections.sort(WidgetList, WidgetComparator);
      Collections.max(WidgetList, WidgetComparator);$\pause$
    \end{java}
  \end{itemize}
\end{PauseHighLight}

\end{slide}


%%%%%%%%%%%%%%%%%%%%%%% Next Slide %%%%%%%%%%%%%%%%%%%%%%%
\Outline
%%%%%%%%%%%%%%%%%%%%%%% Next Slide %%%%%%%%%%%%%%%%%%%%%%%

\begin{slide}
\section[-1]{Iterators}

\begin{PauseHighLight}
  \begin{itemize}
  \item One of the jobs we left undone at the end of the last lecture
    was to iterate through a linked list\pause
  \item Iterating through collections is a common operation\pause
  \item We would like to follow a common pattern\pause
  \item This is known as the ``iterator pattern''\pause
  \item It is used in both Java Collections and C++ STL\pause
  \item In Java it is implemented via an interface\pause
\end{itemize}
\end{PauseHighLight}
\end{slide}

%%%%%%%%%%%%%%%%%%%%%%% Next Slide %%%%%%%%%%%%%%%%%%%%%%%

\begin{slide}
\section[-1]{The Iterator Interface}

\begin{java}
public interface Iterator<T>
{
  /**
   * Determine if Iterator object has a next object
   * @return true - if this object has a next object
   */
   boolean hasNext();

   /**
    * Advances Iterator object and return next element
    * @return next element
    * @throws NoSuchElementException - if no next element
    */
    T next();

   /**
    * Remove the element most recently called by next
    * @throws IllegalStateException - if next() has not been called
    */
    void remove();
}
\end{java}\vspace{-1cm}
\end{slide}

%%%%%%%%%%%%%%%%%%%%%%% Next Slide %%%%%%%%%%%%%%%%%%%%%%%

\begin{slide}
\section[-2]{Javadoc Output}

\begin{center}
  \includegraphics[width=0.95\linewidth]{iterator1}\vspace*{-2cm}
\end{center}
\end{slide}

%%%%%%%%%%%%%%%%%%%%%%% Next Slide %%%%%%%%%%%%%%%%%%%%%%%

\begin{slide}
\section[-2]{Javadoc Output}

\begin{center}
  \includegraphics[width=0.95\linewidth]{iterator2}\vspace*{-2cm}
\end{center}
\end{slide}

%%%%%%%%%%%%%%%%%%%%%%% Next Slide %%%%%%%%%%%%%%%%%%%%%%%

\begin{slide}
\section[-2]{Using Iterators}

\begin{PauseHighLight}
  \begin{itemize}
  \item It is now easy to iterate over any collection, e.g.
    \begin{java}
public class TestIterator
{
    public static void main(String [] args )
    {
        List<Integer> list = new LinkedList<Integer>();
        for (int i=0; i<5 ;i++)
            list.add(i);
        Iterator<Integer> iter = list.iterator();
        while (iter.hasNext())
            System.out.println(iter.next());
    }
}$\pause$
    \end{java}\vspace{-1.5cm}
  \item Note the iterator pattern
    \begin{java}
        Iterator<Integer> iter = list.iterator();
        while (iter.hasNext())
            System.out.println(iter.next());$\pause$
    \end{java}\vspace{-1.5cm}
  \end{itemize}
\end{PauseHighLight}
\end{slide}

%%%%%%%%%%%%%%%%%%%%%%% Next Slide %%%%%%%%%%%%%%%%%%%%%%%

\begin{slide}
\section[-1]{Enhanced \texttt{for} Statement}

\begin{PauseHighLight}
  \begin{itemize}
  \item Java 1.5+ provides an ``enhanced \texttt{for} statements'' that
    work with collections which provide iterators\pause
  \item Rather than
    \begin{java}
        Iterator<Integer> iter = list.iterator();
        while (iter.hasNext())
            System.out.println(iter.next());      
    \end{java}\pause\vspace{-1cm}
  \item We can even more simply write
    \begin{java}
        for(Integer i: list)
            System.out.println(i);
    \end{java}\pause\vspace{-1cm}
  \item This is intended to enhance readibility of code\pause
  \end{itemize}
\end{PauseHighLight}

\end{slide}

%%%%%%%%%%%%%%%%%%%%%%% Next Slide %%%%%%%%%%%%%%%%%%%%%%%

\begin{slide}
\section[-1]{Generic Algorithms}

\begin{PauseHighLight}
  \begin{itemize}
  \item Using iterators allows us to write generic methods
    \begin{java}
      double sum(Collection<Double> c)
      {
        double sum = 0.0;
        for (Double x: c)
            sum += x;
        return sum;
      }$\pause$
    \end{java}\vspace{-1cm}
  \item This will work for any collection which implements the
    \texttt{Collection} interface\pause
  \item This allows us to write algorithms independent of the underlying
    data structure (this is used for instance by the AbstractList
    class)\pause
  \end{itemize}
\end{PauseHighLight}
\end{slide}

%%%%%%%%%%%%%%%%%%%%%%% Next Slide %%%%%%%%%%%%%%%%%%%%%%%

\begin{slide}
\section{Lists}
\vspace{-1cm}
\begin{center}
  \includegraphics[height=14cm]{java-list}
\end{center}
\vspace{-1cm}
\end{slide}

%%%%%%%%%%%%%%%%%%%%%%% Next Slide %%%%%%%%%%%%%%%%%%%%%%%

\Outline
%%%%%%%%%%%%%%%%%%%%%%% Next Slide %%%%%%%%%%%%%%%%%%%%%%%

\begin{slide}
\section[-2]{Implementing An Iterator}

\begin{PauseHighLight}
  \begin{itemize}
  \item We create an iterator using the \jl$iterator()$ method
\begin{java}
public class MyLinkedList<T>
{
    private Node<T> head;
    private int no_elements;

    // methods described in last lectures

    public Iterator<T> iterator()
    {
        return new LLIter();
    }

    private class LLIter implements Iterator<T>
    {
        // Implementation
    }
}$\pause$
\end{java}\vspace{-1cm}
  \item \jl$LLIter$ is a (private) nested class inside
  \jl$MyLinkedList<T>$\pause
  \end{itemize}
\end{PauseHighLight}
\end{slide}

%%%%%%%%%%%%%%%%%%%%%%% Next Slide %%%%%%%%%%%%%%%%%%%%%%%

\begin{slide}
\section[-1.5]{Linked List Iterator}

\begin{PauseHighLight}
\begin{java}
    private class LLIter implements Iterator<T>$\pause$
    {
        private Node<T> next;$\pause$

        public LLIter() { next = head; }$\pause$

        public boolean hasNext() { return next != null; }$\pause$

        public T next() {
            if (next==null)
                throw new NoSuchElementException();
            T theElement = next.element;
            next = next.next;
            return theElement;
        }$\pause$

        public void remove() { // inefficient to implement, i.e. O(n)
            throw new UnsupportedOperationException();
        }$\pause$
    }
\end{java}
\end{PauseHighLight}

\end{slide}

%%%%%%%%%%%%%%%%%%%%%%% Next Slide %%%%%%%%%%%%%%%%%%%%%%%

\begin{slide}
\section[-1]{Notes on Iterator Class}

\begin{PauseHighLight}
  \begin{itemize}
  \item Implementation of constructor, \texttt{hasNext()} and
    \texttt{next()} is trivial\pause
  \item We don't implement \texttt{remove()} as this is inefficient for a
    singly linked list---we need to find the predecessor\pause
  \item Our singly linked list class \texttt{MyLinkedList<T>} doesn't
    implement the \texttt{Collection} interface so we can't use the
    enhanced \texttt{for} loop\pause
  \item The doubly linked list class \texttt{LinkedList<T>} which is part
    of the Collections framework implements the \texttt{ListIterator<T>}
    interface\pause
  \end{itemize}
\end{PauseHighLight}

\end{slide}

%%%%%%%%%%%%%%%%%%%%%%% Next Slide %%%%%%%%%%%%%%%%%%%%%%%

\begin{slide}
\section[-1.5]{ListIterator}

\begin{PauseHighLight}
  \begin{itemize}
  \item Lists allow bidirectional iterators (i.e. ones that go backward
    and forward)\pause
  \item The methods for this class are
    \begin{java}
      public void add(T element)
      public boolean hasNext()
      public boolean hasPrevious()
      public T next()
      public int nextIndex()
      public T previous()
      public int previousIndex()
      public void remove()
      public void set(T element)
    \end{java}\pause\vspace{-1cm}
  \item For all of these the \texttt{Linkedlist} iterator runs in constant
    time\pause---not true for the \texttt{ArrayList} iterator\pause
  \end{itemize}
\end{PauseHighLight}

\end{slide}

%%%%%%%%%%%%%%%%%%%%%%% Next Slide %%%%%%%%%%%%%%%%%%%%%%%

\begin{slide}
\section{Other Iterators}

\begin{PauseHighLight}
  \begin{itemize}
  \item All the standard collections \jl$ArrayList<T>$,
    \jl$LinkedList<T>$, \jl$TreeSet<T>$ and \jl$HashSet<T>$ have iterator
    classes\pause
  \item For \jl$TreeSet<T>$ and \jl$HashSet<T>$ these are less trivial
    than for the lists\pause
  \item Because all these classes follow the same pattern they are easy
    to understand and use\pause
  \end{itemize}
\end{PauseHighLight}
\end{slide}
%%%%%%%%%%%%%%%%%%%%%%% Next Slide %%%%%%%%%%%%%%%%%%%%%%%

\begin{slide}
\section[-2]{Maps}

\begin{PauseHighLight}
  \begin{itemize}
  \item Maps do not implement the collection interface and don't have
    iterators\pause
  \item They are implemented using sets
    \begin{java}
      public class TreeMap<K,V> implements Map<K,V> {
        private TreeSet<Map.Entry<K,V>> hiddenSet;$\pause$

        public put(K key, V value) {
          hiddenSet.add(new Map.Entry<K,V>(key, value));
        }$\pause$

        public TreeSet<Map.Entry<K,V>> setView() {
          return hiddenSet;
        }$\pause$
    \end{java}
  \item Iterate over a map by calling \jl+setView+\pause
  \end{itemize}
\end{PauseHighLight}


\end{slide}

%%%%%%%%%%%%%%%%%%%%%%% Next Slide %%%%%%%%%%%%%%%%%%%%%%%

\begin{slide}
\section[-1]{Lessons}

\begin{PauseHighLight}
  \begin{itemize}
  \item It is good software practice to follow established
    patterns\pause
  \item It requires skill and experience to spot common patterns\pause
  \item It requires discipline to follow the pattern\pause
  \item This is especially true when designing a group of classes,
    e.g. data structure classes\pause
    \begin{itemize}
    \item Using the classes is easier because the methods are
      unified\pause
    \item It is possible to save coding by using generic methods\pause
    \end{itemize}
  \item Iterators are one of the most commonly used patterns---learn to
    use them\pause
  \end{itemize}
\end{PauseHighLight}

\end{slide}
