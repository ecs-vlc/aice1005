%Master File:lectures.tex
\renewcommand{\class}[1]{\textsf{#1}}

\lesson{Settle For Good Solutions}

\vspace{-2cm}
\begin{center}
  \includegraphics[height=10cm]{\figs/GAroulette/GAroulette}
\end{center}
\keywords{neighbourhood search, heuristics, simulated annealing,
  GA}
%%%%%%%%%%%%%%%%%%%%%%% Next Slide %%%%%%%%%%%%%%%%%%%%%%%
\renewcommand{\Outline}{%
\begin{slide}
\section[1]{Outline}

\begin{minipage}{12cm}
  \begin{enumerate}\squeeze
    \outlineitem{Heuristic Search}{heuristic}
    \begin{itemize}
    \item \toplink{constructive}{Constructive algorithms}
    \item \toplink{neighbour}{Neighbourhood search}
    \end{itemize}
    \outlineitem{Simulated Annealing}{simannealtt}
    \outlineitem{Evolutionary Algorithms}{eastt}
   \end{enumerate}
\end{minipage}\hfill
\begin{minipage}{10cm}
  \includegraphics[width=10cm]{\figs/GAroulette/GAroulette}
\end{minipage}
\end{slide}
\addtocounter{outlineitem}{1}
}

\setcounter{outlineitem}{1}

%%%%%%%%%%%%%%%%%%%%%%% Next Slide %%%%%%%%%%%%%%%%%%%%%%%
\Outline %
\toptarget{firstoutline}
%%%%%%%%%%%%%%%%%%%%%%% Next Slide %%%%%%%%%%%%%%%%%%%%%%%

\begin{slide}
\section{Heuristic Algorithms}

\begin{PauseHighLight}
  \begin{itemize}
  \item Given that we know of no efficient algorithms for finding the
    optimal solution to NP-hard problems we must content ourselves with
    either\pause
    \begin{itemize}
    \item Spending a very long time (e.g. using branch and bound)\pause
    \item Accepting good solutions which aren't necessarily optimal\pause
    \end{itemize}
  \item Algorithms for finding good solutions are often called
  \emph{approximation algorithms} or \emph{heuristic algorithms}\pause
  \end{itemize}
\end{PauseHighLight}

\end{slide}

%%%%%%%%%%%%%%%%%%%%%%% Next Slide %%%%%%%%%%%%%%%%%%%%%%%

\begin{slide}
\section{Heuristics}

\begin{PauseHighLight}
  \begin{itemize}
  \item The idea behind heuristic algorithms is to use a rough guide or
    \emph{heuristic} pointing you in reasonable direction\pause
  \item If this heuristic is good you should find good solutions much
    faster than exhaustive search\pause
  \item Two commonly used heuristics are
    \begin{itemize}
    \item A greedy heuristic (take the best move)\pause
    \item Believe that good solutions are `close' to each other\pause
    \end{itemize}
  \end{itemize}
\end{PauseHighLight}

\end{slide}

%%%%%%%%%%%%%%%%%%%%%%% Next Slide %%%%%%%%%%%%%%%%%%%%%%%

\begin{slide}
\section[-1]{Constructive Algorithms}
\toptarget{constructive}

\pb
\begin{minipage}{0.67\linewidth}
  \begin{itemize}
  \item Constructive algorithms build-up a solution\pauseh
  \item They usually rely on a greedy heuristic\pauseh
  \item They are very fast\pauseh
  \item Once you have got a solution that's it\pauseh
  \item They can give reasonable solutions quickly, but they are not
    usually very good\pauseh
  \end{itemize}
\end{minipage}
\begin{minipage}{0.3\linewidth}
  \multipdf[width=0.9\linewidth]{greedy_tsp}\pause
\end{minipage}

\end{slide}

%%%%%%%%%%%%%%%%%%%%%%% Next Slide %%%%%%%%%%%%%%%%%%%%%%%

\begin{slide}
\section{Neighbourhood search}
\toptarget{neighbour}

\begin{PauseHighLight}
  \begin{itemize}
  \item An alternative to constructive algorithms are search algorithms
    relying on good solutions being close to each other\pause
  \item In neighbourhood search we\pause
    \begin{enumerate}
    \item Start from some solution\pause
    \item Examine the neighbouring solutions \label{examine}\pause
    \item Move to a neighbour if it is better or, at least, not worse\pause
    \item Repeat \ref{examine} until some stopping criteria\pause
    \end{enumerate}
  \item If we are maximising this is often called a
    \emph{hill-climber}\pause
  \item If we are minimising it is often called \emph{descent}\pause
  \end{itemize}
\end{PauseHighLight}

\end{slide}

%%%%%%%%%%%%%%%%%%%%%%% Next Slide %%%%%%%%%%%%%%%%%%%%%%%

\begin{slide}
\section{Iterative Improvement at its Best}

\begin{PauseHighLight}
  \begin{itemize}
  \item There are times when a neighbourhood algorithm will find the
    optimal solution\pause
  \item The classic example of this is in \emph{linear programming}
    where the simplex method leads to the optimal solution\pause
  \item Other examples include
    \begin{itemize}
    \item Maximum Flow
    \item Maximum Matching in Bipartite Graphs\pause
    \end{itemize}
  \item Unfortunately, this doesn't always work\pauseb
  \end{itemize}
\end{PauseHighLight}


\end{slide}


%%%%%%%%%%%%%%%%%%%%%%% Next Slide %%%%%%%%%%%%%%%%%%%%%%%

\begin{slide}
\section[-1]{Local Optima}

\begin{PauseHighLight}
  \begin{itemize}
  \item Neighbourhood search is usually much slower than a constructive
    algorithm but tends to find better quality solutions\pause
  \item However, it will often get stuck
  \begin{center}
    \includegraphics[height=8cm]{\figs/twopeaks}\pause
  \end{center}
  \end{itemize}
\end{PauseHighLight}

\end{slide}

%%%%%%%%%%%%%%%%%%%%%%% Next Slide %%%%%%%%%%%%%%%%%%%%%%%

\begin{slide}
\section{Simple Fixes}

\begin{PauseHighLight}
  \begin{itemize}
  \item One simple fix is to restart from many different starting
    positions\pause
  \item Or perturb the current solution and restart\pause
  \item These give improvements over doing nothing, but aren't
    necessarily great strategies\pause
  \item You can also increase the size of the neighbour to decrease the
    chance of getting stuck (e.g. in TSP swap more cites)\pause
  \end{itemize}
\end{PauseHighLight}

\end{slide}

%%%%%%%%%%%%%%%%%%%%%%% Next Slide %%%%%%%%%%%%%%%%%%%%%%%
\Outline % Simulated Annealing
%%%%%%%%%%%%%%%%%%%%%%% Next Slide %%%%%%%%%%%%%%%%%%%%%%%

\begin{slide}
\section{Simulated Annealing}

\begin{PauseHighLight}
  \begin{itemize}
  \item Simulated annealing is an example of a stochastic
    hill-climber\pause 
  \item Sometimes you go in the wrong direction (down-hill)\pause
  \item Historically it is an idea from physics\pause---where you tend
    to minimise energy\pause
  \item Idea is to obtain a (low energy) crystalline material you very
    slowly let the material cool from a liquid state (opposite of
    quenching)\pause
  \end{itemize}
\end{PauseHighLight}

\end{slide}

%%%%%%%%%%%%%%%%%%%%%%% Next Slide %%%%%%%%%%%%%%%%%%%%%%%

\begin{slide}
\section[-1]{Stochastic Descent}

\pb
\begin{itemize}
\item It is easier to fall down hill than to go back up\pause\pauselevel{=1}
  \begin{center}
    \includegraphics[width=\linewidth]{simulated_annealing-0}\mypl{1}
    \multido{\ia=1+1,\ib=2+1}{39}{%
      \llap{\includegraphics[width=\linewidth]{simulated_annealing-\ia}\mypl{\ib}}}
  \end{center}
\end{itemize}

\end{slide}

%%%%%%%%%%%%%%%%%%%%%%% Next Slide %%%%%%%%%%%%%%%%%%%%%%%

\begin{slide}
\section[-1]{Simulated Annealing}

\begin{PauseHighLight}
  \begin{itemize}
  \item Algorithm to minimise energy $E(\bm{X})$ where
    $\bm{X}=(X_1,X_2,\ldots,X_N)$\pause
  \item Starting from a random configuration $\bm{X}$\pause
  \item Choose a neighbour $\bm{X}'$\pause
  \item If the neighbour is better (lower energy) move to it\pause
  \item Otherwise move to the neighbour with some probability\pause
  \item The parameter $\beta$ controls the probability of moving to a
    neighbour\pause
  \item We increase $\beta$ to reduce the probability of going uphill
    over time\pause
  \end{itemize}
\end{PauseHighLight}

\end{slide}

\begin{slide}
\section{Cooling Schedule}

\begin{PauseHighLight}
  \begin{itemize}
  \item The parameter $\beta$ is known as the inverse temperature
    because of an analogy with physics\pause
  \item Over time we have to increase $\beta$ (decrease the temperature)
    so that the system will remain in the low energy state\pause
  \item The way you reduce the temperature (increase $\beta$) is known
    as the cooling schedule\pause
  \item Choosing a good cooling schedule can be critical\pause
  \item Choosing a good cooling schedule is something of a black
    art\pause
  \end{itemize}
\end{PauseHighLight}

\end{slide}

%%%%%%%%%%%%%%%%%%%%%%% Next Slide %%%%%%%%%%%%%%%%%%%%%%%

\begin{slide}
\section{Convergence Theorem}

\begin{PauseHighLight}
  \begin{itemize}
  \item There is a theorem that says if you choose a slow enough cooling
    schedule you will end up in the global minimum eventually\pause
  \item Unfortunately eventually is a very long time\pause
  \item It is quicker to search all possible states\pause
  \item Still people get very excited about convergence proofs\pause
  \end{itemize}
\end{PauseHighLight}

\end{slide}



%%%%%%%%%%%%%%%%%%%%%%% Next Slide %%%%%%%%%%%%%%%%%%%%%%%
\Outline % EAs
%%%%%%%%%%%%%%%%%%%%%%% Next Slide %%%%%%%%%%%%%%%%%%%%%%%

\begin{slide}
\section{Genetic Algorithms}

\begin{PauseHighLight}
  \begin{itemize}
  \item Genetic algorithms are methods to evolve a population of
    potential candidates to find a good solution to an optimisation
    problem\pause
  \item There are a whole set of related methods that go by the name of
    evolutionary algorithms, GAs are a subspecies of EAs\pause
  \item They can be viewed as an engineering approach to solving hard
    problems\pause
  \item I'm going to present my, highly prejudiced view of what's
    important in making a GA work\pause
  \end{itemize}
\end{PauseHighLight}

\end{slide}

%%%%%%%%%%%%%%%%%%%%%%% Next Slide %%%%%%%%%%%%%%%%%%%%%%%

\begin{slide}
\section{A Canonical GA}

\begin{PauseHighLight}
\begin{enumerate}
\item \textit{Initialise population}\pause
\item \textbf{for} $t=1$ \textbf{to} $T$\pause
  \begin{enumerate}
  \item \textit{Evaluate fitness}\pause
  \item \textit{Select} a new population based on fitness\pause
  \item \textit{Mutate} members of the population\pause
  \item \textit{Crossover} members of the population\pause
  \end{enumerate}
\item Return best member of the population\pause
\end{enumerate}
\end{PauseHighLight}

\end{slide}

%%%%%%%%%%%%%%%%%%%%%%% Next Slide %%%%%%%%%%%%%%%%%%%%%%%
\begin{slide}
\section[-2]{E.g. Graph Colouring}
\toptarget{graphcolouring}

\begin{PauseHighLight}
  \begin{center}
    \includegraphics[width=14cm]{ColourConflicts.0}\pause
    \llap{\includegraphics[width=14cm]{ColourConflicts.1}}\pauselevel{=3}\pauseb
  \end{center}
  \vspace*{-2cm}
  \begin{itemize}\pauselevel{=1}
  \item  Given a graph $G=(\mathcal{V},\mathcal{E})$\pause
  \item Assign colours, $c(v)$, to the vertices of the graph
    $v\in\mathcal{V}$\pause 
  \item Minimise the number of edges $e=(v,v')\in\mathcal{E}$ with the same
  coloured vertices $c(v)=c(v')$ (colour conflicts)\pause
  \end{itemize}
\end{PauseHighLight}

\end{slide}

%%%%%%%%%%%%%%%%%%%%%%% Next Slide %%%%%%%%%%%%%%%%%%%%%%%

\begin{slide}
\section[-1]{Initialise Population}

\pb
Generate random colourings\pause. E.g.
\color{TextColor}
\begin{center}
  \includegraphics[width=0.9\linewidth]{graph-colouring/GC_0}\pause
  \llap{\includegraphics[width=0.9\linewidth]{graph-colouring/GC_1}}\pause
\end{center}
\end{slide}


%%%%%%%%%%%%%%%%%%%%%%% Next Slide %%%%%%%%%%%%%%%%%%%%%%%

\begin{slide}
\section{Selection}

\begin{PauseHighLight}
  \begin{itemize}
  \item Select a new population of $P$ members preferentially choosing the
  fitter members\pause
\item Let $w_\alpha$ be a measure of the fitness of member $\alpha$\pause
\item E.g. choose members $\alpha$ with a probability
  \begin{displaymath}
    p_\alpha = \frac{w_\alpha}{\displaystyle \sum_{\alpha'=1}^P
      w_{\alpha'}}\pause
  \end{displaymath}
\item Many different ways of doing this\pause\ (some better than
  others)\pause 
  \end{itemize}
\end{PauseHighLight}

\end{slide}

%%%%%%%%%%%%%%%%%%%%%%% Next Slide %%%%%%%%%%%%%%%%%%%%%%%
\makeatletter
\renewcommand{\@mpm@format}[1]{#1.\the\@mpm@count}
\makeatother

\begin{slide}
\section[-2]{Mutation}

\pb
Change the colour of one or more of the vertices\pause. E.g.
\color{TextColor}
\begin{center}
  \multiinclude[graphics={width=0.9\linewidth}]{GraphMutation}\pause
\end{center}

\end{slide}


%%%%%%%%%%%%%%%%%%%%%%% Next Slide %%%%%%%%%%%%%%%%%%%%%%%

\begin{slide}
\pb
\section{Crossover}
Take two solutions and combine them to form a new solution\pause. E.g.
\color{TextColor}
\begin{center}
  \includegraphics[width=22cm]{\figs/graph-colouring/crossover}\pause
\end{center}

\end{slide}

%%%%%%%%%%%%%%%%%%%%%%% Next Slide %%%%%%%%%%%%%%%%%%%%%%%

\begin{slide}
\section[-1]{Crossover Operators}

\begin{PauseHighLight}
  \begin{itemize}
  \item Single-point crossover
    \begin{itemize}
    \item Take two strings and cut them at some random site
      {\scriptsize\[
        \hspace{-2em}\begin{array}{c}
          (GRBGBR\big|BGGBGBG)\\ (RRBRGB\big|RGRBBGB)
        \end{array} \bigg\}\, \longrightarrow \, \bigg\{
        \begin{array}{c}
          (GRBGBR\big|RGRBBGB)\\
          (RRBRGB\big|BGGBGBG)
        \end{array}
        \pause\]}
    \end{itemize}
  \item Multi-point crossover
    \begin{itemize}
    \item Take two strings and cut them at several sites
      {\scriptsize\[
        \hspace{-2em}\begin{array}{c}
          (GRBGBR\big|BGGB\big|GBG)\\ (RRBRGB\big|RGRB\big|BGB)
        \end{array} \bigg\}\, \longrightarrow \, \bigg\{
        \begin{array}{c}
          (GRBGBR\big|RGRB\big|GBG)\\
          (RRBRGB\big|BGGB\big|BGB)
        \end{array}
        \pause\]}
    \end{itemize}
  \item Uniform Crossover
    \begin{itemize}
    \item Take two strings and create children by a random shuffle
      {\scriptsize\[
        \hspace{-2em}\begin{array}{c}
          (GRBGBRBGGBGBG)\\
          (RRBRGBRGRBBGB)
        \end{array} \bigg\}\, \longrightarrow \, \bigg\{
        \begin{array}{c}
          (GRBRGBRGRGGGB)\\
          (RRBGBRBGGBBBG)
        \end{array}
        \pause\]}
    \end{itemize}
  \item Any of these crossover can be biased towards one parent\pause
  \end{itemize}
\end{PauseHighLight}

\end{slide}

%%%%%%%%%%%%%%%%%%%%%%% Next Slide %%%%%%%%%%%%%%%%%%%%%%%

\begin{slide}
\pb
\section[-2]{Cost of Crossover}
\color{TextColor}
\begin{center}
  \multiinclude[graphics={width=0.8\linewidth}]{graph-colouring/Crossover}
  \pause
\end{center}

\end{slide}

%%%%%%%%%%%%%%%%%%%%%%% Next Slide %%%%%%%%%%%%%%%%%%%%%%%

\begin{slide}
\pausebuild
\section[-1]{GA}

\begin{center}
  \begin{picture}(230,140)
    \put(0,0){\includegraphics[width=23cm]{graph-colouring/GC_0}}\pause
    \put(0,0){\includegraphics[width=23cm]{graph-colouring/GC_1}}\pause
    \put(0,0){\includegraphics[width=23cm]{graph-colouring/GC_2}}\pause
    \put(0,0){\includegraphics[width=23cm]{graph-colouring/GC_3}}\pause
    \put(0,0){\includegraphics[width=23cm]{graph-colouring/GC_4}}\pause
    \put(0,0){\includegraphics[width=23cm]{graph-colouring/GC_5}}\pause
    \put(0,0){\includegraphics[width=23cm]{graph-colouring/GC_6}}\pause
    \put(0,0){\includegraphics[width=23cm]{graph-colouring/GC_7}}\pause
    \put(0,0){\includegraphics[width=23cm]{graph-colouring/GC_8}}\pause
    \put(0,0){\includegraphics[width=23cm]{graph-colouring/GC_9}}\pause
    \put(0,0){\includegraphics[width=23cm]{graph-colouring/GC_10}}\pause
    \put(0,0){\includegraphics[width=23cm]{graph-colouring/GC_11}}\pause
    \put(0,0){\includegraphics[width=23cm]{graph-colouring/GC_12}}\pause
    \put(0,0){\includegraphics[width=23cm]{graph-colouring/GC_13}}\pause
    \put(0,0){\includegraphics[width=23cm]{graph-colouring/GC_14}}\pause
  \end{picture}    
\end{center}
\end{slide}


%%%%%%%%%%%%%%%%%%%%%%% Next Slide %%%%%%%%%%%%%%%%%%%%%%%

\begin{slide}
\section{Bit Simulated Crossover}

\begin{PauseHighLight}
  \begin{itemize}
  \item Choose each variable independently with the probability
    proportional to the frequency of the allele in the population\pause
    \begin{center}
      \begin{tabular}{ccc}
      $(\cdots, B, \cdots)$ & $(\cdots, R, \cdots)$ & $(\cdots, G, \cdots)$ \\
      $(\cdots, R, \cdots)$ & $(\cdots, B, \cdots)$ & $(\cdots, G, \cdots)$ \\
      $(\cdots, B, \cdots)$ & $(\cdots, G, \cdots)$ & $(\cdots, B, \cdots)$ \\
      $(\cdots, B, \cdots)$
      \end{tabular}\pause
    \end{center}
    \begin{displaymath}
      p_i(B) = 0.5, \hspace{1em} p_i(G) = 0.3, \hspace{1em} p_i(R) = 0.2\pause
    \end{displaymath}
  \item New algorithms built on this idea, ``Estimation of Distribution
    Algorithms'' EDAs\pause
  \end{itemize}
\end{PauseHighLight}

\end{slide}

%%%%%%%%%%%%%%%%%%%%%%% Next Slide %%%%%%%%%%%%%%%%%%%%%%%


\begin{slide}
\section[-1]{Galinier and Hao's Crossover Operator}

\begin{PauseHighLight}
  \begin{itemize}\squeeze
  \item Choose two parents\pause
  \item Sort nodes into colour-groups\pause
    \begin{center}
      \begin{tabular}{|c|c|c|} \hline
        & Parent 1 & Parent 2  \\ \hline
        B & \{1,3,4,8,\ldots\} & \{3,5,7,10, \ldots \} \\
        G & \{2,6,7,10,\ldots\} & \{1,11,12,13, \ldots \} \\
        R & \{5,9,11,12,\ldots\} & \{2,3,6,8, \ldots \} \\
        \vdots & \vdots & \vdots \\
        \hline
      \end{tabular}\pause
    \end{center}
  \item Choose largest colour-group in parent~1\pause
  \item Eliminate all nodes from that colour-group in parent~2\pause
  \item Choose largest colour-group in parent~2\pause
  \item etc.\pause
  \end{itemize}
\end{PauseHighLight}

\end{slide}
%%%%%%%%%%%%%%%%%%%%%%% Next Slide %%%%%%%%%%%%%%%%%%%%%%%

\begin{slide}
\section[-2]{Cost of Crossover}

\pb
\color{TextColor}
\begin{center}
  \multiinclude[graphics={width=0.8\linewidth}]{graph-colouring/GalinierHao}
  \pause
\end{center}

\end{slide}

%%%%%%%%%%%%%%%%%%%%%%% Next Slide %%%%%%%%%%%%%%%%%%%%%%%


\begin{slide}
\section[-2]{Other Heuristics}
\toptarget{saga}

\begin{PauseHighLight}
  \begin{itemize}
  \item There are many extensions to neighbourhood search, simulated
    annealing and genetic algorithms\pause
  \item There are other techniques such as Tabu search
    \begin{itemize}
    \item Construct a list of place you cannot go to (usually the last
      few configurations)\pause
    \item Make the best move you are allowed to make\pause
    \item Rather a large number of \textit{ad hoc} rules to make it
      work\pause
    \item Often very fast but runs out of steam\pause
    \end{itemize}
  \item Many other EAs including particle swarm optimisations (PSO), ant
    colony optimisation (ACO), evolutionary strategies, \ldots\pause
  \end{itemize}
\end{PauseHighLight}

\end{slide}


%%%%%%%%%%%%%%%%%%%%%%% Next Slide %%%%%%%%%%%%%%%%%%%%%%%

\begin{slide}
\section{Which Heuristic is Best?}

\begin{PauseHighLight}
  \begin{itemize}
  \item The best heuristic depends on the application\pause
  \item Descent is very fast, but only finds local optima---good
    starting place\pause
  \item Tabu search is often very fast, but sometimes fails to find
    really good solutions\pause
  \item Simulated annealing and Genetic Algorithms are slow, but often
    find good solutions\pause
  \item The best algorithms tend to be special purpose algorithms
    designed for the problem\pause
  \end{itemize}
\end{PauseHighLight}

\end{slide}

%%%%%%%%%%%%%%%%%%%%%%% Next Slide %%%%%%%%%%%%%%%%%%%%%%%

\begin{slide}
\section{Lessons}

\begin{PauseHighLight}
  \begin{itemize}
  \item For many problems the best strategy is to find a good solution,
    but not the best\pause
  \item Iterative search usually give good quality solutions\pause
  \item There are many variants of heuristic search\pause
  \item Heuristic search algorithms aren't fast (don't use these
    techniques in an interactive program if you want to keep
    customers)\pause
  \item For large combinatorial optimisation problems this is often the
    only choice\pause
  \end{itemize}
\end{PauseHighLight}


\end{slide}
