% Created 2024-10-08 Tue 14:41
% Intended LaTeX compiler: pdflatex
\documentclass[11pt]{article}
\usepackage[utf8]{inputenc}
\usepackage[T1]{fontenc}
\usepackage{graphicx}
\usepackage{longtable}
\usepackage{wrapfig}
\usepackage{rotating}
\usepackage[normalem]{ulem}
\usepackage{amsmath}
\usepackage{amssymb}
\usepackage{capt-of}
\usepackage{hyperref}
\usepackage[a4paper,margin=20mm]{geometry}
\usepackage{amsmath}
\usepackage{amsfonts}
\usepackage{bm}
\usepackage{minted}
\usemintedstyle{emacs}
\usepackage[T1]{fontenc}
\usepackage[scaled]{beraserif}
\usepackage[scaled]{berasans}
\usepackage[scaled]{beramono}
\newcommand{\tr}{\textsf{T}}
\newcommand{\grad}{\bm{\nabla}}
\newcommand{\av}[2][]{\mathbb{E}_{#1\!}\left[ #2 \right]}
\newcommand{\Prob}[2][]{\mathbb{P}_{#1\!}\left[ #2 \right]}
\newcommand{\logg}[1]{\log\!\left( #1 \right)}
\newcommand{\e}[1]{{\rm e}^{#1}}
\newcommand{\dd}{\mathrm{d}}
\DeclareMathAlphabet{\mat}{OT1}{cmss}{bx}{n}
\newcommand{\normal}[2]{\mathcal{N}\!\left(#1 \big| #2 \right)}
\newcounter{eqCounter}
\setcounter{eqCounter}{0}
\newcommand{\explanation}{\setcounter{eqCounter}{0}\renewcommand{\labelenumi}{(\arabic{enumi})}}
\newcommand{\eq}[1][=]{\stepcounter{eqCounter}\stackrel{\text{\tiny(\arabic{eqCounter})}}{#1}}
\newcommand{\argmax}{\mathop{\mathrm{argmax}}}
\author{Adam Prugel Bennett}
\date{\today}
\title{Notes on building an array class in C++}
\hypersetup{
 pdfauthor={Adam Prugel Bennett},
 pdftitle={Notes on building an array class in C++},
 pdfkeywords={},
 pdfsubject={},
 pdfcreator={Emacs 29.4 (Org mode 9.6.15)}, 
 pdflang={English}}
\begin{document}

\maketitle

\section{Files}
\label{sec:org409e323}
\begin{itemize}
\item main.cpp
\item array.h
\item array.cpp
\item Makefile
\begin{verbatim}
main: array.h main.cpp array.cpp
     g++ main.cpp array.cpp -o main
     ./main
\end{verbatim}
\end{itemize}

\section{Getting started}
\label{sec:org4d15ff9}
\begin{itemize}
\item Header
\begin{verbatim}
class Array {
public:
  Array(int n);
private:
  int* data;
  int my_size;
};
\end{verbatim}
\item array.cpp
\begin{verbatim}
#include "array.h"

Array::Array(int n) {
  data = new int[n];
  my_size = n;
}

int Array::size() {
  return my_size;
}
\end{verbatim}
\end{itemize}

\section{Getter}
\label{sec:orgaa9d0d1}
\begin{itemize}
\item get command
\end{itemize}
\begin{verbatim}
int& Array::get(int i) {
  return data[i];
}
\end{verbatim}
\begin{itemize}
\item see what we have done
\end{itemize}
\begin{verbatim}
#include <iostream>
#include "array.h"
using namespace std;

int main(int, char**) {
  Array a(3);
  a.get(0) = 22;
  a.get(1) = 111;
  a.get(2) = 33;

  cout << a.get(0) << ", " << a.get(1) << ", " << a.get(2) << endl;
  return 0;
}
\end{verbatim}
\begin{itemize}
\item operator overloading, replace get with []
\end{itemize}
\begin{verbatim}
int& Array::operator[](int i) {
  return data[i];
}
\end{verbatim}

\section{Useful default behaviour}
\label{sec:orgf28af97}
\subsection{Copy constructor}
\label{sec:orgbd0c264}
\begin{itemize}
\item default constructure
\end{itemize}
\begin{verbatim}
void print(Array& a, string name) {
  cout << name << ": " << a[0];
  for(int i=1; i<a.size(); ++i) {
    cout << ", "<< a[i];
  }
  cout << endl;
}

int main(int, char**) {

  Array a(3);

  a[0] = 0;
  a[1] = 11;
  a[2] = 222;

  print(a, "a");

  Array b(a);

  print(b, "b");

  return 0;
}
\end{verbatim}
-- So far so good but
\begin{verbatim}
int main(int, char**) {

  Array a(3);

  a[0] = 0;
  a[1] = 11;
  a[2] = 222;

  print(a, "a");

  Array b(a);

  print(b, "b");

  a[0] = 777;
  cout << "-----------------\n";
  print(a, "a");
  print(b, "b");

  return 0;
}
\end{verbatim}
-- Adding an explicit copy constructor
\begin{verbatim}
Array::Array(const Array& other) {
  data = new int[other.my_size];
  my_size = other.my_size;
  for(int i=0; i< my_size; ++i) {
    data[i] = other.data[i];
  }
}
\end{verbatim}

\subsection{Assignment constuctor}
\label{sec:org75773f9}
\begin{itemize}
\item Replace \texttt{Array a(b);} with \texttt{Array a = b;}
\item Add assignment constructor
\end{itemize}
\begin{verbatim}
Array& Array::operator=(const Array& rhs) {
  data = new int[rhs.my_size];
  my_size = rhs.my_size;
  for(int i=0; i< my_size; ++i) {
    data[i] = rhs.data[i];
  }
  return *this;
}    
\end{verbatim}


\section{Memory leeks and hanging points}
\label{sec:orga55e0bf}
\begin{itemize}
\item Looking at memory
\end{itemize}
\begin{verbatim}
#include <unistd.h>

int main(int, char**){

  for(int i=0; i<500000; ++i) {
    Array a(100000000);
    if (i%10000==0) {
      cout << i << endl;
      sleep(1);
    }
  }
  cout << "Finished\n";

  return 0;
}
\end{verbatim}
\begin{itemize}
\item \texttt{top -c -p \$(pgrep -d',' main)}
\item Add a destructor
\end{itemize}
\begin{verbatim}
Array::~Array() {
  delete data;
}
\end{verbatim}
\begin{itemize}
\item How does it work?
\begin{itemize}
\item Whenever you create an Array object and it goes out of scope the
destructor is called and frees the memory
\end{itemize}
\item Design pattern \textbf{Source allocation is initialisation}
\item Used throughout C++
\item If you do this properly you don't have to worry about memory leaks
\item Used for other resources (open-close files, database tokens, etc.)
\end{itemize}

\section{Const consistency}
\label{sec:org7eb4a2d}
\begin{itemize}
\item The compiler is your friend
\begin{itemize}
\item Compiler errors takes seconds or minutes to fix
\item Bugs in your code can take minutes or hours
\end{itemize}
\item Let's modify \texttt{print}
\end{itemize}
\begin{verbatim}
void print(Array& a, string name) {
  cout << name << ": " << a[0];
  for(int i=1; i<a.size(); ++i) {
    cout << ", "<< a[i];
  }
  cout << endl;
  a[0] = 999;
}
\end{verbatim}
\begin{itemize}
\item We could pass by value
\end{itemize}
\begin{verbatim}
void print(Array a, string name) {
  cout << name << ": " << a[0];
  for(int i=1; i<a.size(); ++i) {
    cout << ", "<< a[i];
  }
  cout << endl;
  a[0] = 999;
}
\end{verbatim}
\begin{itemize}
\item This is expensive
\item I have to copy the whole array, but I'm not changing it
\item Let we declare that the array is not to be modified
\end{itemize}
\begin{verbatim}
void print(const Array& a, string name) {
  cout << name << ": " << a[0];
  for(int i=1; i<a.size(); ++i) {
    cout << ", "<< a[i];
  }
  cout << endl;
  a[0] = 999;
}
\end{verbatim}
\begin{itemize}
\item Need to declare a new access operators
\end{itemize}
\begin{verbatim}
int Array::operator[](int i) const {
    return data[i];
}
\end{verbatim}
\begin{itemize}
\item or
\begin{verbatim}
const int& Array::operator[](int i) const {
    return data[i];
}
\end{verbatim}
\item For integers there is no advantage, but if I modify the array to be
an array of memory intensive objects then the latter is preferred.
\item Note that the final \texttt{const} declares that the member function does
not change the underlying data
\item Need to declare that \texttt{size} is a const function
\item It seems expensive but notice that you can't modify the array within \texttt{print}
\item When you get used to it there is a satisfying feeling of making your
classes const consistent
\item The compiler will usually tell you when you have violated const consistency
\end{itemize}

\subsection{unsigned}
\label{sec:org277a5de}
\begin{itemize}
\item While we are at refactoring our code lets make \texttt{my\_size} be \texttt{unsigned}
\item We can't have negative size arrays
\end{itemize}

\section{Generic programming}
\label{sec:org1701b89}
\begin{itemize}
\item \texttt{Array} is going to be useful, but what if we want to store double or floats
\item It's going to be annoying to write a data structure for every possible type
\item \textbf{Templates} to the rescue
\begin{verbatim}
#include <memory>

template<typename T>
class Array {
public:
  Array(unsigned n);
  Array(const Array<T>& other);
  ~Array();
  Array& operator=(const Array&);
  T& operator[](unsigned i);
  const T& operator[](unsigned i) const;
  unsigned size() const;
private:
  T* data;
  unsigned my_size;
};


template<typename T>
Array<T>::Array(unsigned n) {
    data = new T[n];
    my_size = n;
}

template<typename T>
Array<T>::Array(const Array<T>& other) {
    data = new T[other.my_size];
    my_size = other.my_size;
    for(unsigned i=0; i< my_size; ++i) {
      data[i] = other.data[i];
    }
}

template<typename T>
Array<T>::~Array() {
  delete data;
}

template<typename T>
Array<T>& Array<T>::operator=(const Array<T>& rhs) {
    data = new unsigned[rhs.my_size];
    my_size = rhs.my_size;
    for(unsigned i=0; i< my_size; ++i) {
      data[i] = rhs.data[i];
    }
    return *this;
}

template<typename T>
unsigned Array<T>::size() const {
  return my_size;
}

template<typename T>
T& Array<T>::operator[](unsigned i) {
    return data[i];
}

template<typename T>
const T& Array<T>::operator[](unsigned i) const {
    return data[i];
}
\end{verbatim}
\item We don't write template code in a \texttt{.cpp} file as it is not compiled
\item We need to change \texttt{main.cpp}
\end{itemize}
\begin{verbatim}
#include <iostream>
#include "array.h"

using namespace std;

template<typename T>
void print(const Array<T>& a, string name) {
  cout << name << ": " << a[0];
  for(int i=1; i<a.size(); ++i) {
    cout << ", "<< a[i];
  }
  cout << endl;
}

int main(int, char**) {

  Array<int> a(3);

  a[0] = 0;
  a[1] = 11;
  a[2] = 222;

  print(a, "a");

  Array<int> b = a;

  print(b, "b");

  a[0] = 777;
  cout << "-----------------\n";
  print(a, "a");
  print(b, "b");

  return 0;
}
\end{verbatim}
\begin{itemize}
\item When the compiler finds \texttt{Array<int>} it compiles the code with \texttt{T}
replaced by \texttt{int}
\item Templates take a bit of getting used to but for data-structures they ace
\end{itemize}

\section{Variable Length Arrays}
\label{sec:org1f320f9}
\begin{itemize}
\item Back to data-structures 101
\item How do we make a variable length array?
\item Firstly, why do we need a variable length array?
\begin{itemize}
\item Reading from a file
\end{itemize}
\end{itemize}
\begin{verbatim}
#include <fstream>
using namespace std;

int main() {

  ofstream file;
  file.open("some_numbers.txt");
  Array<int> a
  while (!file.eof()) {
    a.push_back(file.get());
  }

  cout << a.size() << ", " << a[0] << endl;

}
\end{verbatim}
\end{document}
