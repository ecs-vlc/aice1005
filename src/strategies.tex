%Master File:lectures.tex

\lesson{Know What's Important}

\vspace{-2cm}
\begin{center}
  \includegraphics[height=10cm]{strategise}
\end{center}
\keywords{Optimising code, strategies}
%%%%%%%%%%%%%%%%%%%%%%% Next Slide %%%%%%%%%%%%%%%%%%%%%%%
\renewcommand{\Outline}{%
\begin{slide}
\section[1]{Outline}

\begin{minipage}{12cm}
  \begin{enumerate}\squeeze
    \outlineitem{Time Complexity}{timecomp}
    \outlineitem{Strategies}{strategies}
    \begin{itemize}
    \item \toplink{bruteforce}{Brute Force Methods}
    \item \toplink{divideandconq}{Divide and Conquer}
    \item \toplink{greedy}{Greedy Algorithms}
    \item \toplink{dynamicprogramming}{Dynamic Programming}
    \item \toplink{linearprogramming}{Linear Programming}
    \item \toplink{backtracking}{Backtracking}
    \item \toplink{heuristics}{Heuristic Search}
    \end{itemize}
  \end{enumerate}
\end{minipage}\hfill
\begin{minipage}{10cm}
  \includegraphics[width=10cm]{strategise}
\end{minipage}
\end{slide}
\addtocounter{outlineitem}{1}
}

\setcounter{outlineitem}{1}
\Outline
\toptarget{firstoutline}
%%%%%%%%%%%%%%%%%%%%%%% Next Slide %%%%%%%%%%%%%%%%%%%%%%%

\begin{slide}
\section{Designing Algorithms from Scratch}

\begin{PauseHighLight}
  \begin{itemize}
  \item In writing code you will come across new problems\pause---at
    least if your lucky\pauseb
  \item Your first task is to see if you can map this onto a problem
    with a well known solution\pause
  \item If not you are going to have to come up with a solution\pause
  \item Two questions you need to ask yourself are
    \begin{itemize}
    \item How efficient does my solution need to be?\pause
    \item How do I go about solving the problem?\pause
    \end{itemize}
  \end{itemize}
\end{PauseHighLight}

\end{slide}

%%%%%%%%%%%%%%%%%%%%%%% Next Slide %%%%%%%%%%%%%%%%%%%%%%%

\begin{slide}
\section{Is Solving the Problem Time Critical?}

\begin{PauseHighLight}
  \begin{itemize}
  \item The majority of code is not time critical\pause
  \item Even when programs are slow there is usually only one part of the
    code which takes almost all the time\pause
  \item However there are times when solving the problem na\"{\i}vely is going to
    take too long\pause
  \item Advice on improving the performance of your code is not hard to
    come by\ldots\pause
  \end{itemize}
\end{PauseHighLight}

\end{slide}


%%%%%%%%%%%%%%%%%%%%%%% Next Slide %%%%%%%%%%%%%%%%%%%%%%%

\begin{slide}
\section[-2]{Advice on Optimising Code}

\begin{PauseHighLight}
  \begin{itemize} \squeeze
  \item ``More computing sins are committed in the name of efficiency
    (without necessarily achieving it) than any other single
    reason---including blind stupidity'' \hfill W. Wulf\pause
  \item ``We \textit{should} forget about small efficiencies, say about
    97\% of the time: premature optimization is the root of all evil.''
    \hfill D. Knuth\pause
  \item ``We follow two rules in the matter of optimization:
    \begin{itemize}
    \item Rule 1. Don't do it
    \item Rule 2 (for experts only).  Don't do it yet---that is, not
      until you have a perfectly clear and unoptimized solution''
    \end{itemize}
    \ \hfill M. A. Jackson\pause
  \item ``Strive to write good programs rather than fast ones'' \hfill
    J. Block\pause
  \end{itemize}
\end{PauseHighLight}

\end{slide}

%%%%%%%%%%%%%%%%%%%%%%% Next Slide %%%%%%%%%%%%%%%%%%%%%%%

\begin{slide}
\section{Time Complexity}

\begin{PauseHighLight}
  \begin{itemize}
  \item Ignoring this good advice, your next job is to
    decide what is the time complexity of your algorithm and whether you
    can tolerate that time complexity?\pause
  \item Typically your algorithm will be
    \begin{itemize}
    \item constant or (log)-linear time {\Large \Smiley}
    \item quadratic/cubic,\ldots
    \item exponential time {\Large \Frowny}\pause
    \end{itemize}
  \item Lets deal with this in inverse order\pause
  \end{itemize}
\end{PauseHighLight}

\end{slide}

%%%%%%%%%%%%%%%%%%%%%%% Next Slide %%%%%%%%%%%%%%%%%%%%%%%

\begin{slide}
\section{Exponential Time}

\begin{PauseHighLight}
  \begin{itemize}
  \item You can often tolerate this if your problem is small\pause
  \item E.g. you're trying to solve a small puzzle\pause
  \item or planning a short sequence of actions\pause
  \item However, in many cases exponential is just too long
    \begin{itemize}
    \item Look for an efficient solution (e.g. dynamics programming)\pause
    \item Redefine the problem (e.g. use a linear approximation)\pause
    \item Settle for a sub-optimal solution  (e.g. using heuristic
      search)\pause
    \end{itemize}
  \end{itemize}
\end{PauseHighLight}

\end{slide}

%%%%%%%%%%%%%%%%%%%%%%% Next Slide %%%%%%%%%%%%%%%%%%%%%%%

\begin{slide}
\section{Quadratic and Cubic Time}

\begin{PauseHighLight}
  \begin{itemize}
  \item There are algorithms which are cubic (e.g. LP, inverting a
    matrix)\pause
  \item For many applications this is acceptable because there is no
    hurry for the solution or the problem isn't that big\pause
  \item For large data sets these algorithms might just be
    impractical\pause
  \item Often taking advantage of the structure of the problem
    (e.g. sparsity) can speed things up (e.g. a good LP package)\pause
  \item Sometimes quadratic algorithms can be made log-linear using a
    divide and conquer or a greedy strategy\pause
  \end{itemize}
\end{PauseHighLight}

\end{slide}

%%%%%%%%%%%%%%%%%%%%%%% Next Slide %%%%%%%%%%%%%%%%%%%%%%%

\begin{slide}
\section{Sub-Quadratic Algorithm}

\begin{PauseHighLight}
  \begin{itemize}
  \item If you have to run an algorithm on any data of arbitrary size
    you want a sub-quadratic algorithm\pause
  \item We have seen this in practice with sorting\pause
  \item The fast Fourier transform revolutionised digital signal
    processing when it was introduced (reducing a quadratic algorithm to
    a log-linear algorithm)\pause
  \item An active area of research is \emph{big data} where the only
    algorithms that can be used are sub-quadratic algorithms\pause
  \end{itemize}
\end{PauseHighLight}

\end{slide}

%%%%%%%%%%%%%%%%%%%%%%% Next Slide %%%%%%%%%%%%%%%%%%%%%%%

\begin{slide}
\section[-1]{Good Code is Fast Code}

\begin{PauseHighLight}
  \begin{itemize}
  \item Using appropriate data structures and algorithms is by far the
    best way of speeding up code\pause
  \item Many coders wedded to arrays try to simulate sets and maps very
    inefficiently often using code that does not scale\pause
  \item Changing the structure of a program can lead to huge speed
    ups\pause
  \item If you require more speed then concentrate on the inner loop
    where almost all the work is done (usually gives less than a factor
    of two)---optimising code in outer loops is pointless\pause
  \item Using the right strategy will give the biggest speed-up\pause
  \end{itemize}
\end{PauseHighLight}

\end{slide}



%%%%%%%%%%%%%%%%%%%%%%% Next Slide %%%%%%%%%%%%%%%%%%%%%%%
\Outline % Strategies
%%%%%%%%%%%%%%%%%%%%%%% Next Slide %%%%%%%%%%%%%%%%%%%%%%%

\begin{slide}
\section{Algorithmic Strategies}

\begin{PauseHighLight}
  \begin{itemize}
  \item Good algorithms are difficult to invent\pause
  \item However, many algorithms follow particular patterns or
    strategies\pause
  \item Understanding these strategies is important for deriving new
    algorithms\pause
  \item We have seen the classic strategies throughout the course\pause
  \end{itemize}
\end{PauseHighLight}

\end{slide}

%%%%%%%%%%%%%%%%%%%%%%% Next Slide %%%%%%%%%%%%%%%%%%%%%%%

\begin{slide}
\section{Brute Force}
\toptarget{bruteforce}

\begin{PauseHighLight}
  \begin{itemize}
  \item Many problems have obvious \emph{brute force} solutions\pause
  \item E.g. in sorting; selection sort, insertion sort and bubble sort
    are fairly simple algorithm that do the obvious\pause
  \item Similarly searching an array using sequential search provides an
    obvious solution\pause
  \item Sometimes, brute force methods are the best you can
    do\pause---e.g. sequential search on an unordered array\pause
  \end{itemize}
\end{PauseHighLight}

\end{slide}

%%%%%%%%%%%%%%%%%%%%%%% Next Slide %%%%%%%%%%%%%%%%%%%%%%%

\begin{slide}
\section{Exhaustive Search}

\begin{PauseHighLight}
  \begin{itemize}
  \item For optimisation problems such as the travelling salesperson
    problem, the brute force method is to try all possible
    solutions\pause
  \item This \emph{exhaustive search} starts to hurt very quickly and
    becomes intractable for moderate size problems (e.g. tours of length
    20)\pause
  \item Even for sorting, brute force methods become unattractive when
    the inputs are long\pause
  \item We really want to do better\pause
  \end{itemize}
\end{PauseHighLight}

\end{slide}

%%%%%%%%%%%%%%%%%%%%%%% Next Slide %%%%%%%%%%%%%%%%%%%%%%%

\begin{slide}
\section[-1]{Divide and Conquer}
\toptarget{divideandconq}
\begin{PauseHighLight}
  \begin{itemize}
  \item Another strategy we have met is \emph{divide and conquer}\pause
    \begin{itemize}
    \item Divide the problem into two or more parts
    \item Solve the parts
    \item Combine the parts to obtain a full solution\pause
    \end{itemize}
  \item This needs to be quicker than solving the original problem by
    brute force\pause
  \item We can do this division recursively until the problems are
    trivial to solve\pause
  \end{itemize}
\end{PauseHighLight}

\end{slide}

%%%%%%%%%%%%%%%%%%%%%%% Next Slide %%%%%%%%%%%%%%%%%%%%%%%

\begin{slide}
\section[-1]{Divide and Conquer Problems}

\begin{PauseHighLight}
  \begin{itemize}
  \item Algorithms based on this idea include
    \begin{itemize}
    \item Computing the integer power of a number
    \item Binary search
    \item Merge sort
    \item Quick sort
    \item Fast Fourier transform\pause
    \end{itemize}
  \item Often implemented using recursion\pause
  \item It's nice when it works---but not all problems allow this\pause
  \end{itemize}
\end{PauseHighLight}

\end{slide}

%%%%%%%%%%%%%%%%%%%%%%% Next Slide %%%%%%%%%%%%%%%%%%%%%%%

\begin{slide}
\section[-2]{Fast Fourier Transform}
\toptarget{fft}

\begin{PauseHighLight}
  \begin{itemize}
  \item The Fourier Transform provides a different ``view'' of a
    sequence such as a signals, images, etc.
    \begin{align*}
      \tilde{f}(\bm{k}) = \sum_{x_1=0}^{n-1}\cdots \sum_{x_d=0}^{n-1} f(\bm{x})\,\e{2\pi\mathrm{i}
        \bm{k}\cdot\bm{x}/n}\pause
    \end{align*}
  \item The \emph{Fast Fourier Transform} was an algorithm divised by
    John Tukey and James Cooley in 1965 to compute the Fourier
    Transform quickly\pause
  \item Gauss had used exactly this idea in a paper in 1805 to save
    himself work computing a Fourier transform by hand\pause
  \item It is based on a divide-and-conquer strategy and takes
    $O(n\log(n))$ operation compared to the $O(n^2)$ brute force
    method\pause
  \end{itemize}
\end{PauseHighLight}

\end{slide}


%%%%%%%%%%%%%%%%%%%%%%% Next Slide %%%%%%%%%%%%%%%%%%%%%%%

\begin{slide}
\section{Applications of FFT}

\begin{PauseHighLight}
  \begin{itemize}
  \item The application of FFT are enormous\pause
  \item It lies at the heart of digital signal processing\pause
  \item It is frequently used in image analysis\pause
  \item It is a type of wavelet transform used in JPEG\pause
  \item It is even used in fast multiplication of very large
    integers---with important applications in cryptography\pause
  \end{itemize}
\end{PauseHighLight}

\end{slide}

%%%%%%%%%%%%%%%%%%%%%%% Next Slide %%%%%%%%%%%%%%%%%%%%%%%

\begin{slide}
\section[-1]{Greedy Algorithms}
\toptarget{greedy}

\begin{PauseHighLight}
  \begin{itemize}
  \item The \emph{greedy strategy} is to build a solution to a problem by
    choosing the best available option\pause
  \item If you are lucky this will give an optimal solution\pause
  \item Example of optimal algorithms based on the greedy strategy
    include
    \begin{itemize}\squeeze
    \item Constructing Huffman trees
    \item Prim's algorithms
    \item Kruskal's algorithm
    \item Dijkstra's algorithm\pause
    \end{itemize}
  \item Often uses priority queues\pause
  \end{itemize}
\end{PauseHighLight}

\end{slide}

%%%%%%%%%%%%%%%%%%%%%%% Next Slide %%%%%%%%%%%%%%%%%%%%%%%

\begin{slide}
\section[-1]{Non-optimal Greedy Algorithms}

\begin{PauseHighLight}
  \begin{itemize}
  \item Greedy algorithms can also be used to solve optimisation
    problems such as the travelling salesperson problem\pause
  \item In the TSP we can start at some city and move to the nearest
    as-yet-unvisited city\pause
  \item The algorithm is guaranteed to find a solution no longer than
    $0.5(\lfloor \log_2(n)\rfloor +1)$ times the optimal tour length\pause
  \item It usually does substantially better, but it is very unlikely to
    find the optimal for very long tours\pause
  \item It is more the exception rather than the rule that Greedy
    algorithms find an optimal solutions\pause
  \end{itemize}
\end{PauseHighLight}

\end{slide}

%%%%%%%%%%%%%%%%%%%%%%% Next Slide %%%%%%%%%%%%%%%%%%%%%%%

\begin{slide}
\section{Dynamic Programming}
\toptarget{dynamicprogramming}

\begin{PauseHighLight}
  \begin{itemize}
  \item Dynamic programming can be used for solving many problems\pause
  \item It requires some (partial) ordering so that you can assign costs
    to partial solutions from previous solutions\pause
  \item Requires imagination to think how to do this\pause
  \item Used in inexact matching, shortest paths, line breaks, etc.\pause
  \end{itemize}
\end{PauseHighLight}

\end{slide}

%%%%%%%%%%%%%%%%%%%%%%% Next Slide %%%%%%%%%%%%%%%%%%%%%%%

\begin{slide}
\section[-2]{Uses of Dynamic Programming}

\begin{PauseHighLight}
  \begin{itemize}\small\squeeze
  \item Recurrent solutions to lattice models for protein-DNA binding
  \item Backward induction as a solution method for finite-horizon
    discrete-time dynamic optimization problems 
  \item Method of undetermined coefficients can be used to solve the
    Bellman equation in infinite-horizon, discrete-time, discounted,
    time-invariant dynamic optimization problems 
  \item Many string algorithms including longest common subsequence,
    longest increasing subsequence, longest common substring,
    Levenshtein distance (edit distance) 
  \item Many algorithmic problems on graphs can be solved efficiently
    for graphs of bounded treewidth or bounded clique-width by using
    dynamic programming on a tree decomposition of the graph. 
  \item The Cocke–Younger–Kasami (CYK) algorithm which determines
    whether and how a given string can be generated by a given
    context-free grammar 
  \item Knuth's word wrapping algorithm that minimizes raggedness when
    word wrapping text 
  \item The use of transposition tables and refutation tables in
    computer chess 
  \item The Viterbi algorithm (used for hidden Markov models)
  \item The Earley algorithm (a type of chart parser)
  \item The Needleman–Wunsch and other algorithms used in
    bioinformatics, including sequence alignment, structural alignment,
    RNA structure prediction 
  \item Floyd's all-pairs shortest path algorithm
  \item Optimizing the order for chain matrix multiplication
  \item Pseudo-polynomial time algorithms for the subset sum and
    knapsack and partition problems 
  \item The dynamic time warping algorithm for computing the global
    distance between two time series 
  \item The Selinger (a.k.a. System R) algorithm for relational database
    query optimization 
  \item De Boor algorithm for evaluating B-spline curves
  \item Duckworth–Lewis method for resolving the problem when games of
    cricket are interrupted 
  \item The value iteration method for solving Markov decision processes
  \item Some graphic image edge following selection methods such as the
    "magnet" selection tool in Photoshop 
  \item Some methods for solving interval scheduling problems
  \item Some methods for solving word wrap problems
  \item Some methods for solving the travelling salesman problem, either
    exactly (in exponential time) or approximately (e.g. via the bitonic
    tour) 
  \item Recursive least squares method
  \item Beat tracking in music information retrieval
  \item Adaptive-critic training strategy for artificial neural networks
  \item Stereo algorithms for solving the correspondence problem used in
    stereo vision 
  \item Seam carving (content aware image resizing)
  \item The Bellman–Ford algorithm for finding the shortest distance in
    a graph 
  \item Some approximate solution methods for the linear search problem
  \item Kadane's algorithm for the maximum subarray problem
  \end{itemize}
\end{PauseHighLight}

\end{slide}

%%%%%%%%%%%%%%%%%%%%%%% Next Slide %%%%%%%%%%%%%%%%%%%%%%%

\begin{slide}
\section{Linear Programming}
\toptarget{linearprogramming}

\begin{PauseHighLight}
  \begin{itemize}
  \item Look out for problems with linear objectives or where a
    linearisation approximation is acceptable\pause
  \item These can often be turned in linear programs which can be solved
    efficiently\pause
  \item The constraints have to be linear and the variables take on
    continuous values\pause
  \item Sometimes by being careful we can force integer solutions (see
    linear assignment in last lecture)\pause
  \item Applications in planning, but also in many areas of
    optimisation\pause
  \end{itemize}
\end{PauseHighLight}

\end{slide}

%%%%%%%%%%%%%%%%%%%%%%% Next Slide %%%%%%%%%%%%%%%%%%%%%%%

\begin{slide}
\section[-2]{Solving Linear Programming}

\pb\pause \pauselevel{=1}
\begin{minipage}{0.5\linewidth}
\begin{center}
  \multipdf[width=\linewidth]{normalForm}\pause
\end{center}  
\end{minipage}\hfil
\begin{minipage}{0.45\linewidth}
  \begin{itemize}
  \item The basic feasible points for LP problems with $n$ variables and
    $m$ constraints have at least $n-m$ zero variables\pause
  \item Typical number of basic feasible solutions is
    $\binom{n}{m}\geq\left(\frac{n}{m}\right)^m$\pause
  \item Simplex algorithm organises iterative search for global solutions\pause
  \end{itemize}
\end{minipage}

\end{slide}


%%%%%%%%%%%%%%%%%%%%%%% Next Slide %%%%%%%%%%%%%%%%%%%%%%%

\begin{slide}
\section{Backtracking}
\toptarget{backtracking}

\begin{PauseHighLight}
  \begin{itemize}
  \item Backtracking is in many ways a brute force method\pause
  \item It is just a way of exploring a search space\pause
  \item However, when we solving problems with constraints then we can
    vastly reduce the number of solutions that we visit\pause
  \item Used in many game playing, planning, verification, puzzle
    solving situations\pause
  \item For optimisation we can use branch and bound which is
    backtracking using the best solution as a constraint\pause
  \end{itemize}
\end{PauseHighLight}

\end{slide}

%%%%%%%%%%%%%%%%%%%%%%% Next Slide %%%%%%%%%%%%%%%%%%%%%%%

\begin{slide}
\section[-2]{Computer Chess}

\begin{PauseHighLight}
  \begin{itemize}\squeeze
  \item Computer chess algorithms explore the search tree of possible
    moves using backtracking with pruning\pause
  \item Although they cannot look at all possible moves, they can look
    deep enough to play good chess\pause
  \item In 1968 International Master David Levy bet that he would not be
    beaten by a computer in the next decade, a bet he won\pause
  \item He was beaten in 1989 by Deep Thought\pause
  \item In 1997 Deep Blue beat Gary Kasparov the reigning world
    champion\pause
  \item Modern chess engines such as Deep Rybka and Houdini have a chess
    rating of around 3200 (c.f. Magnus Carsen's 2872---the top human
    player)\pause
  \end{itemize}
\end{PauseHighLight}

\end{slide}

%%%%%%%%%%%%%%%%%%%%%%% Next Slide %%%%%%%%%%%%%%%%%%%%%%%

\begin{slide}
\section{Heuristic Search}
\toptarget{heuristics}

\begin{PauseHighLight}
  \begin{itemize}
  \item When all else fails we have to settle for finding a good
    solution, not the best\pause
  \item In fact, because so many problems that we are interested in turn
    out to be NP-hard this is quite common\pause
  \item There are many strategies
    \begin{itemize}\squeeze
    \item Neighbourhood search (hill-climbing, descent)
    \item Simulated annealing
    \item Evolutionary algorithms, etc.\pause
    \end{itemize}
  \end{itemize}
\end{PauseHighLight}

\end{slide}

%%%%%%%%%%%%%%%%%%%%%%% Next Slide %%%%%%%%%%%%%%%%%%%%%%%

\begin{slide}
\section{Lessons}

\begin{PauseHighLight}
  \begin{itemize}
  \item Many applications bring up interesting programming challenges\pause
  \item Some are dealt with by using sensible data structures and
    common algorithms\pause
  \item However, often you face a new challenge requiring thought\pause
  \item Thinking in terms of strategies and having a feel for time
    complexity is part of armoury of a professional programmer\pause
  \end{itemize}
\end{PauseHighLight}

\end{slide}
