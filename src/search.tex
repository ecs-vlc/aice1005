%Master File:lectures.tex

\lesson{Know how to Search}

\vspace{-2cm}
\begin{center}
  \includegraphics[height=10cm]{4queens-18}
\end{center}
\keywords{Backtracking,
  Branch and Bound}

%%%%%%%%%%%%%%%%%%%%%%% Next Slide %%%%%%%%%%%%%%%%%%%%%%%
\renewcommand{\Outline}{%
\begin{slide}
\section[1]{Outline}

\begin{minipage}{12cm}
  \begin{enumerate}\squeeze
    \outlineitem{Search Trees}{searchTrees}
    \outlineitem{Backtracking}{backtrack}
    \outlineitem{Branch and Bound}{branchandbound}
    \outlineitem{Search in AI}{aistuff}
  \end{enumerate}
\end{minipage}\hfill
\begin{minipage}{10cm}
  \includegraphics[width=10cm]{4queens-18}
\end{minipage}
\end{slide}
\addtocounter{outlineitem}{1}
}

\setcounter{outlineitem}{1}
%%%%%%%%%%%%%%%%%%%%%%% Next Slide %%%%%%%%%%%%%%%%%%%%%%%
\Outline % Search Trees
\toptarget{firstoutline}
%%%%%%%%%%%%%%%%%%%%%%% Next Slide %%%%%%%%%%%%%%%%%%%%%%%

\begin{slide}
\section{State Space Representation}
\begin{PauseHighLight}
  \begin{itemize}
  \item Many real world problems involve taking a series of actions to
    manipulate the state of the system\pause
  \item This is the area of planning and search which sits within the
    domain of artificial intelligence\pause
  \item One of the key props to help us develop algorithms is to think
    of the states as nodes of a graph which are linked if there exists
    an action taking us from one state to another\pause
  \item This provides a \emph{state space representation} of the problem
    (we saw this before when we derived a low bound on sorting)\pause
  \end{itemize}
\end{PauseHighLight}

\end{slide}

%%%%%%%%%%%%%%%%%%%%%%% Next Slide %%%%%%%%%%%%%%%%%%%%%%%

\begin{slide}
\section{8-Puzzle Example}

\begin{center}
  \includegraphics[width=\linewidth]{8puzzle}
\end{center}
\end{slide}

%%%%%%%%%%%%%%%%%%%%%%% Next Slide %%%%%%%%%%%%%%%%%%%%%%%

\begin{slide}
\section{Large State Spaces}

\begin{PauseHighLight}
  \begin{itemize}
  \item The search space typically increase exponentially with the
    problem size\pause
  \item We can find the quickest solution to the 8-puzzle (and the 15
    puzzle) using breadth first search, but larger puzzles soon become
    intractable\pause
  \item Nevertheless, a lot of important problems involve very large
    state spaces and we have to find algorithms to explore them\pause
  \end{itemize}
\end{PauseHighLight}

\end{slide}



%%%%%%%%%%%%%%%%%%%%%%% Next Slide %%%%%%%%%%%%%%%%%%%%%%%
\Outline % Backtracking
%%%%%%%%%%%%%%%%%%%%%%% Next Slide %%%%%%%%%%%%%%%%%%%%%%%

\begin{slide}
\section{Backtracking}

\begin{PauseHighLight}
  \begin{itemize}
  \item Backtracking is used to find feasible solutions in large state
    spaces\pause
  \item E.g. solving sudoku\pause
  \item It works by growing partial solutions until either
    \begin{itemize}
    \item a feasible solution is found when we can finish
    \item no feasible solution is found when we backtrack\pause
    \end{itemize}
  \item We often search the state space using depth first search\pause
  \end{itemize}
\end{PauseHighLight}

\end{slide}

%%%%%%%%%%%%%%%%%%%%%%% Next Slide %%%%%%%%%%%%%%%%%%%%%%%

\begin{slide}
\section[-2]{4-Queens Problem}
\pb
\pause\pauselevel{=1}
\begin{center}
  \multipdf[width=0.7\linewidth]{4queens}\pause
\end{center}
\end{slide}

%%%%%%%%%%%%%%%%%%%%%%% Next Slide %%%%%%%%%%%%%%%%%%%%%%%

\begin{slide}
\section[-2]{6-Queens Problem}
\pb
\pause
\begin{center}
  \multipdf[width=0.5\linewidth]{nQueens}\pause
\end{center}
\end{slide}


%%%%%%%%%%%%%%%%%%%%%%% Next Slide %%%%%%%%%%%%%%%%%%%%%%%

\begin{slide}
\section{Implementing $n$-Queens}

\begin{PauseHighLight}
  \begin{itemize}
  \item Implementing backtracking is easily done using recursion\pause
  \item Recall depth-first search is easily implemented using
    recursion\pause
  \item We just need a recursive function \texttt{next(n, row, sol)}
    which for a $n$-Queens problem searches new solutions in
    \texttt{row} given queens in previous rows given in
    \texttt{sol}\pause
  \item Run: \texttt{List sol = nextRow(6, 0, new List());}\pause
  \end{itemize}
\end{PauseHighLight}

\end{slide}

%%%%%%%%%%%%%%%%%%%%%%% Next Slide %%%%%%%%%%%%%%%%%%%%%%%

\begin{slide}
\section[-1]{Code}

\begin{java}
List nextRow(int noRows, int row, List queenPositions) {
  if (row==noRows) {return queenPositions;}$\pause$
  for (int col=0; col<noRows; ++col) {
    if (legalQueen(col, row, queenPositions)) {
      queenPositions.add(col);
      List solution = nextRow(noRows, row+1, queenPositions);
      if (solution!=null)
	return solution;
    }$\pause$
  }
  return null;
}$\pause$

bool LegalQueen(int col, int row, List sol) {
  for(int r=0; r<row: ++r) {
    rf (sol[r] == col || sol[r]-row+r == col || sol[r]+row-r==col) {
      return false;
    }
  }
  return true;
}$\pause$
\end{java}

\end{slide}

%%%%%%%%%%%%%%%%%%%%%%% Next Slide %%%%%%%%%%%%%%%%%%%%%%%

\begin{slide}
\section[-2]{Hamiltonian Circuit}

\begin{PauseHighLight}
  \begin{itemize}
  \item A Hamiltonian cycle is a tour through a graph which visits every
    vertex once only and returns to the start\pause
  \item It is a hard problem in that there are no known algorithms
    that are guaranteed to find a Hamiltonian cycle in polynomial time\pause
  \item For many graphs it is not too hard
  \end{itemize}
  \begin{center}
    \includegraphics[width=0.2\linewidth]{hamiltonCircuit}\pause
  \end{center}
\end{PauseHighLight}

\end{slide}

%%%%%%%%%%%%%%%%%%%%%%% Next Slide %%%%%%%%%%%%%%%%%%%%%%%

\begin{slide}
\section{Hamiltonian Circuit Example}
\pb\pause\pauselevel{=1}
\begin{center}
  \multipdf[width=0.5\linewidth]{hamiltonianCircuit}\pause
\end{center}
\end{slide}

%%%%%%%%%%%%%%%%%%%%%%% Next Slide %%%%%%%%%%%%%%%%%%%%%%%

\begin{slide}
\section{Backtracking}

\begin{PauseHighLight}
  \begin{itemize}
  \item Backtracking is a standard algorithm for solving constraint
    problems with large search spaces\pause
  \item It can take exponential amount of time, however with many
    constraints it will often find solutions relatively quickly\pause
  \item A backtracking algorithm does not solve, for example, sudoku in the
    same way as a human\pause---it uses speed rather than brains\pauseb
  \item We can often speed up backtracking by adding more constraints
    (although, this can make writing the program longer)\pause
  \end{itemize}
\end{PauseHighLight}

\end{slide}


%%%%%%%%%%%%%%%%%%%%%%% Next Slide %%%%%%%%%%%%%%%%%%%%%%%
\Outline % Branch and Bound
%%%%%%%%%%%%%%%%%%%%%%% Next Slide %%%%%%%%%%%%%%%%%%%%%%%

\begin{slide}
\section{Optimisation Problems}

\begin{PauseHighLight}
  \begin{itemize}
  \item In many optimisation problems (TSP, Graph-colouring, etc.) we
    again have a huge search space ($n!$, $k^n$)\pause
  \item However, we don't have hard constraints\pause
  \item If we are interested in finding the optimal then we can use the
    cost as a constraint
    \begin{quote}
      any partial solution has to have a lower cost than the best
      solution we have found so far\pause
    \end{quote}
  \item This allows us to develop a backtracking strategy known as
    branch and bound\pause
  \end{itemize}
\end{PauseHighLight}

\end{slide}


%%%%%%%%%%%%%%%%%%%%%%% Next Slide %%%%%%%%%%%%%%%%%%%%%%%

\begin{slide}
\section{Branch and Bound}

\begin{PauseHighLight}
  \begin{itemize}
  \item Branch and bound is used on optimisation problems where
    efficient strategies just don't work\pause
  \item It beats exhaustive enumeration by eliminate many possible
    solutions without having to enumerate them all\pause
  \item Branch and bound can be slow as the constraints aren't
    necessarily very strong\pause
  \item By working harder we can sometimes strengthen the constraints
    thus eliminating much of the search space\pause
  \item This strategy works quite well on smallish problems, but usually
    fails on large problems\pause
  \end{itemize}
\end{PauseHighLight}

\end{slide}


%%%%%%%%%%%%%%%%%%%%%%% Next Slide %%%%%%%%%%%%%%%%%%%%%%%

\begin{slide}
\section{Cutting the Search Tree}

\begin{PauseHighLight}
\begin{minipage}{15cm}
  \begin{itemize}
  \item We can think of exact enumeration as exploring a giant search
    tree\pause
  \item If we know a partial solution is worse than our bound we cut the
    search tree\pause
  \item The earlier we cut the tree the more we can save\pause
  \end{itemize}
\end{minipage}
\begin{minipage}{8cm}
\includegraphics[width=8cm]{branchandbound.0}
\end{minipage}
\end{PauseHighLight}

\end{slide}

%%%%%%%%%%%%%%%%%%%%%%% Next Slide %%%%%%%%%%%%%%%%%%%%%%%

\begin{slide}
\section{Branch and Bound in Action}

\pb
\pause
\begin{center}
  \includegraphics[width=\linewidth]{bb_0}\mypl{1}
  \multido{\ia=1+1,\ib=2+1}{60}{%
    \llap{\includegraphics[width=\linewidth]{bb_\ia}\mypl{\ib}}}
\end{center}

\end{slide}


%%%%%%%%%%%%%%%%%%%%%%% Next Slide %%%%%%%%%%%%%%%%%%%%%%%

\begin{slide}
\section[-2]{Bound on Partial Solution}

\begin{PauseHighLight}
  \begin{itemize}\squeeze
  \item We know that the partial solution has to include all the
    remaining cities
    \begin{center}
      \includegraphics[height=4cm]{bblb}\pause
    \end{center}
  \item We can use this to obtain a lower bound on the partial
    solution\pause
  \item We know the remaining tour will go through each of the unvisited
    cities and the two edge cities\pause
  \item In fact the remaining part of the tour is a spanning tree of
    these vertices (it connects all the vertices once and has no
    cycles)\pause
  \item But we know a lower bound for this\pause---\emph{the minimum
      spanning tree}\pauseb
  \end{itemize}
\end{PauseHighLight}

\end{slide}

%%%%%%%%%%%%%%%%%%%%%%% Next Slide %%%%%%%%%%%%%%%%%%%%%%%

\begin{slide}
\section{Other Cuts}

\begin{PauseHighLight}
  \begin{itemize}
  \item For 2-D Euclidean TSPs edges should never cross\pause
    \begin{center}
      \includegraphics[width=0.9\linewidth]{twoopt}\pause
    \end{center}
  \item In fact we can check that we cannot perform a 2-opt move\pause
  \item We can also halve the search by considering only one
  direction---for example, by insisting we visit city~1 before city~2\pause
  \end{itemize}
\end{PauseHighLight}

\end{slide}

%%%%%%%%%%%%%%%%%%%%%%% Next Slide %%%%%%%%%%%%%%%%%%%%%%%

\begin{slide}
\section{Good Starting Bound}

\pb

\begin{minipage}{0.4\linewidth}
  \begin{itemize}
  \item It helps to start with a good bound\pauseh
  \item We can use an \textit{incomplete heuristic algorithm} to
    find a good solution which will act as a starting bound\pauseh
  \item One very simple heuristic is a greedy algorithm\pauseh
  \end{itemize}
\end{minipage}\hfil\
\begin{minipage}{0.54\linewidth}
  \setlength{\unitlength}{\linewidth}
  \begin{picture}(1,1)
    \put(0,0){\includegraphics[width=\linewidth]{greedy_0}}\pauselevel{=1 :4}\pause
    \put(0,0){\includegraphics[width=\linewidth]{greedy_1}}\pauselevel{=5 :5}\pause
    \put(0,0){\includegraphics[width=\linewidth]{greedy_2}}\pauselevel{=6 :6}\pause
    \put(0,0){\includegraphics[width=\linewidth]{greedy_3}}\pauselevel{=7 :7}\pause
    \put(0,0){\includegraphics[width=\linewidth]{greedy_4}}\pauselevel{=8 :8}\pause
    \put(0,0){\includegraphics[width=\linewidth]{greedy_5}}\pauselevel{=9 :9}\pause
  \end{picture}
\end{minipage}

\end{slide}

%%%%%%%%%%%%%%%%%%%%%%% Next Slide %%%%%%%%%%%%%%%%%%%%%%%

\begin{slide}
\section{Branch and Bound after Pruning}

\pb
\pause
\begin{center}
  \includegraphics[width=\linewidth]{bbfast_0}\mypl{1}
  \multido{\ia=1+1,\ib=2+1}{16}{%
    \llap{\includegraphics[width=\linewidth]{bbfast_\ia}\mypl{\ib}}}
\end{center}

\end{slide}
%%%%%%%%%%%%%%%%%%%%%%% Next Slide %%%%%%%%%%%%%%%%%%%%%%%

\begin{slide}
\section{Applications of Branch and Bound}

\begin{PauseHighLight}
  \begin{itemize}
  \item Branch and bound works for many optimisation problems\pause
  \item It's drawback is that you often end up still searching an
    exponentially large search space even though it might be massively
    faster than exhaustive enumeration\pause
  \item To make it work well requires considerable work\pause
  \item This is not an instantaneous algorithm, you may be waiting hours
    before you find a solution\pause
  \item For really large problems branch and bound might be too
    slow\pause
  \end{itemize}
\end{PauseHighLight}

\end{slide}

%%%%%%%%%%%%%%%%%%%%%%% Next Slide %%%%%%%%%%%%%%%%%%%%%%%
\Outline % Backtracking
%%%%%%%%%%%%%%%%%%%%%%% Next Slide %%%%%%%%%%%%%%%%%%%%%%%

\begin{slide}
\section{Other Search Strategies}

\begin{PauseHighLight}
  \begin{itemize}
  \item Search is a big topic in AI\pause
  \item The algorithms used depends on the information available\pause
  \item A classic search scenario is when there is ``heuristic''
    information which provides a hint as to where an optimal solution
    lies\pause
  \item Algorithms such as $A^*$ exist which will finds the best route
    given an (admissible) heuristic as efficiently as possible\pause
  \item You should learn about this next year in AI\pause
  \end{itemize}
\end{PauseHighLight}

\end{slide}

%%%%%%%%%%%%%%%%%%%%%%% Next Slide %%%%%%%%%%%%%%%%%%%%%%%

\begin{slide}
\section{Planning and Game Paying}

\begin{PauseHighLight}
  \begin{itemize}
  \item Search is also used to find the best action to take in planning
    problems and game playing (e.g. computer chess)\pause
  \item Again it is useful to think in terms of a search tree\pause
  \item Searching all paths on the search tree is usually infeasible\pause
  \item Look for ways of pruning the search tree to focus on good moves\pause
  \item Strategies include \textit{minimax} and \textit{alpha-beta
      pruning}\pause
  \end{itemize}
\end{PauseHighLight}

\end{slide}

%%%%%%%%%%%%%%%%%%%%%%% Next Slide %%%%%%%%%%%%%%%%%%%%%%%

\begin{slide}
\section[-2]{Minimax with Alpha-Beta Prunning}

\begin{center}
  \includegraphics[width=0.8\linewidth]{minimax-illustration}
\end{center}
\end{slide}


%%%%%%%%%%%%%%%%%%%%%%% Next Slide %%%%%%%%%%%%%%%%%%%%%%%

\begin{slide}
\section{Lessons}

\begin{PauseHighLight}
  \begin{itemize}
  \item Search has many applications\pause
  \item It is helpful to consider the search space as a tree whose
    branch corresponds to possible actions\pause
  \item Backtracking is useful in search trees with constraints\pause
  \item For optimisation problems branch and bound uses backtracking and
    costs of partial solutions as constraints\pause
  \item Widely applicable, but can take too long\pause
  \end{itemize}
\end{PauseHighLight}

\end{slide}
