%Master File:lectures.tex

\lesson{Improve Your General Knowledge}

\vspace{-2cm}
\begin{center}
  \includegraphics[height=10cm]{genknowledge}
\end{center}
\keywords{Computational Geometry, Numerical
  Algorithms, FFT, Linear Programming}

%%%%%%%%%%%%%%%%%%%%%%% Next Slide %%%%%%%%%%%%%%%%%%%%%%%
\renewcommand{\Outline}{%
\begin{slide}
\section[1]{Outline}

\begin{minipage}{12cm}
  \begin{enumerate}\squeeze
    \outlineitem{Computational Geometry}{computationalgeom}
    \outlineitem{Numerical Algorithms}{numericalalg}
    \begin{itemize}
    \item \toplink{fft}{FFT}
    \end{itemize}
    \outlineitem{Linear Programming, etc.}{linearprog}
    \outlineitem{Data Analysis}{dataanalsyis}
    \outlineitem{Problem Solving}{ai}
   \end{enumerate}
\end{minipage}\hfill
\begin{minipage}{10cm}
  \includegraphics[width=10cm]{genknowledge}
\end{minipage}
\end{slide}
\addtocounter{outlineitem}{1}
}

\setcounter{outlineitem}{1}
\Outline
\toptarget{firstoutline}
%%%%%%%%%%%%%%%%%%%%%%% Next Slide %%%%%%%%%%%%%%%%%%%%%%%

\begin{slide}
\section{What Isn't Covered}

\begin{PauseHighLight}
  \begin{itemize}
  \item This course has covered some of the most important and
    frequently used data structures and algorithms\pause
  \item There is a lot we haven't covered\pause
  \item This is an attempt at a whistle-stop tour of algorithms we
    haven't looked at\pause
  \item We start with a subject which appears in many data structure and
    algorithm books, namely \emph{multiway trees}\pause
  \end{itemize}
\end{PauseHighLight}

\end{slide}



%%%%%%%%%%%%%%%%%%%%%%% Next Slide %%%%%%%%%%%%%%%%%%%%%%%
\Outline % Computational Geometry
%%%%%%%%%%%%%%%%%%%%%%% Next Slide %%%%%%%%%%%%%%%%%%%%%%%

\begin{slide}
\section[-1]{Computational Geometry}

\begin{PauseHighLight}
  \begin{itemize}
  \item Computational Geometry is a study of algorithm related to
    geometric objects\pause
  \item There are obvious applications in computer graphics and machine
    vision\pause
  \item A typical problem is given $n$ points in the 2-d plane find the
    closest two points\pause
  \item The brute force method of comparing each pair of points takes
    $O(n^2)$\pause
  \item An algorithm based on divide-and-conquer strategy takes
    $O(n\log(n))$\pause
  \end{itemize}
\end{PauseHighLight}

\end{slide}

%%%%%%%%%%%%%%%%%%%%%%% Next Slide %%%%%%%%%%%%%%%%%%%%%%%

\begin{slide}
\section[-1.5]{Convex Hull}

\begin{PauseHighLight}
  \begin{itemize}
  \item Given a set of points find the smallest convex hull which
    contains all the points
    \begin{center}
      \begin{picture}(75,75)
        \put(0,-10){\includegraphics[height=10cm]{convexhull1}}\pause
        \put(0,-10){\includegraphics[height=10cm]{convexhull2}}\pauseb
      \end{picture}
    \end{center}
  \item The simplest brute force algorithm takes $O(n^3)$
    operations\pause
  \item An algorithm based on divide-and-conquer strategy closely
    resembling quicksort takes $O(n\log(n))$ operations\pause
  \end{itemize}
\end{PauseHighLight}

\end{slide}



%%%%%%%%%%%%%%%%%%%%%%% Next Slide %%%%%%%%%%%%%%%%%%%%%%%
\Outline % Numerical Algorithms
%%%%%%%%%%%%%%%%%%%%%%% Next Slide %%%%%%%%%%%%%%%%%%%%%%%

\begin{slide}
\section[-2]{Numerical Algorithms}

\begin{PauseHighLight}
  \begin{itemize}
  \item The analysis of numerical algorithms is a subject known as
    \emph{numerical analysis}\pause
  \item The classic book covering numerical algrorithms is
    \textit{Numerical Recipes in X} by Press et. al.\pause
  \item Topics covered include
    \begin{itemize}\squeeze
    \item Evaluation of special functions\pause
    \item Linear algebra (operations on matrices)\pause
    \item Interpolation and extrapolation\pause
    \item Computing integrals, solving differential equations\pause
    \item Solving equations (root finding), optimisation (maximising and
      minimising function)\pause
    \end{itemize}
  \end{itemize}
\end{PauseHighLight}

\end{slide}


%%%%%%%%%%%%%%%%%%%%%%% Next Slide %%%%%%%%%%%%%%%%%%%%%%%

\begin{slide}
\section[-1]{Fast Fourier Transform}
\toptarget{fft}

\begin{PauseHighLight}
  \begin{itemize}
  \item Numerical algorithms also include transformations\pause
  \item These provide different ``views'' of functions, sequences,
    signals, images, etc.\pause
  \item By far the most important of these is the Fourier
    Transform\pause
  \item The \emph{Fast Fourier Transform} was an algorithm divised by
    John Tukey and James Cooley in 1965 to compute the Fourier
    Transform\pause
  \item It is based on a divide-and-conquer strategy and takes
    $O(n\log(n))$ operation compared to the $O(n^2)$ brute force
    method\pause
  \end{itemize}
\end{PauseHighLight}

\end{slide}

%%%%%%%%%%%%%%%%%%%%%%% Next Slide %%%%%%%%%%%%%%%%%%%%%%%

\begin{slide}
\section{Applications of FFT}

\begin{PauseHighLight}
  \begin{itemize}
  \item The application of FFT are enormous\pause
  \item It lies at the heart of digital signal processing\pause
  \item It is frequently used in image analysis\pause
  \item It is even used in fast multiplication of very large
    integers---with important applications in cryptography\pause
  \end{itemize}
\end{PauseHighLight}

\end{slide}


%%%%%%%%%%%%%%%%%%%%%%% Next Slide %%%%%%%%%%%%%%%%%%%%%%%
\Outline % Linear Programming, Etc.
%%%%%%%%%%%%%%%%%%%%%%% Next Slide %%%%%%%%%%%%%%%%%%%%%%%

\begin{slide}
\section{Other Problems in P}

\begin{PauseHighLight}
  \begin{itemize}
  \item There are many problems with brute force algorithms that take
    exponential time which actually have polynomial time algorithms\pause
  \item We have only looked at minimum spanning tree and shortest
    path\pause
  \item These algorithms are very important and lie at the heart of
    \emph{Operations Research}\pause
  \end{itemize}
\end{PauseHighLight}

\end{slide}

%%%%%%%%%%%%%%%%%%%%%%% Next Slide %%%%%%%%%%%%%%%%%%%%%%%

\begin{slide}
\section[-1]{Linear Assignment}

\begin{minipage}{12cm}
\begin{PauseHighLight}
  \begin{itemize}
  \item The linear assignment problem is pair two sets of object\pause
  \item There is a cost associated with every pair\pause
  \item The aim is to minimise the overall cost\pause
  \item Brute force method would take $\Theta(n!)$ operations\pause
  \item The best algorithms take $O(n^3)$ operations\pause
  \end{itemize}
\end{PauseHighLight}

\end{minipage}\hfill
\begin{minipage}{10cm}
\includegraphics[width=10cm]{linear_assignment}
\end{minipage}
\end{slide}

%%%%%%%%%%%%%%%%%%%%%%% Next Slide %%%%%%%%%%%%%%%%%%%%%%%

\begin{slide}
\section[-1]{Max Flow}

\begin{PauseHighLight}
  \begin{itemize}
  \item Given a directed graph where each edge has a carrying capacity
    find the maximum flow between a source and target node\pause
  \item Algorithm is $O(\vert V \vert \times \vert E \vert^2)$\pause
  \begin{center}
    \includegraphics[height=10cm]{maxflow}
  \end{center}
  \end{itemize}
\end{PauseHighLight}

\end{slide}

%%%%%%%%%%%%%%%%%%%%%%% Next Slide %%%%%%%%%%%%%%%%%%%%%%%

\begin{slide}
\section[-1.5]{Linear Programming}

\begin{PauseHighLight}
  \begin{itemize}\squeeze
  \item Many problems in class P can be reduced to a \emph{linear
    programming} problem\pause
  \item Given a linear function $f(\bm{x})= \sum_i c_i x_i$\pause
  \item A set of linear constraints of the form $b_1 x_1 + \cdots +
    b_m x_n \leq b_0$\pause
  \item Find the optimal value $\bm{x}$\pause
  \item George Dantzig invented the \emph{simplex algorithm} for solving
    linear programming problems in 1947\pause
  \item The simplex algorithm has an exponential worst-case
    complexity!\pause
  \item However, it is nearly always very fast!---there are now
    polynomial algorithms for linear programming, but they are rarely as
    fast as the simplex algorithm\pause
  \end{itemize}
\end{PauseHighLight}

\end{slide}

%%%%%%%%%%%%%%%%%%%%%%% Next Slide %%%%%%%%%%%%%%%%%%%%%%%

\begin{slide}
\section[-1]{A Simple Linear Programming Problem}

\begin{center}
  \includegraphics[height=14cm]{linearprogramming}
\end{center}
\end{slide}

%%%%%%%%%%%%%%%%%%%%%%% Next Slide %%%%%%%%%%%%%%%%%%%%%%%

\begin{slide}
\section{Application of Linear Programming}

\begin{PauseHighLight}
  \begin{itemize}
  \item There are many applications of linear programming\pause
    \begin{itemize}
    \item Minimising costs in construction projects
    \item Truck routing
    \item Staff scheduling
    \item Obtaining bounds on TSP\pause
    \end{itemize}
  \end{itemize}
\end{PauseHighLight}

\end{slide}


%%%%%%%%%%%%%%%%%%%%%%% Next Slide %%%%%%%%%%%%%%%%%%%%%%%

\begin{slide}
\section{Quadratic Programming}

\begin{PauseHighLight}
  \begin{itemize}
  \item In \emph{quadratic programming} we have a quadratic cost function
    \begin{align*}
      f(\bm{x}) = \sum_{i,j} Q_{i,j} x_i x_j + \sum_i c_i x_i\pause
    \end{align*}
  \item We also have linear constraints similar to linear
    programming\pause
  \item There are polynomial time, $O(n^3)$, algorithms for solving
    this problem provided the matrix $\mat{Q}$ is positive
    semi-definite\pause
  \item Many application including machine learning\pause
  \end{itemize}
\end{PauseHighLight}

\end{slide}


%%%%%%%%%%%%%%%%%%%%%%% Next Slide %%%%%%%%%%%%%%%%%%%%%%%
\Outline % Data Analysis
%%%%%%%%%%%%%%%%%%%%%%% Next Slide %%%%%%%%%%%%%%%%%%%%%%%

\begin{slide}
\section{Data Analysis}

\begin{PauseHighLight}
  \begin{itemize}
  \item There are a large number of applications related to data
    analysis\pause
  \item These include finding stuctures in data\pause
    \begin{itemize}
    \item Cluster detection (K-means clustering)\pause
    \item Find relevant directions in high dimensional data (PCA)\pause
    \end{itemize}
  \item Statistical properties of data\pause
    \begin{itemize}
    \item Hypothesis testing\pause
    \item Independence\pause
    \end{itemize}
\end{itemize}
\end{PauseHighLight}

\end{slide}

%%%%%%%%%%%%%%%%%%%%%%% Next Slide %%%%%%%%%%%%%%%%%%%%%%%

\begin{slide}
\section[-2]{Learning from Data}

\begin{PauseHighLight}
  \begin{itemize}
  \item Given some examples can you extract rules from data\pause
  \item This is the problem of machine learning/pattern
    recognition\pause
  \item Strategies include, neural networks, SVM, decision trees,
    probabilitic reasoning (Bayes rules)\pause
  \item Used in fraud detection, spam filters, medical diognosis,
    machine vision, etc.
  \end{itemize}
  \begin{center}
    \includegraphics[width=0.4\linewidth]{googleCar}\pause
  \end{center}
\end{PauseHighLight}

\end{slide}

%%%%%%%%%%%%%%%%%%%%%%% Next Slide %%%%%%%%%%%%%%%%%%%%%%%
\Outline % Data Analysis
%%%%%%%%%%%%%%%%%%%%%%% Next Slide %%%%%%%%%%%%%%%%%%%%%%%

\begin{slide}
\section{AI}

\begin{PauseHighLight}
  \begin{itemize}
  \item Large number of AI algorithms
    \begin{itemize}
    \item Search algorithms
      \begin{itemize}
      \item Depth first, breadth first, $A^{*}$, \ldots
      \item Used for solving mazes, route planning, etc.
      \end{itemize}
    \item Game Playing Algorithms
      \begin{itemize}
      \item minimax, alpha-beta, pruning algorithms, etc.
      \item reinforcement learning, Q-learning
      \item Used for chess, backgammon, \ldots
      \end{itemize}
    \end{itemize}
  \end{itemize}
\end{PauseHighLight}

\end{slide}


%%%%%%%%%%%%%%%%%%%%%%% Next Slide %%%%%%%%%%%%%%%%%%%%%%%

\begin{slide}
\section[-1.5]{Lessons}

\begin{PauseHighLight}
  \begin{itemize}
  \item There are large number of good algorithms that have been
    developed\pause
  \item Part of being a computer scientist is being able to analyse a
    problem and see if it has been solved before\pause
  \item These algorithms are developed in many advanced course
    \begin{itemize}\squeeze
    \item Intelligent algorithms
    \item Compilers
    \item Artificial Intelligence
    \item Machine Learning
    \item Computer Graphics/Machine Vision
    \item Cryptography\pause
    \end{itemize}
  \end{itemize}
\end{PauseHighLight}

\end{slide}
