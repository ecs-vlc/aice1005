%Master File:lectures.tex

\lesson{Think Graphically}
\vspace{-1cm}
\begin{center}
  \includegraphics[height=10cm]{colour-graph}
\end{center}
\keywords{Graph theory, applications of graphs, graph problems}

%%%%%%%%%%%%%%%%%%%%%%% Next Slide %%%%%%%%%%%%%%%%%%%%%%%
\renewcommand{\Outline}{%
\begin{slide}
\section[1]{Outline}

\begin{minipage}{12cm}
  \begin{enumerate}\squeeze
    \outlineitem{Graph Theory}{gt}
    \outlineitem{Applications of Graphs}{aog}
    \begin{itemize}
    \item \toplink{geometry}{Geometric applications}
    \item \toplink{relational}{Relational applications}
    \end{itemize}
    \outlineitem{Implementing Graphs}{gr}
    \outlineitem{Graph Problems}{gp}
  \end{enumerate}
\end{minipage}\hfill
\begin{minipage}{10cm}
  \includegraphics[width=10cm]{colour-graph}
\end{minipage}
\end{slide}
\addtocounter{outlineitem}{1}
}

\setcounter{outlineitem}{1}
\Outline
\toptarget{firstoutline}

%%%%%%%%%%%%%%%%%%%%%%% Next Slide %%%%%%%%%%%%%%%%%%%%%%%

\begin{slide}
\section{Motivation}

\begin{PauseHighLight}
  \begin{itemize}
  \item Many different problems can be described in terms of
    graphs\pause
  \item This often reveals the true nature of the problem\pause
  \item It unifies many apparently different problems\pause
  \item As much is known about graph problems it often provides a
    pointer to the solution\pause
  \end{itemize}
\end{PauseHighLight}

\end{slide}

%%%%%%%%%%%%%%%%%%%%%%% Next Slide %%%%%%%%%%%%%%%%%%%%%%%

\begin{slide}
\section[-2]{Definition of a Graph}

\begin{PauseHighLight}
  \begin{itemize}\squeeze
  \item A graph, $G$, can be described by\pause
    \begin{itemize}
    \item A set of vertices or nodes $\mathcal{V} = \{1,2,3 \ldots
      n\}$\pause
    \item A set of edges $\mathcal{E} = \{(i,j)| \mbox{vertex $i$ is
    connected to vertex $j$} \}$\pause
    \end{itemize}
  \item The edges may be
    \begin{itemize}
    \item \emph{directed}---sometimes called a \emph{digraph}\pause
    \item \emph{undirected}\pause
    \end{itemize}
  \end{itemize}
  \begin{center}
    \includegraphics[width=0.3\linewidth]{directed-graph.mps}\hfil
    \pauselevel{=4}\pause
    \includegraphics[width=0.3\linewidth]{undirected-graph.mps}
    \pauselevel{=5}\pause
  \end{center}
\end{PauseHighLight}

\end{slide}

%%%%%%%%%%%%%%%%%%%%%%% Next Slide %%%%%%%%%%%%%%%%%%%%%%%

\begin{slide}
\section[-2]{Connected and Unconnected Graphs}


\begin{PauseHighLight}
  \begin{itemize}
  \item A graph is \emph{connected} if you can get from one node to any other
    along a series of edges\pause
  \item Otherwise it is \emph{disconnected}\pause
    \begin{center}
      \multiinclude[graphics={height=10cm}]{disconnected-graph}\pause
    \end{center}
  \end{itemize}
\end{PauseHighLight}

\end{slide}

%%%%%%%%%%%%%%%%%%%%%%% Next Slide %%%%%%%%%%%%%%%%%%%%%%%

\begin{slide}
\section[-2]{Trees}

\begin{PauseHighLight}
  \begin{itemize}
  \item A tree is a connected graphs with no cycles\pause
  \item A tree will have $n-1$ edges\pause
  \end{itemize}
  \begin{center}\color{TextColor}
    \includegraphics[height=10cm]{treeGraph.mps}
  \end{center}
\end{PauseHighLight}

\end{slide}


%%%%%%%%%%%%%%%%%%%%%%% Next Slide %%%%%%%%%%%%%%%%%%%%%%%

\begin{slide}
\section[-2]{Multigraphs}

\begin{PauseHighLight}
  \begin{itemize}
  \item If the collection of edges is a \textit{multiset} then we obtain
    a \emph{multigraphs} where more than one edge is allowed between
    pairs of vertices
  \begin{center}\color{TextColor}
    \includegraphics[height=10cm]{multigraph.mps}\pause
  \end{center}
  \end{itemize}
\end{PauseHighLight}


\end{slide}


%%%%%%%%%%%%%%%%%%%%%%% Next Slide %%%%%%%%%%%%%%%%%%%%%%%

\begin{slide}
\section[-2]{Weighted Graphs}

\begin{PauseHighLight}
  \begin{itemize}
  \item If we assign a number to an edge we obtain a \emph{weighted
      graph}
  \end{itemize}
  \begin{center}\color{TextColor}
    \includegraphics[height=10cm]{weightedGraph.mps}\pause
  \end{center}
\end{PauseHighLight}

\end{slide}

%%%%%%%%%%%%%%%%%%%%%%% Next Slide %%%%%%%%%%%%%%%%%%%%%%%

\begin{slide}
\section{Networks}

\begin{PauseHighLight}
  \begin{itemize}
  \item Sometimes we add more information to the graph\pause
  \item E.g. attributes to the nodes or edges\pause
  \item Graphs with many attributes are often referred to as
    \emph{networks}\pause 
  \end{itemize}
\end{PauseHighLight}

\end{slide}


%%%%%%%%%%%%%%%%%%%%%%% Next Slide %%%%%%%%%%%%%%%%%%%%%%%
\Outline % Applications of Graphs
%%%%%%%%%%%%%%%%%%%%%%% Next Slide %%%%%%%%%%%%%%%%%%%%%%%

\begin{slide}
\section[-2]{Bridges of K\"onigsberg}
\toptarget{geometry}

\pausebuild
Is there a tour around K\"onigsberg going over every bridge once?\pause
\begin{center}
  \begin{picture}(140,120)
    \put(0,0){\includegraphics[height=12cm]{Konigsberg}}
      \pauselevel{=1 :2}\pause
    \put(0,0){\includegraphics[height=12cm]{Konigsberg_bridge}}
      \pauselevel{=2 :3}\pause
    \put(0,0){\includegraphics[height=12cm]{kb3}}
      \pauselevel{=3 :4}\pause
  \end{picture}
\end{center}
In 1736 Euler published a paper answering this question and founding
graph theory\pause
\end{slide}

%%%%%%%%%%%%%%%%%%%%%%% Next Slide %%%%%%%%%%%%%%%%%%%%%%%

\begin{slide}
\section{Representing Distances}

\begin{minipage}{8cm}
  \begin{center}
    \begin{picture}(80,140)
      \put(0,0){\multiinclude[graphics={width=8cm}]{uk1}}\pause
      \put(0,0){\includegraphics[width=8cm]{uk.pdf}}\pause
    \end{picture}    
  \end{center}
\end{minipage}\hfil
\begin{minipage}{14cm}
  \color{TwoColor}
  \begin{itemize}
  \item Consider some graph\hl{1}
  \item With weights representing the distance between nodes\pauseh
  \item What is the shortest distance between $S$ and $I$?\pauseh
  \end{itemize}
\end{minipage}
\end{slide}

%%%%%%%%%%%%%%%%%%%%%%% Next Slide %%%%%%%%%%%%%%%%%%%%%%%

\begin{slide}
\section{Other Applications}

\begin{PauseHighLight}
  \begin{itemize}
  \item We could take the weights to represent the time taken to travel
    between nodes\pause
  \item In a computer network the weights might represent the
    bandwidth\pause 
  \item In a representation of a transport system the weights might
    represent the carrying capacity of the traffic on a road\pause
  \item Graphs can be used to represent other kinds of relationships\pause
  \item E.g. We could create a digraph of links between web pages\pause
  \end{itemize}
\end{PauseHighLight}

\end{slide}


%%%%%%%%%%%%%%%%%%%%%%% Next Slide %%%%%%%%%%%%%%%%%%%%%%%

\begin{slide}{}
\section[-2]{Christmas Card Problem}
\toptarget{relational}

\pausebuild
\color{TwoColor}
\begin{itemize}\squeeze
\item I have four types of Christmas cards\pauseh
\item Some of my friends know each other\pauseh
\begin{center}
  \color{TextColor}
  \begin{picture}(100,100)
    \put(0,0){\includegraphics[height=10cm]{name-colour}}
    \pauselevel{=1}\pause
    \put(0,0){\includegraphics[height=10cm]{colour-graph}}
    \pauselevel{=4}\pause
  \end{picture}
\end{center}
\color{TwoColor}\pauselevel{=3}
\item I don't want to send friends that know each other the same card\pauseh
\end{itemize}
\end{slide}

%%%%%%%%%%%%%%%%%%%%%%% Next Slide %%%%%%%%%%%%%%%%%%%%%%%

\begin{slide}
\section[-2]{A Real World Problem}

\begin{PauseHighLight}
  \begin{itemize}
  \item A food company used different colour bags for each of it
    products\pause
  \item To save money they reduced the stock of bags to 25\pause
  \item They wanted to know what items to put in what bags so that as
    few customers as possible would have items with the same colour
    bags\pause
  \item This can again be reduced to a graph colouring problem\pause
    \begin{itemize}
    \item Each node represents an item\pause
    \item The edges were weighted by the number of customers that took
      both items\pause
    \item The aim was to colour the nodes with 25 colours to minimise
      the weights where the edges shared the same colour\pause
    \end{itemize}
  \end{itemize}
\end{PauseHighLight}

\end{slide}

%%%%%%%%%%%%%%%%%%%%%%% Next Slide %%%%%%%%%%%%%%%%%%%%%%%

\begin{slide}
\section{Frequency Assignment Problem}

\begin{center}
  \includegraphics[width=0.9\linewidth]{frequency_assignment}
\end{center}
\end{slide}


%%%%%%%%%%%%%%%%%%%%%%% Next Slide %%%%%%%%%%%%%%%%%%%%%%%
\Outline % Implementing Graphs
%%%%%%%%%%%%%%%%%%%%%%% Next Slide %%%%%%%%%%%%%%%%%%%%%%%


\begin{slide}
\section{Representations}

\begin{PauseHighLight}
  \begin{itemize}
  \item There is no single way to represent graphs\pause
  \item The best representation depends on the graph\pause
  \item Some books describe a \textit{Graph ADT}---graphs are too varied
    for this to be very useful\pause
  \item An important issue in representing a graph is how to store the
    edge information\pause
  \end{itemize}
\end{PauseHighLight}

\end{slide}

%%%%%%%%%%%%%%%%%%%%%%% Next Slide %%%%%%%%%%%%%%%%%%%%%%%

\begin{slide}
\section[-1]{Adjacency Matrices}

\begin{PauseHighLight}
  \begin{itemize}
  \item One representation of a graph $G=(\mathcal{V},\mathcal{E})$ is
    in term of an $n\times n$ \emph{adjacency matrix} $\mat{A}$ with elements
    \begin{align*}
      A_{ij} = \left\{
          \begin{array}{lll}
            1 & & \mbox{if $(i,j)\in\mathcal{E}$}\\
            0 & & \mbox{if $(i,j)\not\in\mathcal{E}$}\\
          \end{array}\right.
    \end{align*}
    where $n=\vert\mathcal{V}\vert$\pause
  \item For undirected graphs $\mat{A}$ is a symmetric matrix, i.e.
    $\mat{A}=\mat{A}^\tr$\pause
  \item For weighted graphs we often store the \emph{connectivity
      matrix} or \emph{cost-adjacency matrix}, $\mat{C}$, where
    \begin{align*}
      C_{ij} = \left\{
          \begin{array}{lll}
            w_{ij} & & \mbox{if $(i,j)\in\mathcal{E}$}\\
            0 & & \mbox{if $(i,j)\not\in\mathcal{E}$}\\
          \end{array}\right.\pause
    \end{align*}
  \end{itemize}
\end{PauseHighLight}

\end{slide}

%%%%%%%%%%%%%%%%%%%%%%% Next Slide %%%%%%%%%%%%%%%%%%%%%%%

\begin{slide}
\section{Adjacency Lists}

\begin{PauseHighLight}
  \begin{itemize}
  \item For \emph{dense} graphs where the number of edges is $\Theta(n^2)$
    the adjacency matrix is often a useful representation\pause
  \item But in \emph{sparse} graphs where the number of edges is $\Theta(n)$
    the adjacency matrix has a very large number of zeros\pause
  \item A more efficient representation is in terms of the adjacency
    list where the set of outgoing edges is stored for each node\pause
  \item In some applications it is useful to store both the adjacency
    matrix and the adjacency list\pause
  \end{itemize}
\end{PauseHighLight}

\end{slide}


%%%%%%%%%%%%%%%%%%%%%%% Next Slide %%%%%%%%%%%%%%%%%%%%%%%

\begin{slide}
\section{Representing Undirected Graphs}

\begin{center}
  \includegraphics[width=\linewidth]{adjacency-matrix.mps}
  Graph \hfil\hfil Adjacency Matrix \hfil Adjancency List
\end{center}

\end{slide}

%%%%%%%%%%%%%%%%%%%%%%% Next Slide %%%%%%%%%%%%%%%%%%%%%%%

\begin{slide}
\section{Representing Digraphs}

\begin{center}
  \includegraphics[width=\linewidth]{adj-digraph.mps}
  Graph \hfil\hfil Adjacency Matrix \hfil Adjancency List
\end{center}

\end{slide}


%%%%%%%%%%%%%%%%%%%%%%% Next Slide %%%%%%%%%%%%%%%%%%%%%%%
\Outline % Graph Problems
%%%%%%%%%%%%%%%%%%%%%%% Next Slide %%%%%%%%%%%%%%%%%%%%%%%

\begin{slide}
\section[-2]{Hamilton Cycle}
\toptarget{hamiltoncycle}

\pausebuild
\color{TwoColor}
\begin{itemize}\squeeze
\item The Euler path problem is to find a path through a multigraph
  that passes through every edge once---easy to solve\pauseh
\item The Hamilton cycle problem is to find a cycle that goes through
  each vertex exactly once\pauseh
  \begin{center}\pauselevel{=1, :2}\color{TextColor}
    \multiinclude[graphics={height=8.5cm}]{hamiltoncycle}\pauselevel{=3}\pause
  \end{center}\color{TwoColor}
\item There is no known efficient algorithm to solve this\pauseh
\end{itemize}

\end{slide}

%%%%%%%%%%%%%%%%%%%%%%% Next Slide %%%%%%%%%%%%%%%%%%%%%%%

\begin{slide}
\section{Shortest Path and TSP}
\toptarget{shortestpath}

\begin{PauseHighLight}
  \begin{itemize}
  \item The shortest path problem is to find a path between two
    nodes\pause
  \item There is an efficient algorithm---see next lecture\pause
  \item In the travelling salesperson problem the task is to find the
    shortest tour (Hamilton cycle)---we usually assume there is an edge
    between every pair of nodes\pause
  \item There is no know efficient algorithm to solve all TSPs\pause
  \end{itemize}
\end{PauseHighLight}

\end{slide}

%%%%%%%%%%%%%%%%%%%%%%% Next Slide %%%%%%%%%%%%%%%%%%%%%%%

\begin{slide}
\section[-2]{Minimum Spanning Tree}
\toptarget{minspantree}
\pausebuild
\color{TwoColor}
\begin{itemize}
\item Suppose we want to construct pylons connecting a
  number of cities using the least amount of cable\pauseh
  \vspace*{-1cm}
  \begin{center}\pauselevel{=1, :2}\color{TextColor}
    \multiinclude[graphics={height=12cm}]{PrimGraph}\pause
  \end{center}
  \vspace*{-1cm}
\color{TwoColor}
\item We will study an efficient algorithm to solve this in the next
  but one lecture\pauseh
\end{itemize}

\end{slide}

%%%%%%%%%%%%%%%%%%%%%%% Next Slide %%%%%%%%%%%%%%%%%%%%%%%

\begin{slide}
\section[-2]{Graph Partitioning}
\toptarget{partitioning}

\begin{PauseHighLight}
  \begin{itemize}\squeeze
  \item The simplest version of this problem is to cut a graph into two
    equal halves so that you minimise the number of edges you cut\pause
  \item If the edges are weighted then you want to minimise the sum of
    edges that are cut\pause
  \item If the vertices are weighted you want to balance the sum of
    vertex weights in the two partitions\pause
  \item An example of this problem is in dividing up a problem to run on
    a parallel computer\pause
    \begin{itemize}
    \item Nodes are subtasks (weights on nodes are run times)\pause
    \item Edge weights indicate communication cost\pause
    \end{itemize}
  \item There is no known efficient algorithm to solve this\pause
  \end{itemize}
\end{PauseHighLight}

\end{slide}

%%%%%%%%%%%%%%%%%%%%%%% Next Slide %%%%%%%%%%%%%%%%%%%%%%%

\begin{slide}
\section{Graph Isomorphism}
\toptarget{isograph}

\begin{PauseHighLight}
  \begin{itemize}
  \item Do two graphs have the same structure?\pause
    \begin{center}
      \includegraphics[height=8cm]{isomorphism}
    \end{center}
  \item There is no known efficient algorithm to solve this
    problem\pause
  \item Theoretically it is interesting because it is not NP-complete\pause
  \end{itemize}
\end{PauseHighLight}

\end{slide}

%%%%%%%%%%%%%%%%%%%%%%% Next Slide %%%%%%%%%%%%%%%%%%%%%%%

\begin{slide}
\section[-2]{Vertex Cover}

\pausebuild
\color{TwoColor}
\begin{itemize}
\item How many guards do you need to cover all the corridors in a
  museum\pauseh
  \vspace*{-1cm}
  \begin{center}\pauselevel{=1, :2}\color{TextColor}
    \multiinclude[graphics={height=10cm}]{VertexCover}\pause
  \end{center}
  \vspace*{-1cm}
\color{TwoColor}
\item There is no known efficient algorithm to solve this\pauseh
\end{itemize}

\end{slide}


%%%%%%%%%%%%%%%%%%%%%%% Next Slide %%%%%%%%%%%%%%%%%%%%%%%

\begin{slide}
\section[-2]{Graph Colouring}
\toptarget{colouring}

\color{TwoColor}
\begin{itemize}
\item How many colours do I need to colour a graph with no conflicts\pauseh
  \begin{center}\pauselevel{=1, :2}\color{TextColor}
    \begin{picture}(100,100)
      \put(0,0){\includegraphics[height=11cm]{noncolour-graph}}\pause
      \put(0,0){\includegraphics[height=11cm]{colour-graph}}\pause
    \end{picture}
  \end{center}
\color{TwoColor}
\item There is no known efficient algorithm to solve this\pauseh
\end{itemize}

\end{slide}

%%%%%%%%%%%%%%%%%%%%%%% Next Slide %%%%%%%%%%%%%%%%%%%%%%%

\begin{slide}
\section{Other Graph Problems}

\begin{PauseHighLight}
  \begin{itemize}
  \item These are only a sample of the many famous graph
    problems\pause
  \item Others include

    \begin{minipage}{10cm}
      \begin{itemize}
      \item Max-clique (hard)\pause
      \item Maximal independent set (hard)\pause
      \item Maximal flow problem (easy)\pause
      \item Max-cut (hard)\pause
      \end{itemize}
    \end{minipage}\hfill
    \begin{minipage}{11cm}
      \includegraphics[width=0.9\linewidth]{maxclique}
    \end{minipage}
  \end{itemize}
\end{PauseHighLight}

\end{slide}


%%%%%%%%%%%%%%%%%%%%%%% Next Slide %%%%%%%%%%%%%%%%%%%%%%%

\begin{slide}
\section{Lessons}

\begin{PauseHighLight}
  \begin{itemize}
  \item Graphs are an important method for abstracting problems\pause
  \item They appear in a huge number of disparate fields\pause
  \item There are many problems for which efficient algorithms are
    known\pause
  \item There are many problems which are believed to be
    hard---i.e. there aren't any efficient algorithms\pause
  \item Even for hard problems there are good algorithms for finding
    approximated solutions\pause
  \end{itemize}
\end{PauseHighLight}

\end{slide}

