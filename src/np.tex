%Master File:lectures.tex
\renewcommand{\class}[1]{\textsf{#1}}

\lesson{Know What's Possible}

\vspace{-2cm}
\begin{center}
  \includegraphics[height=10cm]{\figs/dp}
\end{center}
\keywords{Combinatorial optimisation, NP-completeness, polynomial reduction}
%%%%%%%%%%%%%%%%%%%%%%% Next Slide %%%%%%%%%%%%%%%%%%%%%%%
\renewcommand{\Outline}{%
\begin{slide}
\section[1]{Outline}

\begin{minipage}{12cm}
  \begin{enumerate}\squeeze
    \outlineitem{Motivation}{opt}
    \outlineitem{P, NP and NP-complete}{npcomplete}
    \outlineitem{Polynomial Reduction}{polyr}
   \end{enumerate}
\end{minipage}\hfill
\begin{minipage}{10cm}
  \includegraphics[width=10cm]{\figs/dp}
\end{minipage}
\end{slide}
\addtocounter{outlineitem}{1}
}

\setcounter{outlineitem}{1}

%%%%%%%%%%%%%%%%%%%%%%% Next Slide %%%%%%%%%%%%%%%%%%%%%%%
\Outline %
\toptarget{firstoutline}
%%%%%%%%%%%%%%%%%%%%%%% Next Slide %%%%%%%%%%%%%%%%%%%%%%%

\begin{slide}
\section{Exponentially Large Search Spaces}

\begin{PauseHighLight}
  \begin{itemize}
  \item We have seen a large number of decision problems and
    optimisation problems involving an exponentially large search
    space\pause
  \item For some of these we have found efficient algorithms (greedy
    algorithms, divide and conquer, dynamic programming, \ldots)\pause
  \item For other problems we have found good algorithms (backtracking,
    branch and bound), but they are not necessarily polynomial\pause
  \item Can we say anything general about how easy they are to solve\pause
  \end{itemize}
\end{PauseHighLight}

\end{slide}

%%%%%%%%%%%%%%%%%%%%%%% Next Slide %%%%%%%%%%%%%%%%%%%%%%%

\begin{slide}
\section{Types of Problems}

\begin{PauseHighLight}
  \begin{itemize}
  \item We concentrate here on two types of problems
    \begin{itemize}
    \item Decision Problems
    \item Combinatorial Optimisation Problems\pause
    \end{itemize}
  \item Decision problems are problems with a true/false answer, e.g. is
    it possible to cross all the bridges of K\"onigsberg once?\pause
  \item We showed earlier that backtracking can be used to find a
    solution which answers the decision problems, e.g. Hamiltonian
    circuit problem\pause
  \item There are many other decision problems, but the most famous is
    satisfiability or SAT\pause
  \end{itemize}
\end{PauseHighLight}

\end{slide}

%%%%%%%%%%%%%%%%%%%%%%% Next Slide %%%%%%%%%%%%%%%%%%%%%%%

\begin{slide}
\section[-1.5]{SAT}

\begin{PauseHighLight}
  \begin{itemize}\squeeze
  \item Given $n$ Boolean variables $X_i\in\{T,F\}$\pause
  \item $m$ disjunctive (or's) clauses, e.g.
    \begin{align*}
      c_1 &= X_1 \vee \neg X_2 \vee X_3 \\
      c_2 &= \neg X_2 \vee  X_3 \vee X_5 \\
      \vdots & \hspace{2cm} \vdots \\
      c_m &= X_2 \vee \neg X_4 \vee \neg X_5\pause
    \end{align*}
  \item Find an assignment, $\bm{X} \in \{T,F\}^n$ which satisfies all
    the clauses\pause
  \item We can view this as finding an assignment that makes the
    formula $f(\bm{X})$ true where
    \begin{align*}
      f(\bm{X}) = c_1 \wedge c_2 \wedge \cdots \wedge c_m\pause
    \end{align*}
  \end{itemize}
\end{PauseHighLight}

\end{slide}

%%%%%%%%%%%%%%%%%%%%%%% Next Slide %%%%%%%%%%%%%%%%%%%%%%%

\begin{slide}
\section{Decisions and Optimisation Problems}

\begin{PauseHighLight}
  \begin{itemize}
  \item Often we can cast a decision problem as an optimisation problem\pause
  \item E.g. the MAX-SAT problem is to find an assignment of variables
    that satisfies the most clauses\pause
  \item If we can solve the MAX-SAT optimisation problem we can solve
    the decision problem\pause
  \item We can also cast optimisation problems as decision problems
    \begin{quote}
      \textit{Does there exist a TSP tour shorter than 200 miles?}\pause
    \end{quote}
  \end{itemize}
\end{PauseHighLight}

\end{slide}


%%%%%%%%%%%%%%%%%%%%%%% Next Slide %%%%%%%%%%%%%%%%%%%%%%%

\begin{slide}
\section[-1]{Combinatorial Optimisation Problems}

\begin{PauseHighLight}
  \begin{itemize}
  \item In the set of discrete optimisation problems an important class
    are those that involve \textit{combinatorial objects} such as
    permutations, binary string, etc.\pause
  \item Optimisation problems involving such objects are termed
    \emph{combinatorial optimisation problems}\pause
  \item Classical examples of such problems include
    \begin{itemize}\squeeze
    \item Travelling Salesperson Problem (TSP)
    \item Graph colouring
    \item Maximum Satisfiability (MAX-SAT)
    \item Many scheduling problems
    \item Bin-packing
    \item Quadratic integer problems\pause
    \end{itemize}
  \end{itemize}
\end{PauseHighLight}

\end{slide}


%%%%%%%%%%%%%%%%%%%%%%% Next Slide %%%%%%%%%%%%%%%%%%%%%%%
\Outline % P, NP and NP-complete
%%%%%%%%%%%%%%%%%%%%%%% Next Slide %%%%%%%%%%%%%%%%%%%%%%%

\begin{slide}
\section{Polynomial Problems}

\begin{PauseHighLight}
  \begin{itemize}
  \item Some optimisation problems are ``easy''---there are known
    polynomial time algorithms to solve them\pause
    \begin{itemize}
    \item Minimum spanning tree
    \item Shortest path
    \item Linear assignment problem
    \item Maximum flow between any two vertices of a directed graph
    \item Linear programming\pause
    \end{itemize}
  \item Many apparently different problems can be mapped onto these
    problems\pause
  \end{itemize}
\end{PauseHighLight}

\end{slide}

%%%%%%%%%%%%%%%%%%%%%%% Next Slide %%%%%%%%%%%%%%%%%%%%%%%

\begin{slide}
\section[-1]{\class{NP-Hard} Problems}
\toptarget{nphard}

\begin{PauseHighLight}
  \begin{itemize}
  \item Is it possible to solve the TSP in polynomial time?\pause
  \item Answer:\pause{} \emph{maybe}\pauseb
  \item However, no one has discovered such an algorithm\pause\ and if they do
    it will have huge implications\pause
  \item TSP is an example of a class of problems called \class{NP-Hard}\pause
  \item If you can solve one of these problems then you can solve a whole
    class of problems\pause
  \item To understand what \class{NP-Hard} means we must backtrack\pause 
  \end{itemize}
\end{PauseHighLight}

\end{slide}

%%%%%%%%%%%%%%%%%%%%%%% Next Slide %%%%%%%%%%%%%%%%%%%%%%%

\begin{slide}
\section[-1]{Decision Problems}

\begin{PauseHighLight}
  \begin{itemize}\squeeze
  \item Decision problems are problems with a true or false answer\pause
  \item E.g. does a TSP have a tour less than $2\,000$ miles?\pause
  \item Algorithmic complexity theory deals with classes of decision problems
    that have some characteristic size, $n$\pause
  \item E.g. $n$ is the number of cities in a TSP\pause
  \item Any decision problem that can be answered with an algorithm that
    runs in polynomial time ($n^a$) on a normal computer (Turing
    machine) is said to be in class \class{P}\pause
  \item E.g. The decision problem ``Is there a path of length less that
    450 miles between Southampton and Glasgow?'' is in class
    \class{P}\pause
  \end{itemize}
\end{PauseHighLight}
\end{slide}

%%%%%%%%%%%%%%%%%%%%%%% Next Slide %%%%%%%%%%%%%%%%%%%%%%%

\begin{slide}
\section[-2]{Turing Machines}

\begin{PauseHighLight}
  \begin{itemize}
  \item  A Turing machine can be viewed as a very dumb computer which
    computes by having a number of states and reading and writing to a
    tape\pause
    \begin{center}
      \includegraphics[width=0.5\linewidth]{turingMachine.png}
    \end{center}
  \item Although dumb, it can do anything that any other computer can do
    (although it may take polynomially more time)\pause
  \end{itemize}
\end{PauseHighLight}

\end{slide}


%%%%%%%%%%%%%%%%%%%%%%% Next Slide %%%%%%%%%%%%%%%%%%%%%%%

\begin{slide}
\section{Non-Deterministic Turing Machines}

\begin{PauseHighLight}
  \begin{itemize}
  \item A non-deterministic Turing machine is a magic machine that can
    guess the answer and then use a normal Turing machine to verify the
    answer\pause
  \item We assume that it guesses right in the first go\pause
  \item No one knows how to simulate a non-deterministic Turing machine
    in polynomial time\pause
  \item We can simulate it in exponential time by trying all possible
    guesses\pause
  \end{itemize}
\end{PauseHighLight}

\end{slide}

%%%%%%%%%%%%%%%%%%%%%%% Next Slide %%%%%%%%%%%%%%%%%%%%%%%

\begin{slide}
\section{Class \class{NP}}

\begin{PauseHighLight}
  \begin{itemize}
  \item Any decision problem that can be answered with an algorithm that
    runs in polynomial time on a non-deterministic Turing machine is
    said to be in class \class{NP}\pause
  \item A whole lot of decision problems belong to class~\class{NP}\pause
  \item All decision problems with polynomial algorithms are also in class
    \class{NP}  (\class{P}$\subset$\class{NP})\pause
  \item ``\textit{Is the game of chess winnable by white?}'' is \emph{not} in class~\class{NP}\pause
  \end{itemize}
\end{PauseHighLight}

\end{slide}

%%%%%%%%%%%%%%%%%%%%%%% Next Slide %%%%%%%%%%%%%%%%%%%%%%%

\begin{slide}
\section[-0.5]{Belonging to \class{NP}}

\begin{PauseHighLight}
  \begin{itemize}
  \item For a problem to belong to class \class{NP} it must
    \begin{itemize}
    \item be a decision problem (true of false)\pause
    \item describable by some polynomial sized string in the length of
      the input $n$\pause
    \item be verifiable if true in polynomial time on a normal Turing
      machine (e.g. a computer)\pause
    \end{itemize}
  \item To be verifiable it is sufficient for the decision problem to
    have a ``\emph{witness}'' which is usually a solution that can be
    checked in polynomial time\pause
  \item E.g. in TSP the witness would be a tour which satisfies the
    condition\pause
  \end{itemize}
\end{PauseHighLight}

\end{slide}


%%%%%%%%%%%%%%%%%%%%%%% Next Slide %%%%%%%%%%%%%%%%%%%%%%%

\begin{slide}
\section{Using Non-Deterministic Turing Machines}

\begin{PauseHighLight}
  \begin{itemize}
  \item A witness is sufficient for a problem to be in \class{NP} since a
    non-deterministic Turing machine can guess the witness in one step
    and then proceed to check the witness is true in polynomial
    time\pause
  \item Thus TSP is also in NP\pause
  \item As are graph-colouring, SAT, Max-Clique, Hamilton cycle and
    countless others\pause
  \end{itemize}
\end{PauseHighLight}

\end{slide}


%%%%%%%%%%%%%%%%%%%%%%% Next Slide %%%%%%%%%%%%%%%%%%%%%%%

\begin{slide}
\section[-2]{Class \class{NP-complete}}

\begin{PauseHighLight}
  \begin{itemize}\squeeze
  \item In 1971 Cook showed that you could represent any \class{NP}
    problem on a non-deterministic Turing machine by a polynomially
    sized SAT (Satisfaction) problem\pause
  \item Thus if you could solve SAT in polynomial time you can use
    that to simulate a non-deterministic Turing machine in polynomial
    time\pause
  \item SAT is therefore an example of one of the hardest
    problems in \class{NP} since if you can solve SAT in
    polynomial time you can solve all problems in \class{NP} in
    polynomial time\pause
  \item These hardest problems in \class{NP} are said to be in
    \class{NP-complete}\pause
  \item If there existed a polynomial time algorithm for SAT then all
    problems in \class{NP} could be performed in polynomial time so that
    \class{NP}=\class{P}\pause
  \end{itemize}
\end{PauseHighLight}

\end{slide}


%%%%%%%%%%%%%%%%%%%%%%% Next Slide %%%%%%%%%%%%%%%%%%%%%%%

\begin{slide}
\section{Idea Behind Cook's Theorem}

\begin{PauseHighLight}
  \begin{itemize}
  \item Cook showed that a non-deterministic Turing machine could be
    encoded as a big SAT formula\pause
  \item The evolution of the state and tape was represented by a big
    tableau ($n^k\times n^k$-table where $n^k$ is the time it takes for
    the Turing machine to verify the answer)\pause
  \item The structure of the clauses reflect the rules the
    Turing machine operates\pause
  \item If the clauses are simultaneously satisfiable then there exists
    an input that satisfies the conditions\pause
  \end{itemize}
\end{PauseHighLight}

\end{slide}


%%%%%%%%%%%%%%%%%%%%%%% Next Slide %%%%%%%%%%%%%%%%%%%%%%%
\Outline % Polynomial Reductions
%%%%%%%%%%%%%%%%%%%%%%% Next Slide %%%%%%%%%%%%%%%%%%%%%%%

\begin{slide}
\section[-1]{Polynomial Reductions}

\begin{PauseHighLight}
  \begin{itemize}
  \item Given two decision problems A and B we say there is a
      \emph{polynomial reduction} from A to B if\pause
    \begin{itemize}
    \item Every instance of A can be mapped to an instance of B:\pause
    \item The truth of the instance A is the same as the
      corresponding instance B\pause
    \end{itemize}
  \item We can therefore use B to solve A\pause
  \item So:  B $\in$ \class{P} $\rightarrow$ A  $\in$ \class{P}\pause
  \item The contrapositive of this statement is
    \begin{quote}
      A $\not\in$ \class{P} $\rightarrow$ B  $\not\in$ \class{P}\pause
    \end{quote}
  \end{itemize}
\end{PauseHighLight}

\end{slide}

%%%%%%%%%%%%%%%%%%%%%%% Next Slide %%%%%%%%%%%%%%%%%%%%%%%

\begin{slide}
\section[-1]{SAT to 3-SAT}

\begin{PauseHighLight}
  \begin{itemize}
  \item We can reduce a clause with 4 variable to a clause with 3
    \begin{align*}
      X_1 \vee \neg X_3 \vee  X_6 \vee \neg X_{10}
      \equiv (X_1 \vee \neg X_3 \vee Z) \wedge (\neg Z \vee  X_6 \vee
      \neg X_{10})\pause
    \end{align*}
  \item In doing so we increase the number of variables and the number
    of clauses to satisfy\pause
  \item We can similarly reduce a clause with more variables
    {\small
      \begin{align*}
        X_1 \vee \neg X_3 \vee  X_6 \vee \neg X_{10} \vee \neg X_{11} 
        \vee X_{15} 
        \equiv \hspace{8cm} \\
        (X_1 \vee \neg X_3 \vee Z_1) \wedge (\neg Z_1 \vee  X_6
        \vee Z_2) \wedge (\neg Z_2 \vee \neg X_{10} \vee Z_3) 
        \wedge (\neg Z_3 \vee \neg X_{11} \vee X_{15})\pause
      \end{align*}}
  \item Because every instance of SAT can be written as a 3-SAT problem
    which is only polynomially larger than the SAT problem, 3-SAT is
    also NP-complete\pause
  \end{itemize}
\end{PauseHighLight}

\end{slide}


%%%%%%%%%%%%%%%%%%%%%%% Next Slide %%%%%%%%%%%%%%%%%%%%%%%

\begin{slide}
\section[-2]{Vertex Cover}
\pausebuild
\color{TwoColor}
\begin{itemize}
\item The vertex cover problem is: ``can we choose $K$ (9) vertices of a
    graph such that every edge is connected to a chosen vertex?''\pauseh
  \vspace*{-1cm}
  \begin{center}\pauselevel{=1, :2}\color{TextColor}
    \multiinclude[graphics={height=12cm}]{VertexCover}\pause
  \end{center}
  \vspace*{-1cm}
\color{TwoColor}
\end{itemize}

\end{slide}

%%%%%%%%%%%%%%%%%%%%%%% Next Slide %%%%%%%%%%%%%%%%%%%%%%%

\begin{slide}
\section{Vertex Cover is \class{NP-complete}}

\begin{PauseHighLight}
  \begin{itemize}
  \item Vertex cover is obviously in \class{NP} as a set of $K$ vertices
    acts as a witness, i.e. it can be checked that it covers all edges\pause
  \item To show vertex cover is \class{NP-complete} we show that every
    instance of 3-SAT is reducible to vertex cover\pause
  \item The idea is to show that any 3-SAT problem with variables
    $\{X_1,X_2, \ldots,X_n\}$ and clauses $\{c_1,c_2,\ldots, c_m\}$ can
    be encoded as a vertex cover problem\pause
  \end{itemize}
\end{PauseHighLight}

\end{slide}

%%%%%%%%%%%%%%%%%%%%%%% Next Slide %%%%%%%%%%%%%%%%%%%%%%%

\begin{slide}
\section{3-SAT to Vertex Cover}

\pb
\pause
\begin{center}
  \includegraphics[width=\linewidth]{vcsat0}\mypl{1}
  \multido{\ia=1+1,\ib=2+1}{13}{%
    \llap{\includegraphics[width=\linewidth]{vcsat\ia}\mypl{\ib}}}
\end{center}


\end{slide}


%%%%%%%%%%%%%%%%%%%%%%% Next Slide %%%%%%%%%%%%%%%%%%%%%%%

\begin{slide}
\section[-1]{Other Examples of \class{NP-complete}}

\begin{PauseHighLight}
  \begin{itemize}
  \item As we can polynomial reduce any instance of SAT to vertex cover
    then vertex cover is also \class{NP-complete}\pause
  \item Lots of problems have been shown to be in class
    \class{NP-complete}\pause---some $10\,000$, or so, to date\pause
  \item These include
    \begin{itemize}
    \item TSP\pause
    \item Graph colouring\pause
    \item Many scheduling problems\pause
    \item Bin-packing\pause
    \item Quadratic integer problems\pause
    \end{itemize}
  \end{itemize}
\end{PauseHighLight}

\end{slide}


%%%%%%%%%%%%%%%%%%%%%%% Next Slide %%%%%%%%%%%%%%%%%%%%%%%

\begin{slide}
\section{Structure of Decision Problems}

\begin{center}
  \includegraphics[height=14cm]{\figs/dp}
\end{center}

\end{slide}

%%%%%%%%%%%%%%%%%%%%%%% Next Slide %%%%%%%%%%%%%%%%%%%%%%%

\begin{slide}
\section[-2]{$P\neq NP$?}

\begin{PauseHighLight}
  \begin{itemize}
  \item No one has proved that any problem in \class{NP} is not solvable in
    polynomial time\pause
  \item If any \class{NP-complete} problem was solved in
    polynomial time all problems in NP would be solve and
    \class{NP}=\class{P}
    \begin{center}
      \includegraphics[width=0.5\linewidth]{npneqp}\pause
    \end{center}
  \end{itemize}
\end{PauseHighLight}

\end{slide}


%%%%%%%%%%%%%%%%%%%%%%% Next Slide %%%%%%%%%%%%%%%%%%%%%%%

\begin{slide}
\section[-1]{\class{NP-Hard}}

\begin{PauseHighLight}
  \begin{itemize}
  \item TSP is not a decision problem---although we can make it into
    one---Is there a tour shorter that $L$?\pause
  \item However, if we can find the shortest tour in polynomial time we
    could solve the TSP decision problem\pause
  \item Thus finding the shortest tour is at least as hard as solving the
    decision problems\pause
  \item Problems that are at least as hard as \class{NP-complete} decision
    problems are said to be in \emph{\class{NP-hard}}\pause
  \item Graph colouring (finding a colouring with the least number of
    conflicts), job scheduling, etc.\ are all examples of \class{NP-hard}
    problems\pause
  \end{itemize}
\end{PauseHighLight}

\end{slide}

%%%%%%%%%%%%%%%%%%%%%%% Next Slide %%%%%%%%%%%%%%%%%%%%%%%

\begin{slide}
\section[-0.5]{Not All Hard Problems are \class{NP-Hard}}

\begin{PauseHighLight}
  \begin{itemize}
  \item Graph isomorphism, GI, (are two graphs identical up to a relabelling
    of the vertices?) has not been proved to be NP-complete---it is
    postulated that 
    \begin{align*}
      GI \in \class{NP} \,\wedge\, GI \not\in\class{P} \,\wedge\, GI
      \not\in\class{NP-complete}\pause
    \end{align*}
  \item Factoring is \textit{not} believed to be \class{NP-hard}, but it
    is believed to be sufficiently hard that most banks use an
    encryption technique based on people not being able to factor large
    numbers easily\pause
  \item For large problems polynomial algorithms can take too
    long\pause
  \end{itemize}
\end{PauseHighLight}

\end{slide}

%%%%%%%%%%%%%%%%%%%%%%% Next Slide %%%%%%%%%%%%%%%%%%%%%%%

\begin{slide}
\section[-2]{Not All \class{NP-Hard} Problem Instances are Hard}

\begin{PauseHighLight}
  \begin{itemize}\squeeze
  \item \class{NP-hard}ness is a worst case analysis\pause
  \item It means there exist some instance of the problem that we don't
    know how to solve in polynomial time\pause
  \item Many instance of the problem might be rather easy to solve\pause
  \item What is the optimal TSP tour for the problem below?
    \begin{center}
      \multiinclude[graphics={height=8cm}]{trivialtsp}\pauseb
    \end{center}
  \end{itemize}
\end{PauseHighLight}

\end{slide}

%%%%%%%%%%%%%%%%%%%%%%% Next Slide %%%%%%%%%%%%%%%%%%%%%%%

\begin{slide}
\section[-1]{Not All \class{NP-Hard} Problems are Hard}

\begin{PauseHighLight}
  \begin{itemize}
  \item For some problems almost all instances appear easy\pause
  \item E.g. The subset-sum problem\pause
    \begin{itemize}
    \item Given a set of numbers find a subset whose sums is as close as
      possible to some constant\pause
    \item Subset-sum is in \class{NP-hard} but there exist a
      ``pseudo-polynomial time'' algorithm which solves almost every
      instance in polynomial time\pause
    \end{itemize}
  \item Many problems including subset-sum are known to be easy to
    approximate\pause
  \item For other problems even finding a good approximation is known to
    be NP-hard\pause
  \end{itemize}
\end{PauseHighLight}

\end{slide}


%%%%%%%%%%%%%%%%%%%%%%% Next Slide %%%%%%%%%%%%%%%%%%%%%%%

\begin{slide}
\section{Lessons}

\begin{PauseHighLight}
  \begin{itemize}
  \item There exist efficient algorithms for many problems\ldots\pause
  \item \ldots but probably not for all\pause
  \item There are no known polynomial algorithm for any NP-complete
    problem\pause
  \item These include many famous problems: TSP, graph-colouring,
    scheduling, \ldots\pause
  \item If you could find a polynomial algorithm for any of these
    problems then you could use it to solve all problems in NP in
    polynomial time\pause
  \end{itemize}
\end{PauseHighLight}

\end{slide}

