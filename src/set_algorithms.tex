%Master File:lectures.tex

\lesson{Use Arrays for Fast Set Algorithms}

\vspace{-2cm}
\begin{center}
  \includegraphics[height=10cm]{newMaze}
\end{center}

\keywords{Equivalent classes, Disjoint Set, Fast Sets}

%%%%%%%%%%%%%%%%%%%%%%% Next Slide %%%%%%%%%%%%%%%%%%%%%%%
\renewcommand{\Outline}{%
\begin{slide}
\section{Outline}
\vfill
\begin{minipage}{10cm}
  \vfill
  \begin{enumerate}
    \outlineitem{Equivalent Classes}{equiv}
    \outlineitem{Disjoint Sets}{disjoint}
    \outlineitem{Fast Sets}{fast}
  \end{enumerate}
  \vfill
\end{minipage}\hfill
\begin{minipage}{10cm}
  \includegraphics[width=7cm]{newMaze1}
\end{minipage}
\vfill
\end{slide}
\addtocounter{outlineitem}{1}
}
\setcounter{outlineitem}{1}

%%%%%%%%%%%%%%%%%%%%%%% Next Slide %%%%%%%%%%%%%%%%%%%%%%%
\Outline % Equivalence Classes
\toptarget{firstoutline}
%%%%%%%%%%%%%%%%%%%%%%% Next Slide %%%%%%%%%%%%%%%%%%%%%%%

\begin{slide}
\section[-2]{Equivalence Relations}
\begin{PauseHighLight}
  \begin{itemize}
  \item Given a set of elements $\mathcal{X}=\{x_1,\, x_2,\, \ldots \}$
    and a binary relationship $\sim$ with the following properties\pause
    \begin{description}
    \item[\textbf{(Reflexivity)}] For every element $x\in\mathcal{X}$,
      $x\sim x$\pause
    \item[\textbf{(Symmetry)}] For every two elements $x, y \in
      \mathcal{X}$ if $x \sim y$ then $y \sim x$\pause
    \item[\textit{(Transitivity)}] For every three elements $x, y, z \in
      \mathcal{X}$ if $x \sim y$ and $y\sim z$ then $x\sim z$\pause
    \end{description}
  \item Then $\sim$ defines a partitioning of the set into
    \emph{equivalence classes}
  \end{itemize}
  \begin{center}
    \includegraphics[width=0.4\linewidth]{partition}\pause
  \end{center}
\end{PauseHighLight}

\end{slide}

%%%%%%%%%%%%%%%%%%%%%%% Next Slide %%%%%%%%%%%%%%%%%%%%%%%

\begin{slide}
\section{Example of Equivalence Classes}

\begin{PauseHighLight}
  \begin{itemize}
  \item Although, equivalent classes sound very mathematical they
    often provide a useful formalisation of the real world\pause
  \item E.g. Pairs of web pages with a link in each direction between them\pause
  \item Consider web pages in the same equivalence class if you can
    get from one to the other by clicking links\pause
  \item Partitions the web into linked domains\pause
  \item Friendship relations in social media\pause
  \end{itemize}
\end{PauseHighLight}

\end{slide}

%%%%%%%%%%%%%%%%%%%%%%% Next Slide %%%%%%%%%%%%%%%%%%%%%%%

\begin{slide}
\section[-1]{Dynamic Equivalence Classes}

\begin{PauseHighLight}
  \begin{itemize}
  \item Finding equivalence classes is rather easy using graph traversal
    algorithms\pause
  \item However, as our web example suggests, there are applications
    where equivalence classes change over time\pause
  \item Adding a link could join two domains which were separate\pause
  \item We will see this is a useful idea both for building mazes
    and (in a later lecture) for finding minimum spanning trees\pause
  \item Building a data structure which finds equivalence classes where
    the equivalence relation changes over time is challenging\pause,
    but fortunately there is an elegant solution to this\pauseb
  \end{itemize}
\end{PauseHighLight}


\end{slide}



%%%%%%%%%%%%%%%%%%%%%%% Next Slide %%%%%%%%%%%%%%%%%%%%%%%
\Outline % Union-Find
%%%%%%%%%%%%%%%%%%%%%%% Next Slide %%%%%%%%%%%%%%%%%%%%%%%

\begin{slide}
\section{Union-Find}

\begin{PauseHighLight}
  \begin{itemize}
  \item In the union-find algorithm we have a set of objects $x \in 
    \mathcal{S}$ which are to be grouped into subsets $\mathcal{S}_1$,
    $\mathcal{S}_2$, \ldots\pause
  \item Initially each object is in its individual subset (no relationships)\pause
  \item We want to make the \emph{union} of two subsets (add
    relationship between elements)\pause
  \item We also want to \emph{find} the subset given an element\pause
  \item This is a common problem for which we will write a class
  \jl$DisjointSets$ to perform fast \texttt{union}s and \texttt{find}s\pause
  \end{itemize}
\end{PauseHighLight}

\end{slide}

%%%%%%%%%%%%%%%%%%%%%%% Next Slide %%%%%%%%%%%%%%%%%%%%%%%

\begin{slide}
\section[-1]{DisjointSets}

\begin{PauseHighLight}
  \begin{itemize}
  \item We want to create a class
    \begin{java}
public class DisjointSets
{
    public DisjointSets(int numElements) {/* Constructor */}

    public int find(int x) {/* Find root */}

    public void union(int root1, int root2) {/* Union */}

    private int[] s;
}$\pause$
    \end{java}
  \item Where \jl$find(x)$ returns a unique identifier for the subset
    which element \jl$x$ belongs to\pause
  \item The array \jl$s$ contains labelling information to implement
    \jl$find(x)$\pause
  \end{itemize}
\end{PauseHighLight}

\end{slide}

%%%%%%%%%%%%%%%%%%%%%%% Next Slide %%%%%%%%%%%%%%%%%%%%%%%

\begin{slide}
\section[-1]{The Union-Find Dilemma}

\begin{PauseHighLight}
  \begin{itemize}
  \item A natural algorithm to perform finds is to maintain an
    array returning a subset label for each element\pause---this makes
    \texttt{find} fast\pause
  \item However, every time we combine two subset we have to change all
    the labels in this array (taking $O(n)$ operations)\pause
  \item If we are unlucky the cost of performing $n$ unions is
    $\Theta(n^2)$\pause
  \item If we ensure that we relabel the smaller subset then the time
    complexity is $\Theta(n \log(n))$\pause
  \item Fast \textit{finds} seems to give slow(ish) \textit{unions}\pause
  \item What about the other way around?\pause
  \end{itemize}
\end{PauseHighLight}

\end{slide}

%%%%%%%%%%%%%%%%%%%%%%% Next Slide %%%%%%%%%%%%%%%%%%%%%%%

\begin{slide}
\section{Fast Union}

\begin{PauseHighLight}
  \begin{itemize}
  \item To achieve fast unions we can represent our disjoint sets as a
    forest (many disjoint trees)\pause
  \item Every time we perform a union we make one of the trees point to
    the head of the other tree\pause
  \item The cost of \texttt{find} depends on the depth of the tree\pause
  \item To make unions efficient we make the shallow tree a subtree of
    the deeper tree\pause
  \end{itemize}
\end{PauseHighLight}

\end{slide}


%%%%%%%%%%%%%%%%%%%%%%% Next Slide %%%%%%%%%%%%%%%%%%%%%%%

\begin{slide}
\section[-2]{Putting it Together}
\pb

\begin{center}
  \multiinclude[graphics={height=14cm}]{disjointset}\pause
\end{center}

\end{slide}

%%%%%%%%%%%%%%%%%%%%%%% Next Slide %%%%%%%%%%%%%%%%%%%%%%%

\begin{slide}
\section[-2]{Smart Union}

\begin{PauseHighLight}
\begin{java}
    public DisjointSets(int numElements)
    {
        s = new int[numElements];
        for(int i=0; i<s.length; i++)
            s[i] = -1;                   // roots are negative number
    }$\pause$

    public void union(int root1, int root2)
    {
        if (s[root2]<s[root1]) {         // root2 is deeper
            s[root1] = root2;            // make root2 the root$\pause$
        } else {
            if (s[root1]==s[root2])      
                s[root1]--;              // update height if same
            s[root2] = root1;            // make root1 new root
        }
    }$\pause$
\end{java}
\begin{center}
  \includegraphics[width=0.9\linewidth]{disjointDiag}
\end{center}
\end{PauseHighLight}
\end{slide}

%%%%%%%%%%%%%%%%%%%%%%% Next Slide %%%%%%%%%%%%%%%%%%%%%%%

\begin{slide}
\section{Path Compression}

\begin{PauseHighLight}
  \begin{itemize}
  \item To speed up \texttt{find} we relabel all nodes we visit during
    \texttt{find} by the root label\pause
    \begin{java}
    public int find(int index)
    {
        if (s[index]<0)
            return index;
        else
            return s[index] = find(s[index]);
    }$\pause$
    \end{java}
  \end{itemize}
  \begin{center}
    \includegraphics[width=0.9\linewidth]{disjointDiag1}
  \end{center}
\end{PauseHighLight}

\end{slide}

%%%%%%%%%%%%%%%%%%%%%%% Next Slide %%%%%%%%%%%%%%%%%%%%%%%

\begin{slide}
\section[-1]{Mazes}

\pb
\begin{minipage}{15cm}
  \begin{itemize}\squeeze
  \item Union-Find is a data structure which can occur in very different
    applications\pauseh
  \item One application is building a maze\pauseh
  \item Start from a complete lattice\pauseh
  \item Remove a randomly chosen edge if it connects two unconnected
    regions\pauseh
  \item Stop when the start and end cell are connected\pauseh
  \item Or better after all cells are connected\pauseh
  \end{itemize}
\end{minipage}\hfill
\begin{minipage}{8cm}
  \begin{picture}(80,150)
    \put(0,0){\includegraphics[width=8cm]{maze.0}}
    \pauselevel{=1}\pause
    \put(0,0){\includegraphics[width=8cm]{maze.1}}
    \pauselevel{=4}\pause
    \put(0,0){\includegraphics[width=8cm]{maze.2}}
    \pauselevel{=5}\pause
    \put(0,0){\includegraphics[width=8cm]{maze.3}}
    \pauselevel{=6}\pause
  \end{picture}
\end{minipage}
\end{slide}


%%%%%%%%%%%%%%%%%%%%%%% Next Slide %%%%%%%%%%%%%%%%%%%%%%%

\begin{slide}
\section{Time Complexity of Union-Find}

\pb
\begin{itemize}
\item If we perform $M$ finds and $N$ unions then the time complexity
  is $O\bigl(M\log_2^*(N)\bigr)$\pauseh
\item Where $\log_2^*(N)$ is the number of times you need to apply the
  logarithm function before you get a number less than 1\pauseh
\item In practice $\log_2^*(N)\leq 5$ for all conceivable $N$\pauseh
  \begin{align*}
    \hspace{3cm}\log_2(10^{80}) = 265.75 \mypl{4}
    \mathllap{\log_2(\log_2(10^{80})) = 8.0539\mypl{5}}
    \mathllap{\log_2(\log_2(\log_2(10^{80}))) = 3.0097\mypl{6}}
    \mathllap{\log_2(\log_2(\log_2(\log_2(10^{80})))) = 1.5896\mypl{7}}
    \mathllap{\log_2(\log_2(\log_2(\log_2(\log_2(10^{80}))))) = 0.66868\pause}
  \end{align*}
\item The proof of this time complexity is rather involved\pause
\end{itemize}

\end{slide}



%%%%%%%%%%%%%%%%%%%%%%% Next Slide %%%%%%%%%%%%%%%%%%%%%%%
\Outline % Fast Set
%%%%%%%%%%%%%%%%%%%%%%% Next Slide %%%%%%%%%%%%%%%%%%%%%%%

\begin{slide}
\section{Comparison of Sets}

\begin{PauseHighLight}
  \begin{itemize}
  \item Binary Search Trees: $O(\log_2(n))$, general purpose\pause
  \item Hash tables: $O(1)$, but need to compute hash, slow iterator
    when sparse, general purpose\pause
  \item B-trees: $O((k-1)\,\log_k(n))$ very complicated, used for large amounts
    of data\pause
  \item Tries: $O(\log_k(n))$ for large $k$ expensive in memory,
    complicated to code efficiently\pause
  \end{itemize}
\end{PauseHighLight}

\end{slide}

%%%%%%%%%%%%%%%%%%%%%%% Next Slide %%%%%%%%%%%%%%%%%%%%%%%

\begin{slide}
\section{What Set to Use?}

\begin{PauseHighLight}
  \begin{itemize}
  \item A PhD student and I were working on writing a fast solver for a
    combinatorial optimisation problem\pause
  \item We had to choose one variable to change out of a small number of
    possible variables\pause
  \item Each time we changed a variable then we had to update the list
    of possible variables (remove some variables add others)\pause
  \item We wanted a data structure which had quick add and remove and
    where we could choose a variable at random\pause---what should we
    use?\pauseb
  \end{itemize}
\end{PauseHighLight}

\end{slide}

%%%%%%%%%%%%%%%%%%%%%%% Next Slide %%%%%%%%%%%%%%%%%%%%%%%

\begin{slide}
\section{Bounded Set}

\begin{PauseHighLight}
  \begin{itemize}
  \item One special feature is that we knew we only wanted the set to
    contain integers between 0 and $n$ (where $n$ might be 100\,000)\pause
  \item This allowed us to use an array to represent whether an integer
    belong to that set\pause
  \item But how do we find a random element of the set quickly?\pause
  \item Use another array of course!\pauseb
  \end{itemize}
\end{PauseHighLight}

\end{slide}

%%%%%%%%%%%%%%%%%%%%%%% Next Slide %%%%%%%%%%%%%%%%%%%%%%%

\begin{slide}
\section{FastSet}

\pb \pause
\begin{center}
  \multipdf[width=0.9\linewidth]{fastSet}\pause
\end{center}
\end{slide}

%%%%%%%%%%%%%%%%%%%%%%% Next Slide %%%%%%%%%%%%%%%%%%%%%%%

\begin{slide}
\section{Implementation}

\pausecolors{TwoColor}{TextColor}{HighLight}%
  \color{TwoColor}\pausehighlight
\begin{java}
public class FastSet extends AbstractSet<Integer> {
    private int[] indexArray;
    private int[] memberArray;
    private int noMembers;$\pause$

    public FastSet(int n) {
        indexArray = new int[n];
        memberArray = new int[n];
        for(int i=0; i<n; i++) {
             indexArray [i] = -1;
        }
        noMembers = 0;
    }$\pause$

    public int size() {
       return noMembers;
    }$\pause$
\end{java}
\end{slide}

%%%%%%%%%%%%%%%%%%%%%%% Next Slide %%%%%%%%%%%%%%%%%%%%%%%

\begin{slide}
\section{Add and Remove}

\begin{java}
   public boolean add(int i) {
      if (indexArray[i]>-1)
         return false;
      memberArray[noMembers] = i;
      indexArray[i] = noMembers;
      ++noMembers;
      return true;
   }$\pause$

   public boolean remove(int i) {
      if (indexArray[i]==-1)
            return false;
      --noMembers;
      memberArray[indexArray[i]] = memberArray[noMembers];
      indexArray[memberArray[noMembers]] = indexArray[i];
      indexArray[i] = -1;
      return true;
   }$\pause$
\end{java}
\end{slide}

%%%%%%%%%%%%%%%%%%%%%%% Next Slide %%%%%%%%%%%%%%%%%%%%%%%

\begin{slide}
\section{Collection Methods}

\begin{java}
   public void clear() {
      for(int i=0; i<noMembers; i++) {
         indexArray[memberArray[i]] = -1;
      }
      noMembers = 0;
   }$\pause$

   public boolean isEmpty() {
      return noMembers==0;
   }$\pause$
  
   public Iterator<Integer> iterator() {
      return new FastSetIterator();
   }$\pause$
\end{java}
\end{slide}

%%%%%%%%%%%%%%%%%%%%%%% Next Slide %%%%%%%%%%%%%%%%%%%%%%%

\begin{slide}
\section[-2]{Iterator}

\begin{java}
   private class FastSetIterator implements Iterator<Integer> {
      int current = 0;$\pause$
      
      public boolean hasNext() {
         return current < noMembers;
      }$\pause$
      
      public Integer next() throws NoSuchElementException {
         if (current>=noMembers) throw new NoSuchElementException();
         current++;
         return memberArray[current-1];
      }$\pause$
      
      public void remove() throws IllegalStateException {
         if (current==0) throw new IllegalStateException();
         indexArray[memberArray[current-1]] = -1;
         noMembers--;
         memberArray[current-1] = memberArray[noMembers];
	 indexArray[memberArray[noMembers]] = current-1;
      }$\pause$
   }
\end{java}
\end{slide}

%%%%%%%%%%%%%%%%%%%%%%% Next Slide %%%%%%%%%%%%%%%%%%%%%%%

\begin{slide}
\section[-1]{And Random?}

\begin{PauseHighLight}
  \begin{itemize}
  \item So far we have just implemented a new \jl$Set<Integer>$ as part
    of the java Collection class\pause
  \item We can add additional methods taking advantage of the classes
    strength
    \begin{java}
   private static Random rand = new Random();
   
   public int getRandomElement() {
      return memberArray[rand.nextInt(noMembers)];
   }$\pause$
    \end{java}
  \item Need to use FastSet signature to use this
    \begin{java}
   FastSet fastSet = new FastSet(n);
   $\vdots$
   int r = fastSet.getRandomElement();$\pause$
    \end{java}
  \end{itemize}
\end{PauseHighLight}

\end{slide}

%%%%%%%%%%%%%%%%%%%%%%% Next Slide %%%%%%%%%%%%%%%%%%%%%%%

\begin{slide}
\section{Speed Up}

\begin{PauseHighLight}
  \begin{itemize}
  \item We compared our algorithm to a very highly regarded
    ``state-of-the-art'' algorithm\pause
  \item For large problems we were over 10 times faster because of this
    data structure\pause
  \item The competitor algorithm used a complex tree structure instead
    of the simple array\pause
  \item Why?\pause\ The array solution isn't in the books\pauseb
  \end{itemize}
\end{PauseHighLight}

\end{slide}

%%%%%%%%%%%%%%%%%%%%%%% Next Slide %%%%%%%%%%%%%%%%%%%%%%%

\begin{slide}
\section{Lessons}

\begin{PauseHighLight}
  \begin{itemize}
  \item If you have a bounded set then using an array is usually going
    to be very fast $O(1)$ (or $O(\log^*(n))$)\pause
  \item These data structures are not general purpose for solving every
    day problems (c.f. \jl$List<T>$, \jl$Set<T>$ and \jl$Map<T>$)\pause
  \item They are ``back pocket'' data structures that solve problems
    that come up often enough that they are worth knowing about\pause
  \item Sometimes good algorithms are not documented, but it doesn't
    mean they don't exist\pause
  \end{itemize}
\end{PauseHighLight}

\end{slide}
