%Master File:lectures.tex

\lesson{Point to where you are going: links}

\vspace*{-1cm}
\begin{center}
  \includegraphics[height=10cm]{pointing}
\end{center}
\keywords{Linked lists}

%%%%%%%%%%%%%%%%%%%%%%% Next Slide %%%%%%%%%%%%%%%%%%%%%%%
\renewcommand{\Outline}{%
\begin{slide}
\section{Outline}
\begin{minipage}{14cm}
  \vfill
  \begin{enumerate}
    \outlineitem{References}{intro}
    \outlineitem{Singly Linked List}{sll}
    \outlineitem{Stacks and Queues}{stacks}
    \outlineitem{Doubly Linked List}{dll}
    \outlineitem{Using Linked Lists}{use}
    \outlineitem{Skip Lists}{skip}
  \end{enumerate}
  \vfill
\end{minipage}\hfill
\begin{minipage}{8cm}
  \includegraphics[width=8cm]{pointing}
\end{minipage}
\end{slide}
\addtocounter{outlineitem}{1}
}

\setcounter{outlineitem}{1}
\Outline
\toptarget{firstoutline}

%%%%%%%%%%%%%%%%%%%%%%% Next Slide %%%%%%%%%%%%%%%%%%%%%%%

\begin{slide}
\section{Non-Contiguous Data}

\begin{PauseHighLight}
  \begin{itemize}
  \item So far we have considered arrays where the data is stored in a
    contiguous chunk of memory\pause
  \item This has the great advantage of allowing random access\pause
  \item It has the disadvantage that it is expensive to add or remove
    data from the middle of the list or to rearrange the data\pause
  \item A different approach is to use units of data that point to
    other units\pause
  \end{itemize}
\end{PauseHighLight}
\end{slide}

%%%%%%%%%%%%%%%%%%%%%%% Next Slide %%%%%%%%%%%%%%%%%%%%%%%

\begin{slide}
\section{Non-Contiguous Data Structures}

\begin{PauseHighLight}
  \begin{itemize}
  \item There are a lot of important data structures using
    non-contiguous memory
    \begin{itemize}
    \item Binary trees
    \item Graphs\pause
    \end{itemize}
  \item In this lecture we consider \emph{linked-lists}\pause
  \item This is a classic data structure\pause, which is almost entirely
    useless\pauseb
  \item However, it serves as a good introduction to much more useful
    data structures\pause
  \end{itemize}
\end{PauseHighLight}

\end{slide}


%

%%%%%%%%%%%%%%%%%%%%%%% Next Slide %%%%%%%%%%%%%%%%%%%%%%%

\begin{slide}
\section[-2]{Self-Referential Classes}

\begin{PauseHighLight}
  \begin{itemize}
  \item The building block for a linked list is a node class
    \begin{java}
      struct Node<T>
      {
        Node(U value, Node<U> *node): value(value), next(node) {}
        T element;
        Node<T> *next;
      }      
    \end{java}\pause
    \vspace{-1cm}
  \item We create new nodes\\
    \jl$Node<int> *node = new Node<int>(10, pt_to_next)$\pause
  \item Note that \jl$node$ is the address of this node\pause
  \item I make it a \texttt{struct} as this is a class where I want
    public access to the element and next\pause
  \item I can make this class a private class of my linked list\pause 
  \end{itemize}
\end{PauseHighLight}
\end{slide}


%%%%%%%%%%%%%%%%%%%%%%% Next Slide %%%%%%%%%%%%%%%%%%%%%%%
\Outline

%%%%%%%%%%%%%%%%%%%%%%% Next Slide %%%%%%%%%%%%%%%%%%%%%%%

\begin{slide}
\section[-1]{Singly Linked List}

\begin{PauseHighLight}
  \begin{itemize}
  \item We can build a linked list by stringing nodes together\pause
    \begin{center}
      \includegraphics[height=1.5cm]{linked-list}\pause
    \end{center}
    We don't show the ``pointer'' to \texttt{element}\pause
  \item A singly linked list has a single ``pointer'' to the next
    element\pause
  \item A doubly linked list has ``pointers'' to the next and previous
    element---we will see this later\pause
  \item We should be able to create a linked list, add elements, remove
    elements, see if an element exists, etc.\pause
  \end{itemize}
\end{PauseHighLight}

\end{slide}

%%%%%%%%%%%%%%%%%%%%%%% Next Slide %%%%%%%%%%%%%%%%%%%%%%%

\begin{slide}
\section{Implementation}

\begin{PauseHighLight}
  \begin{itemize}
  \item We consider a lightweight implementation\pause
  \item The class will have a head, a size counter and have a
  \texttt{Node} as a  nested class
  \begin{cpp}
class MyList {
private:
  template <typename U>
  struct Node{
    Node(U value, Node<U> *node): value(value), next(node) {}
    U value;
    Node<U> *next;
  };
  Node<T> *head;
  unsigned noElements;
  \end{cpp}\pause
  \end{itemize}
\end{PauseHighLight}
\end{slide}

%%%%%%%%%%%%%%%%%%%%%%% Next Slide %%%%%%%%%%%%%%%%%%%%%%%

\begin{slide}
\section{Simple Methods}

\vspace{-1cm}
\begin{PauseHighLight}
  \begin{itemize}
  \item The constructor is simple (and not strictly necessary)
    \begin{cpp}
  MyList(): n(0), head(0) {}
    \end{cpp}\pause \vspace{-1cm}
  \item Other simple methods are
    \begin{cpp}
  unsigned size() const {return noElements;}

  bool empty() const {
    return head == 0;
  }
    \end{cpp}\pause
  \end{itemize}\vspace{-1cm}
\end{PauseHighLight}
\end{slide}

%%%%%%%%%%%%%%%%%%%%%%% Next Slide %%%%%%%%%%%%%%%%%%%%%%%

\begin{slide}
\section{Adding elements}

\pausebuild
\begin{cpp}
    void add(T element)$\hl{3}$ 
    {$\hll$
        Node<T> *newNode = new Node<E>(); $\hl{4}$ 
        newNode.element = element; $\hl{5}$
        newNode.next = head; $\hl{6}$
        head = newNode; $\hl{7}$
        noElements++; $\hl{8}$
        return true;
    }$\hll$
\end{cpp}\pauselevel{=2}
\begin{center}
  \multiinclude[graphics={width=\linewidth}]{linked-list-add}\pause
\end{center}
\end{slide}

%%%%%%%%%%%%%%%%%%%%%%% Next Slide %%%%%%%%%%%%%%%%%%%%%%%

\begin{slide}
\section{Remove Head of List}

\vspace{-1cm}
\begin{java}$\pauselevel{build}$
    void remove_head()$\hl{3}$
    {
        Node<T>* dead = head;$\hll$
        head = head->next$\hl{4}$
        noElements--;
        delete dead;$\hl{5}$
    }$\hll$
\end{java}\pause\pauselevel{=2}
\begin{center}
  \multiinclude[graphics={width=\linewidth}]{linked-list-remove}\pause
\end{center}
\vspace{-1cm}

\end{slide}


%%%%%%%%%%%%%%%%%%%%%%% Next Slide %%%%%%%%%%%%%%%%%%%%%%%

\begin{slide}
\section{Other Methods}

\begin{PauseHighLight}
  \begin{itemize}
  \item We can easily implement many other methods\pause
    \begin{itemize}
    \item \texttt{get(int i)}---return $i^{th}$ item in list\pause
    \item \texttt{remove(T obj)}-remove \texttt{obj} from list\pause
    \item \texttt{insert(int position, element)}
    \end{itemize}
  \item Note that \texttt{get(int i)} requires moving down the list so
    is $O(n)$ (i.e. not random access)\pause
  \end{itemize}
\end{PauseHighLight}
\end{slide}



%%%%%%%%%%%%%%%%%%%%%%% Next Slide %%%%%%%%%%%%%%%%%%%%%%%
\Outline

%%%%%%%%%%%%%%%%%%%%%%% Next Slide %%%%%%%%%%%%%%%%%%%%%%%

\begin{slide}
\section[-2]{Stack}

\begin{PauseHighLight}
  \begin{itemize}
  \item It is easy to implement a stack using a linked list\pause
    \begin{java}
template <typename T>
class Stack<E>
{
  private Mylist<T> list = new mylist<T>();

  boolean push(E obj) {list.add(obj);}
  
  E top() {return list.get_head();} // throw exception

  E pop() {
    T tmp = list.get_head();
    list.remove_head();
    return tmp;
  }

  boolean empty() {return list.empty();}
}
    \end{java}
  \end{itemize}
\end{PauseHighLight}

\end{slide}

%%%%%%%%%%%%%%%%%%%%%%% Next Slide %%%%%%%%%%%%%%%%%%%%%%%

\begin{slide}
\section{Complexity of Stack}

\begin{PauseHighLight}
  \begin{itemize}
  \item All operations of the stack is constant time, i.e. $O(1)$\pause
  \item This is the same time complexity as an array
    implementation\pause
  \item Memory requirement is approximately $2\times n$ reference and
    $n$ objects\pause---same as worst case for an array\pause
  \item However, hidden cost of creating and destroying \texttt{Node}
    objects\pause
  \item The array implementation is therefore slightly faster\pause
  \end{itemize}
\end{PauseHighLight}

\end{slide}


%%%%%%%%%%%%%%%%%%%%%%% Next Slide %%%%%%%%%%%%%%%%%%%%%%%

\begin{slide}
\section{Point to the Back}

\vspace{-1.5cm}
\begin{PauseHighLight}
  \begin{itemize}
  \item To find the end of the queue takes $n$ jumps\pause
  \item Thus our linked list isn't the right data structure to
    implement a queue\pause
  \item However, we could include a pointer to the end of the queue
    \begin{center}
      \includegraphics[height=6cm]{linked-list2}\pause
    \end{center}
  \end{itemize}
\end{PauseHighLight}
\end{slide}

%%%%%%%%%%%%%%%%%%%%%%% Next Slide %%%%%%%%%%%%%%%%%%%%%%%

\begin{slide}
\section{Implementing a Queue}

\begin{PauseHighLight}
  \begin{itemize}
  \item We can then add elements to the tail in constant time\pause
  \item We can the implement a queue in $O(1)$ time by
    \begin{itemize}
    \item \texttt{enqueue}ing at the back\pause
    \item \texttt{dequeue}ing at the head\pause
    \end{itemize}
  \item I leave the implementation details as an exercise for you\pause
  \item Note that although adding an element to the tail is constant
    time, removing an element from the tail is $O(n)$ as we have to find
    the new tail\pause
  \end{itemize}
\end{PauseHighLight}
\end{slide}

% %%%%%%%%%%%%%%%%%%%%%%% Next Slide %%%%%%%%%%%%%%%%%%%%%%%

% \begin{slide}
% \section[-2]{Stacks and Queues}
% \pb
% \begin{itemize}
% \item Stack\pause
%   \begin{center}
%     \includegraphics[width=0.8\linewidth]{llstack0}\mypl{1}
%     \multido{\ia=1+1,\ib=2+1}{15}{%
%       \llap{\includegraphics[width=0.8\linewidth]{llstack\ia}}\mypl{\ib}}
%   \end{center}
% \item Queue\pauselevel{build =1 40, highlight =17}\pause
%   \begin{center}
%     \includegraphics[width=0.8\linewidth]{llqueue0}\mypl{17}
%     \multido{\ia=1+1,\ib=18+1}{15}{%
%       \llap{\includegraphics[width=0.8\linewidth]{llqueue\ia}}\mypl{\ib}}
%   \end{center}
% \end{itemize}
% \end{slide}



%%%%%%%%%%%%%%%%%%%%%%% Next Slide %%%%%%%%%%%%%%%%%%%%%%%
\Outline

%%%%%%%%%%%%%%%%%%%%%%% Next Slide %%%%%%%%%%%%%%%%%%%%%%%

\begin{slide}
\section{Doubly linked list}

\vspace{-1.5cm}
\begin{PauseHighLight}
\begin{itemize}\squeeze
\item In a more powerful linked list we would like to navigate the
  list in either direction\pause
\item To achieve this it uses a doubly-linked lists with elements to
  next and previous
\begin{java}
  class Node<T>
  {
    T element;
    Node<T> *next;
    Node<T> *previous;
  }
\end{java}\pause
\vspace{-2cm}
\begin{center}
  \includegraphics[width=\linewidth]{double-entry}\pause
\end{center}
\vspace{-1cm}
\end{itemize}
\end{PauseHighLight}
\end{slide}

%%%%%%%%%%%%%%%%%%%%%%% Next Slide %%%%%%%%%%%%%%%%%%%%%%%

\begin{slide}
\section{Dummy Node}

\begin{PauseHighLight}
  \begin{itemize}
  \item List includes a dummy node\pause---this make the
    implementations slicker\pause
    \begin{center}
      \includegraphics[width=\linewidth]{doubly-linked-list}\pause
    \end{center}
  \item Symmetric data structure so processing head and tail is equally
    efficient\pause
  \end{itemize}
\end{PauseHighLight}
\end{slide}


%%%%%%%%%%%%%%%%%%%%%%% Next Slide %%%%%%%%%%%%%%%%%%%%%%%

\begin{slide}
\section{Time Complexity}

\pausebuild
\color{TwoColor}
\begin{itemize}
\item \texttt{add} and \texttt{remove} from head and tail $O(1)$\pauseh
\item \texttt{find} $O(n)$ and slow\pauseh
\item \texttt{insert} and \texttt{delete} $O(1)$ (faster than an array
  list) once position is found\pauseh
\end{itemize}\color{TextColor}
\begin{center}
  \multiinclude[graphics={width=\linewidth}]{doubly-linked-list-insert}\pause
\end{center}
\end{slide}


%%%%%%%%%%%%%%%%%%%%%%% Next Slide %%%%%%%%%%%%%%%%%%%%%%%
\Outline



%%%%%%%%%%%%%%%%%%%%%%% Next Slide %%%%%%%%%%%%%%%%%%%%%%%

\begin{slide}
\section{When To Use Linked Lists}

\begin{PauseHighLight}
  \begin{itemize}
  \item It is difficult to think of applications where linked lists are
    the best data structure\pause
  \item lists---variable length arrays are usually better\pause
  \item queues---linked list OK, but circular arrays are probably better\pause
  \item sorted lists---binary trees much better\pause
  \item linked lists have efficient insertion and deletion but it is
    difficult to think of an application where this matters\pause
  \end{itemize}
\end{PauseHighLight}
\end{slide}

%%%%%%%%%%%%%%%%%%%%%%% Next Slide %%%%%%%%%%%%%%%%%%%%%%%

\begin{slide}
\section{Line Editor}

\begin{PauseHighLight}
  \begin{itemize}
  \item One application where efficient insertion and deletion matters
    is a line editor\pause
  \item We are usually working at a particular location in the
    text\pause
  \item We often want to add or delete whole lines\pause
  \item Storing the lines as strings in a linked list would allow a
    fairly efficient implementation\pause
  \end{itemize}
\end{PauseHighLight}
\end{slide}

%%%%%%%%%%%%%%%%%%%%%%% Next Slide %%%%%%%%%%%%%%%%%%%%%%%
\Outline
%%%%%%%%%%%%%%%%%%%%%%% Next Slide %%%%%%%%%%%%%%%%%%%%%%%

\begin{slide}
\section{Skip Lists}
\pb

\begin{itemize}
\item Linked lists have the disadvantage that to get to anywhere in
  the list takes on average $\Theta(n)$ steps\pauseh
\item Even if you kept an ordered list you still need to traverse it\pauseh
\item Skip lists are hierarchies of linked lists which allow binary
  search
  \begin{center}
    \includegraphics[width=0.9\linewidth]{skiplist0}\pauselevel{=1
      , highlight =3 :3}\pause
    \llap{\includegraphics[width=0.9\linewidth]{skiplist1}}\mypl{4}
    \llap{\includegraphics[width=0.9\linewidth]{skiplist2}}\mypl{5}
  \end{center}
\end{itemize}


\end{slide}

%%%%%%%%%%%%%%%%%%%%%%% Next Slide %%%%%%%%%%%%%%%%%%%%%%%

\begin{slide}
\section{Efficiency of Skip Lists}

\begin{PauseHighLight}
  \begin{itemize}
  \item Skip lists provide $\Theta(\log_2(n))$ search as opposed to
    $\Theta(n)$\pause
  \item They have the similar time complexity to binary trees, although
    binary trees are slightly faster\pause
  \item They have one advantage over binary trees---they allow efficient concurrent
    access\pause
  \item The standard template library provides a doubly linked list,
    \jl$list<T>$\pause, as well as a slingly linked-list \jl$slist<T>$\pauseb
  \end{itemize}
\end{PauseHighLight}

\end{slide}


%%%%%%%%%%%%%%%%%%%%%%% Next Slide %%%%%%%%%%%%%%%%%%%%%%%

\begin{slide}
\section{Lessons}

\begin{PauseHighLight}
  \begin{itemize}
  \item Node structures that point to other Node structures are used
    in many important data structures\pause
  \item Linked lists are the simplest examples of this kind of structure
    and consequently has a dominant position in most DSA books\pause
  \item In practice linked lists are seldom the data structure of
    choice\pause---before choosing to use a linked list consider the
    alternatives\pause
  \item There are some important uses for linked lists, e.g. skip lists
    and hash tables (see lecture on hashing)\pause
  \end{itemize}
\end{PauseHighLight}
\end{slide}
