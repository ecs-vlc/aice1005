%Master File:lectures.tex
\renewcommand{\class}[1]{\textsf{#1}}

\lesson{Solving Linear Programs}

\vspace{-2cm}
\begin{center}
  \includegraphics[height=10cm]{Simplex_description.png}
\end{center}
\keywords{linear programming, simplex methods, iterative search}
%%%%%%%%%%%%%%%%%%%%%%% Next Slide %%%%%%%%%%%%%%%%%%%%%%%
\renewcommand{\Outline}{%
\begin{slide}
\section[1]{Outline}

\begin{minipage}{12cm}
  \begin{enumerate}\squeeze
    \outlineitem{Recap}{recap}
    \outlineitem{Basic Feasible Solutions}{bfs}
    \outlineitem{Simplex Method}{simplexMethod}
    \outlineitem{Classic LP Problems}{lpprobs}
   \end{enumerate}
\end{minipage}\hfill
\begin{minipage}{10cm}
  \includegraphics[width=10cm]{Simplex_description.png}
\end{minipage}
\end{slide}
\addtocounter{outlineitem}{1}
}

\setcounter{outlineitem}{1}

%%%%%%%%%%%%%%%%%%%%%%% Next Slide %%%%%%%%%%%%%%%%%%%%%%%
\Outline %
\toptarget{firstoutline}
%%%%%%%%%%%%%%%%%%%%%%% Next Slide %%%%%%%%%%%%%%%%%%%%%%%

\begin{slide}
\section{Recap}

\begin{PauseHighLight}
  \begin{itemize}
  \item Linear programs are problems that can be formulated as follows
    \begin{align*}
      &\min_{\bm{x}} \bm{c}\cdot \bm{x} \\
      \text{subject to}&\\
      &\mat{A}^{\scriptscriptstyle \leq} \bm{x} \leq
      \bm{b}^{\scriptscriptstyle \leq},  \quad
      \mat{A}^{\scriptscriptstyle \geq} 
      \bm{x} \geq \bm{b}^{\scriptscriptstyle \geq}, \quad 
      \mat{A}^{\scriptscriptstyle =} \bm{x} = \bm{b}^{\scriptscriptstyle
        =}, \quad \bm{x} \geq \bm{0}\pause 
    \end{align*}
  \item Where $\bm{x} = (x_1,\,x_2,\,\ldots,\,x_n)$\pause
  \item $\mat{A}^{*}$ are matrices and we interpret the inequalities to
    mean
    \begin{align*}
      \forall k \quad\quad \sum_{j=1}^n A_{kj}^{\scriptscriptstyle \leq}
      x_j \leq  b^{\scriptscriptstyle \leq}_k\pause
    \end{align*}
  \end{itemize}
\end{PauseHighLight}
\end{slide}

%%%%%%%%%%%%%%%%%%%%%%% Next Slide %%%%%%%%%%%%%%%%%%%%%%%

\begin{slide}
\section{Optima and Vertices}

\begin{PauseHighLight}
  \begin{itemize}
  \item Because the objective function is linear ($\bm{c}\cdot \bm{x}$)
    there is a direction where the objective is always improving\pause
  \item Thus, the optima cannot lie in the interior of the search space\pause
  \item When we meet a constraint that limits the direction we can move,
    but we can still move along the constraint\pause
  \item We then meet another constraint which restricts the direction we
    can move by two degrees of freedom\pause
  \item Eventually, we will reach $n$ constraints which defines a vertex
    of the feasible region and is optimal\pause
  \end{itemize}
\end{PauseHighLight}

\end{slide}


%%%%%%%%%%%%%%%%%%%%%%% Next Slide %%%%%%%%%%%%%%%%%%%%%%%

\begin{slide}
\section{Transforming Linear Programs}

\begin{PauseHighLight}
  \begin{itemize}
  \item We can always transform an inequality constraint into an
    equality constraint by adding slack variables
  \item E.g.
    \begin{align*}
      \bm{a}_1\cdot\bm{x} &\geq 0 &\Rightarrow& &
      \bm{a}_1\cdot\bm{x} - z_1 &=0 \quad z_1 \geq 0 \\ 
      \bm{a}_2\cdot\bm{x} &\leq 0 &\Rightarrow& &
      \bm{a}_2\cdot\bm{x} + z_2 &=0 \quad z_2 \geq 0
    \end{align*}
    $z_1$ (the excess) and $z_2$ (the deficit) are known as slack
    variables\pause
  \item A linear program with just equality constraints is said to be in
    normal form\pause
  \end{itemize}
\end{PauseHighLight}

\end{slide}

%%%%%%%%%%%%%%%%%%%%%%% Next Slide %%%%%%%%%%%%%%%%%%%%%%%
\Outline % Basic Feasible Solutions
%%%%%%%%%%%%%%%%%%%%%%% Next Slide %%%%%%%%%%%%%%%%%%%%%%%

\begin{slide}
\section[-2]{Basic Feasible Solution}

\begin{PauseHighLight}
  \begin{itemize}
  \item A \textit{basic feasible solution} or \textit{basic feasible
      point} is a solution that lies at a vertex of the feasible space\pause
  \item To solve a linear program we will start at a basic feasible
    point and move to the neighbour which best improves the objective
    function\pause
  \item When we cannot find a better solution we are at the optimal
    solution\pause
  \item This is an example of an iterative improvement algorithm which
    gives an optimal solution\pause
  \end{itemize}
\end{PauseHighLight}

\end{slide}

%%%%%%%%%%%%%%%%%%%%%%% Next Slide %%%%%%%%%%%%%%%%%%%%%%%

\begin{slide}
\section[-2]{Constraints}

\begin{PauseHighLight}
  \begin{itemize}
  \item There are two types of constraints
    \begin{enumerate}
    \item $n$ non-negativity constraints $x_i>0$
    \item $m$ additional constraints, which we can take to be equalities
      $\mat{A}\bm{x} = \bm{b}$\pause
    \end{enumerate}
  \item Note that some of the variables might be slack variables\pause
  \item We consider the case when there are more variables than
    additional constraints, i.e. $n>m$\pause
  \item This is usually be the case, but\ldots\pause
  \item If this isn't true it turns out you can consider an equivalent
    problem (dual problem) where you have a variable for each constraint
    and a constraint for each variable\pause
  \end{itemize}
\end{PauseHighLight}

\end{slide}

%%%%%%%%%%%%%%%%%%%%%%% Next Slide %%%%%%%%%%%%%%%%%%%%%%%

\begin{slide}
\section{Basic Variable}

\begin{PauseHighLight}
  \begin{itemize}
  \item In total we have $n+m$ constraints\pause
  \item $n$ constraints must be satisfied to be at a vertex of feasible
    region\pause
  \item So at least $n-m$ of the non-negativity constraints are
    satisfied (i.e. $x_i=0$)\pause
  \item The $n-m$ variables that are zero are said to be \emph{non-basic
    variables}\pause
  \item The other $m$ variables are said to be \emph{basic variables}\pause
  \end{itemize}
\end{PauseHighLight}

\end{slide}

%%%%%%%%%%%%%%%%%%%%%%% Next Slide %%%%%%%%%%%%%%%%%%%%%%%

\begin{slide}
\section[-2]{Initial Basic Feasible Solution}

\begin{PauseHighLight}
  \begin{itemize}
  \item One of the tricky bits of tackling a linear program is to find
    an initial feasible solution\pause
  \item We do this in \emph{phase one} of the simplex program\pause
  \item To do this for each additional constraint we add a new
    \emph{auxiliary variable} $\xi_k$, e.g.
    \begin{align*}
     \forall\, k\in\{1,2,\ldots, m\} \quad\quad
     \xi_k + \sum_i A_{ki} x_i &= b_k  \geq 0 \pause
    \end{align*}
  \item We then can find a basic feasible solution by setting $x_i=0$ so
    \begin{align*}
      \xi_k &= b_k  \quad \forall\, k\in\{1,2,\ldots, m\}\pause
    \end{align*}
  \end{itemize}
\end{PauseHighLight}

\end{slide}

%%%%%%%%%%%%%%%%%%%%%%% Next Slide %%%%%%%%%%%%%%%%%%%%%%%

\begin{slide}
\section{Eliminating Auxiliary Variables}

\begin{PauseHighLight}
  \begin{itemize}
  \item In phase one we run a simplex algorithm with an auxiliary cost
    function
    \begin{align*}
      \min f_{\text{\tiny aux}}(\bm{x},\bm{\xi}) = \sum_{k=1}^m \xi_k \pause
    \end{align*}
  \item This should find a solution where all the $\xi_k=0$\pause
  \item If no solution exists it means there is no feasible solution and
    we're finished\pause
  \item If there is a solution then we can eliminate the auxiliary
    variables and we have a feasible solution\pause
  \end{itemize}
\end{PauseHighLight}

\end{slide}

%%%%%%%%%%%%%%%%%%%%%%% Next Slide %%%%%%%%%%%%%%%%%%%%%%%
\Outline % Simplex Method
%%%%%%%%%%%%%%%%%%%%%%% Next Slide %%%%%%%%%%%%%%%%%%%%%%%

\begin{slide}
\section{Phase Two}

\begin{PauseHighLight}
  \begin{itemize}
  \item In phase two we now have an initial basic feasible solution
    (with $n-m$ zero variables)\pause
  \item We then run the simplex algorithm on the original objective
    function $f(\bm{x}) = \bm{c} \cdot \bm{x}$\pause
  \item That is we move to a neighbouring vertex which gives the best
    increase in the objective function\pause
  \item To help organise this search we write the objective function and
    constraints in a \emph{restricted normal form} and then build a
    \emph{tableau} showing the basic variables and the non-basic
    variables\pause
  \end{itemize}
\end{PauseHighLight}

\end{slide}

%%%%%%%%%%%%%%%%%%%%%%% Next Slide %%%%%%%%%%%%%%%%%%%%%%%

\begin{slide}
\section{Restricted Normal Form}

\begin{PauseHighLight}
  \begin{itemize}
  \item To perform the moves between vertices it helps to represent the
    problem in a \emph{restricted normal form}\pause
  \item Starting from a basic feasible point we have a constraint for
    each basic (non-zero) variable\pause
  \item We write the constraints as an equality between basic and
    non-basic (zero valued) variables\pause
  \item Similarly we write the objective function in terms of non-basic
    variables\pause
  \item This is always possible as we can use the constraints to
    eliminate the basic variables\pause
  \end{itemize}
\end{PauseHighLight}

\end{slide}

%%%%%%%%%%%%%%%%%%%%%%% Next Slide %%%%%%%%%%%%%%%%%%%%%%%

\begin{slide}
\section[-2]{Tableau}

\pb
\pause \pauselevel{=1}
\begin{center}
  \multipdf[width=\linewidth]{simplexAlg}\pause
\end{center}
\end{slide}

%%%%%%%%%%%%%%%%%%%%%%% Next Slide %%%%%%%%%%%%%%%%%%%%%%%

\begin{slide}
\section{Awkward Problems}

\begin{PauseHighLight}
  \begin{itemize}
  \item If there are any column with all entries positive then this
    variable can be increase forever---this is a signal that the linear
    programming problem is unbounded\pause
  \item You can also find that a basic variable becomes zero---this is
    known as a degenerate feasible vector\pause
  \item It can by removed by exchanging variables on the left of the
    inequality with variables on the right\pause
  \item This makes the algorithm a bit more complex to implement\pause
  \end{itemize}
\end{PauseHighLight}

\end{slide}
%%%%%%%%%%%%%%%%%%%%%%% Next Slide %%%%%%%%%%%%%%%%%%%%%%%

\begin{slide}
\section[-1]{High Performance Solvers}

\begin{PauseHighLight}
  \begin{itemize}
  \item Although the tableau method is the ``classic solver'' it doesn't
    cut the mustard for large scale problems\pause
  \item The simplex update can also be viewed as solving a linear set of
    equations which is facilitated by performing an LU-decomposition\pause
  \item However, the constraints are often very sparse so good solvers
    try to take advantage of the sparsity\pause
  \item Top end simplex algorithms are rather complex\pause
  \item There is a second approach known as the interior point method
    which is competitive on large problems\pause
  \end{itemize}
\end{PauseHighLight}

\end{slide}

%%%%%%%%%%%%%%%%%%%%%%% Next Slide %%%%%%%%%%%%%%%%%%%%%%%

\begin{slide}
\section{Time Complexity of Simplex}

\begin{PauseHighLight}
  \begin{itemize}
  \item The time complexity of the updates is $O(n^2)$\pause
  \item The critical question is how many updates are necessary\pause
  \item It turns out that typically this is $O(n)$ making the simplex
    algorithm $O(n^3)$\pause
  \item However, it is possible to cook up problems where there is a
    ``long path'' from the initial solution to the optimum which is
    exponentially big\pause
  \item Thus the worst case time is exponential, although this almost
    never happens in practice\pause
  \end{itemize}
\end{PauseHighLight}

\end{slide}

%%%%%%%%%%%%%%%%%%%%%%% Next Slide %%%%%%%%%%%%%%%%%%%%%%%

\begin{slide}
\section[-2]{Interior Point Method}

\begin{PauseHighLight}
  \begin{itemize}
  \item An alternative to the simplex method is the interior point
    method which always remains in the feasible region, away from the
    constraints\pause
  \item These method iterate towards the constraints and are provably
    polynomial\pause
  \item For small linear programming problems they are out-performed in
    practice by the simplex method\pause
  \item On large and very large problems they seem to perform as well if
    not better than the simplex method\pause
  \item The high-end solvers will have a variety of interior point
    methods tailored to the particular problem\pause
  \end{itemize}
\end{PauseHighLight}

\end{slide}

%%%%%%%%%%%%%%%%%%%%%%% Next Slide %%%%%%%%%%%%%%%%%%%%%%%
\Outline % Classic LP Problems
%%%%%%%%%%%%%%%%%%%%%%% Next Slide %%%%%%%%%%%%%%%%%%%%%%%

\begin{slide}
\section{LP Problems}

\begin{PauseHighLight}
  \begin{itemize}
  \item Any problem that can be set up as a linear program can be solved
    in polynomial time\pause
  \item One way is just to feed it to a LP-solver\pause
  \item Sometimes the problems are important enough and have such a
    distinctive formulation that faster specialised algorithms have been
    developed\pause
  \item We consider a couple of classic problems: \textit{maximum flow}
    and \textit{linear assignment}\pause
  \end{itemize}
\end{PauseHighLight}

\end{slide}

%%%%%%%%%%%%%%%%%%%%%%% Next Slide %%%%%%%%%%%%%%%%%%%%%%%

\begin{slide}
\section{Maximum Flow}

\begin{PauseHighLight}
  \begin{itemize}
  \item In maximum flow we consider a directed graph representing a
    network of pipes\pause
  \item We choose one vertex as the source and a second vertex as a
    sink\pause
  \item Each edge has a flow capacity that cannot be exceeded\pause
  \item The problem is to maximise the flow between source of sink\pause
  \item This can be used to model the flow of a fluid, parts in an
    assembly line, current in an electrical circuit or packets through a
    communication network\pause
  \end{itemize}
\end{PauseHighLight}

\end{slide}

%%%%%%%%%%%%%%%%%%%%%%% Next Slide %%%%%%%%%%%%%%%%%%%%%%%

\begin{slide}
\section[-2]{Example}

\begin{PauseHighLight}
  \begin{itemize}
  \item Consider a firm that has to ship haggis from Edinburgh to
    Southampton\pause
  \item The shipping firm transports this in crates which it sends
    through intermediate cities\pause
  \item The number of crates is limited by the size of the lorries it
    uses\pause
    \begin{center}
      \includegraphics[width=0.6\linewidth]{maxFlow}
    \end{center}
  \end{itemize}
\end{PauseHighLight}

\end{slide}


%%%%%%%%%%%%%%%%%%%%%%% Next Slide %%%%%%%%%%%%%%%%%%%%%%%

\begin{slide}
\section[-1]{Flow}

\begin{PauseHighLight}
  \begin{itemize}
  \item We are given a directed graph $\mathcal{G} = (\mathcal{V},
    \mathcal{E})$ where each edge has a capacity $c(i,j)$\pause
  \item We define the flow from $i$ to $j$ as $f(i,j)$ with $0 \leq
    f(i,j) \leq c(i,j)$\pause
  \item For all vertices except the source ($s$) and sink ($t$) we assume
    \begin{align*}
      \forall i \in \mathcal{V}/\{s,t\} \quad\quad
    \sum_{j\in\mathcal{V}| (i,j)\in\mathcal{E}} f(i, j) = 
    \sum_{j\in\mathcal{V}| (i,j)\in\mathcal{E}} f(j, i)
    \end{align*}
    (i.e. no flow is lost from source to sink)\pause
  \item We want to maximise the flow from the source
    \begin{align*}
      \sum_{i\in\mathcal{V}| (s,i)\in\mathcal{E}} f(s,i)\pause
    \end{align*}
  \end{itemize}
\end{PauseHighLight}

\end{slide}

%%%%%%%%%%%%%%%%%%%%%%% Next Slide %%%%%%%%%%%%%%%%%%%%%%%

\begin{slide}
\section[-2]{Solving Maximum Flow}

\begin{PauseHighLight}
  \begin{itemize}
  \item As set up we have a linear objective function with linear
    constraints\pause
  \item We can therefore solve this problem with a LP-solver\pause
  \item (Note the solution will typically involve a fraction flow)\pause
  \item However, this is such a classic problem with a distinctive
    structure that we can solve it more quickly with other
    algorithms\pause
  \item The classic algorithm is the Ford-Fulkerson method with run time
    $O(|\mathcal{E}| \times f_{\max})$ where $f_{\max}$ is the maximum
    flow, although we
    won't cover this in the course\pause
  \end{itemize}
\end{PauseHighLight}

\end{slide}

%%%%%%%%%%%%%%%%%%%%%%% Next Slide %%%%%%%%%%%%%%%%%%%%%%%

\begin{slide}
\section[-1]{Linear Assignment}

\begin{PauseHighLight}
  \begin{minipage}{0.7\linewidth}
    \begin{itemize}
    \item We are given a set of $n$ agents, $\mathcal{A}$, and $n$ tasks,
      $\mathcal{T}$\pause
    \item Each agent has a cost associated with performing a task
      $c(a,t)$\pause
    \item We want to assign an agent to one task so as to minimise the
      total cost\pause
    \item Consider a taxi firm with taxi's at 5 different locations and 5
      requests to fulfil.  The cost is the distance to the clients.  Which
      taxi should go to which client?\pause
    \end{itemize}
  \end{minipage}\hfill
  \begin{minipage}{0.25\linewidth}
    \includegraphics[width=0.9\linewidth]{linear_assignment}
  \end{minipage}
\end{PauseHighLight}

\end{slide}

%%%%%%%%%%%%%%%%%%%%%%% Next Slide %%%%%%%%%%%%%%%%%%%%%%%

\begin{slide}
\section[-1]{LA as LP}

\begin{PauseHighLight}
  \begin{itemize}
  \item The linear assignment problem can be set as a linear programming
    problem
    \begin{align*}
      \min_{\bm{x}} \sum_{a\in\mathcal{A}, t\in\mathcal{T}} c(a,t)
      x_{a,t}\\
      \text{subject to}\\
      \forall a\in\mathcal{A} \quad \quad \sum_{t\in\mathcal{T}} x_{a,t}
      = 1 \\
      \forall  t\in\mathcal{T} \quad \quad \sum_{a\in\mathcal{A}} x_{a,t}
      = 1 \\
      \forall (a,t)\in(\mathcal{A},\mathcal{T}) \quad \quad x_{a,t}\geq 0\pause
    \end{align*}
  \end{itemize}
\end{PauseHighLight}

\end{slide}

%%%%%%%%%%%%%%%%%%%%%%% Next Slide %%%%%%%%%%%%%%%%%%%%%%%

\begin{slide}
\section{Hungarian Algorithm}

\begin{PauseHighLight}
  \begin{itemize}
  \item Linear assignment is another classic problem that is commonly
    encountered\pause
  \item Although it can be solved using a generic LP-solver this is not
    the most efficient algorithm\pause
  \item The most efficient algorithm is the Hungarian algorithms\pause
  \item This is rather complex (having once implemented it I can tell
    you from bitter experience it ain't easy)\pause
  \item Its worst case time is $O(n^3)$ although it frequently takes
    $\Theta(n^2)$\pause
  \end{itemize}
\end{PauseHighLight}

\end{slide}
%%%%%%%%%%%%%%%%%%%%%%% Next Slide %%%%%%%%%%%%%%%%%%%%%%%

\begin{slide}
\section{Quadratic Programming}

\begin{PauseHighLight}
  \begin{itemize}
  \item If we have linear constraints and a quadratic objective function
    then we have a quadratic programming problem\pause
  \item Again this can be solved in polynomial time\pause
  \item Many of the ideas used are the same as for linear programming\pause
  \item This also has important applications in science and
    engineering\pause
  \end{itemize}
\end{PauseHighLight}

\end{slide}

%%%%%%%%%%%%%%%%%%%%%%% Next Slide %%%%%%%%%%%%%%%%%%%%%%%

\begin{slide}
\section{Lessons}

\begin{PauseHighLight}
  \begin{itemize}
  \item Linear programming is a classic problem\pause
  \item We know a huge number of problems are solvable in polynomial
    time because they can be formulated as linear programs\pause
  \item Linear programs occur sufficiently often that they are hugely
    important\pause
  \item They aren't easy to solve, although standard simplex is not
    massively complex\pause
  \item For particular LP problems with distinctive structure there are
    sometimes better algorithms than generic LP-solvers\pause
  \end{itemize}
\end{PauseHighLight}

\end{slide}
