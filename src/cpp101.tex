%Master File:lectures.tex

\lesson{C++ 101}
\vspace*{-2cm}
\begin{center}
    \includegraphics[height=11cm]{Cpp}
\end{center}
\vspace*{-1cm}
\keywords{C with classes, new, overloading, templates}

%%%%%%%%%%%%%%%%%%%%%%% Next Slide %%%%%%%%%%%%%%%%%%%%%%%

\renewcommand{\Outline}{%
\begin{slide}
\section{Outline}
\begin{minipage}{10cm}
  \vfill
  \begin{enumerate}\raggedright
    \outlineitem{C with Classes}{classes}
    \outlineitem{New}{new}
    \outlineitem{Overloading}{overloading}
    \outlineitem{Templates}{templates}
  \end{enumerate}
  \vfill
\end{minipage}\hfill
\begin{minipage}{10cm}
  \includegraphics[width=10cm]{Cpp}
\end{minipage}
\end{slide}
\addtocounter{outlineitem}{1}
}


\setcounter{outlineitem}{1}
\Outline
\toptarget{firstoutline}

%%%%%%%%%%%%%%%%%%%%%%% Next Slide %%%%%%%%%%%%%%%%%%%%%%%

\begin{slide}
\section{C}

\begin{PauseHighLight}
  \begin{itemize}
  \item C was developed in the 1970s by Dennis Ritchie for writing
    UNIX tools\pause
  \item It supported structural programming through functions\pause
  \item It allowed run-time allocation of memory (through
    \texttt{malloc} and \texttt{free})\pause
  \item It allowed manipulation of memory through pointers\pause
  \item This made it efficient\pause, but not safe\pauseb{} or easy to
    use\pauseb
  \end{itemize}
\end{PauseHighLight}

\end{slide}

%%%%%%%%%%%%%%%%%%%%%%% Next Slide %%%%%%%%%%%%%%%%%%%%%%%

\begin{slide}
\section[-2]{Keeping Things Together}

\begin{PauseHighLight}
  \begin{itemize}
  \item As soon as you start programming bigger systems you want to
    keep information together\pause
  \item C facilitated this through C structures \texttt{struct}
    \begin{cpp}
      struct MyStructure {   // $Structure declaration$
        int myNum;           // $Member (int variable)$
        char myLetter;       // $Member (char variable)$
      }; // $End the structure with a semicolon$

      int main() {
        struct myStructure s1;

        s1.myNum = 13;
        s1.myLetter = 'B';

        printf("My number: %d\n", s1.myNum);
        printf("My letter: %c\n", s1.myLetter);
        return 0;
      } $\pause$
    \end{cpp}
  \end{itemize}
\end{PauseHighLight}

\end{slide}

%%%%%%%%%%%%%%%%%%%%%%% Next Slide %%%%%%%%%%%%%%%%%%%%%%%

\begin{slide}
\section[-2]{Estimated Errors in the Mean}

\begin{PauseHighLight}
  \begin{itemize}
  \item When working with empirical data, $\{X_i, i=1,2,\ldots,n\}$,
    we want to compute the \textit{mean} and \textit{variance} (from
    which we can estimate the error in the mean)\pause
  \item We can do this on the fly by storing
    \begin{align*}
      n, && \hat{\mu}_n &= \frac{1}{n} \sum_{i=1}^n X_i,
      & Q_n &= \sum_{i=1}^n (X_i - \hat{\mu}_n)^2 \pause
    \end{align*}
  \item Given $X_{n+1}$ we can update our data using
    \begin{align*}
      \Delta &= \frac{X_{n+1} - \hat{\mu}_n}{n+1},\pause
      & Q_{n+1} &= Q_n + n\,\Delta\,(X_{n+1}-\hat{\mu}_n),\pause
      & \hat{\mu}_{n+1} = \hat{\mu}_n + \Delta\pause
    \end{align*}
    this requires the back of an envelop to verify\pauseb
  \end{itemize}
\end{PauseHighLight}

\end{slide}

%%%%%%%%%%%%%%%%%%%%%%% Next Slide %%%%%%%%%%%%%%%%%%%%%%%

\begin{slide}
\section[-2]{Second Order Statistics in C}
  
\begin{PauseHighLight}
  \begin{itemize}
  \item In C we can use a \texttt{struct} to keep this data together
    \begin{cpp}
      struct Sos {
        unsigned n;
        double mu;
        double Q;
      };$\pause$
    \end{cpp}
  \item We can write functions that update thos
    \begin{cpp}
      void add(struct Sos& sos, double x) {
        double delta = = (x - mu)/(n+1.0);
        Q += n*delta*(x - mu);
        mu += delta;
        n++;
      }$\pause$
    \end{cpp}
  \end{itemize}
\end{PauseHighLight}

\end{slide}

%%%%%%%%%%%%%%%%%%%%%%% Next Slide %%%%%%%%%%%%%%%%%%%%%%%

\begin{slide}
\section{Classes}

\begin{PauseHighLight}
  \begin{itemize}
  \item C++ was developed by Bjarne Stroustrup and released in 1985 as
    ``C with classes''\pause
  \item It was syntactic sugar that compiled down to C\pause{} (as
    such if was intended to be as fast as C)\pause
  \item You are familiar with classes from python and they are very
    much the same thing\pause, except C++ is a lot more elegant than
    python\pauseb
  \item It has grown since 1985, adding templates and a lot of nice
    functionality\pauseb
  \end{itemize}
\end{PauseHighLight}

\end{slide}

%%%%%%%%%%%%%%%%%%%%%%% Next Slide %%%%%%%%%%%%%%%%%%%%%%%

\begin{slide}
\section[-1]{Classes by Example}

\begin{PauseHighLight}
  \begin{itemize}
  \item Define programme in header file \texttt{sos.h}\pause
  \end{itemize}
\end{PauseHighLight}

\begin{cpp}
  class Sos {
  private:               // $encapsulate\pause$
    int n;
    double mu;
    double Q;$\pause$

  public:                 // $interface   \pause$
    Sos();                // $constructor \pause$
    void add(double x);   // $add data  \pause$
    double mean();        // $return mean$
    double var();         // $unbiased estimate of variance$
    double error();       // $estimated error in mean$
  }$\pause$
\end{cpp}
\end{slide}

%%%%%%%%%%%%%%%%%%%%%%% Next Slide %%%%%%%%%%%%%%%%%%%%%%%

\begin{slide}
\section[-1]{Implementation of sos.cc}

\begin{cpp}
  Sos::Sos() {n=0; mu=0.0; Q=0.0;}$\pause$ 

  void Sos::add(double x) {
    double delta = = (x - mu)/(n+1.0);
    Q += n*delta*(x - mu);
    mu += delta;
    n++;
  }$\pause$

  double Sos::mean() const {return mu;} $\pause$
  
  double Sos::var() const
  {
    assert(n>1.0);
    return nvar/(n-1.0);
  }$\pause$

  double error() const
  {
    sqrt(var()/n);
  }$\pause$
\end{cpp}
\end{slide}

%%%%%%%%%%%%%%%%%%%%%%% Next Slide %%%%%%%%%%%%%%%%%%%%%%%

\begin{slide}
\section{Using Classes}

\begin{PauseHighLight}
  \begin{itemize}
  \item Classes are easy to use
    \begin{cpp}
      #include "sos.h"     $\pause$
      #include <iostream>
      using namespace std; $\pause$

      void main() {
        Sos mean;  $\pause$
        for(int i=0; i<n; ++i) {
          // compute X
          mean.add(X);
        }
        cout << mean.mean() << ' ' << mean.error() << endl;$\pause$
      }$\pause$
    \end{cpp}
  \item \texttt{Sos} is the class that I use most (both in C++ and python)\pause
  \end{itemize}
\end{PauseHighLight}
\end{slide}

%%%%%%%%%%%%%%%%%%%%%%% Next Slide %%%%%%%%%%%%%%%%%%%%%%%

\begin{slide}
\section{Libraries}

\begin{PauseHighLight}
  \begin{itemize}
  \item C++ comes with a lot of in built libraries\pause
  \item I include libraries using include statements
    \begin{cpp}
      #include <iostream>
      #include <vector>$\pause$
    \end{cpp}
    \vspace*{-1cm}
  \item This is the same as C, but the C++ libraries don't have
    ``.h\pause
  \item These are known as the standard library or the standard
    template library\pause
  \end{itemize}
\end{PauseHighLight}

\end{slide}


%%%%%%%%%%%%%%%%%%%%%%% Next Slide %%%%%%%%%%%%%%%%%%%%%%%

\begin{slide}
\section[-2]{Namespaces}

\begin{PauseHighLight}
  \begin{itemize}
  \item When you are writing very large programmes (possibly involving
    other peoples code) you might accidentally use the same name for a
    class, function or variable used elsewhere\pause
  \item If you are luck this won't compile, or crash\pause. If you are
    unlucky you will have a weird bug that will be very difficult to
    find\pause
  \item To prevent this, C++ invented a new scope called
    \emph{namespaces}\pause
  \item By default all the standard library classes and functions are
    in namespace \jl+std+\pause
  \item To call the library we write \jl+std::vector<double>+\pause
  \item We can be lazy and write \jl+using namespace std;+\pause
  \end{itemize}
\end{PauseHighLight}

\end{slide}

%%%%%%%%%%%%%%%%%%%%%%% Next Slide %%%%%%%%%%%%%%%%%%%%%%%

\begin{slide}
\section{Print}

\begin{PauseHighLight}
  \begin{itemize}
  \item Rather than pesky \texttt{printf} statements C++ allows us to
    use the opeartor \jl+<<+\pause
  \item When you get used to it, you will love if
    \begin{cpp}
      #include <iostream>      // $header file the defines library$
      using namespace std;

      void main() {
        int i = 5;
        double x = 3.3;$\pause$
        
        cout << "hello there" << i << ' ' << x << endl; 
      }$\pause$
    \end{cpp}
  \end{itemize}
\end{PauseHighLight}

\end{slide}


%%%%%%%%%%%%%%%%%%%%%%% Next Slide %%%%%%%%%%%%%%%%%%%%%%%
\Outline
%%%%%%%%%%%%%%%%%%%%%%% Next Slide %%%%%%%%%%%%%%%%%%%%%%%

\begin{slide}
\section{Pointers}
  
\begin{PauseHighLight}
  \begin{itemize}
  \item In C and C++ we can access an object through its memory
    address
    \begin{cpp}
      int a = 5;    // $creates an object a with value 5\pause$
      int* b = &a;  // $b is the memory address of object a\pause$
      *b = 6        // $*b is now a pseudonym for a\pause$
    \end{cpp}
  \item \jl+b+ is called a pointer\pause
  \item The \textit{dereferencing} operator \jl+*+ turns the pointer
    back into the object\pause
  \end{itemize}
\end{PauseHighLight}

\end{slide}

%%%%%%%%%%%%%%%%%%%%%%% Next Slide %%%%%%%%%%%%%%%%%%%%%%%

\begin{slide}
\section{New Object}
  
\begin{PauseHighLight}
  \begin{itemize}
  \item The operator \jl+new+ will create an object and return a
    reference
    \begin{cpp}
Widget w(arg);                  // $w is an instance of class Widget\pause$
Widget* wpt = new Widget(args); // $pointer to instance of class Widget\pause$
    \end{cpp}
  \item To call a member function of \jl+wp+ use either
    \begin{cpp}
(*wpt).func();  // $dereference object and call member function\pause$
wpt->func();    // $easy to type\pause$
    \end{cpp}
  \end{itemize}
\end{PauseHighLight}

\end{slide}

%%%%%%%%%%%%%%%%%%%%%%% Next Slide %%%%%%%%%%%%%%%%%%%%%%%

\begin{slide}
\section[-2]{Inheritence}

\begin{PauseHighLight}
  \begin{itemize}
  \item C++ allows classes to inherit from other classes\pause
  \item Suppose \jl+Square+ and \jl+Circle+ inherits from \jl+Shape+
  \item If \jl+Shape+ has a (virtual) member function \jl+area+ then
    \jl+Square+ and \jl+Circle+ can redefine this
    \begin{cpp}
      class Square: public Shape $\pause${
        private:
          double l;

        public:
          Square(double len) {l=len;}  // $constructor\pause$
          double area() {return l*l;}  // $define area$
      }$\pause$
    \end{cpp}
  \end{itemize}
\end{PauseHighLight}

\end{slide}

%%%%%%%%%%%%%%%%%%%%%%% Next Slide %%%%%%%%%%%%%%%%%%%%%%%

\begin{slide}
\section{Polymorphism}
  
\begin{PauseHighLight}
  \begin{itemize}
  \item Polymorphism is a way of using inheritance where we
    instantiate a parent pointer with a child class
    \begin{cpp}
      Shape* shape = new Square(2.5);$\pause$

      cout << shape->area() << endl;$\pause$
    \end{cpp}
  \item This provides a clean way of choosing a behaviour depending on
    the object type\pause
  \item It is used in \textit{iterator}s which we will come to later
    in the course\pause
  \end{itemize}
\end{PauseHighLight}

\end{slide}

%%%%%%%%%%%%%%%%%%%%%%% Next Slide %%%%%%%%%%%%%%%%%%%%%%%

\begin{slide}
\section{Arrays}

\begin{PauseHighLight}
  \begin{itemize}
  \item C++ also uses \jl+new+ to return arrays (in place of malloc)
    \begin{cpp}
      int* pt = new int[20];
    \end{cpp}
    creates a pointer to memory location where we can store 20
    integers\pause
  \item We can dereference the $i^{th}$ element using \jl+pt[i]+\pause
    (which is equivalent to \jl$*(pt+i)$)\pauseb---this is the same as C\pauseb
  \item We can free this up with
    \begin{cpp}
      delete[] pt;$\pause$
    \end{cpp}
  \end{itemize}
\end{PauseHighLight}

\end{slide}



%%%%%%%%%%%%%%%%%%%%%%% Next Slide %%%%%%%%%%%%%%%%%%%%%%%

\begin{slide}
\section[-2]{References}
  
\begin{PauseHighLight}
  \begin{itemize}
  \item C and C++ also provides references
    \begin{cpp}
      int a = 5;         // $create a memory location called a\pause$
      int& b = a;        // $b is a pseudonym for a\pause$
      b = 6              // $both b and a are now 6\pause$
    \end{cpp}
  \item References are like dereferenced pointers\pause
  \item There are many uses of references, one is so we can make
    functions change their value
    \begin{cpp}
      void f(int x) {x += 6;}   // $define function f\pause$

      void g(int& x) {x += 2;}  // $define function g\pause$
      
      int a = 5;$\pause$
      
      f(a);                      // $does nothing a=5\pause$
      g(a);                      // $now a=7\pause$
    \end{cpp}
  \end{itemize}
\end{PauseHighLight}

\end{slide}

%%%%%%%%%%%%%%%%%%%%%%% Next Slide %%%%%%%%%%%%%%%%%%%%%%%

\begin{slide}
\section{Saving Copying}

\begin{PauseHighLight}
  \begin{itemize}
  \item When we declare a function \jl+f(Widget w)+ then widget \jl+w+
    is copied to the function (this is known as passed by value)\pause
  \item If widget is big, even if we don't want to change it we might
    not want to copy it
    \begin{cpp}
      void f(const Widget& w);$\pause$
      void g(Widget w);$\pause$
    \end{cpp}
  \item In both cases \jl$w$ is a Widget, but function \jl$f$ avoids
    copying its input\pause
  \end{itemize}
\end{PauseHighLight}

\end{slide}


%%%%%%%%%%%%%%%%%%%%%%% Next Slide %%%%%%%%%%%%%%%%%%%%%%%
\Outline
%%%%%%%%%%%%%%%%%%%%%%% Next Slide %%%%%%%%%%%%%%%%%%%%%%%

\begin{slide}
\section{Overloading}
  
\begin{PauseHighLight}
  \begin{itemize}
  \item C and C++ allow you to define different functions with the
    same name but different arguments
    \begin{cpp}
      void func(int a);     // $called if argument is an \texttt{int}$
      void func(double a);  // $called if argument is a \texttt{double}\pause$
    \end{cpp}
  \item Needs to be used sensibly, but provides flexibility\pause
  \end{itemize}
\end{PauseHighLight}

\end{slide}

%%%%%%%%%%%%%%%%%%%%%%% Next Slide %%%%%%%%%%%%%%%%%%%%%%%

\begin{slide}
\section[-1]{Example}

\begin{PauseHighLight}
  \begin{itemize}
  \item In the second order statistics class we could define a member
    function
    \begin{cpp}
      void add(const Sos& rhs);
    \end{cpp}
  \item With an implementation
    \begin{cpp}
void Sos::add(const Sos& rhs)
{
  double total = n + rhs.n;
  double diff = rhs.mu-mu;
  mu += rhs.n*diff/total;
  Q += rhs.Q + n*rhs.n*diff*diff/total;
  n = total;

  return rhs;
}
    \end{cpp}
  \end{itemize}
\end{PauseHighLight}

\end{slide}

%%%%%%%%%%%%%%%%%%%%%%% Next Slide %%%%%%%%%%%%%%%%%%%%%%%

\begin{slide}
\section{Overloading Continued}
  
  \begin{itemize}
  \item This allows us to add second order statistics
    \begin{cpp}
      Sos total;
      for(int i=0; i<10; ++i) {
        Sos local;
        for(int j=0; j<100; ++j) {
          // compute X
          cout << local.mean() << ',' << local.error() << endl;
          local.add()
        }
        total.add(local)
        cout << total.mean() << ',' << total.error() << endl;
      }
    \end{cpp}
  \end{itemize}

\end{slide}

%%%%%%%%%%%%%%%%%%%%%%% Next Slide %%%%%%%%%%%%%%%%%%%%%%%

\begin{slide}
\section{Opeartor Overloading}

\begin{PauseHighLight}
  \begin{itemize}
  \item C++ like python allows us to overload operators\pause
  \item Rather than using add I might prefer to use
    \begin{cpp}
      class Sos {
        ...
        double operator+=(double x) { add(x); return(x); }
      }$\pause$
    \end{cpp}
  \item Then we can write
    \begin{cpp}
      Sos sos;
      sos += X;$\pause$
    \end{cpp}
  \end{itemize}
\end{PauseHighLight}

\end{slide}

%%%%%%%%%%%%%%%%%%%%%%% Next Slide %%%%%%%%%%%%%%%%%%%%%%%

\begin{slide}
\section{Overloading \jl+<<+}
  
\begin{PauseHighLight}
  \begin{itemize}
  \item To print an object of type \texttt{Sos} we define
    \begin{cpp}
      ostream& operator<<(ostream& out, const Sos& d)
      {
        out << d.mean() << " " << d.error();
        return(out);
      }$\pause$
    \end{cpp}
  \item We can then print
    \begin{cpp}
      Sos sos;
      ...

      cout << sos << endl; $\pause$
    \end{cpp}
  \item I've made \texttt{sos.h} and \texttt{sos.cc} available on the
    web site\pause---I use them a lot, you might want to keep them around\pauseb
  \end{itemize}
\end{PauseHighLight}

\end{slide}



%%%%%%%%%%%%%%%%%%%%%%% Next Slide %%%%%%%%%%%%%%%%%%%%%%%
\Outline
%%%%%%%%%%%%%%%%%%%%%%% Next Slide %%%%%%%%%%%%%%%%%%%%%%%

\begin{slide}
\section{Templates}
  
\begin{PauseHighLight}
  \begin{itemize}
  \item Many algorithms and data structures can be applied to a wide
    range of types\pause
    \begin{cpp}
      vector<double> double_vec; // $resizable array of doubles\pause$
      vector<int>    int_vec;    // $resizable array of int\pause$
      map<string, int>  mymap    // $map with string keys and int values\pause$
    \end{cpp}
  \item C++ allows us to define a template class
    \begin{cpp}
      template <typename T>
      class myclass {
        private T data;
      }$\pause$
    \end{cpp}
  \end{itemize}
\end{PauseHighLight}

\end{slide}

%%%%%%%%%%%%%%%%%%%%%%% Next Slide %%%%%%%%%%%%%%%%%%%%%%%

\begin{slide}
\section{Templates}

\begin{PauseHighLight}
  \begin{itemize}
  \item Templates work very simply\pause
  \item They provide a template for same type (e.g. \jl+T+)\pause
  \item When you ask for an instance of that object
    \begin{cpp}
      myclass<int> instance;$\pause$
    \end{cpp}
    the C++ compiler takes your template and substitutes the \jl+T+
    with \jl+int+\pause
  \item This is both simple and powerful\pause
  \end{itemize}
\end{PauseHighLight}

\end{slide}

%%%%%%%%%%%%%%%%%%%%%%% Next Slide %%%%%%%%%%%%%%%%%%%%%%%

\begin{slide}
\section{Template Functions}

\begin{PauseHighLight}
  \begin{itemize}
  \item As well as classes I can create template functions
    \begin{cpp}
      template <typename T>
      T accumulate(const vector<T>& vec) {
        T sum = 0;
        for(int i=0; i<vec.size(); ++i) {
          sum += vec[i];
        }
        return sum
      }$\pause$
    \end{cpp}
  \item This will work with \jl$vector<int>$, \jl+vector<double>+\pause
  \end{itemize}
\end{PauseHighLight}


\end{slide}

%%%%%%%%%%%%%%%%%%%%%%% Next Slide %%%%%%%%%%%%%%%%%%%%%%%

\begin{slide}
\section{Summary}

\begin{PauseHighLight}
  \begin{itemize}
  \item C++ is a rich language\pause
  \item You should learn some C++ in low-level programming\pause
  \item There are a lot of resources\pause
  \item I'm afraid you will only get good at it by writing
    programs\pause
  \item The lab session are to help you learn C++\pause
  \end{itemize}
\end{PauseHighLight}

\end{slide}

