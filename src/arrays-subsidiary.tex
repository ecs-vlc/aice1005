\documentclass{article}
\usepackage{geometry}
\usepackage[english]{babel}
\usepackage{amsmath}

\begin{document}
\begin{center}
  \Large \bf Details on Array Lecture
\end{center}

\section*{Time Analysis for Resizing}

We want to calculate the number of operations it takes when we add $N$
elements to an array with initial capacity $n$.  This involves resizing
the array $m+1$ times. Each time we double the array.  We thus have arrays
of size $n$, $2n$, $4n$, \ldots, $2^m n$, $2^{m+1}n$ such that $N\leq
2^{m+1}n$ so that all the elements fit.  Clearly, $N>2^{m}n$ otherwise
we would have stopped earlier.  We thus have
\begin{displaymath}
  n\times 2^m < N \leq n \times 2^{m+1}.
\end{displaymath}

How big is $m$?  First let us divide through by $n$
\begin{displaymath}
  2^m < \frac{N}{n} \leq 2^{m+1}.
\end{displaymath}
(The inequality remain the same as $n$ is positive).
If we take log to the base 2 of each sides
\begin{align*}
  \log_2\left( 2^m\right) < \log_2(\frac{N}{n}) 
  \leq \log_2\left(2^{m+1}\right).
\end{align*}
(We have to be a little careful, since in general $a<b$ does not imply
$f(a)<f(b)$ for an arbitrary function.  However, logarithms are
continuous monotonically increasing functions so $a<b$ does imply
$\log(a)<\log(b)$ for any base of logarithm).  Remembering
that $\log_2(2^a) = a$ we find
\begin{align*}
  m < \log_2(\frac{N}{n}) \leq m+1
\end{align*}
thus $m$ is the integer which is the largest integer smaller than
$\log_2(\frac{N}{n})$.  We can write this as
\begin{align*}
  m = \left\lfloor \log_2\left(\strut \frac{N}{n} \right) \right\rfloor
\end{align*}
where $\lfloor x \rfloor$ is the floor function and means the largest
integer smaller than $x$.

Every time we expand the array we must copy all the elements across to
the new array.  Thus we have to do
\begin{align*}
  n + 2n + 4n + \cdots + 2^m n
\end{align*}
copies.  But we can sum this series easily
\begin{align*}
  n + 2n + 4n + \cdots + 2^m n = n\left( \strut 1+2+4+\cdots+ 2^m \right)
  = n\times (2^{m+1}-1).
\end{align*}
Now we can substitute in the value of $m$ we have calculated above
\begin{align*}
  2^{m+1} = 2 \times 2^m = 2 \times 2^{\lfloor \log_2\left(\strut
  \frac{N}{n} \right) \rfloor}.
\end{align*}
Now we know that $\lfloor \log_2(N/n) \rfloor\leq \log_2(N/n)$ so that
\begin{align*}
  2^{\lfloor \log_2\left(\strut \frac{N}{n} \right) \rfloor} \leq
  2^{\log_2\left(\strut \frac{N}{n} \right)} = \frac{N}{n}
\end{align*}
which follows from $2^{\log_2(a)}=a$.  Thus
\begin{align*}
  2^{m+1} \leq \frac{2N}{n}
\end{align*}
and the number of copies is
\begin{align*}
  n \times (2^{m+1}-1) \leq n\times \left(\frac{2N}{n} -n\right) = 2N-n.
\end{align*}

In addition to the copies we also do $N$ adds.  Thus the total number of
add and copy operations is less than or equal to
\begin{align*}
  N+2N-n= 3N-n < 3N.
\end{align*}
Thus the cost of resizing an array is less than 3 times the number of
elements you are putting into the array.

Of course, if you know that you want an array of size $N$ in the first
place you can reserve that much memory using
\begin{verbatim}
List = new ArrayList(N);
\end{verbatim}
In which case you don't have to resize the memory at all.  The cost of
adding is thus $N$ add operations.



\section*{Things You Need to Know About Logarithms}

Logs are the inverse of exponentiation.  In general
\begin{align*}
  \log_a(a^x) = x.
\end{align*}
If we exponentiate this we get
\begin{align*}
  a^{\log_a(a^x)} = a^x
\end{align*}
setting $a^x=y$ we can write this as
\begin{align*}
  a^{\log_a(y)} = y.
\end{align*}

We note that
\begin{align*}
  \log_a(x y) = \log_a(x) + \log_a(y)
\end{align*}
We can show this by exponentiating
\begin{align*}
  a^{\log_a(x \times y)} = a^{\log_a(x) + \log_a(y)}
\end{align*}
But $a^{u+v} = a^u \times a^v$ so
\begin{align*}
  a^{\log_a(x \times y)} &= a^{\log_a(x)} \times a^{\log_a(y)} \\
  x \times y &= x \times y.
\end{align*}
A related property of logarithms is
\begin{align*}
  \log_a(b^c) = c\log(b)
\end{align*}
which clearly follows from the previous result if $c$ is an integer, but
is true even when $c$ isn't an integer.

Finally we note that
\begin{align*}
  \log_a(x) = \log_a(b^{\log_b(x)}) = \log_b(x) \times \log_a(b)
\end{align*}
where we have used $x = b^{\log_b(x)}$.  Thus $\log_a(x) = c \log_b(x)$
where $c=\log_a(b)$.  When using big-O notation it doesn't matter which
logarithm we are using since they are all the same up to a
multiplicative constant (which we ignore).

\end{document}
