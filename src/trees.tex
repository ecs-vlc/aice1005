%Master File:lectures.tex

\lesson{Make Friends with Trees}
 
\vspace*{-1cm}
\begin{center}
  \includegraphics[height=10cm]{tree_hug}\hfil%
  \includegraphics[height=10cm]{tree_hug1}
\end{center}

\keywords{Binary trees, binary search trees, sets, tree iterators}

%%%%%%%%%%%%%%%%%%%%%%% Next Slide %%%%%%%%%%%%%%%%%%%%%%%
\renewcommand{\Outline}{%
\begin{slide}
\section{Outline}

\begin{minipage}{12cm}
  \vfill
  \begin{enumerate}
    \outlineitem{Trees}{intro}
    \outlineitem{Binary Trees}{bt}
    \begin{itemize}
    \item \toplink{impl}{Implementing Binary Trees}
    \end{itemize}
    \outlineitem{Binary Search Trees}{bst}
    \begin{itemize}
    \item \toplink{bstdef}{Definition}
    \item \toplink{implset}{Implementing a Set}
    \end{itemize}
    \outlineitem{Tree Iterators}{treeiter}
  \end{enumerate}
  \vfill
\end{minipage}\hfill
\begin{minipage}{10cm}
  \includegraphics[width=8cm]{tree_hug}
\end{minipage}
\end{slide}
\addtocounter{outlineitem}{1}
}

\setcounter{outlineitem}{1}
\Outline
\toptarget{firstoutline}

%%%%%%%%%%%%%%%%%%%%%%% Next Slide %%%%%%%%%%%%%%%%%%%%%%%

\begin{slide}
\section{Trees}

\begin{PauseHighLight}
\begin{itemize}
\item Trees are one of the major ways of structuring data\pause
\item They are used in a vast number of data structures
  \begin{itemize}\squeeze
  \item Binary search trees
  \item B-trees
  \item splay trees
  \item heaps
  \item tries
  \item suffix trees\pause
  \end{itemize}
\item We shall cover most of these\pause
\end{itemize}
\end{PauseHighLight}
\end{slide}

%%%%%%%%%%%%%%%%%%%%%%% Next Slide %%%%%%%%%%%%%%%%%%%%%%%

\begin{slide}
\section{Defining Trees}

\begin{PauseHighLight}
  \begin{itemize}
  \item Mathematically a tree is an \emph{acyclic undirected
      graph}\pause
    \begin{itemize}
    \item \emph{graph}: a structure consisting of \emph{nodes} or
      \emph{vertices} joined by \emph{edges}\pause
    \item \emph{undirected}: the edges goes both ways\pause
    \item \emph{acyclic}: there are no cycles in the graph\pause
    \end{itemize}
  \end{itemize}
  \begin{center}\pauselevel{=2}
    \multiinclude[graphics={width=\linewidth}]{tree-def}\pause
  \end{center}
\end{PauseHighLight}
\end{slide}

%%%%%%%%%%%%%%%%%%%%%%% Next Slide %%%%%%%%%%%%%%%%%%%%%%%

\begin{slide}
\section[-2]{Borrowing from Nature}

\begin{PauseHighLight}
  \begin{itemize}\squeeze
  \item We often impose an ordering on the nodes (or a direction on the
    edges)\pause---known as a rooted tree\pause
  \item Borrowing from nature, we recognise one node as the
    \emph{root} node\pause
  \item Nodes have \emph{children} nodes living beneath them
  \item Each child has a \emph{parent} node above them except the
    root\pause
  \item Nodes with no children are \emph{leaf} nodes\pause
  \end{itemize}
  \begin{center}\pauselevel{=2}
    \multiinclude[graphics={height=5cm}]{tree-def2}\pause
  \end{center}\vspace{-2cm}
\end{PauseHighLight}
\end{slide}

%%%%%%%%%%%%%%%%%%%%%%% Next Slide %%%%%%%%%%%%%%%%%%%%%%%

\begin{slide}
\section[-1]{Spot the Error}
\pausebuild
\begin{itemize}
\item One small biological inconsistency\pause
\item Yep!, computer scientists draw there trees upside down\pause
  \begin{itemize}
  \item root at the top
  \item leaves at the bottom\pause
  \end{itemize}
  \begin{center}\pauselevel{=2}
    \multiinclude[graphics={height=8cm}]{tree-def3}\pause
  \end{center}
\end{itemize}
\end{slide}


%%%%%%%%%%%%%%%%%%%%%%% Next Slide %%%%%%%%%%%%%%%%%%%%%%%

\begin{slide}
\section{Subtrees}

\begin{PauseHighLight}
  \begin{itemize}
  \item We can think of the tree made up of \emph{subtrees}
  \end{itemize}
  \begin{center}
    \multiinclude[graphics={height=5cm}]{subtree-def}\pause
  \end{center}
\end{PauseHighLight}
\end{slide}

%%%%%%%%%%%%%%%%%%%%%%% Next Slide %%%%%%%%%%%%%%%%%%%%%%%

\begin{slide}
\section[-1]{Level of Nodes}

\pausebuild
\color{TwoColor}
\begin{itemize}
\item It is useful to label different levels of the tree\pauseh
\item We take the \emph{level} of a node in a tree as its distance
  from the root\pauseh
\item We take the \emph{height} of a tree to be the number of
  levels\pauseh
\end{itemize}
\begin{center}\pauselevel{=1}
  \multiinclude[graphics={width=\linewidth}]{tree-depth}\pause
\end{center}

\end{slide}



%%%%%%%%%%%%%%%%%%%%%%% Next Slide %%%%%%%%%%%%%%%%%%%%%%%
\Outline % Binary trees
%%%%%%%%%%%%%%%%%%%%%%% Next Slide %%%%%%%%%%%%%%%%%%%%%%%

%%%%%%%%%%%%%%%%%%%%%%% Next Slide %%%%%%%%%%%%%%%%%%%%%%%

\begin{slide}
\section[-2]{Binary Trees}

\pausebuild
\color{TwoColor}
\begin{itemize}\squeeze
\item A \emph{binary tree} is a tree where each node can have zero,
  one or two children\pauseh
\item The total number of possible nodes at level $l$ is $2^l$\pauseh
\item The total number of possible nodes of a tree of height $h$ is
  \begin{align*}
    1 + 2 + \cdots + 2^{h-1} = 2^{h} -1 \pauseh
  \end{align*}\vspace{-1cm}
\end{itemize}
\begin{center}\pauselevel{=1}
  \multiinclude[graphics={width=0.9\linewidth}]{binary-tree-def}\pause    
\end{center}\vspace{-1cm}
\end{slide}


%%%%%%%%%%%%%%%%%%%%%%% Next Slide %%%%%%%%%%%%%%%%%%%%%%%

\begin{slide}
\section[-1]{Uses of Binary Trees}

\begin{PauseHighLight}
  \begin{itemize}
  \item Binary trees have a huge number of applications\pause
  \item For example, they are used as \emph{expression trees} to
    represent formulae
  \end{itemize}
  \begin{center}
    \includegraphics[width=20cm]{expression-tree} \pause
  \end{center}
\end{PauseHighLight}

\end{slide}



%%%%%%%%%%%%%%%%%%%%%%% Next Slide %%%%%%%%%%%%%%%%%%%%%%%

\begin{slide}
\section{Implementation}
\toptarget{impl}

\begin{PauseHighLight}
  \begin{itemize}
  \item We wish to build a generic binary tree class with each node
    housing an element\pause
  \item Again we use a \texttt{Node<T>} class as the building block
    for our data structure---in this case a node of the tree\pause
  \item The \texttt{Node<T>} class will contain a pointer to left and
    right children\pause
  \item To help navigate the tree each node will contain a pointer to
    its parent\pause
  \end{itemize}
\end{PauseHighLight}
\end{slide}

%%%%%%%%%%%%%%%%%%%%%%% Next Slide %%%%%%%%%%%%%%%%%%%%%%%

\begin{slide}
\section[-2]{C++ Code}

\begin{minipage}{12cm}
\begin{cpp}
template <typename T>    
class binary_tree {
private:
  class Node {
  public:
    T element;
    Node* parent;
    Node* left = 0;
    Node* right = 0;

    Node(const T& value, Node* parent_node) {
      element = value;
      parent = parent_node;
    }
  };$\pause$

  unsigned no_elements = 0;
  Node* root = 0;$\pause$
\end{cpp}
\end{minipage}\hspace{2cm}
\begin{minipage}{9cm}
\vspace{-3cm}
\includegraphics[width=9cm]{tree-entry}
\end{minipage}
\end{slide}

%%%%%%%%%%%%%%%%%%%%%%% Next Slide %%%%%%%%%%%%%%%%%%%%%%%
\Outline % Binary search trees
%%%%%%%%%%%%%%%%%%%%%%% Next Slide %%%%%%%%%%%%%%%%%%%%%%%

%%%%%%%%%%%%%%%%%%%%%%% Next Slide %%%%%%%%%%%%%%%%%%%%%%%

\begin{slide}
\section{Binary Search Trees}
\toptarget{bstdef}

\begin{PauseHighLight}
  \begin{itemize}
  \item We will concentrate on one of the most important binary trees,
    namely the \emph{binary search tree}\pause
  \item The binary search tree keeps the elements ordered\pause
  \item We can define a binary search tree recursively\pause
    \begin{enumerate}
    \item Each element in the left subtree is less than the root element\pause
    \item Each element in the right subtree is greater than the root
      element\pause
    \item Both left and right subtrees are binary search trees\pause
    \end{enumerate}
  \end{itemize}
\end{PauseHighLight}
\end{slide}

%%%%%%%%%%%%%%%%%%%%%%% Next Slide %%%%%%%%%%%%%%%%%%%%%%%

\begin{slide}
\section[-1]{Example Binary Search Tree}

\begin{PauseHighLight}
  \begin{center}
    \multiinclude[graphics={width=\linewidth}]{bst}\pause
  \end{center}
\end{PauseHighLight}

\end{slide}

%%%%%%%%%%%%%%%%%%%%%%% Next Slide %%%%%%%%%%%%%%%%%%%%%%%

\begin{slide}
\section{Searching A Binary Search Tree}

\pausebuild
\color{TwoColor}
\begin{minipage}{10cm}
  \begin{itemize}
  \item Searching a binary search tree is easy\pauseh
  \item Start at the root
    \pauselevel{=1, highlight =3 :3}\pause
  \item Compare with element\pauselevel{=1, highlight =4 :8}\pause
    \begin{itemize}
    \item If less than element go left
      \pauselevel{=1, highlight =4 :5, highlight =7 :7}\pause
    \item If greater than element go right
      \pauselevel{=1, highlight =6 :6}\pause
    \item If equal to element found
      \pauselevel{=1, highlight =8 :8}\pause
    \end{itemize}
  \end{itemize}
\end{minipage}\hspace{0.5cm}
\begin{minipage}{13cm}
\color{TextColor}
  \begin{center}\pauselevel{=1}
    \multiinclude[graphics={width=13cm}]{bst-find}\pause
  \end{center}
\end{minipage}
\end{slide}

%%%%%%%%%%%%%%%%%%%%%%% Next Slide %%%%%%%%%%%%%%%%%%%%%%%

\begin{slide}
\section[-1]{Speed of Search}

\begin{PauseHighLight}
  \begin{itemize}
  \item The number of comparisons necessary to find an element in a
    binary tree depends on the level of the node in the tree\pause
  \item The worst case number of comparisons is therefore the height of
    the tree\pause
  \item This depends on the density of the tree\pause
  \end{itemize}
  \begin{center}
    \multiinclude[graphics={width=\linewidth}]{tree_shape}\pause
  \end{center}
\end{PauseHighLight}
\end{slide}


%%%%%%%%%%%%%%%%%%%%%%% Next Slide %%%%%%%%%%%%%%%%%%%%%%%

\begin{slide}
\section{Implementing a Set}
\toptarget{implset}

\begin{PauseHighLight}
  \begin{itemize}
  \item A set is a fundamental \emph{abstract data type}\pause
  \item It is a collection of things with no repetition and no
    order\pause
  \item Ironically because order doesn't matter we can order the
    elements
    \begin{align*}
      \{1,3,5,5,3,4\} = \{5, 3, 4, 1\} = \{1, 3, 4, 5\}\pause 
    \end{align*}
  \item This allows rapid search---a feature we care about\pause
  \item Binary trees are one of the efficient ways of implementing a set\pause
  \end{itemize}
\end{PauseHighLight}
\end{slide}


%%%%%%%%%%%%%%%%%%%%%%% Next Slide %%%%%%%%%%%%%%%%%%%%%%%

\begin{slide}
  \section[-1]{Fitting In}

  \begin{PauseHighLight}
    \begin{itemize}
    \item The standard template library provides a class \texttt{std:set<T>}\pause
    \item This contains many functions like
      \begin{itemize}
      \item Constructors
      \item \texttt{size()}
      \item \texttt{insert(T\,o)}
      \item \texttt{find(Object\,o)}
      \item \texttt{erase(Object\, o)}
      \item \texttt{begin()} and \texttt{end()}
      \end{itemize}\pause
    \end{itemize}
  \end{PauseHighLight}
\end{slide}

%%%%%%%%%%%%%%%%%%%%%%% Next Slide %%%%%%%%%%%%%%%%%%%%%%%

\begin{slide}
\section[-1]{Comparable}

\begin{PauseHighLight}
  \begin{itemize}
  \item To sort any objects they must be comparable\pause
  \item In the STL the set implementation has a second template
    parameter: \jl$std::set<T, Compare = less<T> >$\pause
  \item by default this is defined to be \jl$less<T>$ (which is a function
    already defined for most common types) which you can define\pause
  \item If you have a set of complex objects you will have to define
    \texttt{Compare}
    \begin{cpp}
      bool MyCompare(MyObject left, MyObject right) {
        return something
      }

      mySet = set<MyObject, MyCompare>;
    \end{cpp}\pause
  \end{itemize}
\end{PauseHighLight}

\end{slide}

%%%%%%%%%%%%%%%%%%%%%%% Next Slide %%%%%%%%%%%%%%%%%%%%%%%

\begin{slide}
\section{Find an Element}

\begin{PauseHighLight}
  \begin{itemize}
  \item One of the core operations of a binary tree is to find a node\pause
    \begin{cpp}
   iterator find(const T& element) {
    Node* current = root;$\pause$
    while (current!=0) {
      if (current->element == element) {
	return iterator(current);
      }$\pause$
      if (element < current->element) {
	current = current->left;
      } else {
	current = current->right;
      }$\pause$
    }
    return iterator(0);
  }$\pause$
    \end{cpp}
  \end{itemize}
\end{PauseHighLight}

\end{slide}

%%%%%%%%%%%%%%%%%%%%%%% Next Slide %%%%%%%%%%%%%%%%%%%%%%%

\begin{slide}
\section{Add an Element}

\begin{cpp}
  pair<iterator, bool> insert(const T& element) {
    if (no_elements==0) {
      root = new Node(element, 0);
      ++no_elements;
      return pair<iterator, bool>(iterator(root), true);
    }$\pause$
    Node* parent = 0;
    Node* current = root;
    while(current != 0) {
      if (current->element == element) {
	return pair<iterator, bool>(iterator(0), false);
      }$\pause$
      parent = current;
      if (element < current->element) {
	current = current->left;
      } else {
	current = current->right;
      }
    }$\pause$

    




    
    current = new Node(element, parent);
    if (element < parent->element) {
      parent->left = current;
    } else {
      parent->right = current;
    }
    ++no_elements;
    return pair<iterator, bool>(iterator(current), true);
  }$\pause$
\end{cpp}
\end{slide}

%%%%%%%%%%%%%%%%%%%%%%% Next Slide %%%%%%%%%%%%%%%%%%%%%%%

\begin{slide}
\section[-1]{Tree in Action}
\pb\pause
\begin{center}
  \includegraphics[width=0.9\linewidth]{binarytree-add0.eps}\mypl{1}
  \multido{\ia=1+1,\ib=2+1}{19}{%
    \llap{\includegraphics[width=0.9\linewidth]{binarytree-add\ia.eps}}\mypl{\ib}}
\end{center}
\end{slide}



%%%%%%%%%%%%%%%%%%%%%%% Next Slide %%%%%%%%%%%%%%%%%%%%%%%

\begin{slide}
\section{Shape of Tree}

\begin{PauseHighLight}
  \begin{itemize}
  \item The structure of the tree depends on the order in which we add
    elements to it\pause
  \item Suppose we add
    \begin{quote}\it
      To be, or not to be: that is the question:\\
      Whether 'tis nobler in the mind to suffer\\
      The slings and arrows of outrageous fortune,\\
      Or to take arms against a sea of troubles,\pause
    \end{quote}
  \item Ignoring punctuation we get the following tree\pause
  \end{itemize}
\end{PauseHighLight}

\end{slide}

%%%%%%%%%%%%%%%%%%%%%%% Next Slide %%%%%%%%%%%%%%%%%%%%%%%

\begin{slide}
\section{Hamlet}

\begin{center}
  \includegraphics[width=\linewidth]{hamlet.eps}
\end{center}
\end{slide}



%%%%%%%%%%%%%%%%%%%%%%% Next Slide %%%%%%%%%%%%%%%%%%%%%%%
\Outline % Tree Iterators
%%%%%%%%%%%%%%%%%%%%%%% Next Slide %%%%%%%%%%%%%%%%%%%%%%%

%%%%%%%%%%%%%%%%%%%%%%% Next Slide %%%%%%%%%%%%%%%%%%%%%%%

\begin{slide}
\section[-2]{Tree Iterators}

\begin{PauseHighLight}
  \begin{itemize}
  \item As with most container classes it is very useful to define
    iterators\pause
  \item \texttt{begin()} should return a ``pointer'' to the start of
    the tree\pause
  \item \texttt{end()} provides a ``pointer'' past the end\pause
  \item \texttt{operator*()} returns the element\pause
  \item \texttt{opeator++()} increments the ``pointer''\pause
  \item \texttt{operator!=(lhs, rhs)} is used to compare iterators\pause
    \begin{cpp}
      set<int> mySet;
      ...
      for(auto pt=mySet.begin(), pt!=mySet.end(), ++pt) {
        cout << *pt;
      }
    \end{cpp}\pause
  \end{itemize}
\end{PauseHighLight}
\end{slide}

%%%%%%%%%%%%%%%%%%%%%%% Next Slide %%%%%%%%%%%%%%%%%%%%%%%

\begin{slide}
\section[-2.5]{Successor}

\begin{PauseHighLight}
  \begin{itemize}\squeeze
  \item To find the successor we first start in the left most branch
    \pauselevel{=1, =6}\pause
  \item We follow two rules\pause
    \begin{enumerate}
    \item \textbf{If} right child exist \textbf{then} move right once
      and then move as far left as possible
      \pauselevel{=3, =8, =10, =11, =14, =15, =17}\pause
    \item \textbf{else} go \textit{up} to the left as far as possible
      and then move up right
      \pauselevel{=4, =7, =9, =12, =13, =16, =18}\pause
    \end{enumerate}
  \end{itemize}\pauselevel{=5}
  \begin{center}
    \multiinclude[graphics={height=7cm}]{bst-traverse}\pause\\
    \pauselevel{=6} $\textcolor{TextColor}{\{}15\pause\ 25\pause\
    28\pause\ 30\pause\ 32\pause\
    36\pause\ 37\pause\ 50\pause\ 53\pause\ 59\pause\ 61\pause\
    68\pause\ 75\textcolor{TextColor}{\}}\pause$
  \end{center}
\end{PauseHighLight}

\end{slide}


%%%%%%%%%%%%%%%%%%%%%%% Next Slide %%%%%%%%%%%%%%%%%%%%%%%

\begin{slide}
\section{Lessons}

\begin{PauseHighLight}
  \begin{itemize}
  \item Trees and particularly binary trees are one of the most
    important tools of a computer scientist\pause
  \item Conceptually they are quite simple\pause
  \item However, there are a lot of details that need to be
    understood\pause
  \item Coding even simple trees needs great care\pause
  \item As we will see things get more complicated\pause
  \end{itemize}
\end{PauseHighLight}
\end{slide}
