%Master File:lectures.tex

\lesson{Recurse!}
\vspace{-1cm}
\begin{center}
  \includegraphics[height=10cm]{spk_carpet}
\end{center}
\keywords{Induction, integer power, towers of hanoi, analysis}

%%%%%%%%%%%%%%%%%%%%%%% Next Slide %%%%%%%%%%%%%%%%%%%%%%%
\renewcommand{\Outline}{%
\begin{slide}
\section{Outline}

\begin{minipage}{12cm}
  \vfill
  \begin{enumerate}
    \outlineitem{Simple Recursion}{intro}
    \outlineitem{Programming Recursively}{prog}
    \begin{itemize}
    \item \toplink{examples}{Simple Examples}
    \item \toplink{progthink}{Thinking about Recursion}
    \end{itemize}
    \outlineitem{Analysis of Recursion}{progan}
    \begin{itemize}
    \item \toplink{power}{Integer Powers}
    \item \toplink{hanoi}{Towers of Hanoi}
    \end{itemize}
  \end{enumerate}
  \vfill
\end{minipage}\hfill
\begin{minipage}{10cm}
  \includegraphics[width=8cm]{spk_carpet}
\end{minipage}
\end{slide}
\addtocounter{outlineitem}{1}
}

\setcounter{outlineitem}{1}
\Outline
\toptarget{firstoutline}

%%%%%%%%%%%%%%%%%%%%%%% Next Slide %%%%%%%%%%%%%%%%%%%%%%%

\begin{slide}
\section[-2]{Recursion}
\pb
\begin{itemize}
\item Recursion is a strategy whereby we reduce a problem to a smaller
  problem of the same type\pauseh
\item We repeat this until we reach a trivial case we can solve by
  some other means\pauseh
\item Recursion can also be used to describe situations in a succinct
  manner using references to itself.\pauseh\ E.g.
  \begin{minipage}{0.6\linewidth}
    \begin{itemize}\squeeze
    \item Definition of factorial:\\ $n! = n \times (n-1)!$ with $0!=1$\pauseh
    \item Definition of ancestor: X is the ancestor of Y if X is the
      parent of Y or Z is the parent of Y and X is the ancestor of Z\pauseh
    \end{itemize}    
  \end{minipage}\hfill
  \begin{minipage}{0.3\linewidth}
    \multipdf[width=0.7\linewidth]{ancestors}\pause
  \end{minipage}
\end{itemize}

\end{slide}

%%%%%%%%%%%%%%%%%%%%%%% Next Slide %%%%%%%%%%%%%%%%%%%%%%%

\begin{slide}
\section{Structure of Recursion}

\begin{PauseHighLight}
  \begin{itemize}
  \item Notice that these are \textit{self-referential} definitions\pause
  \item A recursive definition consists of two elements\pause
    \begin{itemize}
    \item \emph{The Base Case:} or boundary cases where the problem is
      trivial\pause
    \item \emph{The Recursive Clause:} which is a self-referential part
      driving the problem towards the base case\pause
    \end{itemize}
  \item This should be reminiscent of proofs by induction\pause---indeed
    the two are very closely related (many mathematical functions are
    defined recursively and there properties are proved by
    induction)\pause
  \end{itemize}
\end{PauseHighLight}
\end{slide}

%%%%%%%%%%%%%%%%%%%%%%% Next Slide %%%%%%%%%%%%%%%%%%%%%%%

\begin{slide} 
\section[-2]{Recap on Proof by Induction}

\pausebuild
\begin{itemize}\squeeze
\item Let us prove
  \begin{align*}
    S_n = \sum_{i=0}^n 2^i\pause = 1 + 2 + 4 +\cdots + 2^n\pause = f(n)\pause =
    2^{n+1}-1 \pause
  \end{align*}
\item We use induction\pause
  \begin{itemize}
  \item \emph{Base Case:} $S_0 = \sum_{i=0}^0 2^i = 1\pause$ and $f(0)
  = 2^{0+1} -1\pause=1$ \pause so $S_0 = f(0)$\pause\tick\pause 
  \item \emph{Recursive Case:} We assume $S_n = f(n)\pause = 2^{n+1}
    -1$\pause, we want to prove that $S_{n+1} = f(n+1)\pause =
    2^{(n+1)+1} -1\pause= 2^{n+2}-1$\pause\ now 
    \begin{align*}
      S_{n+1} &= \sum_{i=0}^{n+1} 2^i\pause = \sum_{i=0}^n 2^i +
      2^{n+1}\pause = S_n + 2^{n+1}\pause = f(n) + 2^{n+1}\pause\\ &= 2^{n+1} -1 +
      2^{n+1}\pause = 2\times2^{n+1} -1\pause = 2^{n+2} -1 \pause
      \hspace{1em} \mbox{\tick}\pause 
    \end{align*}
  \end{itemize}
\end{itemize}

\end{slide}


%%%%%%%%%%%%%%%%%%%%%%% Next Slide %%%%%%%%%%%%%%%%%%%%%%%
\Outline
%%%%%%%%%%%%%%%%%%%%%%% Next Slide %%%%%%%%%%%%%%%%%%%%%%%

\begin{slide}
\section[-1]{Programming Recursively}

\begin{PauseHighLight}
  \begin{itemize}
  \item Most modern programming languages, including C++, allow you to
    program recursively\pause
  \item That is they allow functions/methods to be defined in terms of
    themselves
    \begin{cpp}
    long factorial(long n)
    {
        if (n<0)
            throw new IllegalArgumentException();
        else if (n==0)
            return 1;
        else
            return n*factorial(n-1);
    }$\pause$
    \end{cpp}
  \item This will work, is very intuitive, but is certainly not the best
    way to code a factorial function\pause
  \end{itemize}
\end{PauseHighLight}
\end{slide}

%%%%%%%%%%%%%%%%%%%%%%% Next Slide %%%%%%%%%%%%%%%%%%%%%%%

\begin{slide}
\section[-2]{Integer Powers}
\toptarget{examples}

\pausebuild
\begin{itemize}\squeeze
\item How do we compute $0.95^{25}$?\pause
\item One way is to multiply together 0.95 twenty five times\pause
\item A more efficient way is to observe
  \begin{align*}
    x^{2n} &= (x^n)^2 \\
    x^{2n+1} &= x\times x^{2n}\pause
  \end{align*}\vspace{-1cm}
\item We can repeat this until we reach $x^1=x$\pause
  \begin{align*}
    \hspace{-1cm}0.95^{25}\pause &= 0.95\times(0.95)^{24}\pause 
    = 0.95\times\bigl((0.95)^{12}\bigr)^2\pause
    = 0.95\times\Bigl(\bigl((0.95)^6\bigr)^2\Bigr)^2\pause \\
    &=0.95\times\Bigl(\Bigl(\bigl((0.95)^3\bigr)^2\Bigr)^2\Bigr)^2\pause
    =0.95\times\Bigl(\Bigl(\bigl(0.95\times(0.95)^2\bigr)^2\Bigr)^2\Bigr)^2
    \pause
  \end{align*}
  six multiplications rather than 24!\pause
\end{itemize}
\end{slide}



%%%%%%%%%%%%%%%%%%%%%%% Next Slide %%%%%%%%%%%%%%%%%%%%%%%

\begin{slide}
\section[-1]{Implementing Integer Power}

\begin{PauseHighLight}
  \begin{itemize}
  \item Integer power looks rather intimidating to code\pause
  \item However, the recursive definition is easy\pause
  \item We can easily code this function recursively
    \begin{java}
    double power(double x, long n)   //  (Overflow is possible)
    {
        return  n <  0  ? 1 / power(x,-n)          //  Negative power 
            :  n == 0   ? 1                        //  Special case
            :  n == 1   ? x                        //  Base case
            :  n%2 == 0 ? (x = power(x, n/2)) * x  //  Even power
            :              x * power(x, n-1);      //  Odd power
    }$\pause$
    \end{java}
  \end{itemize}
\end{PauseHighLight}
\end{slide}

%%%%%%%%%%%%%%%%%%%%%%% Next Slide %%%%%%%%%%%%%%%%%%%%%%%

\begin{slide}
\section[-2]{Helper Functions}

\begin{PauseHighLight}\squeeze
  \begin{itemize}
  \item This is a slick implementation from the web, but not
    terribly efficient\pause
  \item We only need to do the first two checks once\pause
  \item A more efficient implementation would use a helper function
    \begin{java}
    double power(double x, long n) { //  (Overflow is possible)
        return  n <  0  ? power_recurse(1.0/x,-n)  //  Negative power 
            :  n == 0   ? 1                        //  Special case
            : power_recurse(x,n);

    }$\pause$

    double power_recurse(double x, long n) {
       return  n == 1  ? x                               //  Base case
           : n%2 == 0 ? (x = power_recurse(x, n/2)) * x  //  Even power
           :             x * power_recurse(x, n-1);      //  Odd power
    }$\pause$
    \end{java}
  \end{itemize}
\end{PauseHighLight}

\end{slide}



%%%%%%%%%%%%%%%%%%%%%%% Next Slide %%%%%%%%%%%%%%%%%%%%%%%

\begin{slide}
\section{Writing Recursive Programs}
\toptarget{progthink}

\begin{PauseHighLight}
\begin{itemize}
\item You need to make sure that you catch the base case \textbf{before}
  you recurse\pause
\item The recursive case can call itself, possibly many times, provided
  the inductive argument is closer to the base case\pause
\item That is,
  \begin{itemize}
  \item Ensure that you use a `smaller problem'\pause
  \item Assume that you can solve the `smaller problem'\pause
  \end{itemize}
\end{itemize}
\end{PauseHighLight}
\end{slide}

%%%%%%%%%%%%%%%%%%%%%%% Next Slide %%%%%%%%%%%%%%%%%%%%%%%

\begin{slide}
\section[-1]{The Cost of Recursion}

\begin{PauseHighLight}
  \begin{itemize}
  \item Recursion acts just like any other function call\pause
  \item The values of all local variables in scope are put on a stack\pause
  \item The function is called and\pause
    \begin{itemize}
    \item returns a value\pause\ or
    \item change some variable or object\pause
    \end{itemize}
  \item When the function returns, the values stored on the stack are
    popped and the local variables restored to their original state\pause
  \item Although this operation is well optimised it is time consuming\pause
  \end{itemize}
\end{PauseHighLight}

\end{slide}

%%%%%%%%%%%%%%%%%%%%%%% Next Slide %%%%%%%%%%%%%%%%%%%%%%%

\begin{slide}
\section{Unrolling Recursion}

\begin{PauseHighLight}
  \begin{itemize}
  \item Recursion can frequently be replaced\pause
  \item E.g. we can easily write a factorial function
    \begin{java}
    long factorial(long n)
    {
        if (n<0)
            throw new IllegalArgumentException();
        long res = 1;
        for (int i=2; i<=n; i++)
            res *= i;
        return res;
    }$\pause$
    \end{java}
    with no function calls this will run much faster than the recursive
    version\pause
  \end{itemize}
\end{PauseHighLight}
\end{slide}
%%%%%%%%%%%%%%%%%%%%%%% Next Slide %%%%%%%%%%%%%%%%%%%%%%%

\begin{slide}
\section[-1]{The Greatest Common Denominator}

\begin{PauseHighLight}
  \begin{itemize}
  \item One of the most famous algorithms is Euclid's algorithm for
    calculating the greatest common denominator\pause
  \item The greatest common denominator of $A$ and $B$ is the largest
    integer, $C$, which exactly divides $A$ and $B$\pause
  \item E.g. the greatest common denominator of 70 and 25 is 5\pause
  \item Euclid's algorithms uses the fact that
    \begin{itemize}
    \item $\gcd(A, B) = \gcd(B, A \mod B)$
    \item $\gcd(A, 0) = A$\pause
    \end{itemize}
  \item Thus $\gcd(70, 25)\pause = \gcd(25, 20)\pause = \gcd(20,
  5)\pause = \gcd(5, 0)\pause =5\pause$ 
  \end{itemize}
\end{PauseHighLight}
\end{slide}

%%%%%%%%%%%%%%%%%%%%%%% Next Slide %%%%%%%%%%%%%%%%%%%%%%%

\begin{slide}
\section{Implementation of GCD}

\begin{itemize}
  \item The implementation of gcd is trivial using recursion
    \begin{java}
      long gcd(long a, long b)
      {

           if (b==0) {
              return a;
           }
           return gcd(b, a%b);
      }
    \end{java}
\end{itemize}

\end{slide}

%%%%%%%%%%%%%%%%%%%%%%% Next Slide %%%%%%%%%%%%%%%%%%%%%%%

\begin{slide}
\section{Implementation of GCD}

\begin{itemize}
  \item The implementation of gcd is trivial using recursion
    \begin{java}
      long gcd(long a, long b)
      {

           if (b==0) {
              return a;
           }
           long c = a%b;
           a = b;
           b = c;
           return gcd(a, b);
      }
    \end{java}
\end{itemize}

\end{slide}

%%%%%%%%%%%%%%%%%%%%%%% Next Slide %%%%%%%%%%%%%%%%%%%%%%%

\begin{slide}
\section{Implementation of GCD}

\begin{itemize}
  \item The implementation of gcd is trivial using recursion
    \begin{java}
      long gcd(long a, long b)
      {
        while(true) {
           if (b==0) {
              return a;
           }
           long c = a%b;
           a = b;
           b = c;
         }
      }
    \end{java}
  \item Example of tail recursion
\end{itemize}

\end{slide}



%%%%%%%%%%%%%%%%%%%%%%% Next Slide %%%%%%%%%%%%%%%%%%%%%%%

\begin{slide}
\section[-1]{When Definitely not to Recurse}

\begin{PauseHighLight}
\begin{itemize}
\item A classic recursively defined sequence is the Fibonacci series
  \begin{itemize}
  \item $f_n = f_{n-1} + f_{n-2}$
  \item $f_1 = f_2 = 1$\pause
  \end{itemize}
\item We might be tempted to write a recursive function to define the
  series
  \begin{java}
    long fibonacci(long n)
    {
      if (n<=2)
         return 1;
      return fibonacci(n-1) + fibonacci(n-2);
    }$\pause$
  \end{java}
  \emph{Why shouldn't you want to do this?}\pause
\end{itemize}
\end{PauseHighLight}
\end{slide}

%%%%%%%%%%%%%%%%%%%%%%% Next Slide %%%%%%%%%%%%%%%%%%%%%%%

\begin{slide}
\section[-1]{Fibonacci}

\pb
\begin{java}
  long fib(long n)
  {
    if (n<=2)
      return 1;
    return fib(n-1) + fib(n-2);
  }
\end{java}\pause

\begin{center}
  \multipdf[width=0.8\linewidth]{fibonacci}\pause
\end{center}

\end{slide}


%%%%%%%%%%%%%%%%%%%%%%% Next Slide %%%%%%%%%%%%%%%%%%%%%%%

\begin{slide}
\section[-1]{Why Use Recursion At All?}

\begin{PauseHighLight}
  \begin{itemize}
  \item Both \texttt{factorial} and \texttt{gcd} could be written
    without using recursion\pause
  \item The programs would probably run faster\pause
  \item The \texttt{gcd} program would be less clear\pause
  \item The cost of the additional function calls is often
    insignificant\pause
  \item It would considerably harder to write many programs such as
    \jl$power$ non-recursively\pause
  \item Later we will see algorithms like \texttt{quick sort} which rely on
    recursion\pause
  \end{itemize}
\end{PauseHighLight}
\end{slide}

%%%%%%%%%%%%%%%%%%%%%%% Next Slide %%%%%%%%%%%%%%%%%%%%%%%
\Outline
%%%%%%%%%%%%%%%%%%%%%%% Next Slide %%%%%%%%%%%%%%%%%%%%%%%

\begin{slide}
\section[-2]{Analysis}

\begin{PauseHighLight}
\begin{itemize}\squeeze
\item We can use recursion to compute the time complexity of a recursive
  program!\pause
\item To do this we denote the time taken to solve a problem of size $n$
  by $T(n)$\pause
\item To compute the time complexity of factorial, we note that to
  compute $n!$ we have to multiply $n$ by $(n-1)!$\pause
\item That is, the number of multiplications we need to compute is
  \begin{align*}
    T(n) = T(n-1) +1\pause
  \end{align*}
\item Now $T(0)=0$ so
  \begin{align*}
    T(n) = T(n-1) +1\pause = T(n-2) +2\pause = \cdots = T(0) +n\pause =
    n\pause
  \end{align*}
\end{itemize}
\end{PauseHighLight}
\end{slide}

%%%%%%%%%%%%%%%%%%%%%%% Next Slide %%%%%%%%%%%%%%%%%%%%%%%

\begin{slide}
\section[-1]{Time to Compute Power}
\toptarget{power}

\begin{PauseHighLight}
  \begin{itemize}\squeeze
  \item How long does it take to compute $x^n$?\pause
  \item Remember
  \begin{align*}
    x^{2n} &= (x^n)^2 \\
    x^{2n+1} &= x\times x^{2n}\pause
  \end{align*}\vspace{-1cm}
  \item Thus
    \begin{align*}
      T(n) &= \left\{
      \begin{array}{lll}
        T(n/2) + 1 & \quad & \mbox{if $n$ is even}\\
        T((n-1)/2) + 2 & \quad & \mbox{if $n$ is odd}\pause
      \end{array}\right. \\
    &\leq T(\lfloor n/2 \rfloor) + 2
    \end{align*}
  \item Where $T(1) = 0$\pause
  \end{itemize}
\end{PauseHighLight}
\end{slide}

%%%%%%%%%%%%%%%%%%%%%%% Next Slide %%%%%%%%%%%%%%%%%%%%%%%

\begin{slide}
\section[-1]{How many times?}

\begin{PauseHighLight}
  \begin{itemize}
  \item We want to solve $T(n)\leq T(\lfloor n/2 \rfloor) + 2$ with
    $T(1)=0$\pause
  \item How many times do we divide $n$ by two until we reach 1?\pause
  \item Denoting $n$ by a binary number $n=b_m b_{m-1} \cdots b_2 b_1$
    \begin{itemize}\squeeze
    \item $b_i\in\{0,1\}$\pause
    \item $b_m=1$ \pause
    \item $m$ is the number of digits in the binary representation of
    $n$\pause
    \item $\lfloor n/2 \rfloor=b_m b_{m-1} \cdots b_2$\pause
    \item After $m-1$ `divides' we reach $1$\pause
    \end{itemize}  
  \item Thus $T(n) \leq 2 (m-1)$\pause
  \end{itemize}
\end{PauseHighLight}
\end{slide}

%%%%%%%%%%%%%%%%%%%%%%% Next Slide %%%%%%%%%%%%%%%%%%%%%%%

\begin{slide}
\section{How Big is $m$}

\begin{PauseHighLight}
  \begin{itemize}
  \item How many binary digits do you need to represent an integer $n$\pause
  \item Note that an $m$ digit number can represent a number from $2^m$ to
    $2^{m+1}-1$\pause
  \item Thus
    \begin{align*}
      2^m &\leq n < 2^{m+1} \pause\\
      m &\leq \log_2(n) < m+1\pause
    \end{align*}
  \item But $T(n) \leq 2 (m-1)\pause \leq 2 (\log_2(n)-1)\pause = \Theta(\log_2(n))$\pause
  \end{itemize}
\end{PauseHighLight}
\end{slide}

%%%%%%%%%%%%%%%%%%%%%%% Next Slide %%%%%%%%%%%%%%%%%%%%%%%

\begin{slide}
\section[-2]{Towers of Hanoi}
\toptarget{hanoi}

\begin{PauseHighLight}
\begin{quote}
In an ancient city, so the legend goes, monks in a temple had to move a
pile of 64 sacred disks from one location to another. The disks were
fragile; only one could be carried at a time. A disk could not be placed
on top of a smaller, less valuable disk. In addition, there was only one
other location in the temple (besides the original and destination
locations) sacred enough for a pile of disks to be placed there.\pause

Using the intermediate location, the monks began to move disks back and
forth from the original pile to the pile at the new location, always
keeping the piles in order (largest on the bottom, smallest on the
top). According to the legend, before the monks could make the final
move to complete the new pile in the new location, the temple would turn
to dust and the world would end. \pause
\end{quote}
\end{PauseHighLight}

\end{slide}

%%%%%%%%%%%%%%%%%%%%%%% Next Slide %%%%%%%%%%%%%%%%%%%%%%%

\begin{slide}
\section{A Smaller Tower of Hanoi}
\pb
\pausebuild
\begin{itemize}
\item Here is a smaller problem of just four disks\pauseh
\end{itemize}
\begin{center}
  \multipdf[width=\linewidth]{hanoi}\pause
\end{center}
\end{slide}

%%%%%%%%%%%%%%%%%%%%%%% Next Slide %%%%%%%%%%%%%%%%%%%%%%%

\begin{slide}
\section[-1]{Algorithms in the Real World}

\begin{PauseHighLight}
  \begin{itemize}
  \item We require an algorithm to solve the towers of Hanoi\pause
  \item Algorithms don't just apply to computers!\pause
  \item If you try to solve the problem by hand you will discover that
    its quite fiddly\pause
  \item There is a simple recursive solution which turns out to be
    optimal\pause
  \item Let \jl$move(X,Y)$ denote the procedure of moving the top disk from
  peg \jl$X$ to peg \jl$Y$\pause
  \item Let \jl$hanoi(n,X,Y,Z)$ denote the procedure of moving the top
    \jl$n$ disks from peg \jl$X$ to peg \jl$Z$ using peg \jl$Y$\pause
  \end{itemize}
\end{PauseHighLight}

\end{slide}

%%%%%%%%%%%%%%%%%%%%%%% Next Slide %%%%%%%%%%%%%%%%%%%%%%%

\begin{slide}
\section[-1]{Solving Towers of Hanoi}
\pausebuild
\begin{java}
  hanoi(n, A, B, C)
  {
    if (n>0) {$\hll$
       hanoi(n-1, A, C, B);$\hl{3}$
       move(A, C);$\hl{4}$
       hanoi(n-1, B, A, C);$\hl{5}$
    }
  }$\hll$
\end{java}\pauselevel{=2}
\begin{center}
  \multipdf[width=\linewidth]{solve_hanoi}\pause
\end{center}
\end{slide}

%%%%%%%%%%%%%%%%%%%%%%% Next Slide %%%%%%%%%%%%%%%%%%%%%%%

\begin{slide}
\section{Optimality of Solution}

\begin{PauseHighLight}
  \begin{itemize}
  \item This is optimal\pause\ because
    \begin{itemize}
    \item You have to move the largest disk from peg A to peg C\pause
    \item We do this only once\pause
    \item To make this move all the other disks must be on peg B\pause
    \item Assuming that we solve the $n-1$ disk problem optimally then we
      solve the $n$ disk problem optimally\pause
    \item We solve the one disk problem optimally (i.e. we move it to
      where it should go)\pause
    \item This completes a proof by induction!\pause 
    \end{itemize}
  \end{itemize}
\end{PauseHighLight}
\end{slide}


%%%%%%%%%%%%%%%%%%%%%%% Next Slide %%%%%%%%%%%%%%%%%%%%%%%

\begin{slide}
\section{How Long Does It Take?}
\pausebuild
\begin{itemize}
\item How many moves does it take to transfer $n$ disks?\pause
\item We can use the same procedure as before
\begin{java}
  hanoi(n, A, B, C)
  {
    if (n>0) {
       hanoi(n-1, A, C, B);
       move(A, C);
       hanoi(n-1, B, A, C);
    }
  }$\pause$
\end{java}
\begin{itemize}
\item $T(n) = 2\,T(n-1) + 1$
\item $T(0) = 0$\pause
\end{itemize}
\end{itemize}
\end{slide}

%%%%%%%%%%%%%%%%%%%%%%% Next Slide %%%%%%%%%%%%%%%%%%%%%%%

\begin{slide}
\section{Lets Enumerate}

\begin{itemize}\squeeze
\item $T(n) = 2\,T(n-1) + 1$\pause
\item $T(0) = 0$\pause
\item $T(1) = 2\times0 + 1 = 1$\pause
\item $T(2) = 2\times1 + 1 = 2+1 = 3$\pause
\item $T(3) = 2\times3 + 1 = 6+1 = 7$\pause
\item $T(4) = 2\times7 + 1 = 14+1 = 15$\pause
\item Looks like $T(n) = 2^n-1$\pause
\end{itemize}
\end{slide}

%%%%%%%%%%%%%%%%%%%%%%% Next Slide %%%%%%%%%%%%%%%%%%%%%%%

\begin{slide}
\section{Proof by Induction}

\begin{itemize}
\item $T(n) = 2\,T(n-1) + 1$
\item $T(0) = 0$\pause
\item We want to prove $T(n) = 2^n-1$\pause
\item Base case: $T(0) = 2^0-1\pause=1-1\pause=0$\pause\,\tick\pause
\item Recursive case:\pause\ Assume $T(n-1) = 2^{n-1}-1$\pause\ then
  \begin{align*}
    T(n) &= 2\,T(n-1) + 1\pause\\
    T(n) &= 2\times(2^{n-1}-1) + 1\pause\\
    T(n) &= 2^n - 2 + 1 \pause = 2^n - 1\pause\,\mbox{\tick}\pause
  \end{align*}
\end{itemize}
\end{slide}

%%%%%%%%%%%%%%%%%%%%%%% Next Slide %%%%%%%%%%%%%%%%%%%%%%%

\begin{slide}
\section[-1]{Time Complexity of Recursion}

\begin{PauseHighLight}
  \begin{itemize}
  \item The time complexity for recursion can be tricky to
    calculate\pause
  \item The procedure is to calculate the time, $T(n)$, taken for a
    problem of size $n$ in terms of the time taken for a smaller
    problem\pause
  \item The difficulty is to solve the recursion\pause
  \item Recursive programs can be very quick (e.g. $O(\log n)$ for
    computing integer powers)\pause
  \item Recursive programs can also be very slow, (e.g. $O(2^n)$ for
    towers of Hanoi)\pause
  \item In case your interested, if it takes 1 second to move a disk it
    will take almost $585\,000\,000\,000$ years to move 64 disks\pauseb
  \end{itemize}
\end{PauseHighLight}

\end{slide}


%%%%%%%%%%%%%%%%%%%%%%% Next Slide %%%%%%%%%%%%%%%%%%%%%%%

\begin{slide}
\section[-1]{Lessons}

\begin{PauseHighLight}
  \begin{itemize}
  \item Recursion is a powerful tool for writing algorithms\pause
  \item It often provides simple algorithms to otherwise complex
    problems\pause
  \item Recursion comes at a cost (extra function calls)\pause
  \item There are times when you should avoid recursion (computing
    Fibonacci numbers)\pause
  \item You need to be able to analyse the time complexity of
    recursion\pause
  \item Used appropriately, recursion is fantastic!\pause
  \end{itemize}
\end{PauseHighLight}
\end{slide}
