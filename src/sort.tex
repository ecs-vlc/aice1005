%Master File:lectures.tex

\lesson{Sort Wisely}

\vspace{-2cm}
\begin{center}
  \includegraphics[height=10cm]{sorting}
\end{center}

\keywords{Merge sort, quick sort and radix sort}

%%%%%%%%%%%%%%%%%%%%%%% Next Slide %%%%%%%%%%%%%%%%%%%%%%%
\renewcommand{\Outline}{%
\begin{slide}
\section{Outline}
\vfill
\begin{minipage}{10cm}
  \vfill
  \begin{enumerate}
    \outlineitem{Merge Sort}{merge}
    \outlineitem{Quick Sort}{quick}
    \outlineitem{Radix Sort}{radix}
  \end{enumerate}
  \vfill
\end{minipage}\hfill
\begin{minipage}{10cm}
  \includegraphics[width=10cm]{sorting}
\end{minipage}
\vfill
\end{slide}
\addtocounter{outlineitem}{1}
}

\setcounter{outlineitem}{1}
\Outline
\toptarget{firstoutline}

%%%%%%%%%%%%%%%%%%%%%%% Next Slide %%%%%%%%%%%%%%%%%%%%%%%

\begin{slide}
\section{Merge Sort}

\begin{PauseHighLight}
  \begin{itemize}
  \item Merge sort is an example of sort performed in log-linear
    (i.e. $O(n\log(n))$) time complexity\pause
  \item It was invented in 1945 by John von Neumann\pause
  \item It is an example of a divide-and-conquer strategy\pause
    \begin{itemize}
    \item That is, the problem is divided into a number of parts
      recursively\pause
    \item The full solution is obtained by recombining the parts\pause
    \end{itemize}
  \end{itemize}
\end{PauseHighLight}

\end{slide}

%%%%%%%%%%%%%%%%%%%%%%% Next Slide %%%%%%%%%%%%%%%%%%%%%%%

\begin{slide}
\section[2]{Algorithm}

\begin{minipage}{10cm}
  \begin{pseudo}
#\textsc{MergeSort}#($\bm{a}$)
{
  if $n>1$
    copy $\bm{a}[1:\lfloor n/2 \rfloor]\,\,$ to $\bm{b}$
    copy $\bm{a}[\lfloor n/2 \rfloor+1:n]\,\,$ to $\bm{c}$
    #\textsc{MergeSort}#($\bm{b}$)
    #\textsc{MergeSort}#($\bm{c}$)
    #\textsc{Merge}#($\bm{b}$,$\bm{c}$,$\bm{a}$)
  endif
} $\pause$
  \end{pseudo}
\end{minipage}\hfil
\begin{minipage}{12cm}
  \includegraphics[width=13cm]{mergesort8.mps}
\end{minipage}
\end{slide}

%%%%%%%%%%%%%%%%%%%%%%% Next Slide %%%%%%%%%%%%%%%%%%%%%%%

\begin{slide}
\section[-2]{Merge}
\pausebuild
\begin{minipage}{10cm}
  \begin{pseudo}
#\textsc{Merge}#($\bm{b}[1:p]$,$\bm{c}[1:q]$,$\bm{a}[1:p+q]$)
{
  i:=1
  j:=1
  k:=1
  while $i \leq p\,\,$ and $j\leq q\,\,$ do
    if $b_i \leq c_j$
      $a_k$ := $b_i$
      i := i+1
    else
      $a_k$ := $c_j$
      j := j+1
    endif
    k := k+1
  end
  if i=p
    copy $\bm{c}[j:q]\,\,$ to $\bm{a}[k:p+q]$
  else
    copy $\bm{c}[i:q]\,\,$ to $\bm{a}[k:p+q]$
} #\pauseh#
  \end{pseudo}
\end{minipage}\hfil
\begin{minipage}{12cm}
  \multiinclude[graphics={width=14cm}]{merge}\pause
\end{minipage}
\end{slide}

%%%%%%%%%%%%%%%%%%%%%%% Next Slide %%%%%%%%%%%%%%%%%%%%%%%

\begin{slide}
\section{Properties of Merge Sort}

\begin{PauseHighLight}
  \begin{itemize}
  \item Merge sort is stable provided we merge carefully (i.e. it
    preserves the order of two entries with the same value)\pause
  \item Merge sort isn't in-place\pause---we need an array of at most
    size $n$ to do the merging\pause
  \item Merging is quick.\pause\  Given two arrays of size $n$ the most number
    of comparisons we need to perform is $n-1$\pause
  \end{itemize}
\end{PauseHighLight}

\end{slide}


%%%%%%%%%%%%%%%%%%%%%%% Next Slide %%%%%%%%%%%%%%%%%%%%%%%

\begin{slide}
\section[1]{Time Complexity of Merge Sort}

\pausebuild
\begin{center}
  \begin{picture}(230,100)(10,-5)
    \put(0,20){\includegraphics[width=10cm]{mergesort8.mps}}\pause
    \put(0,10){\line(0,-1){10}}
    \put(100,10){\line(0,-1){10}}
    \put(35,5){\vector(-1,0){35}}
    \put(65,5){\vector(1,0){35}}
    \put(50,5){\makebox(0,0){$O(n)$}}\pause
    \put(110,20){\line(1,0){10}}
    \put(110,90){\line(1,0){10}}
    \put(110,55){\line(1,0){10}}
    \put(115,30){\vector(0,-1){10}}
    \put(115,45){\vector(0,1){10}}
    \put(125,37.5){\makebox(0,0){$O\left(\strut\log_2(n)\right)$}}
    \put(115,65){\vector(0,-1){10}}
    \put(115,80){\vector(0,1){10}}
    \put(125,72.5){\makebox(0,0){$O\left(\strut\log_2(n)\right)$}}
    \pause    
  \end{picture}
  $n=8$
\end{center}
\end{slide}

%%%%%%%%%%%%%%%%%%%%%%% Next Slide %%%%%%%%%%%%%%%%%%%%%%%

\begin{slide}
\section[1]{Time Complexity of Merge Sort}

\begin{center}
  \begin{picture}(230,100)(10,-5)
    \put(0,5){\line(0,-1){10}}
    \put(220,5){\line(0,-1){10}}
    \put(95,0){\vector(-1,0){95}}
    \put(125,0){\vector(1,0){95}}
    \put(110,0){\makebox(0,0){$O(n)$}}
    \put(225,10){\line(1,0){10}}
    \put(225,100){\line(1,0){10}}
    \put(225,55){\line(1,0){10}}
    \put(230,20){\vector(0,-1){10}}
    \put(230,45){\vector(0,1){10}}
    \put(235,32.5){\makebox(0,0){$O\left(\strut\log_2(n)\right)$}}
    \put(230,65){\vector(0,-1){10}}
    \put(230,90){\vector(0,1){10}}
    \put(235,77.5){\makebox(0,0){$O\left(\strut\log_2(n)\right)$}}
    \put(0,9){\includegraphics[width=22cm]{mergesort16.mps}}\pause
  \end{picture}
  $n=16$
\end{center}
\end{slide}

%%%%%%%%%%%%%%%%%%%%%%% Next Slide %%%%%%%%%%%%%%%%%%%%%%%

\begin{slide}
\section[-1]{Time Complexity}

\begin{PauseHighLight}
  \begin{itemize}
  \item We again measure the complexity in the number of comparisons\pause
  \item From the above argument $C(n) = O(n \times \log_2(n))$\pause
  \item We can be a bit more formal
    \begin{align*}
      C(n) &= 2 C(\lfloor n/2 \rfloor) + C_{\mbox{merge}}(n) & \mbox{for $n>1$}\\
      C(0) &= 1 &\pause
    \end{align*}
  \item But in the worst case $C_{\mbox{merge}}(n)=n-1$\pause
  \item Leads to $C_{\mbox{worst}}(n) = n \log_2(n) -n + 1$\pause
  \end{itemize}
\end{PauseHighLight}

\end{slide}

%%%%%%%%%%%%%%%%%%%%%%% Next Slide %%%%%%%%%%%%%%%%%%%%%%%

\begin{slide}
\section[-1.5]{General Time Complexity}

\begin{PauseHighLight}
  \begin{itemize}
  \item In general if we have a recursion formula
    \begin{align*}
      T(n) = a T(n/b) + f(n)\pause
    \end{align*}
    with $a\geq1$, $b>1$\pause
  \item If $f(n) \in \Theta(n^d)$ where $d\geq 0$ then
    \begin{align*}
      T(n) \in \left\{
        \begin{array}{lll}
          \Theta\left(n^d \right) & & \mbox{if $a<b^d$} \\
          \Theta\left(n^d \log(n)\right) & & \mbox{if $a=b^d$} \\
          \Theta\left(n^{\log_d(a)}\right) & & \mbox{if $a>b^d$}
        \end{array}\right.\pause
    \end{align*}
  \item Analogous results hold for the family $O$ and $\Omega$\pause
  \end{itemize}
\end{PauseHighLight}
\end{slide}

%%%%%%%%%%%%%%%%%%%%%%% Next Slide %%%%%%%%%%%%%%%%%%%%%%%

\begin{slide}
\section[-2]{Mixing Sort}

\begin{PauseHighLight}
  \begin{itemize}
  \item For very short sequences it is faster to use insertion sort than
  to pay the overhead of function calls\pause
  \end{itemize}
  \begin{center}\color{TextColor}
    \includegraphics[width=\linewidth]{mergeinsort.mps}\pause
  \end{center}
\end{PauseHighLight}

\end{slide}


%%%%%%%%%%%%%%%%%%%%%%% Next Slide %%%%%%%%%%%%%%%%%%%%%%%
\Outline % Quick Sort
%%%%%%%%%%%%%%%%%%%%%%% Next Slide %%%%%%%%%%%%%%%%%%%%%%%

\begin{slide}
\section{Quicksort}

\begin{PauseHighLight}
  \begin{itemize}
  \item The most commonly used fast sorting algorithm is
    \emph{quicksort}\pause 
  \item It was invented by the British computer scientist  by
    C.~A.~R.~Hoare in 1962\pause
  \item It again uses the divide-and-conquer strategy\pause
  \item It can be performed in-place, but it is \textbf{not}
    stable\pause
  \item It works by splitting an array into two depending on whether the
    elements are less than or greater than a \emph{pivot} value\pause
  \item This is done recursively until the full array is sorted\pause
  \end{itemize}
\end{PauseHighLight}

\end{slide}

%%%%%%%%%%%%%%%%%%%%%%% Next Slide %%%%%%%%%%%%%%%%%%%%%%%

\begin{slide}
\section[-1]{Partition}

\pausebuild
\begin{itemize}
\item We need to partition the array around the pivot $p$ such that
  \begin{center}
    \includegraphics[height=1cm]{apartition}\pauseh
  \end{center}
\end{itemize}
\begin{minipage}{6cm}
\begin{pseudo}
#\textsc{Partition}#($\bm{a}$, p, left, right)
{
  i := left
  j := right
  repeat {
    while $a_i < p$
       i++
    while $a_j$ >= $p$
       j--
    if i >= j
       break
    #\textsc{Swap}#($a_i$,$a_j$)
  }
}$\pauseh$
\end{pseudo}
\end{minipage}\hfill
\begin{minipage}{16cm}
\vfill
\begin{center}
  \multiinclude[graphics={width=16cm}]{partition}\pause
\end{center}
\vfill
\end{minipage}
\end{slide}

%%%%%%%%%%%%%%%%%%%%%%% Next Slide %%%%%%%%%%%%%%%%%%%%%%%

\begin{slide}
\section[-1]{Optimising Partitioning}

\begin{PauseHighLight}
  \begin{itemize}
  \item There are different ways of performing the partitioning\pause
  \item We want to minimise the time taken on the inner loop\pause
  \item This means we want to perform as few checks as possible\pause
  \item One method of doing this is to place \textit{sentinels} at the
  ends of the array\pause
  \item We can also reduce work by placing the partition in its correct
    position
  \begin{center}
    \includegraphics[height=1cm]{anotherpartition}\pause
  \end{center}
  \end{itemize}
\end{PauseHighLight}

\end{slide}

%%%%%%%%%%%%%%%%%%%%%%% Next Slide %%%%%%%%%%%%%%%%%%%%%%%

\begin{slide}
\section{Choosing the Pivot}

\begin{PauseHighLight}
  \begin{itemize}
  \item There are different strategies to choosing the pivot\pause
  \item Choose the first element in the array\pause
  \item Choose the median of the first, middle and last element of the
    array\pause
  \item This increases the likelihood of the pivot being close to the
    median of the whole array\pause
  \item For large arrays (above 40) the median of 3 medians is often
    used\pause
  \end{itemize}
\end{PauseHighLight}

\end{slide}

%%%%%%%%%%%%%%%%%%%%%%% Next Slide %%%%%%%%%%%%%%%%%%%%%%%

\begin{slide}
\section[-2]{Quicksort}

\begin{PauseHighLight}
  We recursively partition the array until each partition is small
  enough to sort using insertion sort\pause
  \begin{pseudo}
#\textsc{QuickSort}#($\bm{a}$, left, right) {
   if (right-left < threshold)
      #\textsc{InsertionSort}#($\bm{a}$, left, right)
   else
      pivot = #\textsc{ChoosePivot}#($\bm{a}$, left, right)
      part = #\textsc{Partition}#($\bm{a}$, pivot, left, right)
      #\textsc{QuickSort}#($\bm{a}$, left, part-1)
      #\textsc{QuickSort}#($\bm{a}$, part+1, right)
   endif$\pause$
}
  \end{pseudo}
\vspace{-1cm}
  \color{TextColor}
  \begin{center}
    \includegraphics[width=0.6\linewidth]{quickSortSchematic}\pause
  \end{center}
\end{PauseHighLight}

\end{slide}

%%%%%%%%%%%%%%%%%%%%%%% Next Slide %%%%%%%%%%%%%%%%%%%%%%%

\begin{slide}
\section{Time Complexity}

\begin{PauseHighLight}
  \begin{itemize}
  \item Partitioning an array of size $n$ takes $\Theta(n)$ operations\pause
  \item If we split the array in half then number of partitions we need
    to do is $\lceil \log_2(n) \rceil$\pause
  \item This is the best case thus quicksort is $\Omega\left(\strut
      n\log(n) \right)$\pause
  \item If the pivot is the minimum element of the array then we have to
    partition $n-1$ times\pause
  \item This is the worst case so quicksort is $O\left(n^2\right)$\pause
  \item This worst case will happen if the array is already sorted and
    we choose the pivot to be the first element in the array!\pause
  \end{itemize}
\end{PauseHighLight}

\end{slide}




%%%%%%%%%%%%%%%%%%%%%%% Next Slide %%%%%%%%%%%%%%%%%%%%%%%

\begin{slide}
\section[-1]{QuickSort}
\pb
\pause

\begin{center}
  \includegraphics[width=0.9\linewidth]{quickSort-0}\mypl{1}
  \multido{\ia=1+1,\ib=2+1}{75}%
  {\llap{\includegraphics[width=0.9\linewidth]{quickSort-\ia}\mypl{\ib}}}
\end{center}

\end{slide}


%%%%%%%%%%%%%%%%%%%%%%% Next Slide %%%%%%%%%%%%%%%%%%%%%%%

\begin{slide}
\section[-1]{Sort in Practice}

\begin{PauseHighLight}
  \begin{itemize}
  \item The STL in C++ offers three sorts
    \begin{itemize}
    \item \jl$sort()$ implemented using quicksort\pause
    \item \jl$stable_sort()$ implemented using mergesort\pause
    \item \jl$partial_sort()$ implemented using heapsort\pause
    \end{itemize}
  \item Java uses
    \begin{itemize}
    \item Quicksort to sort arrays of primitive types\pause
    \item Mergesort to sort Collections of objects\pause
    \end{itemize}
  \item Quicksort is typically fastest but has worst case quadratic time
    complexity\pause
  \end{itemize}
\end{PauseHighLight}

\end{slide}

%%%%%%%%%%%%%%%%%%%%%%% Next Slide %%%%%%%%%%%%%%%%%%%%%%%

\begin{slide}
\section{Selection}

\begin{PauseHighLight}
  \begin{itemize}
  \item A related problem to sorting is selection\pause
  \item That is we want to select the $k^{th}$ largest element\pause
  \item We could do this by first sorting the array\pause
  \item A full sort is not however necessary\pause---we can use a
    modified quicksort where we only continue to sort the part of the
    array we are interested in\pause
  \item This leads to a $\Theta(n\log(n))$ algorithm which is
    considerably faster then sorting\pause
  \end{itemize}
\end{PauseHighLight}

\end{slide}


%%%%%%%%%%%%%%%%%%%%%%% Next Slide %%%%%%%%%%%%%%%%%%%%%%%
\Outline % Radix Sort
%%%%%%%%%%%%%%%%%%%%%%% Next Slide %%%%%%%%%%%%%%%%%%%%%%%

\begin{slide}
\section{Radix Sort}

\begin{PauseHighLight}
  \begin{itemize}
  \item Can we get a sort algorithm to run faster than $O\left(\strut
      n\log(n) \right)$?\pause
  \item Our proof that this was optimal assumed we were performing
    binary decisions (is $a_i$ less than $a_j$?)\pause
  \item If we don't perform pairwise comparisons then the proof doesn't
    apply\pause
  \item Radix sort is the classic example of a sort algorithm that
  doesn't use pairwise comparisons\pause
  \end{itemize}
\end{PauseHighLight}

\end{slide}

%%%%%%%%%%%%%%%%%%%%%%% Next Slide %%%%%%%%%%%%%%%%%%%%%%%

\begin{slide}
\section{Sorting Into Buckets}

\begin{PauseHighLight}
  \begin{itemize}
  \item The idea behind radix sort is to sort the elements of an array
    into some number of buckets\pause
  \item This is done successively until the whole array is sorted\pause
  \item Consider sorting integers in decimals (base 10 or radix 10)\pause
  \item We can successively sort on the digits\pause
  \item The sort finishes when we have got through all the digits\pause
  \end{itemize}
\end{PauseHighLight}

\end{slide}

%%%%%%%%%%%%%%%%%%%%%%% Next Slide %%%%%%%%%%%%%%%%%%%%%%%

\begin{slide}
\section{Radix Sort in Action}

\pausebuild
\color{TextColor}
\begin{center}
  \multiinclude[graphics={width=\linewidth}]{radixSort}\pause
\end{center}
\end{slide}

%%%%%%%%%%%%%%%%%%%%%%% Next Slide %%%%%%%%%%%%%%%%%%%%%%%

\begin{slide}
\section{Time Complexity of Radix Sort}

\begin{PauseHighLight}
  \begin{itemize}
  \item We need not use base 10 we could use base $r$ (the radix)\pause
  \item If the maximum number to be sorted is $N$ then the number of
    iterations of radix sort is $\log_r(N)$\pause
  \item Each sort involves $n$ operations\pause
  \item Thus the total number of operations is $O\left(\strut n \lceil \log_r(N)
    \rceil \right)$\pause
  \item Since $N$ does not depend on $n$ we can write this as
  $O\left(n\right)$ \pause
  \end{itemize}
\end{PauseHighLight}

\end{slide}

%%%%%%%%%%%%%%%%%%%%%%% Next Slide %%%%%%%%%%%%%%%%%%%%%%%

\begin{slide}
\section{Bucket Sort}

\begin{PauseHighLight}
  \begin{itemize}
  \item A closely related sort is bucket sort where we divide up the
    inputs into buckets based on the most significant figure\pause
  \item We then sort the buckets on less significant figures\pause
  \item Quicksort is a bucket sort with two buckets, but where we choose
    a pivot to determine which bucket to use\pause
  \end{itemize}
\end{PauseHighLight}

\end{slide}

%%%%%%%%%%%%%%%%%%%%%%% Next Slide %%%%%%%%%%%%%%%%%%%%%%%

\begin{slide}
\section{Minimum Time for Sort}

\begin{PauseHighLight}
  \begin{itemize}
  \item Can we do better?\pause
  \item In any sort we need to examine all possible elements in the
    array\pause
  \item If there is an element that isn't examined then we don't know
    where to put it\pause
  \item Thus the lower bound on any sort algorithm is $\Omega(n)$\pause
  \end{itemize}
\end{PauseHighLight}

\end{slide}

%%%%%%%%%%%%%%%%%%%%%%% Next Slide %%%%%%%%%%%%%%%%%%%%%%%

\begin{slide}
\section[-0.5]{Practical Sort}

\begin{PauseHighLight}
  \begin{itemize}
  \item In practice, radix sort or bucket sort are rarely used\pause
  \item The overhead of maintaining the buckets make them less efficient
    than they might appear\pause
  \item Radix sort is harder to generalise to other data types than
    comparison based sorts\pause
  \item In practice quick sort and merge sort are usually
    preferred\pause
  \item Having said that there are some very neat implementations of
    radix sort\pause
  \end{itemize}
\end{PauseHighLight}

\end{slide}

%%%%%%%%%%%%%%%%%%%%%%% Next Slide %%%%%%%%%%%%%%%%%%%%%%%

\begin{slide}
\section[-2]{Comparison of Sort Algorithms}

\begin{center}
  \includegraphics[height=16cm]{sort-times}
\end{center}

\end{slide}


%%%%%%%%%%%%%%%%%%%%%%% Next Slide %%%%%%%%%%%%%%%%%%%%%%%

\begin{slide}
\section{Lessons}

\begin{PauseHighLight}
  \begin{itemize}
  \item Sort is important---it is one of the commonest high level
    operations\pause
  \item Merge sort and quick sort are the most commonly used sort\pause
  \item There are sorts that have a better time complexity that
    quicksort\pause
  \item In practice it is difficult to beat quicksort\pause
  \end{itemize}
\end{PauseHighLight}

\end{slide}
