%Master File:lectures.tex

\lesson{Learn to Traverse Graphs}
\vspace{-1cm}
\begin{center}
  \includegraphics[height=10cm]{bipartiteGraph1}
\end{center}
\keywords{Breadth First Search, Depth First Search, Topological Sort}


%%%%%%%%%%%%%%%%%%%%%%% Next Slide %%%%%%%%%%%%%%%%%%%%%%%
\renewcommand{\Outline}{%
\begin{slide}
\section[1]{Outline}

\begin{minipage}{12cm}
  \begin{enumerate}\squeeze
    \outlineitem{Breadth First Search}{bfs}
    \begin{itemize}
    \item \toplink{bfsapp}{BFS applications}
    \end{itemize}
    \outlineitem{Depth First Search}{dfs}
    \begin{itemize}
    \item \toplink{dfsapp}{DFS applications}
    \end{itemize}
    \outlineitem{Topological Sort}{topsort}
  \end{enumerate}
\end{minipage}\hfill
\begin{minipage}{10cm}
  \includegraphics[width=10cm]{graphArticulation2}
\end{minipage}
\end{slide}
\addtocounter{outlineitem}{1}
}

\setcounter{outlineitem}{1}
\Outline
\toptarget{firstoutline}

%%%%%%%%%%%%%%%%%%%%%%% Next Slide %%%%%%%%%%%%%%%%%%%%%%%

\begin{slide}
\section{Basic Graph Algorithms}

\begin{PauseHighLight}
  \begin{itemize}
  \item Graphs provide an abstraction for a huge number of real world
    processes: social networks, compute network, road networks, etc.\pause
  \item Increasing applications focus on very large (sparse) graphs
    (usually implemented using an adjacency list)\pause
  \item Require (near) linear time algorithms\pause
  \item Basic building block are graph traversal algorithms
    \begin{itemize}
    \item Breadth First Search
    \item Depth First Search\pause
    \end{itemize}
  \end{itemize}
\end{PauseHighLight}

\end{slide}

%%%%%%%%%%%%%%%%%%%%%%% Next Slide %%%%%%%%%%%%%%%%%%%%%%%

\begin{slide}
\section[-2]{Advanced Generic}

\begin{PauseHighLight}
  \begin{itemize}
  \item To make these algorithm general purpose (generic) we allow
    ourselves to call arbitrary functions to act on the vertices and
    edges at different points in the algorithm\pause
  \item This introduces a new level of generics which makes the
    algorithms very powerful\pause
  \item Increases the steepness of the learning curve to use these
    algorithms\pause
  \item Once you get familiar with using these algorithms this level of
    generics starts to pay off\pause
  \item Libraries which does this include Boost Graph Library and LEDA
    in C++; JDSL and  JGraphT in Java\pause
  \end{itemize}
\end{PauseHighLight}

\end{slide}

%%%%%%%%%%%%%%%%%%%%%%% Next Slide %%%%%%%%%%%%%%%%%%%%%%%

\begin{slide}
\section[-2]{Graph Traversal}

\begin{PauseHighLight}
  \pb
  \begin{itemize}
  \item To traverse a graph we start at a (arbitrary) \textit{root}
    vertex\pauseh
  \item We the follow edges to create a tree\pauseh
\begin{center}
  \includegraphics[width=0.7\linewidth]{graphTraversal0}\mypl{1}
  \multido{\ia=1+1,\ib=2+1}{12}%
  {\llap{\includegraphics[width=0.7\linewidth]{graphTraversal\ia}\mypl{\ib}}}
\end{center}

  \end{itemize}
\end{PauseHighLight}

\end{slide}

%%%%%%%%%%%%%%%%%%%%%%% Next Slide %%%%%%%%%%%%%%%%%%%%%%%

\begin{slide}
\section[-2]{Breadth First Search}
\pb
\pause
\begin{center}
  \includegraphics[width=0.9\linewidth]{graphBfs0}\mypl{1}
  \multido{\ia=1+1,\ib=2+1}{52}%
  {\llap{\includegraphics[width=0.9\linewidth]{graphBfs\ia}\mypl{\ib}}}
\end{center}

\end{slide}

%%%%%%%%%%%%%%%%%%%%%%% Next Slide %%%%%%%%%%%%%%%%%%%%%%%

\begin{slide}
\section[-1]{Applications of Breadth First Search}
\toptarget{bfsapp}

\begin{PauseHighLight}
  \begin{itemize}
  \item Breadth first search can be used to find the shortest path from
    a source node to a destination node for an \emph{unweighted} graph
    \begin{itemize}\squeeze
    \item Run \texttt{bfs(graph,source)}
    \item Use parent information to find path from destination back to
      source\pause
    \end{itemize}
  \item BFS (as well as DFS) can be used to find connected components
    \begin{itemize}
    \item Use \textit{processVertexEarly} to mark vertices connected to
      the current connected component
    \item Run \textit{bfs} from all vertices that are not labelled\pause
    \end{itemize}
  \end{itemize}
\end{PauseHighLight}

\end{slide}

%%%%%%%%%%%%%%%%%%%%%%% Next Slide %%%%%%%%%%%%%%%%%%%%%%%

\begin{slide}
\section[-1]{Bipartite Graphs}

\begin{PauseHighLight}
  \begin{itemize}
  \item Bipartite graphs are graphs where the vertices can be split into
    two sets so that there are no edges between vertices in the same
    graph
    \begin{center}
      \includegraphics[width=0.8\linewidth]{bipartiteGraph0}\pause
      \llap{\includegraphics[width=0.8\linewidth]{bipartiteGraph1}}\pauseb
    \end{center}
  \end{itemize}
\end{PauseHighLight}

\end{slide}

%%%%%%%%%%%%%%%%%%%%%%% Next Slide %%%%%%%%%%%%%%%%%%%%%%%

\begin{slide}
\section[-2]{Checking Bipartiteness (Two-colourability)}

\begin{PauseHighLight}
  \begin{itemize}
  \item Each edge must connect nodes from different sets\pause
    \begin{java}
    isBipartite(graph) {
       colour = List(graph.noNodes(), "uncoloured");
       bipartite = true;$\pause$

       foreach node in graph {
          if (colour[node] == "uncoloured") {
             colour[node] = white;
             bfs(graph, node);
          }
       }
       return bipartite;
    }$\pause$

    processEdge(node1, node2) {
       if (colour[node1] == colour[node2])
           bipartite = false;
       colour[node2] = (colour[node1]=="white")? "black":"white";
    }$\pause$
    \end{java}
  \end{itemize}
\end{PauseHighLight}

\end{slide}

%%%%%%%%%%%%%%%%%%%%%%% Next Slide %%%%%%%%%%%%%%%%%%%%%%%
\Outline % Depth First Search
%%%%%%%%%%%%%%%%%%%%%%% Next Slide %%%%%%%%%%%%%%%%%%%%%%%

\begin{slide}
\section{Depth First Search}

\begin{PauseHighLight}
  \begin{itemize}
  \item Depth first search is essentially like breadth first search
    except we replace the queue by a stack\pause
  \item In practice it is often implemented using recursion rather than
    a stack\pause
  \item It proves useful to keep a record of the traversal \emph{time}
    for each vertex
    \begin{itemize}
    \item the clock ticks each time a vertex is entered or exited\pause
    \end{itemize}
  \end{itemize}
\end{PauseHighLight}


\end{slide}

%%%%%%%%%%%%%%%%%%%%%%% Next Slide %%%%%%%%%%%%%%%%%%%%%%%

\begin{slide}
\section[-2]{Depth First Search}
\pb
\pause
\begin{center}
  \includegraphics[width=0.9\linewidth]{graphDfs-0}\mypl{1}
  \multido{\ia=1+1,\ib=2+1}{64}%
  {\llap{\includegraphics[width=0.9\linewidth]{graphDfs-\ia}\mypl{\ib}}}
\end{center}

\end{slide}

%%%%%%%%%%%%%%%%%%%%%%% Next Slide %%%%%%%%%%%%%%%%%%%%%%%

\begin{slide}
\section{Applications of DFS}
\toptarget{dfsapp}

\begin{PauseHighLight}
  \begin{itemize}
  \item Depth first search has many applications\pause
  \item Suppose we want to check if the graph is a tree (i.e. has no cycles)\pause
  \item The only edges that are allowed are parent edges
    \begin{pseudo}
      processEdges(node1, node2) {
         if (parent[node1] != node2) {
            isTree := false
            finish := true
         }
      }
    \end{pseudo}
    (note that we set \jl$finish$ to stop DFS prematurely)\pause
  \end{itemize}
\end{PauseHighLight}

\end{slide}

%%%%%%%%%%%%%%%%%%%%%%% Next Slide %%%%%%%%%%%%%%%%%%%%%%%

\begin{slide}
\section{Biconnected Graphs and Articulation Vertices}

\begin{PauseHighLight}
  \begin{itemize}
  \item If we removed a vertex (and all its edges) from a graph would
    the graph become disconnected?\pause
  \item Such nodes are called \emph{articulation vertices}\pause
  \item Graphs with no articulation vertices are said to be
    \emph{biconnected}\pause
  \item In many applications articulation vertices are important,
    e.g. they represent single point of failures in communication
    networks\pause
  \item A brute force method for identifying articulation vertices is to
    remove each and check for connectivity---this would take $O(n(m+n))$
    time\pause. Can we do this any faster?\pauseb 
  \end{itemize}
\end{PauseHighLight}

\end{slide}

%%%%%%%%%%%%%%%%%%%%%%% Next Slide %%%%%%%%%%%%%%%%%%%%%%%

\begin{slide}
\section{Single Pass Algorithm}


\begin{minipage}{0.55\linewidth}
  \begin{PauseHighLight}
    \begin{itemize}
    \item In DFS we divide the edges into tree edges that define the
      search tree (edges between nodes and parents) and back edges
      which take us back to vertices we have already seen\pause
    \item Without the back edges we have a tree where all non-leaf nodes
      are articulation nodes\pause
    \item The back edges secure the edges to the rest of the tree\pause
    \end{itemize}
  \end{PauseHighLight}
\end{minipage}\hfill
\begin{minipage}{0.4\linewidth}
  \includegraphics[width=\linewidth]{graphDfsEdges}
\end{minipage}

\end{slide}

%%%%%%%%%%%%%%%%%%%%%%% Next Slide %%%%%%%%%%%%%%%%%%%%%%%

\begin{slide}
\section[-2]{Reachable Ancestors}

\begin{PauseHighLight}
  \begin{itemize}\squeeze
  \item Need to check if there exist back edges to nodes that have been
    visited earlier\pause
  \item We maintain an array noting the reachable ancestors of all nodes\pause
  \item This is initialised in the \texttt{processVertexEarly} method
    to the node itself\pause
  \item In the \texttt{processEdge} method, if the edge is a
    \begin{description}
    \item[back edge] we update the reachable ancestor (we check the
      \textit{entryTime} to see if the edge leads to a vertex which was
      discovered earlier)\pause
    \item[tree edge] we maintain a count of the number of tree edges
      connected to the vertex (used to determine if we are at a leaf
      node)\pause
    \end{description}
  \end{itemize}
\end{PauseHighLight}

\end{slide}


%%%%%%%%%%%%%%%%%%%%%%% Next Slide %%%%%%%%%%%%%%%%%%%%%%%

\begin{slide}
\section[-1]{Types of Articulated Vertices}

\begin{minipage}{0.55\linewidth}
  \begin{PauseHighLight}\small
  \begin{itemize}\squeeze\raggedright
  \item Key is to recognise that articulated vertices only occur in
    three version\pause
    \begin{description}
    \item[Root cut-nodes] Occur when the root has more than one child\pause
    \item[Bridge cut-nodes] Occurs when the earliest reachable vertex
      (not including the tree edge to the parent) is the vertex
      itself. The parent will be an articulation node as will be the
      node itself if it is not a leaf node\pause
    \item[Parent cut-nodes] If the earliest reachable vertex is it
      parent then the parent is an articulation node\pause
    \end{description}
  \item These are determined in \texttt{processVertexLate} method.\pause
  \end{itemize}
\end{PauseHighLight}

\end{minipage}\hfill
\begin{minipage}{0.4\linewidth}
  \includegraphics[width=0.9\linewidth]{graphArticulation0}
\end{minipage}
\end{slide}

%%%%%%%%%%%%%%%%%%%%%%% Next Slide %%%%%%%%%%%%%%%%%%%%%%%

\begin{slide}
\section{Biconnectivity Summary}

\begin{minipage}{0.55\linewidth}
  \begin{PauseHighLight}
  \begin{itemize}\raggedright
  \item Algorithmic details are not too important\pause
  \item One pass ($O(n+m)$) algorithm\pause
  \item Uses \texttt{processVertexEarly}, \texttt{procesEdge} and
    \texttt{processVerexLate} methods\pause
  \item Bridge cuts also shows articulation edges\pause
  \end{itemize}
\end{PauseHighLight}

\end{minipage}\hfill
\begin{minipage}{0.4\linewidth}
  \includegraphics[width=0.9\linewidth]{graphArticulation1}\pause
  \llap{\includegraphics[width=0.9\linewidth]{graphArticulation2}}\pauseb
\end{minipage}

\end{slide}

%%%%%%%%%%%%%%%%%%%%%%% Next Slide %%%%%%%%%%%%%%%%%%%%%%%
\Outline % Topological Sort
%%%%%%%%%%%%%%%%%%%%%%% Next Slide %%%%%%%%%%%%%%%%%%%%%%%

\begin{slide}
\section[-2]{DAGs}

\begin{PauseHighLight}
  \begin{itemize}
  \item Directed acyclic graphs or DAGs are directed graphs without
    cycles\pause
  \item They are often used to represent complex processes
    \begin{itemize}
    \item Vertices are processes
    \item Directed edge $(i,j)$ indicates process $i$ needs to occur
      before process $j$\pause
    \end{itemize}
  \end{itemize}
  \begin{center}
    \includegraphics[width=0.6\linewidth]{dag}
  \end{center}
\end{PauseHighLight}

\end{slide}



%%%%%%%%%%%%%%%%%%%%%%% Next Slide %%%%%%%%%%%%%%%%%%%%%%%

\begin{slide}
\section{Program Compilation}

\begin{PauseHighLight}
  \begin{itemize}
  \item One example of a DAG is in compiling programs\pause
  \item Some programs depend on other programs so they need compiling first\pause
  \end{itemize}
  \begin{center}
    \includegraphics[width=0.9\linewidth]{compilationDAG}
  \end{center}
\end{PauseHighLight}

\end{slide}

%%%%%%%%%%%%%%%%%%%%%%% Next Slide %%%%%%%%%%%%%%%%%%%%%%%

\begin{slide}
\section[-2]{Other Applications}

\begin{PauseHighLight}
  \begin{itemize}
  \item The same problem occurs in compiling classes
    \begin{itemize}
    \item The implementation of a class can depend on the implementation
      of other graphs\pause
    \item What order should you compile the classes?\pause
    \end{itemize}
  \item In taking a degree various modules have other modules as
    prerequisites\pause---what order should a student study the modules
    in?\pauseb
  \item In your graduation ceremony there is the VC, Provost, Dean, HOS,
    Professors, etc.\pause---in what order should they process?\pauseb 
  \item If the graphs were not acyclic it would be impossible process them!\pauseb
  \end{itemize}
\end{PauseHighLight}

\end{slide}

%%%%%%%%%%%%%%%%%%%%%%% Next Slide %%%%%%%%%%%%%%%%%%%%%%%

\begin{slide}
\section{Topological Sort}

\begin{PauseHighLight}
  \begin{itemize}
  \item Given a DAG a topological sort outputs an ordered list of
    vertices which respects the ordering imposed by the edges\pause
  \item That is, for each edge $(i,j)$, vertex $i$ will occur before
    vertex $j$\pause
  \item Any DAG will have at least one topological sort, but most DAGs
    will have many topological sorts\pause
  \item Topological sort is not a ``sort'', but it is a useful algorithm
    for some applications\pause
  \end{itemize}
\end{PauseHighLight}

\end{slide}

%%%%%%%%%%%%%%%%%%%%%%% Next Slide %%%%%%%%%%%%%%%%%%%%%%%

\begin{slide}
\section[-1]{Performing a Topological Sort}

\pb
\pause
\begin{itemize}
\item A topological sort is generated by a reversed order list of how
    DFS processes nodes\pauseh
\end{itemize}
\begin{center}
  \includegraphics[width=0.9\linewidth]{topologicalSort0}\mypl{2}
  \multido{\ia=1+1,\ib=3+1}{28}%
  {\llap{\includegraphics[width=0.9\linewidth]{topologicalSort\ia}\mypl{\ib}}}
\end{center}

\end{slide}

%%%%%%%%%%%%%%%%%%%%%%% Next Slide %%%%%%%%%%%%%%%%%%%%%%%

\begin{slide}
\section{DFS on Digraphs}

\pb\pause
  \includegraphics[height=12cm]{directedDfs0}\mypl{1}
  \llap{\includegraphics[height=12cm]{directedDfs1}\pause}
\end{slide}

%%%%%%%%%%%%%%%%%%%%%%% Next Slide %%%%%%%%%%%%%%%%%%%%%%%

\begin{slide}
\section{Implementing Topological Sort}

\begin{PauseHighLight}
  \begin{itemize}
  \item Given our DFS programme we define
    \begin{pseudo}
topologicalSort(graph) {
   Stack stack
   for node $\in$ graph.vertexSet()
      if (!discovered[node])
          dfs(graph, node)
  
   List topSortList
   while (!stack.isEmpty())
      topSortList.add(stack.pop())

   return topSortList
}
    \end{pseudo}
  \end{itemize}
\end{PauseHighLight}

\end{slide}

%%%%%%%%%%%%%%%%%%%%%%% Next Slide %%%%%%%%%%%%%%%%%%%%%%%

\begin{slide}
\section{Enhance DFS}

\begin{PauseHighLight}
  \begin{itemize}
  \item Requires us to define a couple of helper function
\begin{pseudo}
  processVertexLast(node) {
     stack.push(node)
  }$\pause$

  processEdge(currentNode, neighbour) {
     if (state[neighbour] == "processed") {
        print "error: graph not a DAG"
        finished = true
     }
  }$\pause$
\end{pseudo}
  \end{itemize}
\end{PauseHighLight}

\end{slide}

%%%%%%%%%%%%%%%%%%%%%%% Next Slide %%%%%%%%%%%%%%%%%%%%%%%

\begin{slide}
\section{Implementation Issues}

\begin{PauseHighLight}
  \begin{itemize}
  \item Most awkward part of the implementation is that the
    \texttt{topologicalSort} algorithm needs access to dfs structures
    (\texttt{discovered[]})\pause
  \item \texttt{processVertexLast(node)} needs access to the stack\pause
  \item Need to be able to redefine \texttt{processVertexFirst},
    \texttt{processEdge} and \texttt{processVertexLast}\pause
  \item Different languages and libraries cope with this differently
    \begin{itemize}
    \item Java: JDSL, JGraphT
    \item C++: Boost Graph Library, LEDA\pause
    \end{itemize}
  \end{itemize}
\end{PauseHighLight}

\end{slide}

%%%%%%%%%%%%%%%%%%%%%%% Next Slide %%%%%%%%%%%%%%%%%%%%%%%

\begin{slide}
\section[-2]{Other Applications}

\pb
\begin{itemize}
\item DFS is used for many other classic problems\pauseh
\item Euler Cycles \pauseh \hfil
  \begin{minipage}{0.3\linewidth}
    \begin{center}
      \includegraphics[width=0.9\linewidth]{eulerCycle0}\mypl{2}
      \multido{\ia=1+1,\ib=3+1}{11}%
      {\llap{\includegraphics[width=0.9\linewidth]{eulerCycle\ia}\mypl{\ib}}}
      \llap{\includegraphics[width=0.9\linewidth]{eulerCycle12}\pause}
    \end{center}
  \end{minipage}
\item Strongly Connected Components\pauseh\raisebox{-3cm}{\includegraphics[width=0.4\linewidth]{stonglyConnectedComponent}}\pause
\end{itemize}

\end{slide}

%%%%%%%%%%%%%%%%%%%%%%% Next Slide %%%%%%%%%%%%%%%%%%%%%%%

\begin{slide}
\section{Lessons}

\begin{PauseHighLight}
  \begin{itemize}
  \item Breadth first and depth first search are different methods for
    traversing graphs\pause
  \item They are used as part of many specific algorithms for discovering
    graph properties\pause
  \item Breadth first search is particularly important for finding
    shortest paths in unweighted graphs\pause
  \item Depth first search is used in a whole host of applications
    (finding articulation points, Euler cycles, strongly connected
    components)\pause
  \item One of the most used application is in topological sort (finding
    an ordering of processes represented by a DAG)\pause
  \end{itemize}
\end{PauseHighLight}

\end{slide}
