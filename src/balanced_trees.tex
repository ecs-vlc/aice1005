%Master File:lectures.tex

\begin{slide}
\lesson{Keep Trees Balanced}

\vspace*{-1cm}
\begin{center}
  \includegraphics[height=10cm]{unbalanced_tree}
\end{center}

\keywords{AVL trees, red-black trees, TreeSet, TreeMap}
\end{slide}

%%%%%%%%%%%%%%%%%%%%%%% Next Slide %%%%%%%%%%%%%%%%%%%%%%%
\renewcommand{\Outline}{%
\begin{slide}
\section{Outline}

\begin{minipage}{10cm}
  \vfill
  \begin{enumerate}
    \outlineitem{Deletion}{deletion}
    \outlineitem{Balancing Trees}{balance}
    \begin{itemize}
    \item \toplink{rotations}{Rotations}
    \end{itemize}
    \outlineitem{AVL}{avl}
    \outlineitem{Red-Black Trees}{rb}
    \begin{itemize}
    \item \toplink{treeset}{TreeSet}
    \item \toplink{treemap}{TreeMap}
    \end{itemize}
  \end{enumerate}
  \vfill
\end{minipage}\hfill
\begin{minipage}{14cm}
  \includegraphics[width=14cm]{Lopsided_tree}
\end{minipage}
\end{slide}
\addtocounter{outlineitem}{1}
}

\setcounter{outlineitem}{1}
\Outline
\toptarget{firstoutline}

%%%%%%%%%%%%%%%%%%%%%%% Next Slide %%%%%%%%%%%%%%%%%%%%%%%

\begin{slide}
\section{Recap}

\begin{PauseHighLight}
  \begin{itemize}
  \item Binary search trees are commonly used to store data because we
    need to only look down one branch to find any element\pause
  \item We saw how to implement many methods of the binary search tree
    \begin{itemize}
    \item \texttt{find}
    \item \texttt{insert}
    \item \texttt{successor} (in outline)\pause
    \end{itemize}
  \item One method we missed was \texttt{remove}\pause
  \end{itemize}
\end{PauseHighLight}

\end{slide}


%%%%%%%%%%%%%%%%%%%%%%% Next Slide %%%%%%%%%%%%%%%%%%%%%%%

\begin{slide}
\section[-2]{Deletion}

\pausebuild
\color{TwoColor}
\begin{itemize}
\item Suppose we want to delete some elements from a tree\pauseh
\item It is relatively easy if the element is a leaf node (e.g. 50)
  \pauselevel{=1, highlight =2 :4}\pause
\item It is not so hard if the node has one child (e.g. 20)
  \pauselevel{=1, highlight =5 :7}\pause
\end{itemize}
\begin{center}\color{TextColor}\setlength{\unitlength}{1mm}
  \begin{picture}(160,80)
    \put(0,0){\includegraphics[width=16cm]{tree1.mps}}\pauselevel{=1 :3}\pause
    \put(-10,70){\jl$remove(50)$}\pauselevel{=3 :3}\pause
    \put(80,43){\circle{12}}\pauselevel{=3 :3}\pause
    \put(0,0){\includegraphics[width=16cm]{tree2.mps}}\pauselevel{=4 :6}\pause
    \put(-10,70){\jl$remove(20)$}\pauselevel{=6 :6}\pause
    \put(62,61){\circle{12}}\pauselevel{=6 :6}\pause
    \put(0,0){\includegraphics[width=16cm]{tree3.mps}}\pauselevel{=7}\pause
  \end{picture}
\end{center}
\end{slide}

%%%%%%%%%%%%%%%%%%%%%%% Next Slide %%%%%%%%%%%%%%%%%%%%%%%

\begin{slide}
\section{Code to remove Node $n$}

\begin{PauseHighLight}
\begin{minipage}{11cm}
\begin{java}
  if (n->left==0 && n->right==0) {
     if (n == n->parent->left)
        n->parent->left = 0;
     else
        n->parent->right = 0;$\pause$
  } else if (n->right==0) {
     if (n == n->parent->left)
        n->parent->left = n->left;
     else
        n->parent->right = n->left;
     n->left->parent = n->parent;$\pause$
  } else if (n->left==0) {
     if (n == n->parent->left)
        n->parent->left = n->right;
     else
        n->parent->right = n->right;
     n->right->parent = n->parent;
  }$\pause$
  delete n;$\pause$
\end{java}
\end{minipage}\hspace{1cm}\pauselevel{=1}
\begin{minipage}{11cm}
  \multiinclude[graphics={width=10cm}]{tree-remove}\pause
\end{minipage}
\end{PauseHighLight}

\end{slide}

%%%%%%%%%%%%%%%%%%%%%%% Next Slide %%%%%%%%%%%%%%%%%%%%%%%

\begin{slide}
\section{Removing Element with Two Children}

\pausebuild
\color{TwoColor}
\begin{itemize}
\item If an element has two children then
  \pauselevel{=1, highlight =1}\pause
  \begin{itemize}
  \item replace that element by its successor
  \pauselevel{=1, highlight =2, highlight =5 :6}\pause
  \item  and then remove the successor using the above procedure
  \pauselevel{=1, highlight =2, highlight =7}\pause
  \end{itemize}
\begin{center}\color{TextColor}\setlength{\unitlength}{1mm}
  \begin{picture}(160,80)
    \thicklines
    \put(0,0){\includegraphics[width=14cm]{tree3.mps}}\pauselevel{=3 :7}\pause
    \put(-10,80){\jl$remove(80)$}\pauselevel{=4 :7}\pause
    {    \color{red}
    \put(82,69){\circle{10}}\pauselevel{=4 :7}\pause
    \put(96,38){\circle{10}}\pauselevel{=5 :7}\pause
    \put(94,43){\vector(-1,2){11}}\pauselevel{=6 :6}\pause
    \put(82,68){\makebox(0,0){\cross}}
    \put(92,68){\makebox(0,0){\bf 90}}\pauselevel{=6 :7}\pause
    \put(96,38){\makebox(0,0){\cross}}
    \put(106,26){\line(1,4){6}}\pauselevel{=7 :7}\pause
    } \color{TextColor}
    \put(0,0){\includegraphics[width=14cm]{tree4.mps}}\pauselevel{=8 :8}\pause
  \end{picture}
\end{center}

\end{itemize}



\end{slide}


%%%%%%%%%%%%%%%%%%%%%%% Next Slide %%%%%%%%%%%%%%%%%%%%%%%
\Outline % Balancing trees
%%%%%%%%%%%%%%%%%%%%%%% Next Slide %%%%%%%%%%%%%%%%%%%%%%%

\begin{slide}
\section[-2]{Why Balance Trees}

\begin{PauseHighLight}
  \begin{itemize}
  \item The number of comparisons to access an element depends on the
    depth of the node\pause
  \item The average depth of the node depends on the shape of the
    tree\pause
    \begin{center}
      \multiinclude[graphics={width=\linewidth}]{tree_shape}\pause
    \end{center}
  \item The shape of the tree depends on the order the elements are
    added\pause
  \end{itemize}
\end{PauseHighLight}

\end{slide}

%%%%%%%%%%%%%%%%%%%%%%% Next Slide %%%%%%%%%%%%%%%%%%%%%%%

\begin{slide}
\section{Time Complexity}

\begin{PauseHighLight}
  \begin{itemize}
  \item In the best situation (a full tree) the number of elements in a
    tree is $n=\Theta(2^l)$ the depth is $l$ so that the maximum depth is
    $\log_2(n)$\pause
  \item It turns out for random sequences the average depth is
    $\Theta(\log(n))$\pause
  \item In the worst case (when the tree is effectively a linked list),
    the average depth is $\Theta(n)$\pause
  \item Unfortunately, the worst case happens when the elements are
    added \textit{in order} (not a rare event)\pause
  \end{itemize}
\end{PauseHighLight}
\end{slide}


%%%%%%%%%%%%%%%%%%%%%%% Next Slide %%%%%%%%%%%%%%%%%%%%%%%

\begin{slide}
\section[-2]{Rotations}
\toptarget{rotations}

\begin{PauseHighLight}
  \begin{itemize}
  \item To avoid unbalanced trees we would like to modify the
    shape\pauseh
  \item This is possible as the shape of the tree is not uniquely
    defined (e.g. we could make any node the root)\pauseh
  \item We can change the shape of a tree using \emph{rotations}\pauseh
  \item E.g. left rotation
  \begin{center}
    \includegraphics[height=8cm]{left-rotation}\pause
  \end{center}
  \end{itemize}
\end{PauseHighLight}

\end{slide}

%%%%%%%%%%%%%%%%%%%%%%% Next Slide %%%%%%%%%%%%%%%%%%%%%%%

\begin{slide}
\section[-2]{Types of Rotations}

\begin{PauseHighLight}
  \begin{itemize}
  \item We can get by with 4 types of rotations\pause
    \begin{itemize}
    \item Left rotation (as above)\pause
    \item Right rotation (symmetric to above)
    \begin{center}
      \includegraphics[height=8cm]{right-rotation}\pause
    \end{center}
    \item Left-right double rotation\pause
    \item Right-left double rotation\pause
    \end{itemize}
  \end{itemize}
\end{PauseHighLight}

\end{slide}

%%%%%%%%%%%%%%%%%%%%%%% Next Slide %%%%%%%%%%%%%%%%%%%%%%%

\begin{slide}
\section{Coding Rotations}

\begin{minipage}{10cm}
\begin{java}
void rotateLeft(Node<T>* e)$\pause$
{
  Node<T>* r = e->right;$\pause$
  e->right = r->left;$\pause$
  if (r->left != 0)
     r->left->parent = e;$\pause$
  r->parent = e->parent;$\pause$
  if (e->parent == 0)
     root = r;
  else if (e->parent->left == e)
     e->parent->left = r;
  else
     e->parent->right = r;$\pause$
  r->left = e;
  e->parent = r;
}$\pause$
\end{java}
\end{minipage}
\begin{minipage}{14cm}
\vspace*{-1ex}
\includegraphics[width=14cm]{left-rotation1}
\end{minipage}
\end{slide}

%%%%%%%%%%%%%%%%%%%%%%% Next Slide %%%%%%%%%%%%%%%%%%%%%%%

\begin{slide}
\section{When Single Rotations Work}

\begin{PauseHighLight}
  \begin{itemize}
  \item Single rotations balance the tree when the unbalanced subtree is
    on the outside
    \begin{center}
      \includegraphics[width=0.45\linewidth]{left-rotation}\hfil
      \includegraphics[width=0.45\linewidth]{right-rotation}\pause
    \end{center}
  \end{itemize}
\end{PauseHighLight}

\end{slide}

%%%%%%%%%%%%%%%%%%%%%%% Next Slide %%%%%%%%%%%%%%%%%%%%%%%

\begin{slide}
\section{Double Rotations}

\begin{PauseHighLight}
  \begin{itemize}
  \item If the unbalanced subtree is on the inside we need a double
    rotation
    \begin{center}
      \includegraphics[width=\linewidth]{left-right-rotation}\pause
    \end{center}
    \begin{java}
      leftRotation(C.left);
      rightRotation(C);$\pause$
    \end{java}
  \end{itemize}
\end{PauseHighLight}


\end{slide}


%%%%%%%%%%%%%%%%%%%%%%% Next Slide %%%%%%%%%%%%%%%%%%%%%%%
\Outline % AVL Trees
%%%%%%%%%%%%%%%%%%%%%%% Next Slide %%%%%%%%%%%%%%%%%%%%%%%

\begin{slide}
\section{Balancing Trees}

\begin{PauseHighLight}
  \begin{itemize}
  \item There are different strategies for using rotations for balancing
    trees\pause
  \item The three most popular are
    \begin{itemize}
    \item AVL-trees
    \item Red-black trees
    \item Splay trees\pause
    \end{itemize}
  \item They differ in the criteria they use for doing rotations\pause
  \end{itemize}
\end{PauseHighLight}

\end{slide}

%%%%%%%%%%%%%%%%%%%%%%% Next Slide %%%%%%%%%%%%%%%%%%%%%%%

\begin{slide}
\section{AVL Trees}

\begin{PauseHighLight}
  \begin{itemize}
  \item AVL-trees were invented in  1962 by two Russian mathematicians
    Adelson-Velski and Landis\pause
  \item In AVL trees
    \begin{enumerate}
    \item The heights of the left and right subtree differ by at most
      1\pause
    \item The left and right subtrees are AVL trees\pause
    \end{enumerate}
  \item This guarantees that the worst case AVL tree has logarithmic
    depth\pause
  \end{itemize}
\end{PauseHighLight}

\end{slide}

%%%%%%%%%%%%%%%%%%%%%%% Next Slide %%%%%%%%%%%%%%%%%%%%%%%

\begin{slide}
\section{Minimum Number of Nodes}

\begin{PauseHighLight}
  \begin{itemize}
  \item Let $m(h)$ be the minimum number of nodes in a tree of height
  $h$\pause
  \item This has to be made up of two subtrees: one of height $h-1$; and, in
    the worst case, one of height $h-2$\pause
  \item Thus, the least number of nodes in a tree of height $h$ is
    \begin{minipage}{12cm}
   \begin{align*}
      m(h) = m(h-1) + m(h-2) + 1
    \end{align*}
    \end{minipage}
    \begin{minipage}{10cm}
      \begin{center}
        \includegraphics[width=9cm]{min-nodes}\pause
      \end{center}
    \end{minipage}
  \item with $m(1)=1$, $m(2)=2$\pause
  \end{itemize}
\end{PauseHighLight}

\end{slide}


%%%%%%%%%%%%%%%%%%%%%%% Next Slide %%%%%%%%%%%%%%%%%%%%%%%

\begin{slide}
\section[-2]{Proof of Exponential Number of Nodes}

\begin{PauseHighLight}
  \begin{itemize}
  \item We have $m(h) = m(h-1) + m(h-2) + 1$ with $m(1)=1$, $m(2)=2$\pause
  \item This gives us a sequence $1, 2, 4, 7, 12, \cdots$\pause
  \item Compare this with Fibonacci $f(h)=f(h-1) + f(h-2)$, with
    $f(1)=f(2)=1$ \pause
  \item This gives us a sequence $1, 1, 2, 3 ,5, 8, 13, \cdots$\pause
  \item It looks like $m(h) = f(h+2) -1$\pause
  \item Proof by substitution\pause
  \end{itemize}
\end{PauseHighLight}

\end{slide}


%%%%%%%%%%%%%%%%%%%%%%% Next Slide %%%%%%%%%%%%%%%%%%%%%%%

\begin{slide}
\section[-2]{Proof of Logarithmic Depth}

\begin{PauseHighLight}
  \begin{itemize}\squeeze
  \item $m(h) = m(h-1) + m(h-2) + 1$ with $m(1)=1$, $m(2)=2$\pause
  \item We can prove by inductions, $m(h) \geq (3/2)^{h-1}$\pause
  \item $m(1)=1 \geq (3/2)^0 = 1$, $m(2)=2 \geq (3/2)^1 =3/2$\pause\,
    \tick\pauseb
    \begin{align*} \scriptstyle
      m(h) \geq \left(\frac{3}{2}\right)^{h-3} \left(\frac{3}{2} +1 +
      \left(\frac{3}{2}\right)^{3-h} \right) \pause \geq
      \left(\frac{3}{2}\right)^{h-3} \frac{5}{2} \pause =
      \left(\frac{3}{2}\right)^{h-3} \frac{10}{4} \pause
      \geq \left(\frac{3}{2}\right)^{h-3} \frac{9}{4} \pause =
      \left(\frac{3}{2}\right)^{h-1} \pause \, \mbox{\tick}\pauseb
    \end{align*}
  \item Taking logs: $\log(m(h)) \geq (h-1) \log(3/2)$ or
    \begin{align*}
      h \leq \frac{\log(m(h))}{\log(3/2)} +1 =
      O\left(\strut\log(m(h))\right)\pause 
    \end{align*}
  \item The number of elements, $n$, we can store in an AVL tree is
    $n\geq m(h)$ thus
    \begin{align*}
      h \leq O(\log(n))\pause
    \end{align*}
  \end{itemize}
\end{PauseHighLight}

\end{slide}

%%%%%%%%%%%%%%%%%%%%%%% Next Slide %%%%%%%%%%%%%%%%%%%%%%%

\begin{slide}
\section[-2]{Implementing AVL Trees}

\pb
\begin{itemize} \small
\item In practice to implement an AVL tree we include additional
  information at each node indicating the balance of the subtrees\pauseh
  \begin{align*}
    \mbox{\texttt{balanceFactor}} = \left\{
      \begin{array}{rcl}
        -1 & & \mbox{right subtree deeper than left subtree} \\
        0  & & \mbox{left and right subtrees equal}\\
        +1 & & \mbox{left subtree deeper than right subtree}\pauseh
      \end{array}\right.
  \end{align*}
\end{itemize}
\begin{center}
  \includegraphics[width=0.8\linewidth]{avl-0.pdf}\mypl{3}
  \multido{\ia=1+1,\ib=4+1}{36}{%
    \llap{\includegraphics[width=0.8\linewidth]{avl-\ia.pdf}}\mypl{\ib}}
\end{center}


\end{slide}

%%%%%%%%%%%%%%%%%%%%%%% Next Slide %%%%%%%%%%%%%%%%%%%%%%%

\begin{slide}
\section{Balancing AVL Trees}

\begin{PauseHighLight}
  \begin{itemize}
  \item When \texttt{add}ing an element to an AVL tree\pause
    \begin{itemize}
    \item Find the location where it is to be inserted\pause
    \item Iterate up through the parents re-adjusting the
      \texttt{balanceFactor}\pause
    \item \textbf{If} the balance factor exceeds $\pm1$ \texttt{then}
      re-balance the tree and stop\pause
    \item \texttt{else if} the balance factor goes to zero \textbf{then}
      stop\pause
    \end{itemize}
    \begin{center}
      \includegraphics[width=\linewidth]{avl-left-rotation}\pause
    \end{center}
\end{itemize}
\end{PauseHighLight}
\end{slide}

%%%%%%%%%%%%%%%%%%%%%%% Next Slide %%%%%%%%%%%%%%%%%%%%%%%

\begin{slide}
\section{AVL Deletions}

\begin{PauseHighLight}
  \begin{itemize}
  \item AVL deletions are similar to AVL insertions\pause
  \item One difference is that after performing a rotation the tree may
    still not satisfy the AVL criteria so higher levels need to be
    examined\pause
  \item In the worst case $\Theta(\log(n))$ rotations may be necessary\pause
  \item This may be relatively slow\pause---but in many applications
    deletions are rare\pause
  \end{itemize}
 \end{PauseHighLight}


\end{slide}


%%%%%%%%%%%%%%%%%%%%%%% Next Slide %%%%%%%%%%%%%%%%%%%%%%%

\begin{slide}
\section{AVL Tree Performance}

\begin{PauseHighLight}
  \begin{itemize}
  \item Insertion, deletion and search in AVL trees are, at worst,
    $\Theta(\log(n))$\pause
  \item The height of an average AVL tree is $1.44\log_2(n)$\pause
  \item The height of an average binary search tree is
    $2.1\log_2(n)$\pause
  \item Despite being more compact insertion is slightly slower in AVL
    trees than binary search trees without balancing (for random input
    sequences)\pause
  \item Search is, of course, quicker\pause
  \end{itemize}
\end{PauseHighLight}


\end{slide}


%%%%%%%%%%%%%%%%%%%%%%% Next Slide %%%%%%%%%%%%%%%%%%%%%%%
\Outline % Red-Black Trees
%%%%%%%%%%%%%%%%%%%%%%% Next Slide %%%%%%%%%%%%%%%%%%%%%%%

\begin{slide}
\section[-2]{Red-Black Trees}

\begin{PauseHighLight}
  \begin{itemize}\squeeze
  \item Red-black trees are another strategy for balancing trees\pause
  \item Nodes are either \textit{red} or \textit{black}\pause
  \item Two rules are imposed\pause
    \begin{description}
    \item[\textit{Red Rule:}] the children of a red node must be
      black\pause
  \item[\textit{Black Rule:}] the number of black elements must be the
    same in all paths from the root to elements with no children or with
    one child\pause
    \end{description}
  \end{itemize}
  \begin{center}
    \includegraphics[height=7cm]{red-black-tree}\pauselevel{build =1}\pause
  \end{center}
\end{PauseHighLight}

\end{slide}

%%%%%%%%%%%%%%%%%%%%%%% Next Slide %%%%%%%%%%%%%%%%%%%%%%%

\begin{slide}
\section[-2]{Restructuring}

\pausebuild
\color{TwoColor}
\begin{itemize}\squeeze
\item When inserting a new element we first find its position\pause
\item If it is the root we colour it black otherwise we colour it
  red\pauseh
\item If its parent is red we must either relabel or restructure the
  tree\pauseh
\end{itemize}
\begin{center}\color{TextColor}
  \multiinclude[graphics={height=10cm}]{red-black-tree-insert}\pause
\end{center}

\end{slide}

%%%%%%%%%%%%%%%%%%%%%%% Next Slide %%%%%%%%%%%%%%%%%%%%%%%

\begin{slide}
\section{Performance of Red-Black Trees}

\begin{PauseHighLight}
  \begin{itemize}
  \item Red-black trees are slightly more complicated to code than AVL
    trees\pause 
  \item Red-black trees tend to be slightly less compact than AVL
    trees\pause
  \item However, insertion and deletion run slightly quicker\pause
  \item Both Java Collection classes and C++ STL use red-black
    trees\pause
  \end{itemize}
\end{PauseHighLight}

\end{slide}


%%%%%%%%%%%%%%%%%%%%%%% Next Slide %%%%%%%%%%%%%%%%%%%%%%%

\begin{slide}
\section{Set}
\toptarget{treeset}

\begin{PauseHighLight}
  \begin{itemize}
  \item The standard template library (STL) has a class
    \jl$std::set<T>$\pause
  \item It also has a \jl$std::underordered_set<T>$ class (which uses a
    hash table covered later)\pause
  \item As well as \jl$std::multiset<T>$ that implements a multiset
    (i.e. a set, but with repetitions)\pause
  \item Using sets you can also implement \emph{maps}\pause
  \end{itemize}
\end{PauseHighLight}

\end{slide}

%%%%%%%%%%%%%%%%%%%%%%% Next Slide %%%%%%%%%%%%%%%%%%%%%%%

\begin{slide}
\section[-2]{Maps}
\toptarget{treemap}

\begin{PauseHighLight}
  \begin{itemize}
  \item One major abstract data type (ADT) we have not encountered is
    the map class\pause
  \item The map class \jl$std:map<Key,V>$ contain key-value pairs \jl$pair<Key,V>$\pause
    \begin{itemize}
    \item The first element of type \texttt{Key} is the \emph{key}\pause
    \item The second element of type \texttt{V} is the \emph{value}\pause
    \end{itemize}
  \item Maps work as content addressable arrays\pause
    \begin{java}
      map<string, int> students;
      student["John Smith"] = 89;
      student["Terry Jones"] = 98;
      cout << students["John Smith"];
    \end{java}\pause
  \end{itemize}
\end{PauseHighLight}

\end{slide}

%%%%%%%%%%%%%%%%%%%%%%% Next Slide %%%%%%%%%%%%%%%%%%%%%%%

\begin{slide}
\section[-2]{Implementing a Map}

\begin{PauseHighLight}
  \begin{itemize}
  \item Maps can be implemented using a set by making each node
    hold a \texttt{pair<K,V>} objects\pause
    \begin{java}
      class pair<K,V> 
      {
        public:
        K first;
        V second;
      }$\pause$
    \end{java}
  \item We can count words using the key for words and value to count
  \end{itemize}
  \begin{center}\color{TextColor}
    \includegraphics[width=\linewidth]{hamlet-cnt.eps}\pause
  \end{center}
\end{PauseHighLight}

\end{slide}

%%%%%%%%%%%%%%%%%%%%%%% Next Slide %%%%%%%%%%%%%%%%%%%%%%%

\begin{slide}
\section{Lessons}

\begin{PauseHighLight}
  \begin{itemize}
  \item Binary search trees are very efficient (order $\log(n)$
    insertion, deletion and search) provided they are balanced\pause
  \item Balanced trees are achieved by performing rotations\pause
  \item There are different strategies for deciding when to rotate
    including
    \begin{itemize}
    \item AVL trees
    \item Red-black trees\pause
    \end{itemize}
  \item Binary trees are used for implementing \emph{sets} and
    \emph{maps}\pause
  \end{itemize}
\end{PauseHighLight}

\end{slide}
