%Master File:lectures.tex
\renewcommand{\class}[1]{\textsf{#1}}

\lesson{Use Linear Programmings}

\vspace{-2cm}
\begin{center}
  \includegraphics[height=10cm]{feasibleSolutions-4}
\end{center}
\keywords{linear programming, applications}
%%%%%%%%%%%%%%%%%%%%%%% Next Slide %%%%%%%%%%%%%%%%%%%%%%%
\renewcommand{\Outline}{%
\begin{slide}
\section[1]{Outline}

\begin{minipage}{12cm}
  \begin{enumerate}\squeeze
    \outlineitem{Examples}{applications}
    \outlineitem{Linear Programs}{linearp}
    \outlineitem{Properties of Solution}{properties}
    \outlineitem{Normal Form}{normalform}
   \end{enumerate}
\end{minipage}\hfill
\begin{minipage}{10cm}
  \includegraphics[width=10cm]{Simplex_description.png}
\end{minipage}
\end{slide}
\addtocounter{outlineitem}{1}
}

\setcounter{outlineitem}{1}

%%%%%%%%%%%%%%%%%%%%%%% Next Slide %%%%%%%%%%%%%%%%%%%%%%%
\Outline % Examples
\toptarget{firstoutline}
%%%%%%%%%%%%%%%%%%%%%%% Next Slide %%%%%%%%%%%%%%%%%%%%%%%

%%%%%%%%%%%%%%%%%%%%%%% Next Slide %%%%%%%%%%%%%%%%%%%%%%%

\begin{slide}
\section{Going Shopping}

\begin{PauseHighLight}
  \begin{itemize}
  \item Suppose we have a number of food stuffs which we label with
    indices $f\in\mathcal{F}$\pause
  \item The price of food stuff $f$ per kilogram we denote $p_f$\pause
  \item We are interested in buying a selection of foods $\bm{x} = (x_f
    | f\in\mathcal{F})$ where $x_f$ is the quantity (in kg) of food
    $f$\pause
  \item We want to minimise the total price $\sum_f p_f\, x_f =
    \bm{p}\cdot\bm{x}$\pause
  \item However we want to ensure that the food has enough vitamins\pause
  \end{itemize}
\end{PauseHighLight}

\end{slide}

%%%%%%%%%%%%%%%%%%%%%%% Next Slide %%%%%%%%%%%%%%%%%%%%%%%

\begin{slide}
\section{Nutrition}

\begin{PauseHighLight}
  \begin{itemize}
  \item We consider the set of vitamins $\mathcal{V}$\pause
  \item Let $A_{vf}$ be the quantity of vitamin $v$ in food stuff $f$\pause
  \item Let $b_v$ be the minimum daily requirement of vitamin $v$\pause
  \item We therefore require
    \begin{align*}
      \forall v \in \mathcal{V} \quad\quad \sum_{f\in\mathcal{F}} A_{vf}\,x_f
      \geq b_v\pause
    \end{align*}
  \end{itemize}
\end{PauseHighLight}

\end{slide}

%%%%%%%%%%%%%%%%%%%%%%% Next Slide %%%%%%%%%%%%%%%%%%%%%%%

\begin{slide}
\section{Optimisation Problem}

\begin{PauseHighLight}
  \begin{itemize}
  \item We can write the food shopping problem as
    \begin{align*}
      \min_{\bm{x}} \bm{p} \cdot \bm{x}  \quad \text{subject to}
      \quad \mat{A} \bm{x} \geq \bm{b} \quad \text{and} \quad \bm{x} \geq \bm{0}\pause
    \end{align*}
  \item Note that the inequalities involving vectors means that each
    component must be satisfied, i.e.
    \begin{align*}
     \mat{A} \bm{x} \geq \bm{b} \quad &\Rightarrow \quad  \forall v \in  \mathcal{V}
     \quad \sum_{f\in\mathcal{F}} A_{vf}\,x_f \geq b_v \\
     \bm{x} \geq \bm{0} \quad  &\Rightarrow \quad \forall f \in \mathcal{F}
     \quad x_f \geq 0 \pause
    \end{align*}
  \item This is an example of a ``\emph{linear program}''\pause
  \end{itemize}
\end{PauseHighLight}

\end{slide}

%%%%%%%%%%%%%%%%%%%%%%% Next Slide %%%%%%%%%%%%%%%%%%%%%%%

\begin{slide}
\section{Transportation}

\begin{PauseHighLight}
  \begin{itemize}
  \item We consider a set of factories $\mathcal{F}$ producing a set of
    commodities $\mathcal{C}$\pause
  \item The amount of commodity $c$ produced by factory $f$ we denote by
    $x_{cf}$\pause
  \item The shipping cost of commodity $c$ from factory $f$ to the
    retailer of $c$ we denote by $p_{cf}$\pause
  \item We want to choose $x_{cf}$ to minimise the transportation costs
    \begin{align*}
      \sum_{c\in\mathcal{C}, f\in\mathcal{F}} p_{cf}\, x_{cf}\pause
    \end{align*}
  \item However, we have constraints\ldots\pause
  \end{itemize}
\end{PauseHighLight}

\end{slide}

%%%%%%%%%%%%%%%%%%%%%%% Next Slide %%%%%%%%%%%%%%%%%%%%%%%

\begin{slide}
\section{Constraints}

\begin{PauseHighLight}
  \begin{itemize}
  \item Each factory can only produce a certain overall tonnage of
    commodities
    \begin{align*}
      \sum_{c\in\mathcal{C}} x_{cf} \leq b_f \quad \quad \forall f \in \mathcal{F}\pause
    \end{align*}
    where $b_f$ is the maximum production capacity of factory $f$\pause
  \item The total demand for each commodity is $d_c$ so
    \begin{align*}
      \sum_{f\in\mathcal{F}} x_{cf} = d_c \quad \quad \forall c \in \mathcal{C}\pause
    \end{align*}
  \item We can only produce positive amounts, i.e. $x_{cf}\geq0$\pause
  \end{itemize}
\end{PauseHighLight}

\end{slide}

%%%%%%%%%%%%%%%%%%%%%%% Next Slide %%%%%%%%%%%%%%%%%%%%%%%

\begin{slide}
\section{Linear Program}

\begin{PauseHighLight}
  \begin{itemize}
  \item We can write the full problem as
    \begin{align*}
     & \min_{\bm{x}} \sum_{c\in\mathcal{C}, f\in\mathcal{F}} p_{cf}\,
      x_{cf} \pause \\
      \text{subject to}& \\
      &\sum_{c\in\mathcal{C}} x_{cf} \leq b_f \quad \quad \forall f \in
      \mathcal{F} \\
      &\sum_{f\in\mathcal{F}} x_{cf} = d_c \quad \quad \forall c \in
      \mathcal{C} \\
      &x_{cf} \geq 0 \quad \quad \forall c \in \mathcal{C}, \quad
      \forall f \in \mathcal{F}\pause
    \end{align*}
  \end{itemize}
\end{PauseHighLight}

\end{slide}

%%%%%%%%%%%%%%%%%%%%%%% Next Slide %%%%%%%%%%%%%%%%%%%%%%%
\Outline % Linear Programs
%%%%%%%%%%%%%%%%%%%%%%% Next Slide %%%%%%%%%%%%%%%%%%%%%%%

\begin{slide}
\section{General Linear Programs}

\begin{PauseHighLight}
  \begin{itemize}
  \item Linear programs are problems that can be formulated as follows
    \begin{align*}
      &\min_{\bm{x}} \bm{c}\cdot \bm{x} \\
      \text{subject to}&\\
      &\mat{A}^{\scriptscriptstyle \leq} \bm{x} \leq
      \bm{b}^{\scriptscriptstyle \leq},  \quad
      \mat{A}^{\scriptscriptstyle \geq} 
      \bm{x} \geq \bm{b}^{\scriptscriptstyle \geq}, \quad 
      \mat{A}^{\scriptscriptstyle =} \bm{x} = \bm{b}^{\scriptscriptstyle
        =}, \quad \bm{x} \geq \bm{0}\pause 
    \end{align*}
  \item Note in the previous example it was convenient to use two
    indices $c$ and $f$ to denote the components $x_{cf}$, however, it
    still has this structure\pause
  \end{itemize}
\end{PauseHighLight}

\end{slide}


%%%%%%%%%%%%%%%%%%%%%%% Next Slide %%%%%%%%%%%%%%%%%%%%%%%

\begin{slide}
\section{Maximising}

\begin{PauseHighLight}
  \begin{itemize}
  \item We can also maximise rather than minimise\pause
  \item Whether we want to maximise or minimise will depend on the
    application\pause
  \item Note that
    \begin{align*}
      \max_{\bm{x}}\bm{c}\cdot \bm{x} \quad \equiv \quad \min_{\bm{x}}
      (-\bm{c}) \cdot \bm{x}\pause
    \end{align*}
  \item We can thus always reformulate a maximisation problem as a
    minimisation problem and vice versa\pause
  \end{itemize}
\end{PauseHighLight}

\end{slide}

%%%%%%%%%%%%%%%%%%%%%%% Next Slide %%%%%%%%%%%%%%%%%%%%%%%

\begin{slide}
\section{Linear Program Applications}

\begin{PauseHighLight}
  \begin{itemize}
  \item A huge number of problems can be mapped to linear programming problems\pause
  \item Or modelled as linear (even when they're not, e.g. oil
    extraction)\pause
  \item Realistic problems might have many more constraints and large
    number of variables\pause
  \item State of the art solvers can deal with problems with hundreds of
    thousands or even millions of variables\pause
  \end{itemize}
\end{PauseHighLight}

\end{slide}


%%%%%%%%%%%%%%%%%%%%%%% Next Slide %%%%%%%%%%%%%%%%%%%%%%%

\begin{slide}
\section{Key Features}

\begin{PauseHighLight}
  \begin{itemize}
  \item There are three key features of linear programs
    \begin{enumerate}
    \item The cost (objective function) is linear in $x_i$ ($\bm{c}\cdot
      \bm{x}$)
    \item The constraints are linear in $x_i$ (e.g. $\mat{A}_1 \bm{x}
      \leq b_1 $)
    \item The component of $\bm{x}$ are non-negative (i.e. $x_i \geq 0)$\pause
    \end{enumerate}
  \item These are very special features, very often they don't apply,
    but a surprising large number of problems can be formulated as
    linear programming problems\pause
  \end{itemize}
\end{PauseHighLight}

\end{slide}

%%%%%%%%%%%%%%%%%%%%%%% Next Slide %%%%%%%%%%%%%%%%%%%%%%%

\begin{slide}
\section{History}

\begin{PauseHighLight}
  \begin{itemize}
  \item Linear programming was ``invented'' by Leonid Kantorovich in
    1939 to help Soviet Russia maximise its production\pause
  \item It was kept secret during the war, but was finally made public
    in 1947 when George Dantzig published the \emph{simplex method}
    which still today is a standard method for solving linear
    programs\pause
  \item John von Neumann developed the idea of duality (you can turn a
    maximisation problem for a set of variables $\bm{x}$ into a
    minimisation problem for a dual set of variables $\bm{\lambda}$
    associated with each constraint)\pause
  \item von Neumann used this idea as the basis for ``game theory''\pause
  \end{itemize}
\end{PauseHighLight}

\end{slide}

%%%%%%%%%%%%%%%%%%%%%%% Next Slide %%%%%%%%%%%%%%%%%%%%%%%
\Outline % Linear Programs
%%%%%%%%%%%%%%%%%%%%%%% Next Slide %%%%%%%%%%%%%%%%%%%%%%%

\begin{slide}
\section{Structure of Linear Programs}

\begin{PauseHighLight}
  \begin{itemize}
  \item Before we go into the details of solving linear programs its
    useful to consider the structure of the solutions\pause
  \item The set of $\bm{x}$ that satisfy all the constraints is known as
    the set of \emph{feasible solutions}\pause
  \item The set of feasible solutions may be empty in which case it is
    impossible to satisfy all the constraints\pause
  \item This is rather disappointing, but usually doesn't happen if we
    have formulated a sensible problem\pause
  \end{itemize}
\end{PauseHighLight}

\end{slide}

%%%%%%%%%%%%%%%%%%%%%%% Next Slide %%%%%%%%%%%%%%%%%%%%%%%

\begin{slide}
\section{The Space of Feasible Solutions}

\pb
\pause \pauselevel{=1}
\begin{center}
  \multipdf[width=\linewidth]{feasibleSolutions}\pause
\end{center}
\end{slide}

%%%%%%%%%%%%%%%%%%%%%%% Next Slide %%%%%%%%%%%%%%%%%%%%%%%

\begin{slide}
\section{Vertices of Polytope}

\begin{PauseHighLight}
  \begin{itemize}
  \item The space of feasible solutions is a polyhedra or polytope\pause
  \item The maximum or minimum solution will always lie at a vertex of
    the polytope\pause
  \item Our solution policy will be to start at a vertex and move
    to a neighbouring vertex that gives the best improvement in cost\pause
  \item When this isn't possible then we are finished\pause
  \item However, there is still a lot of work to realise this solution
    strategy\pause
  \end{itemize}
\end{PauseHighLight}

\end{slide}


%%%%%%%%%%%%%%%%%%%%%%% Next Slide %%%%%%%%%%%%%%%%%%%%%%%

\begin{slide}
\section{Optimal Solution}

\pb
\pause \pauselevel{=1}
\begin{center}
  \multipdf[width=\linewidth]{lpImprove}\pause
\end{center}
\end{slide}


%%%%%%%%%%%%%%%%%%%%%%% Next Slide %%%%%%%%%%%%%%%%%%%%%%%

\begin{slide}
\section[-2]{Unbounded Solutions}

\pb \pause
\begin{itemize}
\item If you are unlucky you might not have a bounded solution\pauseh \pauselevel{=1}
  \begin{center}
    \multipdf[width=0.7\linewidth]{unboundedSolution}\pause
  \end{center}
\item But usually this would not happen because of the problem definition\pause
\end{itemize}
\end{slide}

%%%%%%%%%%%%%%%%%%%%%%% Next Slide %%%%%%%%%%%%%%%%%%%%%%%

\begin{slide}
\section[-2]{Multiple Solutions}

\pb \pause
\begin{itemize}
\item You can also get multiple solutions if a constraint is orthogonal
  to the objective function\pauseh \pauselevel{=1}
  \begin{center}
    \multipdf[width=0.7\linewidth]{multipleSolutions}\pause
  \end{center}
\item Nevertheless the optimal will be at a vertex\pause
\end{itemize}
\end{slide}




%%%%%%%%%%%%%%%%%%%%%%% Next Slide %%%%%%%%%%%%%%%%%%%%%%%
\Outline % Normal Form
%%%%%%%%%%%%%%%%%%%%%%% Next Slide %%%%%%%%%%%%%%%%%%%%%%%

\begin{slide}
\section{Converting Linear Programs}

\begin{PauseHighLight}
  \begin{itemize}
  \item Solving full linear programs is difficult\pause
  \item However, it is much easier to solve linear programs in
    \emph{normal form}\pause
  \item This is basically a form where we get rid of all
    inequalities and rewriting the equalities\pause
  \item Fortunately its rather easy to convert linear programs to normal
    form\pause
  \end{itemize}
\end{PauseHighLight}

\end{slide}

%%%%%%%%%%%%%%%%%%%%%%% Next Slide %%%%%%%%%%%%%%%%%%%%%%%

\begin{slide}
\section[-2]{Slack Variables}

\begin{PauseHighLight}
  \begin{itemize}
  \item We can change an inequality into an equality by introducing a
    new ``\emph{slack}'' variable\pause
  \item E.g.
    \begin{align*}
      \bm{a}_1\cdot\bm{x} &\geq 0 &\Rightarrow& &
      \bm{a}_1\cdot\bm{x} - z_1 &=0 \quad z_1 \geq 0 \\ 
      \bm{a}_2\cdot\bm{x} &\leq 0 &\Rightarrow& &
      \bm{a}_2\cdot\bm{x} + z_2 &=0 \quad z_2 \geq 0
    \end{align*}
    $z_1$ (the excess) and $z_2$ (the deficit) are known as slack
    variables\pause
  \item We eliminate inequalities at the expense of increasing the
    number of variables\pause
  \item We can treat the slack variables on an equivalent footing to the
    normal variables (they just provide a different way of describing
    the original problem)\pause
  \end{itemize}
\end{PauseHighLight}

\end{slide}

%%%%%%%%%%%%%%%%%%%%%%% Next Slide %%%%%%%%%%%%%%%%%%%%%%%

\begin{slide}
\section{Normal Form}

\begin{PauseHighLight}
  \begin{itemize}
  \item A linear program with only equality constraints is said to be in
    \emph{normal form}\pause
  \item We will find in the next lecture that this is a convenient form
    for solving linear programs\pause
  \item An equality constraint restricts the solutions to a subspace
    (some lower dimensional space)\pause
  \end{itemize}
\end{PauseHighLight}

\end{slide}

%%%%%%%%%%%%%%%%%%%%%%% Next Slide %%%%%%%%%%%%%%%%%%%%%%%

\begin{slide}
\section[-2]{Solving Linear Programming}

\pb\pause \pauselevel{=1}
\begin{minipage}{0.5\linewidth}
\begin{center}
  \multipdf[width=\linewidth]{normalForm}\pause
\end{center}  
\end{minipage}\hfil
\begin{minipage}{0.45\linewidth}
  \begin{itemize}
  \item The basic feasible points for LP problems with $n$ variables and
    $m$ constraints have at least $n-m$ zero variables\pause
  \item Typical number of basic feasible solutions is
    $\binom{n}{m}\geq\left(\frac{n}{m}\right)^m$\pause
  \item Simplex algorithm organises iterative search for global solutions\pause
  \end{itemize}
\end{minipage}

\end{slide}



%%%%%%%%%%%%%%%%%%%%%%% Next Slide %%%%%%%%%%%%%%%%%%%%%%%

\begin{slide}
\section{Lessons}
 
\begin{PauseHighLight}
  \begin{itemize}
  \item There are a huge number of problems that can be set up as linear
    programs\pause
  \item They are particularly useful in resource allocation where the
    resources are all positive\pause
  \item The solution to linear programming problems is at the vertex of
    the feasible space (intersection of constraints)\pause
  \item We can search for solutions by moving from vertex to vertex\pause
  \item We can transform inequality constraints to equality constraints
    using slack variables\pause
  \end{itemize}
\end{PauseHighLight}

\end{slide}

